%Begin








%%%%%%%%%%%%%%%%%%%%%%%%%%%%%%%%%%%%%%%%%%%%%%%%%%%%%%%%%%%%%%%
\chapter*{\S \space 1. --- TOPOS MULTIGALOISIENS}\thispagestyle{empty}
\addcontentsline{toc}{section}{1. Topos multigaloisiens}
\label{sec:1}
\section*{}

{
Proposition ({\bf 1.1}). --- \it Soit $E$ une catégorie. Conditions équivalentes :
\begin{enumerate}
    \item[a)] $E$ est un topos, et tout objet de $E$ est localement constant.
    \item[b)] $E$ est équivalent à une catégorie $\widehat{C}$, où $C$ est un groupoïde 
    (N.B. On verra plus loin que l'on peut choisir $C$ \emph{canoniquement}).
    \item[b')] Il existe une famille $(G_i)_{i \in I}$ de groupes et une équivalence de catégories
    $$
    E \xlongrightarrow{\approx} \prod_{i \in I}\Ens(G_i)
    $$
    \item[c)] Conditions d'exactitudes ad-hoc, du type de celles données dans SGA 1\dots
\end{enumerate}
}
\vskip .5cm

{\bf Démonstration} : b) $\Rightarrow$ b') $\Rightarrow$ a) immédiat. Pour a) $\Rightarrow$ b) je suis moins sur, peut être faut-il supposer que $E$ est localement connexe, et qu'il a suffisamment de foncteurs fibres i.e. suffisamment de points.
\vskip .5cm
{
Définition ({\bf 1.2}). --- \it Si les conditions équivalentes b), b') ci-dessus sont satisfaites, on dit que 
$C$ est un \emph{topos multigaloisien} (ou une \emph{catégorie multigaloisienne}).
}

\vskip .5cm
{
Proposition ({\bf 1.3}). --- \it
\begin{enumerate}
    \item[a)] Si $E$ est multigaloisien tout topos induit $C/S$ aussi.
    \item[b)] Toute somme de topos multigaloisiens (i.e. tout produit de catégories multigaloisiennes)
    est un topos multigaloisien.
\end{enumerate}
}

\vskip .5cm

{
Proposition ({\bf 1.4}). --- \it Soient $E$ un topos, $C$ la catégorie des points de $E$, (opposée à 
la catégorie des foncteurs fibres sur $E$). Le foncteur canonique $E \times C^{\circ} \to \Ens$ induit 
un foncteur canonique
$$
E \to \Hom(E^\circ, \Ens) \defeq \widehat{C}
$$
Ceci posé [$E$ étant multigaloisien]
\begin{enumerate}
    \item[a)] $C$ est un groupoïde (appelé groupoïde fondamental du topos multigaloisien $E$ et souvent
    noté $\Pi_1(E)$).
    \item[b)] $E \to \widehat{C}$ est une équivalence de catégories.
\end{enumerate}
}
\vskip .5cm

Un objet $S$ d'un topos $E$ est dit $0$-connexe s'il est $\neq \emptyset_E$ (i.e. n'est pas objet initial) et s'il est connexe (i.e. $S \isom S' \amalg S''$ implique $S'$ ou $S'' \isom \emptyset_E$) - cela signifie aussi que le topos induit $E/S$ est $0$-connexe i.e. n'est pas le topos initial (``topos vide'', équivalent à la catégorie finale) et qu'il est connexe, i.e. \dots

On dit que $S$ est \emph{$1$-connexe} (ou \emph{simplement connexe}) s'il est $0$-connexe et si tout objet $S'$ de $E/S$ localement constant est constant - ce qui ne dépend encore que du topos induit $E/S$, qui sera dit alors \emph{$1$-connexe}.

\vskip .5cm

{
Proposition ({\bf 1.5}). --- \it Soit $E$ un topos multigaloisien, et soit $S$ un objet de $E$. Conditions équivalentes :
\begin{enumerate}
    \item[a)] $S$ est $1$-connexe
    \item[b)] $S$ est $0$-connexe et projectif
    \item[c)] Le foncteur covariant représenté par $S$
    $$
    T \mapsto \Hom_E(S, T)
    $$
    est un foncteur fibre, ou encore (comme il est déjà exact à gauche) il commute aux $\varinjlim$
    inductives quelconques (N.B. il suffit qu'il commute aux sommes et aux passages aux quotients \dots)
    \item[d)] $E/S$ est équivalent au topos ponctuel
    \item[e)] (Si $E = \widehat{C}$, $C$ un groupoïde) le foncteur $S$ sur $C$ est représentable.
\end{enumerate}
}
\vskip .5cm

{
Définition ({\bf 1.6}). --- \it On dit alors, parfois que $S$ (ou mieux, le topos $E/S$) est un \emph{revêtement universel} du topos multigaloisien $E$.
}

\vskip .5cm

{
Proposition ({\bf 1.7}). --- \it Soit $E_\circ$ la sous-catégorie pleine de $E$ formée des objets
$1$-connexes (de la catégorie multigaloisienne $E$), et $\Pi_1(E)$ le groupoïde fondamental de $E$. On a par ({\bf 1.5}). un foncteur canonique
$$
E_\circ \to \Pi_1(C)
$$
(associant à tout $S \in Ob(E_\circ)$ le foncteur fibre qu'il représente, ou plutôt le ``point'' correspondant de $C$), qui est (non seulement pleinement fidèle mais même) une équivalence de catégorie :
tout foncteur fibre sur $E$ est représentable (par un objet ($1$-connexe) essentiellement unique comme de juste \dots)
}

\vskip .5cm
{
Corollaire ({\bf 1.8}). --- \it Soit $P$ un ``point'' de $E$ (associé à un foncteur fibre $F_P$). Il existe un objet $1$-connexe $S$ de $E$ et un relèvement
\[\begin{tikzcd}
	P && {E/S} \\
	& E
	\arrow[from=1-1, to=2-2]
	\arrow[from=1-3, to=2-2]
	\arrow["\alpha", from=1-1, to=1-3]
\end{tikzcd}\]
(i.e. $\alpha \in F(S)$) et cela détermine $(S, \alpha)$ à isomorphisme près.

En fait, $S$ est l'unique objet de $E$ qui représente $F_P$ \dots
}
\vskip .5cm

{
Définition ({\bf 1.9}). --- \it On dit que $S$ (ou $E/S$) est le \emph{revêtement universel ponctué au dessus} de    $P$ déterminé par le point $P$.
}
\vskip .5cm
{
Scholie ({\bf 1.10}). --- \it Se donner un ``point'' du topos multigaloisien $E$, ou se donne un revêtement
universel, revient essentiellement au même : chacun détermine l'autre \dots
}

\vskip .5cm

{
Proposition ({\bf 1.11}). --- \it Soient $E$, $E'$ deux topos multigaloisiens, $\Pi_1(E)$, $\Pi_1(E')$
leur groupoïdes fondamentaux. Le foncteur évident
$$
\underline{\Hom}_{\text{top}}(E, E') \to \underline{\Hom}(\Pi_1(E), \Pi_1(E'))
$$
est une équivalence de catégories ; posant $C = \Pi_1(E)$, $C' = \Pi_1(E')$ on trouve une équivalence 
quiasi-inverse en composant
$$\diagram
\underline{\Hom}(C, C') & \rArr & \underline{\Hom}_{\text{top}}(\widehat{C}, \widehat{C'})
&~~~ \xlongrightarrow{\approx} ~~& \underline{\Hom}_{\text{top}}(E, E') \\
& & \cap & &  & &  & & \\
 &  & \underline{\Hom}(\widehat{C'}, \widehat{C})
&  &   \\
\enddiagram$$
}

\vskip .5cm

({\bf 1.12}). {\bf Explicitation du cas ou $E$, $E'$ sont $0$-connexes et ponctués, donc donnés comme
$E \isom \Ens(G)$, $E' \isom \Ens(G')$ \dots}












%%%%%%%%%%%%%%%%%%%%%%%%%%%%%%%%%%%%%%%%%%%%%%%%%%%%%%%%%%%%%%%
\chapter*{\S \space 2. --- APPLICATIONS AUX REVÊTEMENTS DES TOPOS}\thispagestyle{empty}
\addcontentsline{toc}{section}{2. Application aux revêtements des topos}
\label{sec:2}
\section*{}

{
Théorème ({\bf 2.1}). --- \it Soit $E$ un topos localement connexe (i.e. dont tout objet est somme d'objets connexes) et localement simplement connexe (i.e. admettant un système de générateurs qui sont $1$-connexes)\footnote{N.B. peut-être faut-il supposé que $E$ ait ``assez de points'' i.e. assez de foncteurs libres\dots}. Alors la catégorie $E_{lc}$ des objets localement constants de $E$ est un topos multigaloisien, et l'inclusion
$$
E_{lc} \hookrightarrow E \leqno{(2.1.1.)}
$$
commute aux $\varprojlim$ finies (N. B. en fait sans hypothèses sur le topos $E$, $E_{lc}$ est stable par $\varinjlim$ finies) \emph{et} aux $\varinjlim$ quelconques.
}

\vskip .5cm

{
Définition ({\bf 2.2}). --- \it On dénote ce topos par $E_{[1]}$, on l'appelle l'enveloppe multigaloisienne
de $E$ et le morphisme de topos transposé de l'inclusion (2.1.1.) : 
$$
E \to E_{[1]}
$$
prend le nom de \emph{morphisme canonique}.
}

\vskip .5cm

N.B. C'est l'équivalent en théorie des topos de l'opération de ``tuage des $\pi_i$ pour $i \geq 2$''.

On peut définir aussi $E_{[0]}$ et une suite
$$
E \to E_{[1]} \to E_{[0]}
$$
($E_{[0]}$ est le topos \emph{discret} défini par $\pi_0(E)$, qui a un sens satisfaisant dès que $E$
localement connexe \dots).

Moyennant des hypothèses convenables sur $E$ (du type ``locale contractibilité''), on doit pouvoir définir les $E_{[i]}$ pour tout $i \in \mathbf{N}$, et des morphismes canoniques
$$
E \to \dots E_{[i]} \to E_{[i-1]} \to \dots E_{[1]} \to E_{[0]}
$$

{\bf 2.3}. Le \emph{groupoïde fondamental} de $E$ se définit comme ayant pour objets les points de $E$ (qui induisent des points de $E_{[1]}$ grâce à $E_{[1]} \to E$), et comme morphismes les morphismes \emph{de points} de $E_{[1]}$. On a donc des foncteurs canoniques
$$
\underline{\Pt}(E) \xlongrightarrow{\alpha} \Pi_1(E) \xlongrightarrow[\approx]{\beta} \underline{\Pt}(E_{[1]}) \defeq \Pi_1(E_{[1]})
$$
où $\beta$ est une équivalence (mais pas surjectif sur les objets), et où bien sur $\alpha$ n'est pas 
nécessairement une équivalence ni même pleinement fidèle, ou seulement fidèle.

Par exemple si $E$ est $1$-connexe (i.e. $\Pi_1(E)$ équivalent à la catégorie ponctuelle, il ne s'ensuit 
pas nécessairement que les $\Hom$ dans $\underline{\Pt}(E)$ soient tous de cardinal $\leq 1$ !)

Comme un point $P$ de $E$ définit un point (noté encore $P$ par abus) de $E_{[1]}$, on peut donc définir 
le \emph{revêtement universel de} $E$ basé en ce point, comme un objet $S$ $1$-connexe de $E_{[1]}$ - il 
est caractérisé dans $E$ par le fait d'être localement constant, $1$-connexe, et muni d'un relèvement
$$
P \to E/S
$$
Mais comme $\alpha$ n'est pas une équivalence de catégorie (bien qu'il soit essentiellement surjectif si 
on suppose que $E$ a suffisamment de points \dots) on \emph{ne peut pas} dire que tout revêtement universel 
de $E$ soit défini à isomorphisme unique près \dots.















%%%%%%%%%%%%%%%%%%%%%%%%%%%%%%%%%%%%%%%%%%%%%%%%%%%%%%%%%%%%%%%
\chapter*{\S \space 3. --- VARIANTES ``PRO-MULTIGALOISIENNES'', RESPECTIVEMENT PROFINIES}\thispagestyle{empty}
\addcontentsline{toc}{section}{3. Variantes pro-multigaloisiennes}
\label{sec:3}
\section*{}

(en se bornant, pour simplifier, au cas des topos localement connexes \dots)













%%%%%%%%%%%%%%%%%%%%%%%%%%%%%%%%%%%%%%%%%%%%%%%%%%%%%%%%%%%%%%%
\chapter*{\S \space 4. --- COMPLÉMENT-REMORD SUR LES CATÉGORIES MULTIGALOISIENNES,}\thispagestyle{empty}
\addcontentsline{toc}{section}{4. Compléments, remords}
\label{sec:4}
\section*{}



Qui précise l'intention que pour un topos $E$, la donnée d'un objet $S \in E$ définit un topos induit
$E/S \to E$, et que $S$ se reconstitue à isomorphisme près par la connaissance du topos induit en 
tant que topos \emph{au dessus} de $E$. Ici, $E$ étant multigaloisien, $E/S$ aussi - et il se pose la 
question quand un morphisme de topos multigaloisien $E' \to E$ peut être considéré comme un morphisme
d'induction. Si $C = \Pi_1(E)$, $C' = \Pi_1(E')$, la donnée de $E' \to E$ équivaut à la donnée d'un 
foncteur $C' \to E$.

On trouve que $E' \to E$ est un morphisme d'induction si et seulement si $C' \to C$ est \emph{fidèle}.
Ainsi, on trouve une équivalence entre la catégorie $E$ (des objets $S$ de la catégorie multigaloisienne
$E \isom \widehat{C}$, où $C$ est un groupoïde quelconque si on y tient\footnote{un peu vif !}) et la
catégorie dont les objets sont les ``groupoïdes $C'$ au dessus de $C$'', avec un foncteur structural 
$C' \to C$ \emph{fidèle}, les morphismes de $C'_1$ dans $C'_2$ étant les \emph{classes d'isomorphie}\footnote{préciser les isomorphismes entre couples $(f, \alpha)$ et $(g, \beta)$ \dots} de couples
$(f, \alpha)$ d'un foncteur $f: C'_1 \to C'_2$ et d'un isomorphisme de foncteurs 
$\alpha: p_1 \isommap p_2 \circ f$
\[\begin{tikzcd}
	{C'_1} && {C'_2} \\
	& C
	\arrow["f", from=1-1, to=1-3]
	\arrow[""{name=0, anchor=center, inner sep=0}, "{p_2}", from=1-3, to=2-2]
	\arrow[""{name=1, anchor=center, inner sep=0}, "{p_1}"', from=1-1, to=2-2]
	\arrow["\alpha", shift right=2, draw=none, from=1, to=0]
\end{tikzcd}\]

Dans le cas où par exemple $C$ est la catégorie réduite à un seul objet, avec groupe d'automorphisme $G$,
cette description de la catégorie $E = \Ens(G)$ est évidemment un peu lourde, mais elle s'insère bien dans certains contextes plus bas.

Ainsi, si $k$ est un corps de base, la catégorie $E$ des schémas étales sur $k$ se décrit, en terme d'une
clôture séparable $k_s$ de $k$ et du groupe profini $\Gamma = \Gal(k_s/k)$, comme les groupoïdes profinis 
au dessus du groupoïde profini $(\pt, \Gamma)$ \dots Nous voulons insérer cette description dans une 
``description'' ``galoisienne'' de [certains] schémas [lisses quasi-projectifs de dimension $\leq 1$]
sur $k$, du moins si $k$ corps de type fini sur $\mathbf{Q}$.
















%%%%%%%%%%%%%%%%%%%%%%%%%%%%%%%%%%%%%%%%%%%%%%%%%%%%%%%%%%%%%%%
\chapter*{\S \space 5. --- INTRODUCTION DU CONTEXTE ARITHMÉTIQUE; ``CONJECTURE ANABÉLIENNE'' FONDAMENTALE}\thispagestyle{empty}
\addcontentsline{toc}{section}{{\bf 5.} Introduction du contexte arithmétique ; ``conjecture anabélienne'' fondamentale}
\label{sec:5}
\section*{}

Soit $K$ une extension de type fini de $\mathbf{Q}$, et choisissons une clôture algébrique $\overline{K}$ de $K$. On pose $\Gamma = \Gal(\overline{K}/K)$.

{\bf 5.1}. Nous considérons des couples $(X, S)$, où :
\begin{enumerate}
    \item[a)] $X$ est un schéma projectif et lisse sur $K$, de dimension $\leq 1$ ;
    \item[b)] $S$ est sous schéma fini réduit de $X$ (donc fini étale sur $K$) contenu dans la réunion
    des composantes irréductibles de dimension $1$ de $X$.
\end{enumerate}

Les morphismes $(X', S') \to (X, S)$ seront par définition les morphismes de schémas
$$
f: X' \to X
$$
tels que
$$
S' = f^{-1}(S)_{\text{\text{red}}}
$$
i.e. tels que $\text{supp} S' = f^{-1}(\text{supp} S)$.

Nous cherchons une ``description galoisienne'' de cette catégorie, ou tout au moins d'une 
sous-catégorie pleine $V_K$ que nous allons définir maintenant.

\vskip .5cm

{
Lemme {\bf (5.2)}. --- \it Soit $\Omega$ un corps algébriquement clos, $X$ une courbe projective lisse
connexe sur $\Omega$, $S$ une partie finie de $X(\Omega)$, $U = X \textbackslash S$, $g$ le genre 
de $X$ et $n = \card S$. Conditions équivalentes :
\begin{enumerate}
    \item[a)] $\pi_1(U)$ non abélien,
    \item[b)] $\Aut(U)$ fini,
    \item[c)] pour tout schéma connexe réduit $X$ de type fini sur $\Omega$, l'ensemble des morphismes \emph{non constants} de $X$ dans $U$ est fini,
    \item[d)] on est dans l'un des trois cas suivant : $1 ^{\circ})$ $g \geq 2$ $2 ^{\circ})$ $g = 1$, $n \geq 1$ $3 ^{\circ})$ $g = 0$, $n \geq 3$
    \item[e)] (si $\Omega \subset  \mathbf{C}$) le revêtement universel de $X(\mathbf{C}) \textbackslash S(\mathbf{C})$ est isomorphe au demi plan de Poincaré,
    \item[f)] (??) (si $\Omega = \overline{\mathbf{Q}}$, $S \neq \emptyset$) Le revêtement universel de $X \textbackslash S = U$ est isomorphe à celui de $\mathbb{P}^1_{\Omega} \textbackslash \{ 0, 1, \infty \}$.
\end{enumerate}
}

\vskip .5cm
{
Définition {\bf (5.3)}. --- \it On dit alors que $(X, S)$ est anabélien.
}
\vskip .5cm

Comme cette condition est (par d) par exemple) invariante par extension du corps de base
algébriquement clos, on étend cette définition au cas d'un couple $(X, S)$, avec $(X, S)$ comme dans
(5.1) (N.B. On regarde séparément les composantes connexes de $X_{\overline{K}}$\dots). Dorénavant, dans (5.1) nous allons nous borner au cas de couples $(X, S)$ anabéliens.

{\bf (5.4)}. A un couple $(X, S)$ (pas nécessairement anabélien) - plus généralement à tout schéma $X$ localement de type fini sur $S$ - on associe un objet ``de nature galoisienne'' [à] savoir le groupoïde fondamental profini $\Pi(X)$ de $X$ (fermé (?) si on veut des revêtements universel de $X$), \emph{muni} d'un foncteur canonique
\[\begin{tikzcd}
	{\Pi_1(X)} & {\Pi_1(K)} \\
	& {[\Tors (\Gamma)]}
	\arrow["\approx"', from=1-2, to=2-2]
	\arrow[from=1-1, to=1-2]
\end{tikzcd}\]

Un morphisme de $K$-schémas
$$
X' \xlongrightarrow{f} X
$$
définit un foncteur
$$
\Pi_1(X') \xlongrightarrow{\Pi_1(f)} \Pi_1(X)
$$
et un isomorphisme de commutation $\alpha$:
\[\begin{tikzcd}
	{\Pi_1(X')} && {\Pi_1(X)} \\
	& {\Pi_1(K)}
	\arrow[""{name=0, anchor=center, inner sep=0}, from=1-1, to=2-2]
	\arrow[""{name=1, anchor=center, inner sep=0}, from=1-3, to=2-2]
	\arrow["{\Pi_1(f)}", from=1-1, to=1-3]
	\arrow["\alpha", shorten <=12pt, shorten >=12pt, from=0, to=1]
\end{tikzcd}\]

On trouve ainsi un foncteur, de la catégorie des schémas localement de type fini $X$ sur $K$, dans la ``catégorie des groupoïdes profinis sur $\Pi_1(K)$'', définie comme au $n^{\circ} 4$.

Quand on passe à la catégorie des schémas localement de type fini connexes, \emph{munis d'un point géométrique au dessus de $\overline{K}/K$}\footnote{il vaut mieux dire : munis d'un revêtement universel\dots} (i.e. d'un $x \in X$, d'une clôture séparable $\overline{k(x)}$ de $k(x)$ et d'un $K$-morphisme $\overline{K} \hookrightarrow \overline{k(x)}$), cela correspond à un foncteur des $K$-schémas localement de type fini et connexes, ponctués sur $\overline{K}/K$ (au ses précédent) vers la catégorie des groupes profinis $\Pi$ munis d'un homomorphisme (de groupes profinis)
$$
\Pi \to \Gamma
$$
(dont l'image sera d'ailleurs nécessairement ouverte donc d'indice fini, pour des objets provenant de $X$ comme [ci-]dessus).

\vskip .5cm
{
Conjecture {\bf (5.5)}\footnote{c'est un peu faux cf $n^{\circ} 9$} --- \it La restriction du foncteur précédent $X \mapsto (\Pi_1(X)~\text{sur}~\Pi_1(K))$ aux schémas projectifs lisses de dimension $\geq 1$ et anabéliens (i.e. tels que $(X, S)$ soit anabélien, où $S$ est la réunion des composantes de dimension $0$) est pleinement fidèle.
}
\vskip .5cm
Il revient au même de dire ceci:

\vskip .5cm
{
Définition {\bf (5.5 bis)}. --- \it Le foncteur qui, à tout $X$ comme dans (5.5.) et de plus \emph{connexe}, (de dimension $0$ ou $1$), muni d'un point géométrique $\xi$ au dessus de $\overline{K}$, associe le groupe profini $\pi_1(X, \xi)$ sur $\Gamma = \pi_1(K, \xi)$, est un foncteur pleinement fidèle.
}
\vskip .5cm

Il faut quand même expliciter les morphismes $(X, \xi) \to (X', \xi')$ dans la catégorie de départ : morphismes de $K$-schémas $X \xlongrightarrow{f} X'$, munis d'un morphisme (ou classe de chemins) $f(\xi) \isom \xi'$.

Ces conjectures se réduisent à la théorie de Galois, pour des $X$ de dimension $0$. Pour des $X$ de dimension $1$, elles ne concernent que des $X$ tels que les composantes connexes de $X_{\overline{K}}$ soient de genre $\geq 2$ (ou, ce qui revient au même, introduisant l'extension finie $K' = \mathrm{H}^0(X, \underline{\cO}_X)$) de $K$, de sorte que $X$ soit géométriquement connexe sur $K'$, tel que $X$ comme courbe algébrique sur $K'$ soit de genre $\geq 2$. On voit aisément (prenant $X' = \Spec(K)$, $X = \mathbb{P}^1_K$ courbe elliptique sur $K$) qu'elles deviennent fausses sinon - c'est pourquoi il a fallu introduire $S$, plus l'hypothèse anabélienne sur $(X, S)$, pour associer à $(X, S)$ une structure plus riche que $\Pi_1(X)$ sur $\Pi_1(K)$. On trouvera des conjectures (par exemple) pour $X$ courbe géométriquement connexe sur $K$ de genre $1$ (resp. $0$), \emph{pourvu} que $S$ soit de degré $\geq 1$ (resp. $\geq 3$).










%%%%%%%%%%%%%%%%%%%%%%%%%%%%%%%%%%%%%%%%%%%%%%%%%%%%%%%%%%%%%%%
\chapter*{\S \space 6. --- ANALYSE LOCALE DE $(X,S)$ EN UN $s\in S$}\thispagestyle{empty}
\addcontentsline{toc}{section}{6. Analyse locale de $(X,S)$ en un $s\in S$}
\label{sec:6}
\section*{}

On s'intéresse au cas où dim$_s(X) = 1$, i.e. où $s$ n'est pas point isolé dans $X$.

Soit $\underline{\cO}_s$ le hensélisé (ou le complété, si on y tient) de $\underline{\cO}_{X, s}$, $K_s$ son corps de fractions, $D^*_s = \Spec (K_s)$, on identifie $s$ à $\Spec k(s)$ ($k(s)$ est le corps résiduel de l'anneau-jauge $\underline{\cO}_s$). Considérons $D_s = \Spec (\underline{\cO}_s)$ (``disque arithmétique relatif à $k(s)$''), donc $D^*_s = D_s$ \textbackslash $s = ~(\text{``disque épointé''}) \to D_s$, on a :
\[\begin{tikzcd}
	{\Pi_1(D^*_s)} & {\Pi_1(D_s)} & {\Pi_1(K)} \\
	& {\Pi_1(s)}
	\arrow["{\sigma_s}"', from=1-1, to=2-2]
	\arrow["\approx", from=2-2, to=1-2]
	\arrow["{j_s}"', from=2-2, to=1-3]
	\arrow["{q_s}"', from=1-2, to=1-3]
	\arrow["{\text{fidèle}}", from=1-2, to=1-3]
	\arrow["{i_s}"', from=1-1, to=1-2]
	\arrow["{\text{épi sur} \Hom}", from=1-1, to=1-2]
	\arrow["{p_s}", curve={height=-12pt}, draw=none, from=1-1, to=1-3]
\end{tikzcd}\leqno{(6.1)} \]

Pour le choix d'un point géométrique $\xi_s$ de $D^*_s$ sur $\overline{K}/K$ (i.e. d'une clôture algébrique $\overline{K}_s$ de $K_s$ et d'une $K$-injection $\overline{K} \hookrightarrow \overline{K}_s$), ce diagramme de groupoïdes se reflète en un homomorphisme de groupes \footnote{N.B. Le choix de $\overline{K}_s$ implique un choix de $\overline{k}_s$ - c'est la flèche pointillée (6.1).} de
\[\begin{tikzcd}
	{\pi_1(D^*_s, \xi_s)} && {\pi_1(k(s), \xi_s)} && \Gamma \\
	\\
	{\Gal(\overline{K_s}/K_s)} && {\Gal(\overline{k(s)}/k(s))}
	\arrow["\sim"', from=1-3, to=3-3]
	\arrow["{\text{surjectif}}", from=1-1, to=1-3]
	\arrow["{\text{injectif}}", hook, from=1-3, to=1-5]
	\arrow["\sim", from=3-1, to=1-1]
\end{tikzcd}\leqno{(6.2)}
\]

dont le noyau, on le sait par Kummer, est canoniquement isomorphe à $T(\overline{k}_s) \isom T(\overline{K}_s)$ [$\isom T(\overline{K})$].

On veut exprimer la donnée de cet isomorphisme prévilégié comme une structure supplémentaire sur (6.1) - i.e. sur le groupoïde $\Pi_1(D^*_s)$ sur $\Pi_1(s)$ (ou sur $\Pi_1(K)$) - On peut le dire ainsi : si à tout $\xi \in \Pi_1(D^*_s)$, on associe le noyau de 
$$
\Aut(\xi) \to \Aut(i(\xi))
$$
(qui est aussi le noyau des composés
$$
\Aut(\xi) \to \Aut(i(\xi)) \to \Aut(p_s(\xi) = q_s(i_s(\xi)))
$$
on trouve un groupe \emph{abélien}, qui ne dépend (à isomorphisme près) que de $i(\xi) = \xi'$ [ceci, et la suite de la phrase, marche chaque fois qu'on a un foncteur de groupoïdes connexes à noyau abélien et surjectif sur les $\Hom$], et pour $\xi'$ variable forme un système local sur $\Pi_1(D_s)$, qu'on peut appeler le $\pi_1$ \emph{relatif} du groupoïde $\Pi_1(D^*_s)$ sur le groupoïde $\Pi_1(D_s)$.

Ceci dit, on a un isomorphisme de systèmes locaux de groupes
$$
\pi_1(\Pi_1(D^*_s)~\text{sur}~\Pi_1(D_s)) \isom q^*_s(T_K)
$$
où $T_K$ est le système local de Tate sur $K$.

Posons maintenant
$$
D_S = \amalg_{s \in S} D_s \quad \text{(``multidisque arithmétique en}~S\text{''})
$$
$$
D^*_S = \amalg_{s \in S} D^*_s \quad \text{(``multicouronne arithmétique en}~S\text{''})
$$
On a un homomorphisme de groupoïdes
$$
\Pi_1(D^*_s) \xlongrightarrow{\sigma_s} \Pi_1(S) \quad (\xlongrightarrow{j_s} \Pi_1(K))
$$
et un isomorphisme canonique
$$
\Pi_1(D^*_s) / \Pi_1(S)) \isom j^*_S(T(K))
$$
%review
Ceci posé, on a aussi un morphisme
$$
D^*_S \xlongrightarrow{\rho_S} X \textbackslash S
$$
induisant
$$
\Pi_1(D^*_S) \xlongrightarrow{\Pi_1(\rho_S)} \Pi(X \textbackslash S).
$$












%%%%%%%%%%%%%%%%%%%%%%%%%%%%%%%%%%%%%%%%%%%%%%%%%%%%%%%%%%%%%%%
\chapter*{\S \space 7. --- REFORMULATION ``BORDÉLIQUE'' DE LA CONJECTURE (LE PURGATOIRE NÉCESSAIRE \dots)}\thispagestyle{empty}
\addcontentsline{toc}{section}{7. Reformulation ``bordélique'' de la conjecture (le purgatoire nécessaire\dots)}
\label{sec:7}
\section*{}

Ainsi, à $(X, S)$ comme dans ({\bf 5.1.}), on associe :
\begin{enumerate}
    \item[1$^{\circ})$] Trois groupoïdes (profinis) $\Pi_U$, $\Pi_D$, $\Pi_{D^*}$ (en plus de $\Pi_e = \Pi_1(\Spec(K)) (\isom \Tors(\Gamma))$).
    \item[2$^{\circ})$] Quatre foncteurs (de groupoïdes profinis) :
    $$
    \begin{array}{c}
    \xymatrix{ & \Pi_{D^*} \ar[dl]_{\rho} \ar[dr]^{\sigma} && \\
    \Pi_U \ar[dr]_{\phi} && \Pi_D \ar[dl]^{\psi}& \\
    & \Pi_e && }
    \end{array}
    $$
    \item[3$^{\circ})$] Un isomorphisme de commutation 
    $$
    \alpha: \phi \rho \isom \psi \sigma
    $$
    (qui est même l'identité dans le cas de système provenant de $(X, S)$, mais il vaut mieux oublier qu'il en soit ainsi). Ces données satisfaisant aux conditions préliminaires
\end{enumerate}
\begin{enumerate}
    \item[a)] $\sigma$ induit un isomorphisme sur les $\pi_0$, et des épimorphismes sur les $\Hom$, et il est à noyau abélien ; 
    \item[b)] $\psi$ est fidèle.
    \item[[ c)] $\rho$ est fidèle\dots
    \item[d)] $\phi$ est épimorphique modulo groupes finis sur les $\Aut$\dots ]
\end{enumerate}
La condition a) permet déjà de définir le $\pi_1$ relatif $\pi_1(\sigma) = \pi_1(\Pi_{D^*}/\Pi_D)$, qui est un système local de groupes abéliens sur $\Pi_D$ i.e. un foncteur $(\Pi_D)^{\circ} \to \Ens$, et la dernière donnée
\begin{enumerate}
    \item[4$^{\circ})$] Un isomorphisme kummérien 
    $$
    \kappa: \pi_1(\sigma) \isom \psi^*(T)
    $$
\end{enumerate}
Si on a deux systèmes de cette nature $\Pi = (\Pi_U, \Pi_D, \Pi_{D^*}, \phi, \psi, \rho, \sigma, \alpha, \kappa)$ et $\Pi' = (\Pi_U,\dots)$, un \emph{morphisme} de $\Pi'$ dans $\Pi$ est un système de trois foncteurs 
$$
f_U: \Pi'_U \to \Pi_U
$$
$$
f_D: \Pi'_D \to \Pi_D
$$ 
$$
f_{D^*}: \Pi'_{D^*} \to \Pi_{D^*}
$$ 
et de quatre isomorphismes de commutation $\alpha_{D^*, D}$, $\alpha_{D^*, U}$, $\alpha_{U, e}$, $\alpha_{D, e}$, pour les quatre faces du prisme :
\[\begin{tikzcd}
	&&&&& {\Pi'_{D^*}} \\
	&&&& {\Pi'_U} && {\Pi'_D} \\
	& {\Pi_{D^*}} &&&& {\Pi_e} \\
	{\Pi_U} && {\Pi_D} \\
	& {\Pi_e}
	\arrow["{\psi'}", from=2-7, to=3-6]
	\arrow["{\sigma'}", from=1-6, to=2-7]
	\arrow["{\rho'}", from=1-6, to=2-5]
	\arrow["{\varphi'}", from=2-5, to=3-6]
	\arrow[from=3-6, to=5-2]
	\arrow["\psi"', from=4-3, to=5-2]
	\arrow["\varphi"', from=4-1, to=5-2]
	\arrow["\rho"', from=3-2, to=4-1]
	\arrow["\sigma", from=3-2, to=4-3]
	\arrow["{f_U}"', from=2-5, to=4-1]
	\arrow["{f_{D^*}}"', from=1-6, to=3-2]
	\arrow["{f_D}"', from=2-7, to=4-3]
\end{tikzcd}\]
satisfaisant une équation de compatibilité avec $\alpha$, $\alpha'$ que je n'écris pas - signifiant que les \emph{deux} isomorphismes $u$, $v$ de foncteurs
\[\begin{tikzcd}
	{\Pi'_{D^*}} & {\Pi_e}
	\arrow[""{name=0, anchor=center, inner sep=0}, "{\phi \circ \rho \circ f_{D^*}}", shift left=3, from=1-1, to=1-2]
	\arrow[""{name=1, anchor=center, inner sep=0}, "{\id \circ \phi' \circ \sigma'}"', shift right=3, from=1-1, to=1-2]
	\arrow["v", shift left=1, shorten <=2pt, shorten >=2pt, from=0, to=1]
	\arrow["u"', shift right=1, shorten <=2pt, shorten >=2pt, from=0, to=1]
\end{tikzcd}\]
obtenus respectivement, $u$ en utilisant successivement $\alpha$, $\alpha_{D^*, D}$, $\alpha_{D, e}$, $v$ en utilisant successivement $\alpha_{D^*, U}$, $\alpha_{U, e}$, $\alpha'$, sont égaux. [Cette compatibilité pourrait s'exprimer en interprétant la donnée de $\Pi$ comme celle d'une catégorie fibrée $\Pi$ sur la ``catégorie carrée''
    $$
    \begin{array}{c}
    \xymatrix{ & D^* \ar[dl]_{\rho_0} \ar[dr]^{\sigma_0} && \\
    U \ar[dr]_{\phi_0} && D \ar[dl]^{\psi_0}& \\
    & e && }
    \end{array}
    \leqno{Q:}
    $$
(où $\phi_0 \rho_0 = \psi_0 \sigma_0$) à restriction à $\{ e \}$ imposée, et en prenant des foncteurs cartésiens entre catégories fibrées\dots].

De plus, on exige autre une compatibilité, savoir que l'homomorphisme de systèmes locaux en groupes abéliens sur $\Pi'_D$
$$
\pi_1(\sigma') \to (f_D)^* (\pi_1(\sigma))
$$
défini à l'aide de $f_{D^*}$, $f_D$, $\alpha_{D^*, D}$ rende commutatif le diagramme suivant d'isomorphismes de systèmes locaux sur $\Pi'_D$\footnote{\emph{non} cela ne marche que pour le cas de morphismes étales, sinon il faut faire intervenir la multiplication par les ``degrés de ramifications'' $d_{i'}$ $(i' \in \pi_0(\Pi'_{D^*})$.} :
\[\begin{tikzcd}
	{(f_{D^*})^*(\pi_1(\sigma))} & {\pi_1(\sigma')} \\
	{(f_{D^*})^*(\psi^* (T))} & {\psi'(T)} \\
	{(\psi f_D)^* (T)}
	\arrow[from=1-2, to=1-1]
	\arrow["\sim", from=2-2, to=1-2]
	\arrow["{\kappa'}"', from=2-2, to=1-2]
	\arrow["x"', from=2-2, to=3-1]
	\arrow["{\text{can.}}"', from=2-1, to=3-1]
	\arrow["\sim", from=2-1, to=3-1]
	\arrow["{(f_{D^*})^*(\kappa)}"', from=1-1, to=2-1]
	\arrow["\sim", from=1-1, to=2-1]
\end{tikzcd}\]
(où $x$ est déduit de $\alpha_ {D, e}: \psi' \isom \psi \circ f_D$)

Pour $\Pi$, $\Pi^*$ fixés, les systèmes $(f_\alpha = (f_U, f_D, f_{D^*}, \alpha_{D^*, D}, \alpha_{D^*, U}, \alpha_{U, e}, \alpha_{D, e}))$ précédents forment une catégorie de fa\c{c}on naturelle - en fait un groupoïde - en prenant comme morphismes $\mu$ de $(f', \alpha')$ dans $(f, \alpha)$ les triplets de morphismes (foncteurs profinis)
$$
f'_U \xlongrightarrow{\mu_U} f_U, \quad f'_D \xlongrightarrow{\mu_D} f_D, \quad f'_{D^*} \xlongrightarrow{\mu_{D^*}} f_D
$$
satisfaisant quatre conditions de compatibilité avec $\alpha_{D^*, D}$ et $\alpha'_{D^*, D}$, avec $\alpha_{D^*, U}$ et $\alpha'_{D^*, U}$ avec $\alpha_{U, e}$ et $\alpha'_{U, e}$, avec $\alpha_{D, e}$ et $\alpha'_{D, e}$ respectivement (i.e. on travaille avec une sous-catégorie pleine de la catégorie des $\underline{\Hom}$ entre catégories fibrées sur $Q$, à fibre en $e$ fixée\dots).

J'ai l'impression que le groupoïde $\underline{\Hom}((f', \alpha'), (f, \alpha))$ est toujours \emph{rigide}, i.e. $\Aut(f, \alpha)$ est toujours réduit au groupe unité - j'ai la flemme de vérifier - donc que si $(f', \alpha')$ et $(f, \alpha)$ son isomorphes, l'isomorphisme en question est unique. Quoi qu'il en soit, on posera
$$
\Hom((f', \alpha'), (f, \alpha)) = \pi_0 \underline{\Hom}((f', \alpha'), (f, \alpha))
$$

D'où une \emph{catégorie} des systèmes $\Pi = (\Pi_U, \Pi_D, \Pi_{d^*}, \phi, \psi, \rho, \sigma, \alpha, \kappa)$.

On a un foncteur des couples $(X, S)$\footnote{[les] morphismes $(X', S') \to (X, S)$ sont les morphismes $f: X' \to X$ tels que $f^{-1}(S)_{\text{red}} = S'$} (où $X$ schéma localement de type fini sur $K$, $S$ sous schéma fermé de $X$ étale sur $K$, tels que $\forall s \in S$, $X$ soit lisse de dimension relative 1 en $s$) vers cette catégorie bordélique $B$.

\vskip .5cm
{
Conjecture bordélique {\bf (7.1)}. --- \it Quand on se borne aux $(X, S)$ tels que $X$ projectif lisse de dimension $\le 1$, et qui de plus sont anabéliens, alors le foncteur précédent est pleinement fidèle\footnote{{\bf N.B.} La fidélité est facile\dots}.
}
\vskip .5cm

Description de la catégorie bordélique en termes de théorie de groupes.

Soit $I = \pi_0(\Pi_{D^*}) \isom \pi_0(\pi_D)$. Choisissons un élément $D^*_i$ dans chaque composante de $\Pi_{D^*}$, et soit $D_i = \sigma(D^*_i)$. Pour tout $i$, choisissons un isomorphisme
\[\begin{tikzcd}
	{\psi(D_i)} & {\Spec(\overline{K})}
	\arrow["{\lambda_i}", from=1-1, to=1-2]
	\arrow["\sim"', from=1-1, to=1-2]
\end{tikzcd}\]

Quitte à remplacer l'objet par un ``sous-objet'' isomorphe on peut supposer que $\Pi_{D^*}$ est la catégorie somme de catégories $[E_i]$ définis par les $E_i = \Aut(D^*_i)$, $\Pi_D$ la catégorie somme des catégories $[\Gamma_i]$ définies par les $\Gamma_i = \Aut(D_i)$, le foncteur $\sigma$ s'exprimant par un système d'homomorphismes
$$
\sigma_i: E_i \to \Sigma_i \quad (i \in I)
$$
qui sont surjectifs de noyaux abéliens, le foncteur $\psi$ par un système \emph{d'inclusions} $\psi_i: \Gamma_i \hookrightarrow \Gamma$ et la donnée de $\kappa$ équivaut en fait à des isomorphismes
$$
\kappa_i: \ker \sigma_i \isom T(\overline{K})
$$
compatibles avec les opérations de $\Gamma_i$ et de $\Gamma$ sur les deux [?] respectivement, et les inclusions $\psi_i$. On peut dire que la donnée de $(\sigma, \psi, k)$ est exprimée par la donnée du système $(E_i)_{i \in I}$, d'un système de suites exactes
$$
\boxed{1 \to T(\overline{K}) \xlongrightarrow{\kappa_i} \to E_i \xlongrightarrow{p_i} \Gamma}
$$
telles que les $\Sigma_i = \text{Im} p_i$ soient ouverts, et que $\kappa_i$ soit compatible avec les opérations de $\Sigma_i \isom \text{Coker} \kappa_i$ (ou de $E_i$ et $\Gamma$ sur les deux termes respectivement).

N.B. Les extensions des $\Gamma_i$ par $T(\overline{K})$ obtenues par des situations géométriques splittent le choix d'une uniformisante en $s_i$ définit un splittage, et même [seulement ?] le choix d'une base de l'espace tangent en $s$\dots

Re N.B. Deux bases différents définissent des scindages différents !

Supposant d'autre part (pour simplifier) $\Pi_U$ connexe, et choisissant un élément $\widetilde{U}$ de $\Pi_U$ et un isomorphisme
$$
\phi(\widetilde{U}) \xlongrightarrow{\lambda_U} \Spec(\overline{k})
$$
donc (quitte à remplacer $\Pi_U$ par un groupoïde équivalent) on peut supposer $\Pi_U$ réduit à $\widetilde{U}$, et $\Pi_U$ est donné alors par un groupe $E$, et $\phi$ par un homomorphisme de groupes
$$
\boxed{E \xlongrightarrow{\phi_{\widetilde{U}}, \lambda_U, \text{(ou}~p)} \Gamma}
$$
dont l'image $\Sigma \subset  \Gamma$ est encore un sous-groupe ouvert de $\Gamma$, et le noyau sera noté $\pi$
$$
\boxed{1 \to \pi \xlongrightarrow{\kappa} E \to \Sigma \to 1}
$$
N.B. Dans la situation géométrique cette extension de \emph{noyaux de groupes splitte}, i.e. il existe un sous-groupe ouvert $\Sigma_{\circ}$ dans $\sigma$, et un relèvement $\Sigma_{\circ} \to E$\dots

Ayant remplacé $\Pi_U$ initial par une sous-catégorie pleine plus petite, on sera obligé de modifier $\rho$ à isomorphisme près, pratiquement en choisissant pour chaque $i \in I$ un isomorphisme
$$
\rho(D_i) \xlongrightarrow{\mu_i} \widetilde{U}
$$
moyennant quoi $\rho$ s'explicite simplement par des homomorphismes de groupes
$$
\boxed{\rho_i: E_i \to E}
$$
Il reste à expliciter l'isomorphisme de commutation
$$
\alpha: \phi\rho \to \psi\sigma
$$
qui est défini par un système d'éléments
$$
\boxed{\gamma_i \in \Gamma \quad (i \in I)}
$$
tels que
$$
\phi \circ \rho_i = \text{int}(\gamma_i) \circ \Pi_i
$$
ce qui implique d'ailleurs que $\rho_i$ applique ker$p_i$ dans ker$\rho$, i.e\dots induit un homomorphisme de suites exactes
\[\begin{tikzcd}
	1 & {T(\overline{K})} & {E_i} & {\Gamma_i} & 1 \\
	1 & \pi & E & {\Gamma_0} & 1
	\arrow[from=1-4, to=1-5]
	\arrow[from=2-4, to=2-5]
	\arrow[from=1-4, to=2-4]
	\arrow["{\rho_i}", from=1-3, to=2-3]
	\arrow[from=1-3, to=1-4]
	\arrow[from=2-3, to=2-4]
	\arrow[from=2-2, to=2-3]
	\arrow[from=1-2, to=1-3]
	\arrow[from=1-2, to=2-2]
	\arrow[from=2-1, to=2-2]
	\arrow[from=1-1, to=1-2]
\end{tikzcd}\]
avec un homomorphisme induit $\Gamma_i \to \Gamma_0$ injectif - et ceci posé, la relation de compatibilité devient une relation sur des monomorphismes de groupes $\Gamma_i \hookrightarrow \Gamma_O \hookrightarrow \Gamma$ [Les ``objets bordéliques simplifiés'' sont donc les systèmes d'homomorphismes de suites exactes $$\begin{tikzcd}
1 & {T(\overline{K})} & {E_i} & {\Gamma_i} & {} \\
	1 & \pi & E & {\Gamma_0}
	\arrow["{\text{int}(\gamma_i)}", from=1-4, to=2-4]
	\arrow["{\rho_i}", from=1-3, to=2-3]
	\arrow["{p_i}", from=1-3, to=1-4]
	\arrow["p", from=2-3, to=2-4]
	\arrow["\kappa", from=2-2, to=2-3]
	\arrow["{\kappa_i}", from=1-2, to=1-3]
	\arrow["{\rho_i^\circ}", from=1-2, to=2-2]
	\arrow[from=2-1, to=2-2]
\arrow[from=1-1, to=1-2]
\end{tikzcd}$$].

Ainsi, on a une description relativement simple des systèmes bordéliques ``réduit'' (i.e. où dans les groupoïdes $\Pi_U$, $\Pi_{D^*}$, $\Pi_D$, chaque composante connexe a exactement un objet, et où de plus on force en quelque sorte $\Pi(K)$ a n'avoir que l'objet $\Spec(\overline{K})$. Mais on est retrouvé en [ne] tournant pas la détermination des \emph{morphismes}
$$
G = (I, G_i, p_i, K_i, E, \phi (\text{ou}~p?), \rho_i, \gamma_i) \quad \text{et} \quad G'= (I', G'_i,\dots),
$$
disons dans le sens $f: G' \to G$. Il faut donc (pour $f_{D^*}$) une application
$$
\boxed{\tau = \tau_f: I' \to I}
$$
et pour tout $i$ un homomorphisme 
$$
\boxed{G'_{i'} \xlongrightarrow{f_{i'}} G_{\tau i'}}
$$
induisant (compte tenu de $f_D$) par passage au quotient des homomorphismes
$$
\Gamma'_{i'} \xlongrightarrow{g_i} \Gamma_{\tau i'}
$$
de sous groupes ouverts de $\Gamma$ et la donnée $f_{D^*}, f_D, \alpha_{D^*, D}, \alpha_{D, e}$ équivaut donc à la donnée d'éléments $\alpha_i$ $(i \in I)$ de $\Gamma$, tels que 
$$
g_{i'}(\gamma') = \text{Int}_{(\alpha_{i'})}(\gamma') \quad \forall i' \in I', \gamma' \in \Gamma'_{i'}
$$
La donnée de $f_U$ équivaut à la donnée d'un homomorphisme de groupes celle de $\alpha_{D, e}$ équivaut à la donnée de
$$
\boxed{\alpha \in \Gamma}
$$
tel que (*):
\[
\phi f_E = \text{Int}(\alpha)\phi %\leqno{(*)}
\]
de sorte que $f_E$ induit un homomorphisme de suites exactes
\[\begin{tikzcd}
	1 & {\pi'} & {E'_i} & {\Gamma'_0} & 1 \\
	1 & \pi & E & {\Gamma_0} & 1
	\arrow["{f_{\sigma_0}}", from=1-4, to=2-4]
	\arrow["{f_E}"', from=1-3, to=2-3]
	\arrow[from=1-3, to=1-4]
	\arrow[from=2-3, to=2-4]
	\arrow[from=2-2, to=2-3]
	\arrow[from=1-2, to=1-3]
	\arrow[from=1-2, to=2-2]
	\arrow[from=2-1, to=2-2]
	\arrow[from=1-1, to=1-2]
	\arrow[from=2-4, to=2-5]
	\arrow[from=1-4, to=1-5]
\end{tikzcd}\]
et moyennant cela, la condition dite $\alpha$ devient une condition sur des inclusions de sous-groupes de $\gamma$ :
$$
f_{\sigma_0}(\gamma') = \text{int}(\alpha)\gamma' \quad \text{si} \quad \gamma' \in \Gamma'_0.
$$

Il faut expliciter encore (en plus de $f_{D^*}$, $f_D$, $f_U$, $\alpha_{D^*, D}$, $\alpha_{D, e}$, $\alpha_{U, e}$ déjà explicités) la donnée de commutation $\alpha_{D^*, U}$ et écrire les conditions de compatibilités avec $\alpha$, $\alpha'$ et $\kappa$, $\kappa'$. La donnée de $\alpha_{D^*, U}$ équivaut à celle de systèmes d'éléments
$$
\boxed{\beta_{i'} \in E'} \quad (i' \in I')
$$
tels que l'on ait dans le diagramme
\[\begin{tikzcd}
	{G'_{i'}} & {E'} \\
	{G_{\tau i'}} & E
	\arrow["{f_{i'}}"', from=1-1, to=2-1]
	\arrow["{\rho'_{i'}}", from=1-1, to=1-2]
	\arrow["{\rho_{\tau i'}}"', from=2-1, to=2-2]
	\arrow["{f_E}", from=1-2, to=2-2]
\end{tikzcd}\]
la relation
$$
f_E \circ \rho'_{i'} = (\text{int}(\beta_i)\rho_{\tau i'}) \circ f_{i'}
$$
Reste à exprimer les deux compatibilités de $f_{D^*}, f_{D^1}, F_U, \alpha_{D^*, D}, \alpha_{D, e}, \alpha_{D^*, U}, \alpha_{U, e}$ avec lui même et avec $\kappa$ - la deuxième compatibilité est simplement la compatibilité des $f_{i'}$ avec les $\kappa'_{i'}$, $\kappa_i$, i.e.
$$
f'_{i'} \circ \kappa'_{i'} = \kappa_i \quad (i = \tau i')
$$
et la première \emph{sauf erreur} s'exprime par la ``commutativité''
$$
\boxed{\alpha_{i'} = \gamma^{-1}_i p(\beta^{-1}_{i'}) \alpha \gamma'_{i'}} \quad \forall i' \in I'
$$
En résumé les homomorphismes, dans un système de diagrammes commutatifs
\[\begin{tikzcd}
	1 & {T(\overline{K})} & {E_i} & \Gamma \\
	1 & \pi & E & \Gamma
	\arrow[from=1-1, to=1-2]
	\arrow[from=2-1, to=2-2]
	\arrow[from=1-2, to=2-2]
	\arrow["{\kappa_i}", from=1-2, to=1-3]
	\arrow["{p_i}", from=1-3, to=1-4]
	\arrow["{\rho_i}", from=1-3, to=2-3]
	\arrow[from=2-2, to=2-3]
	\arrow["p", from=2-3, to=2-4]
	\arrow["{\text{int}(\gamma_i) \quad \quad \quad (i \in I)}", from=1-4, to=2-4]
\end{tikzcd}\]
d'un système analogue, relatif à un ensemble d'indices $I'$, est donné par une application
$$
\tau: I' \to I
$$
et pour tout $i' \in I'$, posant $i = \tau(i')$ , d'un système de flèches verticales $f_{i'}: E'_{i'} \to E_i$, et d'une flèche verticale $f_E: E' \to E$, enfin d'un système d'éléments $\beta_{i'} \in E$ et d'un $\alpha \in \Gamma$, s'insérant dans le système de diagrammes\footnote{N.B. Comme les $E_i \to E$ sont injectifs, $f_{i'}$ est connu quand on connaît $f_E$ et $\beta_{i'}$, (l'existence de $f_{i'}$ est donc une condition sur les couples $f_E$, $\beta_{i'}$ savoir que $\text{int}(\beta_{i'})^{-1}f_E \rho_{i'}$ applique $E_{i'}$ dans $p_i(E_i)$. On peut supposer les $\gamma_{i'}$, $\gamma_i$ égaux à 1 (en choisissant l'isomorphisme de $\psi(\sigma(D^*_i))$ avec $\Spec(\overline{K})$ via l'isomorphisme de $\varphi(\rho(D^*_i)) = \varphi(\widetilde{U})$ avec $\Spec \overline{K}$ [?])}
\[\begin{tikzcd}
	1 & {T(\overline{K})} && {E'_{i'}} && \Gamma \\
	1 & {} & {\pi'} && {E'} && \Gamma \\
	1 & {T(\overline{K})} && {E_i} && \Gamma \\
	& 1 & \pi && E && \Gamma
	\arrow[from=1-1, to=1-2]
	\arrow[from=3-1, to=3-2]
	\arrow["{d_{i'}}"'{pos=0.2}, from=1-2, to=3-2]
	\arrow["{\kappa_i}"{pos=0.2}, from=3-2, to=3-4]
	\arrow["{\kappa'_{i'}}", from=1-2, to=1-4]
	\arrow["{f_{i'}}"'{pos=0.2}, from=1-4, to=3-4]
	\arrow["{p'_{i'}}", from=1-4, to=1-6]
	\arrow["{\text{int}(\alpha_{i'})}"{pos=0.7}, from=1-6, to=3-6]
	\arrow["{p_i}"{pos=0.2}, from=3-4, to=3-6]
	\arrow["{f_{\epsilon}}"'{pos=0.2}, from=2-5, to=4-5]
	\arrow[from=4-3, to=4-5]
	\arrow["{f_{i'}}"'{pos=0.2}, from=2-3, to=4-3]
	\arrow[from=2-3, to=2-5]
	\arrow[from=4-2, to=4-3]
	\arrow[from=2-1, to=2-3]
	\arrow["{p'}"{pos=0.2}, from=2-5, to=2-7]
	\arrow["p", from=4-5, to=4-7]
	\arrow["{\text{int}(\alpha)}", from=2-7, to=4-7]
	\arrow["{\text{int}(\gamma'_{i'})}", from=1-6, to=2-7]
	\arrow["{\text{int}(\gamma_i)}"{pos=0.2}, from=3-6, to=4-7]
	\arrow["{\rho'_{i'}}", from=1-4, to=2-5]
	\arrow["{\rho_i}", from=3-4, to=4-5]
	\arrow["{\epsilon_i}"', from=3-2, to=4-3]
	\arrow["{\epsilon'_{i'}}", from=1-2, to=2-3]
\end{tikzcd}\]
où on a posé 
$$
\alpha'_i = \gamma^{-1}_i p(\beta^{-1}_i) \alpha\gamma'_{i'} \quad \text{i.e.} \quad \alpha \gamma'_{i'} = p(\beta_i)\gamma_i \alpha_{i'}
$$
et où la face verticale postérieure des prismes est commutative (deux conditions, sur deux carrés), la face verticale antérieure aussi (c'est \emph{une} condition, sur le carré de droite, l'autre carré commutatif s'en déduit par définition de $\pi' \to \pi$ comme induit par $f_{E}$\dots), la face verticale gauche du cube de droite étant commutative modulo l'isomorphisme de commutativité $\text{int}(\beta_{i'})$, et la face verticale droite étant commutative (non seulement, par la condition précédente, sur $\sum'_{i'} = \text{Im}(E'_{i'} \xlongrightarrow{p'_{i'}} \Gamma)$, mais sur $\Gamma$ tout entier) en vertu de la relation plus précise.

\vskip .3cm
\begin{center}
    \line(1, 0){3cm}
\end{center}
%\section*{}

\vskip .5cm
{
Conjecture bordélique précisée (correspondant aux $\Pi_U$ connexes). --- \it Ses objets sont les homomorphismes de groupes profinis
$$
E \xlongrightarrow{p} \Gamma
$$
donnant naissance à une suite exacte
$$
1 \to \pi \xlongrightarrow{\kappa} E \xlongrightarrow{p} \Gamma
$$
si un ensemble fini de sous-groupes (indexés par un ensemble $I$ d'indices)
\[\begin{tikzcd}
	{} \\
	& {E_i} & E
	\arrow["{\rho_i}", hook, from=2-2, to=2-3]
\end{tikzcd}\]
et d'isomorphismes
\[\begin{tikzcd}
	{} & {E_i \cap \pi} & {T(\overline{K})}
	\arrow["{\kappa_i}", from=1-2, to=1-3]
	\arrow["\sim"', from=1-2, to=1-3]
\end{tikzcd}\]
Un homomorphisme d'un système $(E', p', (E'_{i'})_{i' \in I'}, (\kappa'_{i'})_{i' \in I'})$ dans un système $(E, p, (E_i)_{i \in I},\dots)$ est donné par un homomorphisme
$$
f: E' \to E
$$
et des $\beta_{i'} \in E$ ($i' \in I'$)\footnote{Les $\beta_{i'}$ \emph{pas} uniques (si $\beta_{i'}$ convient aussi $\gamma \beta_{i'}$, avec $\gamma \in \text{Im} \kappa_i$) Mais $\alpha$ unique ??}, $\alpha \in \Gamma$, tels que l'on ait les conditions :
\begin{enumerate}
    \item[$1^\circ$)] $p \circ f = \text{int}(\alpha)p'$
    \item[$2^\circ$)] $\forall i' \in I'$, $\exists i \in I$ (unique !) tel que
    $$
    \text{int}(\beta_{i'})^{-1} f(E'_{i'}) = E_i
    $$
    et un entier $d_{i'} \in \mathbf{N}^*$\footnote{N. B. $d_{i'}$ est aussi unique\dots} tel que
    $$
\text{int}(\beta^{-1}_{i'})f \kappa'_{i'} = \kappa_i \circ (d_i, \id_{T(\overline{K})})
$$
\end{enumerate}
}
\begin{center}
\hdashline
\end{center}


Je me a'per\c{c}ois qu'il vaut mieux remplacer les $\beta_{i'}$ par les $\beta^{-1}_{i'}$, i.e. prendre l'isomorphisme de commutation plutôt dans le sens
$$
f_\epsilon \rho'_{i'} \xlongrightarrow{\beta_{i'}} \rho_i f_{i'}
$$
qu'en sens inverse. De plus, conceptuellement le diagramme
\[\begin{tikzcd}
	& {\Pi_{D^*}} \\
	{\Pi_U} && {\Pi_D} \\
	& {\Pi_e}
	\arrow[""{name=0, anchor=center, inner sep=0}, "\phi"', from=2-1, to=3-2]
	\arrow[""{name=1, anchor=center, inner sep=0}, "\psi", from=2-3, to=3-2]
	\arrow["\rho"', from=1-2, to=2-1]
	\arrow["\sigma", from=1-2, to=2-3]
	\arrow["\alpha", shorten <=6pt, shorten >=6pt, from=0, to=1]
\end{tikzcd}\]
est trop compliqué, il suffit de se donner
$$
\boxed{\Pi_{D^*} \xlongrightarrow{\rho} \Pi_U \xlongrightarrow{\phi} \Pi_e}
$$
et de déduire $\Pi_D$ par factorisation canonique de l'homomorphisme de groupoïdes $\Pi_{D^*} \to \Pi_e$ en un homomorphisme qui induit un isomorphisme sur $\pi_0$ et un épimorphisme sur les $\pi_1$, suivi d'un homomorphisme qui est [épi. sur $\pi_0$, et pour cause, et qui est] \emph{fidèle}. Alors les 1-morphismes de
$$
\Pi'_{D^*} \xlongrightarrow{\rho'} \Pi'_U \xlongrightarrow{\phi'} \Pi_e
$$
dans
$$
\Pi_{D^*} \xlongrightarrow{\rho} \Pi_U \xlongrightarrow{\phi} \Pi_e
$$
sont les quintuplés de 2 foncteurs et 3 isomorphismes de foncteurs $(f_{D^*}, f_U, \alpha_{D^*, U}, \alpha_{U, e}, \kappa)$ donnant un diagramme avec données de commutation\footnote{de plus, $\pi_1(\varphi \rho) = \pi_1(\Pi_{D^*}/\Pi_e)$ peut se définir comme un système local de groupes (commutatifs) sur $\Pi{D^*}$, et on peut alors définir $\kappa: (\varphi \rho)^* T'_K \isom \pi_1(\varphi \rho)$.}\footnote{avec \emph{une} condition sur le morphisme $\pi_1(\varphi' \rho') \isom f_{D^*}(\pi_1(\varphi \rho))$ induit, qui modulo les isomorphismes de Kummer doit être la multiplication par un $d$ (indice de ramification) qui est un application $\pi_0(\Pi'_{D^*}) \to \mathbf{N}^*$.} 
\[\begin{tikzcd}
	{\Pi'_{D^*}} & {\Pi'_U} & {\Pi_e} \\
	{\Pi_{D^*}} & {\Pi_U} & {\Pi_e}
	\arrow[""{name=0, anchor=center, inner sep=0}, "\sim", from=1-3, to=2-3]
	\arrow["{\phi'}", from=1-2, to=1-3]
	\arrow["{\rho'}", from=1-1, to=1-2]
	\arrow["{f_{D^*}}"', from=1-1, to=2-1]
	\arrow[""{name=1, anchor=center, inner sep=0}, "{f_U}", from=1-2, to=2-2]
	\arrow[""{name=2, anchor=center, inner sep=0}, "\rho"', from=2-1, to=2-2]
	\arrow[""{name=3, anchor=center, inner sep=0}, "\phi"', from=2-2, to=2-3]
	\arrow["{\alpha_{D^*, U}}", shorten <=4pt, shorten >=4pt, from=2, to=1]
	\arrow["{\alpha_{U, e}}", shorten <=4pt, shorten >=4pt, from=3, to=0]
\end{tikzcd}\]
et si on a deux tels 1-morphismes $f = (f_{D^*}, f_U, \alpha_{D^*, U}, \alpha_{U, e})$ et $g = (g_{D^*}, g_U, \beta_{D^*, U}, \beta_{U, e})$ un 0-morphisme de $f$ dans $g$ est formé d'un couple $(\mu_{D^*}, \mu_U)$ d'isomorphismes de foncteurs
$$
\mu_{D^*}: f_{D^*} \isommap g_{D^*}, \quad \mu_U: f_U \isommap g_U
$$
compatibles avec $\alpha_{D^*, U}$, $\beta_{D^*, U}$ (deux conditions de compatibilités sur les deux carrés).

Si on se borne à des $\Pi_U$ connexes, on trouve en choisissant comme plus haut un objet $\tilde{U}$ de $\Pi_U$, et un isomorphisme
$$
\varphi(\widetilde{U}) \xlongrightarrow[\sim]{\lambda} \Omega
$$
(où $\Omega = \Spec(\overline{K})$ est l'objet référence de $\Pi_e$\dots), un groupe $E$ et un homomorphisme
$$
E \xlongrightarrow{\phi} \Gamma
$$
(qui est changé par automorphisme intérieure si on change $\lambda$, et $E$ lui même remplacé par un groupe isomorphe, l'isomorphisme défini modulo isomorphisme intérieur, si on change l'objet de référence $\tilde{U}$). De même, choisissant un $\tilde{U}_i$ dans chaque composante $i \in \pi_0(\Pi_{D^*})$, et un isomorphisme
$$
\lambda_i: \rho(\tilde{U}_i) \isommap \tilde{U},
$$
on trouve des groupes $E_i$ et des homomorphismes
$$
E_i \xlongrightarrow{\rho_i} E
$$

Si on change $\lambda_i$, $\rho_i$ est changé par automorphisme intérieur de $E$. Si on change $\widetilde{U_i}$, $E_i$ est remplacé par un groupe isomorphe, l'isomorphisme défini à automorphisme près. 

Considérons les composés
\[\begin{tikzcd}
	{E_i} & E \\
	& \Gamma
	\arrow[from=1-1, to=1-2]
	\arrow["p", from=1-2, to=2-2]
	\arrow["{p_i}"', from=1-1, to=2-2]
\end{tikzcd}\]
d'où [un ?] noyau, $\kappa$ est défini par un système d'isomorphismes
$$
T(\overline{K}) \xlongrightarrow[\sim]{\kappa_i} \Ker p_i \quad i \in I
$$
s'insèrent dans une famille de suites exactes
$$
1 \to T(\overline{K}) \xlongrightarrow{\kappa_i} E_i \xlongrightarrow{p_i} \Gamma
$$
(N. B. l'image de $p_i$ est un sous-groupe ouvert) $\kappa_i$ étant compatible aux actions de $E_i$, quand on fait opérer $E_i$ sur $T(\overline{K})$ via l'action de $\Gamma$ sur $T(\overline{K})$, et su lui même par automorphismes intérieures. Introduisant également $\pi = \Ker p$, on trouve donc un homomorphisme de suites exactes (D) :
\[\begin{tikzcd}
	1 & {T(\overline{K})} & {E_i} & \Gamma \\
	1 & \pi & E & \Gamma
	\arrow[from=1-1, to=1-2]
	\arrow["{\kappa_i}", from=1-2, to=1-3]
	\arrow["{p_i}", from=1-3, to=1-4]
	\arrow["\mid", from=1-4, to=2-4]
	\arrow["{\rho_i}", from=1-3, to=2-3]
	\arrow["{\epsilon_i}", from=1-2, to=2-2]
	\arrow[from=2-1, to=2-2]
	\arrow["\kappa", from=2-2, to=2-3]
	\arrow["p", from=2-3, to=2-4]
\end{tikzcd}
\]
On peut dire que $\Pi_U$ définit un ``groupe extérieur'' $[E]$, et $\Pi_U \to \Pi_e$ un homomorphisme de groupes extérieures 
$$
[E] \to [\Gamma]
$$
de même $\Pi_{D^*}$ définit un système de ``groupes extérieures'' $[E_i]$, et $\Pi_{D^*} \to \Pi_U$ un système d'homomorphismes extérieurs
$$
[E_i] \to [E],
$$
mais comment en termes de groupes extérieurs exprimer les données de Kummer $\kappa_i$ ?

Revenant aux systèmes de diagrammes $D$ (relatif à un ensemble d'indices $I$) et à un $D'$ analogue (avec un ensemble d'indices $I'$) D :
\[\begin{tikzcd}
	1 & {T(\overline{K})} & {E'_{i'}} & \Gamma \\
	1 & \pi & E' & \Gamma & i' \in I'
	\arrow[from=1-1, to=1-2]
	\arrow["{\kappa'_{i'}}", from=1-2, to=1-3]
	\arrow["{p_{i'}}", from=1-3, to=1-4]
	\arrow["\mid", from=1-4, to=2-4]
	\arrow["{\rho'_{i'}}", from=1-3, to=2-3]
	\arrow["{\epsilon'_{i'}}", from=1-2, to=2-2]
	\arrow[from=2-1, to=2-2]
	\arrow["\kappa'", from=2-2, to=2-3]
	\arrow["p'", from=2-3, to=2-4]
\end{tikzcd}
\]
un homomorphisme $f$ de $D'$ dans $D$ s'explicite par 
\begin{enumerate}
    \item[a)] Un homomorphisme\footnote{décrit le foncteur $f_U$}
    $$
    \boxed{f_E: E' \to E}
    $$
    \{ s'explicite en termes du choix d'un isomorphisme
    $$
    \nu: f(\widetilde{U}') \isom \widetilde{U}
    $$
    (un autre choix modifie $f_E$ par un $\text{int}(\beta), \beta \in E$) \}
    \item[b)] Une application\footnote{décrit le foncteur $f_{D^*}$} $\tau: I' \to I$, et pour tout $i' \in I'$, posant $i = \tau(i')$, un homomorphisme de groupes
    $$
    \boxed{E_i \xlongrightarrow{f'_{i'}} E_i}
    $$
    [s'explicite en terme du choix d'un isomorphisme $f_{D^*}(\tilde{U_{i'}}) \xlongrightarrow[\isom]{\nu_i} \Pi_{U_i}$, et modifié par des $\text{int}(\beta_{i'})$, $\beta_{i'} \in E_i$, si on change $\nu_{i'}$]
    \item[c)] Une donnée de commutation\footnote{$\alpha$ décrit $\alpha_{U, e}$} pour
    \[\begin{tikzcd}
	{E'} & \Gamma \\
	E & \Gamma
	\arrow["{f_E}"', from=1-1, to=2-1]
	\arrow["{p'}", from=1-1, to=1-2]
	\arrow[""{name=0, anchor=center, inner sep=0}, "p"', from=2-1, to=2-2]
	\arrow[""{name=1, anchor=center, inner sep=0}, "\alpha"', from=1-2, to=2-2]
	\arrow["\sim", from=1-2, to=2-2]
	\arrow[shorten <=4pt, shorten >=4pt, from=0, to=1]
    \end{tikzcd} \quad\quad \boxed{\alpha \in \Gamma}
    \]
    i.e.
    $$
    \boxed{pf_E = \text{int}(\alpha)p'}
    $$
    \item[d)] $\forall i' \in I$ une donnée de commutation\footnote{$\alpha_{i'}$ décrit $\alpha_{D^*, U}$} pour
    \[\begin{tikzcd}
	{E'_{i'}} & E' \\
	{E_i} & E
	\arrow["{f_{i'}}"', from=1-1, to=2-1]
	\arrow["{p'_{i'}}", from=1-1, to=1-2]
	\arrow[""{name=0, anchor=center, inner sep=0}, "{p_i}"', from=2-1, to=2-2]
	\arrow[""{name=1, anchor=center, inner sep=0}, "{\alpha_{i'}}"', from=1-2, to=2-2]
	\arrow["\sim", from=1-2, to=2-2]
	\arrow[shorten <=4pt, shorten >=4pt, from=0, to=1]
    \end{tikzcd} \quad\quad \boxed{\alpha_{i'} \in \Gamma} \quad (i' \in I')
    \]
    i.e.
    $$
    \boxed{p_i f_{i'} = \text{int}(\alpha_{i'})f_E \rho'_{i'}}
    $$
\end{enumerate}

Notons que c), d) ensemble définissent une donnée de commutation
\[\begin{tikzcd}
	{E'_{i'}} & \Gamma \\
	{E_i} & \Gamma
	\arrow["{f_{i'}}"', from=1-1, to=2-1]
	\arrow["{p'_{i'}}", from=1-1, to=1-2]
	\arrow[""{name=0, anchor=center, inner sep=0}, "{p_i}"', from=2-1, to=2-2]
	\arrow[""{name=1, anchor=center, inner sep=0}, "{\beta_{i'}}"', from=1-2, to=2-2]
	\arrow["\mid", from=1-2, to=2-2]
	\arrow[shorten <=4pt, shorten >=4pt, from=0, to=1]
    \end{tikzcd} \quad\quad ~\text{avec}~\beta_{i'} = p (\alpha_{i'}) \alpha
    \]
i.e.
$$
p_if_{i'} = \text{int}(\beta_{i'})p'_{i'}
$$
i.e. on a commutativité dans
\[\begin{tikzcd}
	{E'_{i'}} & \Gamma \\
	{E_i} & \Gamma
	\arrow["{f_{i'}}"', from=1-1, to=2-1]
	\arrow["{p'_{i'}}", from=1-1, to=1-2]
	\arrow["{p_i}"', from=2-1, to=2-2]
	\arrow["\sim"', from=1-2, to=2-2]
	\arrow["{\text{int}(\beta_{i'})}", from=1-2, to=2-2]
\end{tikzcd}\]
donc $f_{i'}$ induit un homomorphisme de suites exactes
\[\begin{tikzcd}
	1 & {T(\overline{K})} & {E'_{i'}} & \Gamma \\
	1 & {T(\overline{K})} & {E_i} & \Gamma
	\arrow[from=1-1, to=1-2]
	\arrow[from=2-1, to=2-2]
	\arrow["{f^\circ_{i'}}", from=1-2, to=2-2]
	\arrow["{f_{i'}}", from=1-3, to=2-3]
	\arrow["{\text{int}(\beta_{i'})}", from=1-4, to=2-4]
	\arrow["{\kappa'_{i'}}", from=1-2, to=1-3]
	\arrow["{p_{i'}}", from=1-3, to=1-4]
	\arrow["{\kappa_i}"', from=2-2, to=2-3]
	\arrow["{p_i}"', from=2-3, to=2-4]
\end{tikzcd}\]
et il faut exprimer la fonctorialité de $f^{\circ}_{i'}$, avec l'indice de ramification $d_{i'} \in \mathbf{N}$, on trouve
$$
\boxed{f^\circ_{i'} = d_{i'} \chi(\beta_{i'})\id_{T(\overline{K})}}
$$
où
$$
\chi: \Gamma \to \widehat{\mathbf{Z}}^*
$$
est le caractère canonique.

Ceci posé, déterminons, pour un 1-morphisme $f = (f_E, \tau, (f_{i'}), \alpha, (\alpha_i))$ et un autre $f' = (f'_E, \tau', (f'_{i'}), \alpha', (\alpha'_i))$, les 0-morphismes de l'un dans l'autre correspondants à de systèmes d'isomorphismes $\mu_{D^*}: f_{D^*} \to f'_{D^*}$, $\mu_U: f_U \to g_U$. Cela correspond [à la] condition
$$
\tau' = \tau, \quad \underbrace{d'_{i'} = d'_i \quad \forall i' \in I'}_{\text{\hspace*{-5mm}ceci résulte des autres conditions\hspace*{-5mm}}}
$$
et à la donnée de
\begin{enumerate}
    \item[a)] $\boxed{\mu \in E}$\footnote{décrit $\mu_U$} définissant un isomorphisme entre $f_E: E' \to E$ et $f'_E: E' \to E$ i.e. tel que
    $$
    \boxed{f'_E = \text{int}(\mu)f_E}
    $$
    \item[b)] $\boxed{\mu_{i'} \in E_i}$\footnote{décrit $\mu_{D^*}$} ($i' \in  I', i = \tau(i')$)
    satisfaisant
    $$
    \boxed{f'_{i'} = \text{int}(\mu_{i'})f_{i'}}
    $$
    [ceci implique déjà $d_{i'} = d'_{i'}$] les $\mu \in E$, $\mu_{i'} \in E_i$ étant liés aux $\alpha$, $\alpha'$ $(\in E)$, aux $\alpha_{i'}$, $\alpha'_{i'}$ $(\in \Gamma)$ de la fa\c{c}on suivante
    $$
    \boxed{\alpha' = p(\mu) \alpha}
    $$
    $\forall i' \in I'$
    $$
    \boxed{\alpha'_{i'} = \rho_i(\mu_{i'})\alpha'}
    $$
    ou encore
    $$
    \boxed{p(\mu) = \alpha' \alpha^{-1}, \rho_i(\mu_{i'}) = \alpha'_{i'}\alpha^{-1}_{i'}}
    $$
    Notons que, comme $\rho_i$ est injectif, les deuxièmes relations, donnant les $\rho_i(\mu_{i'})$, déterminent les $\mu_{i'}$ de \emph{fa\c{c}on unique}, l'existence de ces $\mu_{i'}$ équivaux aux relations
    $$
    \boxed{\alpha_{i'} \alpha^{-1}_{i'} \in \rho_i(E_i)} \quad\quad i' \in I'
    $$
    Quant à la relation $p(\mu) = \alpha' \alpha^{-1}$, elle détermine $\mu$ modulo multiplication (à droite disons) par un $\mu_0 \in \Ker p = \pi$, mais qui doit être tel que l'on ait encore
    $$
    f'_E = \text{int}(\mu\mu_0)f_E
    $$
    (en plus de $f'_E = \text{int}(\mu)f_E$) ce qui signifie que
    $$
    \text{int}(\mu_0)f_E = f_E
    $$
    i.e.
    $$
    \boxed{\mu_0 \in \Centr_{\pi}f_E(E')}
    $$
    C'est donc cela l'indétermination exacte dans le choix d'un 0-morphisme [?] 1-morphismes $f$ et $f'$ de $D'$ dans $D$.
\end{enumerate}
Si par exemple on est dans les conditions où $\pi \to \pi$ induit par $f_E: E' \to E$ a une image d'indice fini (image ouverte) - cas d'un ``morphisme non constant'' ! - alors $\mu_0$ doit centraliser un sous-groupe ouvert de $\pi$ - sauf erreur cela aussi implique $\mu_0 = 1$, donc il semble bien que (dans les cas correspondants à des homomorphismes dominants de courbes algébriques) un $0$-morphisme d'un $f$ dans un $f'$ (s'il existe) est unique, i.e. $\underline{\Hom}(D', D)$ est une catégorie \emph{discrète}.

On peut interpréter $D$ comme $E$ muni de \emph{sous-groupes} $E_i \subset  E$, et de $p:E \to \Gamma$, (noyau $\pi$) enfin d'isomorphismes
$$
\kappa_i: T(\overline{K}) \isom E_i \cap \pi = L_i
$$
(commutant à l'action de $E_i$).

Si on a une autre système $(E', (E'_{i'})_{i' \in I'}, p': E' \to \Gamma, (\kappa'_{i'})_{i' \in I'})$, un morphisme du premier dans le second est \emph{défini} par un 
$$
f: E' \to E
$$
satisfaisant les conditions suivantes
\begin{enumerate}
    \item[a)] $\forall i' \in I'$, $\exists i \in I$ tel que $f(E'_i)$ soit contenu dans un conjugué de $E_i$ [ soit $\alpha_{i'} \in E$ tel que $f(E'_i) \subset  \text{int}(\alpha^{-1}_{i'}(E_i))$ ]
    \item[b)] $p_f$ est conjugué de $f'$ [soit $\alpha \in \Gamma$ tel que $pf = \text{int}(\alpha)p'$]
\end{enumerate}

N. B. Le $i \in I$ correspondant à $i' \in I'$ est unique, d'où $\tau: I' \to I$. Si les $\alpha_{i'}$ sont choisis, on déduit $\forall i'$ homomorphisme induit $f_{i'}: E'_{i'} \to E_i$, $f_{i'} = (\text{int}(\alpha_{i'}) \circ f)/_{E'_{i'}}$, qui induit dès lors, via les $\kappa'_{i'}$, $\kappa_i$, un endomorphisme $f^\circ_{i'}: T(\overline{K}) \to T(\overline{K})$\footnote{N. B. A priori $f^\circ_{i'}$ est la multiplication par un $\chi_{i'} \in \widehat{\mathbf{Z}}^*$ ; le centralisateur dans $\Gamma$ d'un sous-groupe ouvert est réduit à l'unité.}. On exige que $\alpha \in \Gamma$ (qui conjugue $f_E$ en $f'_E$, et est probablement uniquement déterminé par cette condition) et $\alpha_{i'}$ (qui est sans doute déterminé modulo composition à gauche avec élément du normalisateur de $E_i$ (du moins des transporteurs des $E_i$ d'un sous-groupe ouvert de $E_i$ - c'est itou si $f$ est iso) puissent être choisis de telle fa\c{c}on que, posant $\beta_{i'} = p(\alpha_{i'})\alpha$, on ait
$$
f^\circ_{i'} = (d_{i'} \chi (\beta_{i'})) \id_{T(\overline{K})} \quad \text{i.e.} \quad \chi_{i'} = d_{i'} \chi(\beta_{i'})
$$
où $d_{i'} \in \mathbf{N}^*$ est un entier naturel (évidement uniquement déterminé quand $\alpha$ et $\alpha_{i'}$ sont choisis)\footnote{Cette condition sur les systèmes des $\alpha_{i'}$ ne change pas ($\alpha$ restant fixé) si on change $\alpha_{i'}$ en $\mu_{i'} \alpha_{i'} \in E_i$.}.










%%%%%%%%%%%%%%%%%%%%%%%%%%%%%%%%%%%%%%%%%%%%%%%%%%%%%%%%%%%%%%%
\chapter*{\S \space 8. --- RÉFLEXIONS TAXONOMIQUES}\thispagestyle{empty}
\addcontentsline{toc}{section}{8. Réflexions taxonomiques}
\label{sec:8}
\section*{}

Le cas dim $X = 0$ se traduit sur le paradigme du système ($\begin{tikzcd}
	{T(\overline{K})} & {E_i} & E & \Gamma
	\arrow["p", from=1-3, to=1-4]
	\arrow["{\kappa_i}", from=1-1, to=1-2]
	\arrow["{\rho_i}", hook, from=1-2, to=1-3]
\end{tikzcd}$) 
par la condition $p$ injectif et donc $I = \emptyset$ - donc ce cas décrit simplement par la donnée d'un \emph{sous-groupe} ouvert de $E \hookrightarrow \Gamma$, i.e. une sous-extension finie $L$ de $\overline{K}/K$ (N.B. on aura bien sûr $X' = \Spec L$, et le choix faits sur $X$ - i.e. sur $\Pi_{D^*}(= \emptyset) \to \Pi_U \to \Pi_e$ - aboutissant à cette description, reviennent ici au choix d'une telle $K$-immersion de $L = \mathrm{H}^0(X, O_X)$ dans $\overline{K}/K$).

Le cas dim$X = 1$ se traduit par le fait que $E \to \Gamma$ n'est pas injectif - quand à $I$, il peut être vide ou non dans ce cas. S'il est vide, la description se fait donc simplement en termes d'un homomorphisme de groupes profinis $E \xlongrightarrow{p} \Gamma$ (ou $\Gamma$ donnée d'avance $= \Gal(\overline{K}/K)$) $= \Aut_{\Pi_e}(\Omega)$).

Supposons données maintenant à la fois $D = (E, p, I, (\rho_i: E_i \hookrightarrow E), (\kappa_i: T(\overline{K}) \to E_i))$ et $D' = (E', p', I', (\rho_i'), (\kappa_i'))$, et revenons à la question de la description des morphismes de $D'$ dans $D$. On va distinguer quatre cas suivant les deux valeurs possibles 0, 1 de $n = \text{dim} D$ et $n' = \text{dim} D'$ respectivement.
\begin{enumerate}
    \item[I)] $n = 0$, $n' = 1$, $I = I' = \emptyset$ et les données se réduisent à des sous-groupes ouverts $E \xhookrightarrow{p} \Gamma$, $E' \xhookrightarrow{p'} \Gamma$. Dans la pesante description plus haut, les considérations relatives à $I$, $I'$ tombent et un morphisme $D' \to D$ revient à une classe de couples $(f, \alpha)$, où $f: E' \to E$ et $\alpha \in \Gamma$, tels que l'on ait $pf = \text{int}(\alpha) f$
    \[\begin{tikzcd}
	{E'} & E \\
	\Gamma & \Gamma
	\arrow["f", from=1-1, to=1-2]
	\arrow["{p'}"', from=1-1, to=2-1]
	\arrow["p", from=1-2, to=2-2]
	\arrow["\sim", from=2-1, to=2-2]
	\arrow["{\text{int}(\alpha)}"', from=2-1, to=2-2]
    \end{tikzcd}\]
    - ce qui implique que $f$ est déjà déterminé par $\alpha$, étant induit par $\text{int}(\alpha)$ [$\alpha$ assujetti à $\alpha \in \Trans_{\Gamma}(E', E) = \Trans_\Gamma (L, L')$] [qui induit bien un homomorphisme de $L \subset  \overline{K}$ dans $L' \subset  \overline{K}$, donc un homomorphisme $\Spec(L') \to \Spec(L)$ en sens inverse].
    
    La condition que $\alpha$, $\alpha' \in \Gamma$ définissent le même morphisme $D' \to D$ (ou encore $\Spec L' \to \Spec L$) se décrit par l'existence d'un $\mu \in E$ tel que $\alpha' = \mu \alpha$. Le fait qu'on trouve une correspondance 1-1 entre $\Hom_K(X', X) \isom \Hom_K(L, L')$ avec $\Hom(D', D)$ explicité aussi, est clair.
    \item[II)] $n' = 1$, $n = 0$. Ici $n = 0$ implique déjà $I = \emptyset$. Pour que $\Hom(X', X) \neq \emptyset$, il faut que l'on ait $I' = \emptyset$ (par la condition $F^{-1}(S)(= f^{-1}(\emptyset) = \emptyset) = S'$ qui implique $S' = \emptyset$, si $f: (X', S') \to (X, S)$) - dans la description des morphismes $D' \to D$, cela correspond au fait qu'on ne peut avoir d'application $\tau: I' \to I (= \emptyset)$ que si $I' = \emptyset$. Se bornant donc au cas $I' = \emptyset$ (sinon $\Hom(X', X) = \Hom(D', D) = \emptyset$ et on est heureux), on trouve donc que $D$ revient à la donnée de $E \xhookrightarrow{p} \Gamma$ sous-groupe ouvert, et $D'$ à la donnée d'un homomorphisme de groupes profinis $p': E' \to \Gamma$, dont l'image sera un sous-groupe ouvert de $\Gamma$, soit $\Gamma'_0 \subset  \Gamma$, correspondant à une sous-extension finie $L'$ de $\overline{K}/K$. Les choix faits relatifs à $X'$ implique qu'on a un isomorphisme fixé $L' \isom \mathrm{H}^0(X', \underline{\cO}_{X'})$. Ceci dit, un $K$-morphisme de $(X', \emptyset)$ dans $(X, \emptyset)$ i.e. de $X' \to X$ revient (comme $X$ affine) à la donnée d'un $K$-morphisme $\Spec L' \to X$ $(\isom \Spec L)$ de l'enveloppe affine, i.e. d'un $K$-homomorphisme $L \to L'$ (ce qui ne dépend que de $L$, $L'$ ou encore que des sous-groupes $E \hookrightarrow \Gamma$ et $\Gamma_0 \hookrightarrow \Gamma$ de $\Gamma$).
    
    D'autre part la description en termes des diagrammes $D$, $D'$ revient à dire qu'un morphisme est déterminé par un couple $(f, \alpha)$, $f: E' \to E$ et $\alpha \in \Gamma$, tels que l'on ait encore commutativité
    \[\begin{tikzcd}
	{E'} & E && {pf = \text{int}(\alpha)p'} \\
	\Gamma & \Gamma
	\arrow["f", from=1-1, to=1-2]
	\arrow["{p'}"', from=1-1, to=2-1]
	\arrow["p", from=1-2, to=2-2]
	\arrow["{\text{int}(\alpha)}", from=2-1, to=2-2]
    \end{tikzcd}\]
    ce qui implique encore que $f$ est déterminé par $\alpha$ (savoir, induit par $\text{int}(\alpha)$), et $\alpha$ étant sujet à la seule condition
    $$
    \alpha \in \Transp_\Gamma (\Gamma'_0, E)
    $$
    - et $\alpha$, $\alpha'$ décrivant le même homomorphisme si et seulement si $\exists \mu \in E$ tel que
    $$
    \alpha' = \mu \alpha
    $$
    donc on a encore
    $$
    \Hom(D', D) \isom _E \textbackslash \Transp_\Gamma (\Gamma'_0, \Gamma)
    $$
    $$
    \isom \Transp_\Gamma (L, L')/_{\Fix_\Gamma(L)}
    $$
    $$
    \isom \Hom_{K-\text{ext}}(L, L')
    $$
    donc on trouve encore dans ce cas le pleine fidélité.
    
    En fait, on aurait pu traiter ensemble les deux cas $(n' = 0, n' = 1)$ où $n = 0$, et la description de $\Hom(X, X')$ en termes de $D$, $D'$ est valable d'ailleurs, sans faire des hypothèses draconiennes (projective, lisse, dimension $\leq 1$, anabélienne) sur $X'$.
    
    En effet, comme $X$ est fini étale sur $K$, son image inverse $X_{X'} = X' \times_K X$ sur $X'$ est fini étale sur $X'$, et 
    $$
    \Hom_K(X, X') \isom \Sections (X_{X'}/X')
    $$
    peut se décrire en termes de la catégorie $\widehat{\Pi_1(X')} (= \Pi_U)$ des systèmes locaux finis sur $X'$ i.e. des systèmes locaux finis sur $\Pi_U$. Suivant cette description, on aboutit à la description commune donnée dans I, II, sans aucune hypothèse autre sur $X$ que la 0-connexité (pour pouvoir décrire $\Pi_U = \Pi_{X'}$ sur $\Pi_e$ en termes d'un homomorphisme de groupe $p': E' \to \Gamma$\dots)
    
    Quand on se donne $(X', S')$, avec $S'$ pas nécessairement vide, $X'$ 0-connexe (et $X'$ lisse de dimension 1 aux points de $S'$ (il suffirait même que $S'$ ne disconnecte pas localement pas localement (ét) $X'$, donc $X' \textbackslash S'$ 0-connexe\dots), alors la donnée $\Pi_1(X' \textbackslash S') \to \Pi_1 (e_K)$ es décrite encore par $E' \xlongrightarrow{p'} \Gamma$ (indépendamment de la considération des $E_i$ etc) et par suite $\Hom_K(X' \textbackslash S', X)$ est décrit comme on vient de dire. Mais on a (avec les hypothèses de normalité sur $X'$ aux points de $S'$) $\Hom_K(X' \textbackslash S', X) = \Hom_K(X', X)$. Ceci encourage à modifier la définition des morphismes $(X', S') \to (X, S)$ donnée au début, via $f: X' \to X$ avec $f^{-1}(S)_{\text{red}} = S'$,  en exigeant ceci non pour $f$, mais seulement pour la restriction de $f$ à la réunion des composantes irréductibles de $X'$ qui son envoyées dans des composantes irréductibles de $X$ non discrète i.e. non réduites à un point. Dans la description correspondante $D'$, $D$, on spécifiait que l'on n'exige la donnée d'une application $I' \to I$ (et des données correspondantes $f_{i'}$, $\alpha_{i'}$) \emph{que} si $n \neq 0$ i.e. $p_{si}: E \to \Gamma$ pas injectif - dans le cas contraire (impliquant $I = \emptyset$) si par hasard $I' \neq \emptyset$, on laisse tomber la connaissance des $I'$ et des éléments de structure correspondants.
    \item[III)] $n' = 0$, $n = 1$. Ici on a donc $I' = \emptyset$, $p': E' \hookrightarrow \Gamma$ i.e. $D'$ se réduit à la donnée d'un sous-groupe ouvert $E \hookrightarrow \Gamma$ de $\Gamma$.
    
    [\dots ?] $(X', S' = \emptyset)$ et $(X, S)$, la condition $f^{-1}(S)_{\text{red}} = S' = \emptyset$ sur le $K$-morphisme $f: X' \to X$ implique que $f$ se factorise par $X' = \Spec(L') \to X \textbackslash S$, sans préjudice si $S = \emptyset$ ou non, i.e. si $I = \emptyset$ ou non. Donc il s'agit de décrire $\Hom_{K - \text{sch}}(X', S \textbackslash S)$, en termes de $\Pi_1(X') \to \Pi_e = \Pi_1(e_K)$ et de $\Pi_1(X \textbackslash S) (= \Pi_U) \to \Pi_e$. Ici les éventuelles donnés supplémentaires relatives à $I$ ne servent pas explicitement à la description, en termes de diagrammes de groupoïdes.
    
    Sauf erreur, une réduction facile (descente galoisienne) nous ramène au cas où $X' = \Spec K$ i.e. $E' \hookrightarrow \Gamma$. La description des homomorphismes $D' \to D$ est alors encore en terme des couples $(f, \alpha)$, où $f: \Gamma \to E$ et où $\alpha \in \Gamma$ avec les sempiternelles conditions $pf = \text{int} (\alpha) p'$, qui devient (comme $p' = \id$))
    $$
    pf = \text{int}(\alpha)
    $$
    Ici bien sûr $\alpha$ ne décrit plus $f$ (mais l'inverse est vraie, ici la connaissance de $f$ de $pf \in \Aut(\Gamma)$ implique celle de $\alpha$, i.e. $\Centr(\Gamma) = \{ 1 \}$\dots). Les couples $(f, \alpha)$ et $(f', \alpha')$ définissent le même morphisme de diagramme, si et seulement si $\exists \mu \in E$ tel que $f'= \text{int}(\mu) \circ f$, $\alpha' = p(\mu)\alpha$. Comme (si $\exists(f, \alpha)$) $p$ surjectif, on veut que, quitte à choisir $\mu$ au-dessus de $\alpha^{-1}$, et de remplacer $(f, \alpha)$ par $(f', \alpha') = (\text{int}(\mu)f, p(\mu)\alpha)$, on peut toujours décrire un morphisme $D' \to D$ par $(f, \alpha)$ avec $\alpha = 1$ donc par une \emph{section} $f$ de l'homomorphisme $E \to \Gamma$, et deux sections $f$, $f'$ (correspondant à $(f, 1)$, $(f', 1)$) définissent le même homomorphisme de $D'$ dans $D$, si et seulement si $\exists \mu \in E$ tel que $p(\mu) = 1$, i.e. $\mu \in \pi = \Ker(p)$, et tel que $f' = \text{int}(\mu) \circ f$.
    
    Ainsi, les morphismes de $D'$ (correspondant à $X' = \Spec K$) dans $D$ correspondent exactement aux classes de scindages de $E \xlongrightarrow{p} \Gamma$, modulo automorphismes intérieures par des $\mu \in K = \Ker p$. La ``conjecture bordelique'' dans le cas III équivaut donc à ceci:
    \[\begin{tikzcd}
	{\Gamma((X, S)/K)} && {\text{Classes de}~\pi-\text{conjugaison de sections de}~E \to \Gamma \\
	&& \text{i.e de sections de}~\Pi_1(X) \to \Pi_1(e_K)}
	\arrow[from=1-1, to=1-3]
    \end{tikzcd}\]
    est bijective (si $(X, S)$ est un couple permis ``anabélien'')\footnote{Faux tel que, cf. n$^\circ 9$ ci-dessus - il faut des conditions supplémentaires sur les $f$\dots Peut-être si $I = \emptyset$.}.
    
    Je sais en tout cas que cette application est \emph{injective} - ceci vaut chaque fois qu'on a un schéma $U$ sur $K$ (ici $X \textbackslash S$) qui se plonge dans un groupe algébrique $G$ extension d'une variété abélienne par un tore, et résulte alors du théorème de Mordell-Weil ``absolu'' : $G(K)$ est un $\mathbb{Z}$-module de type fini. En fait, il suffit de connaître la classe de splittage de l'extension $E^\natural$ de $\Gamma$ par $\pi_{ab} = \pi/\overline{[\pi, \pi]} \isom \mathrm{H}_1(U)$, correspondant à un point de $U$ rationnel sur $K$ pour connaître ce point. Donc le résultat de fidélité est obtenu, avec des hypothèses moins draconiennes sur $X$ que l'hypothèse anabélienne (avec les notations du lemme 5.2, cela signifie que $g = 0 \Rightarrow n \geq 2$ i.e. qu'il n'y a pas de composante irréductible de $U_{\overline{K}}$ qui soit isomorphe à $\mathbb{P}^1_{\overline{K}}$ ou $\mathbb{E}^1_{\overline{K}}$\dots).
    
    N.B. Ceci suggère une approche de la conjecture de Mordell, via une meilleure connaissance des extensions de $\Gamma$ par $\pi$ : le fait qu'il n'a ait (pour $\geq 2$) qu'un nombre fini de classes de $\pi$-conjugaison de scindages d'une telle extension\dots
    
    Itou pour une approche du théorème de Fermat, via une bonne connaissance d'une extension de $\Gamma$ par $\pi_1(\mathbb{P}\frac{1}{\mathbf{Q}}\textbackslash \{ 0, 1, \infty \})$\dots
    \item[IV)] $n = 1$, $n' = 1$ i.e. $X$, $X'$ de dimension 1.
    \begin{enumerate}
        \item[a)] Si $I = I' = \emptyset$, je n'ai rien à ajouter à la description des morphismes $f: X \to X'$ en termes de morphismes de diagrammes de groupoïdes. Ici la condition $f^{-1}(S)_{\text{réd}} = S'$ n'est pas une restriction sur $f$ - on n'exclut donc pas des applications constantes. Celles-ci (grâce à l'étude du cas II) sont d'ailleurs décrites de fa\c{c}on ``pleinement fidèle'' pas des homomorphismes de diagrammes - ils correspondent aux cas $f: E' \to E$ qui sont nuls sur $\pi'$.
        \item[b)] Le cas $I' = \emptyset$, $I \neq \emptyset$ (i.e. $S' = \emptyset$, $S \neq \emptyset$) signifie, avec la condition $f^{-1}(S)_{\text{réd}} = S'$ i.e. $f^{-1}(S) = \emptyset$, que $f$ doit appliquer $X'$ dans $X \textbackslash S$, donc que tout $f$ correspond à un morphisme de $(X', \emptyset)$ dans $(X, S)$ soit constant. La conjecture bordélique dans ce cas exprime donc essentiellement qu'au niveau des homomorphismes $E' \xlongrightarrow{p'} \Gamma$ et $E \xlongrightarrow{p} \Gamma$, tout homomorphisme des groupes $E' \xlongrightarrow{p} E$ tel que $pf = \text{int}(\alpha)p'$ pour $\alpha \in \Gamma$ [on est ramené au cas où $\alpha = 1$, $p$, $p'$ surjectifs, i.e. aux homomorphismes d'extensions de $\Gamma$ par des groupes $\pi = \pi_1 (\overline{X \textbackslash X})$ et $\pi' = \pi_1(\overline{X'})$] est trivial sur $\pi'$ [i.e. ``est'' une section]. Il faudrait essayer de vérifier ce point directement, qui (modulo le cas III) établirait la ``conjecture bordélique'' dans ce cas-là.
        \item[c)] Le cas $I' \neq \emptyset$, $I = \emptyset$ implique que $\Hom((X, S), (X', S')) = \emptyset$ (cas $f^{-1}(S) = f^{-1}(\emptyset) = \emptyset \neq S'$ !), ce qui correspond bien à $\Hom(D', D) = \emptyset$ puisque $\nexists \tau: I' \to I$.
        \item[d)] Dans le cas $I' \neq \emptyset$, $I \neq \emptyset$, la condition $f^{-1}(S)_{\text{réd}} = S'$ implique que $f$ n'est pas constante (sinon on aurait $f^{-1}(S)_{\text{réd}} = \emptyset$ ou $X'$), donc $f$ est dominant et fini, génériquement étale. [Donc, de tels $f$ ne peuvent intervenir que si $I$ et $I'$ sont :
        
        tous deux $\emptyset$ (cas a)], soit tous deux $\neq \emptyset$ (cas actuel d)). Le afit que dans ces cas-là, la description diagrammatique galoisienne soit \emph{fidèle} (peut-être pas pleinement) résulte aisément du résultat analogue dans le cas III. La pleine fidélité par contre est chose mystérieuse, même si le résultat correspondant dans le cas III était acquis (et encore semble loin, même si $K$ fini sur $\mathbf{Q}$\dots).
    \end{enumerate}
\end{enumerate}

{\bf Réductions élémentaires.} Pour prouver la ``conjecture bordélique'', on est ramené (par des extensions finies du corps $K$) au cas où $E \to \Gamma$ (et, si on y tient, aussi $E' \to \Gamma$) est épimorphique. Sauf erreur, la connaissance du cas III pour des extensions de type fini de $K$, implique le cas général pour $K$ (avec suffisamment de sueurs techniques\dots).

Revenant cependant au cas général quand $E \to \Gamma$ est surjectif, dans toute classe d'équivalence de systèmes $(f: E' \to E, \alpha \in \Gamma, \tau: I' \to I, (\alpha_i \in E_{\tau(i') = i})_{i' \in I'})$, on peut trouver un système avec $\alpha = 1$, de sorte que $f: E' \to E$ soit compatible avec les homomorphismes dans $\Gamma$ (i.e. les structures d'extension). Quand on se borne à de tels systèmes, l'équivalence par conjugaison se fait par un système $(\mu, (\mu_{i'})_{i' \in I'})$ avec $\mu \in \pi (= \Ker p)$, et les $\mu_{i'} \in E_{i = \tau(i')}$ comme avant. Ainsi, $f: E' \to E$ est un homomorphisme d'extensions, défini modulo automorphisme intérieur par un élément de $\pi = \Ker (E \to \Gamma)$, et satisfaisant à des conditions explicitées par ailleurs. Il se pourrait (comme on a déjà remarqué) que la connaissance de la classe de $\pi$-conjugaison de $f$ suffise à déterminer le ``morphisme bordélique'' dans lequel $f$ s'insère (si $I, I' \neq \emptyset$). [C'est lié à la question de savoir si $\forall d \in \mathbf{N}^*$, $\Trans_E(L^d_i, L_i)$ est réduit à $E_i$ (où $L_i = \Ker (p_i: E_i \to \Gamma) = \text{Im}(\kappa_i: T(\overline{K}) \to E_i)$). Dans ce cas, $E_i$ serait connu en termes de $L_i \subset  \pi$, et même en termes du noyau de sous-groupes qu'il définit dans $\pi$, et a fortiori en termes des noyaux d'homomorphismes définis par $\kappa_i: T(\overline{K}) \to \pi$\dots

{\bf Cas où $E_i \to \Gamma$ sont surjectifs} (i.e. les $k(s_i) = K$ i.e. $s_i$ rationnel sur $K$).

Alors si $(f, \tau, (\alpha_{i'}))$ est un homomorphisme de $D'$ dans $D$, quitte à corriger par des $\mu_i \in E_i$ ayant même image que les $\alpha_{i'}$ dans $\Gamma$, on peut supposer $\alpha_{i'} \in \pi$, en plus de $\alpha = 1$ (obtenu en corrigeant par $\mu$ convenable). Donc on décrit l'homomorphisme par $(f, 1, \tau, (\alpha_{i'}))$, les $\alpha_{i'}$ appartenant à $\pi$. La condition ici est que $f: E' \to E$ soit un homomorphisme de groupes sur $\Gamma$, que $f(E'_{i'}) \subset  \text{int}(\alpha^{-1}_{i'} E_{\tau(i') = i})$, et que l'homomorphisme induit $E'_i \to E_{i'}$ par $\text{int}(\alpha_{i'}) f$ induit sur $T(\overline{K})$ (via $\kappa_{i'}, \kappa_i$) \emph{l'homomorphisme de multiplication par $d_i$ sans} plus - que c'est beau !

Deux systèmes $(f, \tau, (\alpha_{i'}))$ et $(f', \tau', (\alpha'_{i'}))$ définissent le même morphisme, si et seulement si $\tau = \tau'$ et s'il existe $\boxed{\mu \in \pi, \mu_{i'} \in L_{i = \tau (i')} = \kappa_i(T(\overline{K}))}$ tels que
$$
f' = \text{int}(\mu)f \quad \alpha'_{i'} = \mu_{i'} \alpha_{i'}
$$

Notons qu'à priori, pour $(f, \tau)$ fixé, les $\alpha_{i'}$ sont déterminés modulo multiplication à droite par des éléments de $\Transp_\pi ((L)^{d_{i'}}_i, L_i)$, qui est sans doute $= L_i = \Ker(\kappa_i: E_i \to \Gamma) = 1 = \text{Im}(\pi_i: T(\pi) \to E_i \pi)$. Donc on trouve que la classe de $\pi$-conjugaison de groupes sur $\Gamma$ de $f: E' \to E$ suffit à déterminer l'homomorphisme $D' \to D$ dans la catégorie bordélique. Je présume qu'un peu de sueur permettrait de prouver que cela marche encore dans le cas général (sans supposer les 
$\Gamma_i = \text{Im}(E_i \to \Gamma)$ égaux à $\Gamma$).

Cela signifie (dans le cas actuel) que la structure essentielle qui décrit $(X, S)$ est celle \emph{d'extension extérieure} de $\Gamma$ par un groupe $\pi$ [homomorphisme extérieur surjectif d'un groupe, de [?] du $\mathbf{R}$], avec $\pi$ muni d'une \emph{structure à lacets i.e. extérieurs $\kappa_i: T(= T(\overline{K}) \isom \hat{\mathbf{Z}}) \to \pi$} (satisfaisant certaines conditions), les homomorphismes de $E' \to \Gamma$ (noyau $\pi'$) dans $E \to \Gamma$ (noyau $\pi$) étant les classes de $\pi$-conjugaisons d'homomorphismes $E' \to E$ de groupes sur $\Gamma$, induisant des homomorphismes $\pi' \to \pi$ compatibles avec la structure à lacets.









%%%%%%%%%%%%%%%%%%%%%%%%%%%%%%%%%%%%%%%%%%%%%%%%%%%%%%%%%%%%%%%
\chapter*{\S \space 9. --- STRUCTURE TANGENTIELLE EN LES $s\in S$ \\ Les sections d'extensions ``de deuxième type''}\thispagestyle{empty}
\addcontentsline{toc}{section}{9. Structure tangentielle en les $s\in S$}
\label{sec:9}
\section*{}

Revenant à la description des $(X, S)$ par $\Pi_{D^*} \to \Pi_U \to \Pi_e$, un $s_i \in S$ correspond donc à une composante connexe de $\Pi_{D^*_i}$ dans $\Pi_{D^*}$, ou un $\Pi_{D_i}$ de son quotient $\Pi_D$.

Considérons $\Pi_{D^*_i} \xlongrightarrow{\sigma_i = \sigma_{D^*_i D_i}} \pi_{D_i}$. On va décrire certaines sections (à isomorphisme près) de ce morphisme de groupoïdes, i.e. des couples $(\gamma, \lambda)$ ou $\gamma$ est un foncteur $\Pi_{D_i} \xlongrightarrow{\gamma} \Pi_{D^*_i}$ et $\lambda$ un isomorphisme $\sigma_i \circ \gamma \isom \id$.

Si on choisit un $\widetilde{D^*_i} \in \Ob \prod_{D^*_i}$ (revêtement universel de $D^*_i$) d'où [un] homomorphisme surjectif $E_i = \Aut(\widetilde{D^*_i}) \to \Gamma_i = \Aut(\sigma_i (\widetilde{D^*_i}) = \widetilde{D_i})$, la donnée de $(\gamma, \lambda)$, et d'un isomorphisme $\gamma(\widetilde{D_i}) \isom \widetilde{D^*_i}$ donnant l'isomorphisme identique par application des $\sigma_i$, revient à celle d'un homomorphisme $\Gamma_i \to E_i$ - et les classes d'isomorphismes des couples $(\gamma, \lambda)$ correspondent aux classes de $L_i$-conjugaison $(L_i = \Ker(E_i \to \Gamma_i))$ de sections de l'extension $E_i$ de $\Gamma_i$ par $L_i$.

Ceci posé, on a 
$$
\Gamma_i = \Gal(\overline{K}_i/K_i)
$$
où $K_i = K(s_i)$, $\overline{K}_i$ est la clôture algébrique de $K_i$ définie par $\widetilde{D_i}$ (à priori, elle n'est pas canoniquement isomorphe à $\overline{K}$ sur $K$\dots), et les classes de scindages d'extensions forment un \emph{torseur} $\Sigma_i$ (s'il y en a, et on sait qu'il y en a\dots) sous $\mathrm{H}^1(\Gamma_i, L_i)$, ou encore, comme $\kappa_i: L_i \xlongleftarrow{\sim} T(\overline{K}_i)$, sous $\mathrm{H}^1(\Gal(\overline{K}_i/K_i), T(\overline{K}_i)) \isom \mathrm{H}^1(K_i, T_l(\mathbb{G}_m)) \isom \varprojlim_n (K_i, \mu_n)$.

Or la suite exacte de Kummer donne
\[\begin{tikzcd}
	0 & {\mathrm{H}^0(K_i, \mathbb{G}_m)_n} & {\mathrm{H}^1(K_i, \mu_n)} & {_n \mathrm{H}^1(K_i, \mathbb{G}_m)} & 0 \\
	& {(K^*_i)_n} && 0
	\arrow[shift left=1, no head, from=1-2, to=2-2]
	\arrow[shift right=1, no head, from=1-2, to=2-2]
	\arrow[shift left=1, no head, from=1-4, to=2-4]
	\arrow[shift right=1, no head, from=1-4, to=2-4]
	\arrow[from=1-2, to=1-3]
	\arrow[from=1-3, to=1-4]
	\arrow[from=1-1, to=1-2]
	\arrow[from=1-4, to=1-5]
\end{tikzcd}\]
i.e. $\mathrm{H}^1(K_i, \mu_n) \isom (K^*_i)_n$, donc $\mathrm{H}^1(K_i, T(\mathbb{G}_m)) \isom \varprojlim_n (K^*_i)_n$

Notons qu'on a un homomorphisme kummérien
$$
K^*_i \to \underbrace{\mathrm{H}^1(K_i, T(\mathbb{G}_m))}_{\hspace*{-5mm}H_i\hspace*{-5mm}} \isom \varprojlim (K^*_i)_n
$$
dont le noyau est formé des $x \in K^*_i$ tel que pour tout $n \in \mathbf{N}$, on ait $x \in K^{*n}_i$. Comme $K_i$ est de type fini sur $\mathbf{Q}$, cet homomorphisme est injectif, i.e. $K^*_i$ s'identifie à un sous-groupe de $H_i$.

Ceci posé, on va définir, dans le torseur $\Sigma_i$ sous $H_i$ des scindages de $D^*_i$ sur $D_i$, un sous-$K^*_i$-torseur (i.e. un élément de $\Sigma_i/K^*_i$).

Pour ceci, considérons plus généralement le cas d'un corps $k$ (= $K_i$) de caractéristique $0$, et d'une $k$-algèbre $O$ qui est une jauge hensélien de corps résiduel $k$. Soit $L$ le corps des fractions de $O$. On va, pour tout uniformisante $t$ de $O$, définir une section $(\gamma, \lambda)$ du groupoïde des clôtures algébriques de $L$, vers celui ses revêtements universels de $\Spec O$ ou, ce qui revient au même, des clôtures algébriques de $k$. Pour ceci, soit $O(n, t) = O[T_n]/(T^n_n - t)$, et $O(\infty, t) = \varinjlim_n O (n, t)$, par $T_n \mapsto T^m_{mn}$.

Soit $L(n, t) = O(n, t) \otimes_O K =$ corps de fractions de $O(n, t)$, $L(\infty, t) = \varinjlim L(n, t)$le corps des fractions de $O(\infty, t)$. C'est une extension algébrique de $L$ et pour tout extension finie ou ind-finie $k'$ de $k$, posant $O' = O \otimes_k k'$ (qui est aussi une jauge hensélienne sur $k'$ de corps résiduel $k'$, de corps des fractions $L' \isom L \otimes_k k'$), on a des isomorphismes canoniques
$$
O(n, t) \otimes_O O' \isom O'(n, t)
$$
$$
O(\infty, t) \otimes_O O' \isom O'(\infty, t)
$$
d'où
$$
L(n, t) \otimes_k k' \isom L'(n, t)
$$
$$
L(\infty, t) \otimes_k k' \isom L'(\infty, t)
$$
Or si $k'$ est algébriquement clos $O'$ strictement hensélien, on sait que $L'(\infty, t)$ est alors algébriquement clos. D'ailleurs comme $K' \subset  O'(\infty, t) \subset  L'(\infty, t)$ dans ce cas $k$ [?] s'identifie à la clôture algébrique $\overline{k}$ de $k$ dans $L'(\infty, t) = \overline{L}$. Donc on a bien trouvé une section de $\Pi_{D^*}$ sur $\Pi_D$.

Mais si on remplace $s$ par $s' = su^n$, avec $u \in O^*$, on trouve des isomorphismes
$$
\lambda(n, u): O(n, s)(= O[T_n]/(T^n_n - s)) \to O(n, s')(= O[T_n]/(T^n_n - s'))
$$
$$
\text{par}~T_n \mapsto uT_n
$$
d'où par passage à la limite
$$
\lambda(\infty, n): O(\infty, s) \to O(\infty, s')
$$
induisant
$$
L(\infty, s) \isommap L(\infty, s')
$$
isomorphisme d'extension de $L$, d'où un isomorphisme entre les sections correspondantes des $\Pi_L$ sur $\Pi_k$. Ceci implique que si $s$, $s'$ diffèrent par un $v \in O^*$ tel que son image dans $k$ soit 1 (i.e. $v \in 1 + m$ [qui est un sous-groupe divisible de $O^*$] alors $s$, $s'$ définissent des sections isomorphes).

On trouve une application canonique
\[\begin{tikzcd}
	{\begin{matrix} (m/m^2)^* \\ \text{(ensemble des bases de}~m/m^2) \\ \text{qui est un torseur sous}~k^* \end{matrix}} & {\begin{matrix} \text{(classes d'isomorphie de sections de}~\Pi_L~\text{sur}~\Pi_k \\ \text{i.e. de}~\Gal(\overline{L}/L) \to \Gal(\overline{k}/k) \\ \text{qui est un torseur sous}~\mathrm{H}^1(\Gal(\overline{k}/k), T(\overline{k})) \\ \text{i.e. sous}~\mathrm{H}^1(k, T(\mathbb{G}_m))\dots) \end{matrix}}
	\arrow[from=1-1, to=1-2]
\end{tikzcd}\]

On constate aussitôt que cette application est compatible avec l'homomorphisme
$$
\theta: k^* \to \HH^1(k, T(\mathbb{G}_m))
$$
sur les groupes d'opérateurs de ces torseurs.

Dans le cas où cet homomorphisme est injectif (par exemple $k$ de type fini sur $\mathbf{Q}$) on trouve donc un plongement de $m/m^2$ comme un sous $\theta (k^*)$-torseur du $\mathrm{H}^1(k, T(\mathbb{G}_m))$-torseur des scindages de $\Pi_{D^*}$ ($\isom$ groupoïde des clôtures algébriques de $L$) sur $\Pi_D$ ($\isom$ groupoïde des clôtures algébriques de $k$).

Revenant au cas des $\Pi_{D^*} \to \Pi_U \to \Pi_e$, $K$\dots et des $D: T(\overline{K}^*) \xlongrightarrow{\kappa_i} E_i \hookrightarrow E \xlongrightarrow{p} \Gamma$ qui les explicitent, on trouve donc une structure supplémentaire sur ce $D$ via les extensions $E_i$ de $\Gamma_i \subset  \Gamma$ par $T(\overline{K}^*)$, savoir $\forall i \in I$, un sous-$K^*_i$-torseur dans le torseur sous $\mathrm{H}^1(K_i, T(\mathbb{G}_m)) \isom \varprojlim (K^*_i)_n$ des classes de scindages de $E_i$. Il faudrait expliciter le comportement de cette structure supplémentaire relativement à des morphismes (provenant de situations géométriques) et la relation avec la norme des 1-formes différentielles - j'ai la flemme de l'expliciter en long en en large !

La chose nouvelle que je retiens surtout, c'est que pour $I \neq \emptyset$ et les $E_i \to \Gamma$ surjectifs (pour simplifier), on trouve $\forall i$ une ``famille trancendante'' de scindages de $E \to \Gamma$ (via les scindages de $E_i$ sur $\Gamma$) - essentiellement [?] par $\mathrm{H}^1(K, T(\mathbb{G}_m)) \isom \varprojlim (K^*)_n$ (plus précisément, par un torseur sous le dit) - dans cette famille, une sous-famille de scindages qu'on peut qualifier de ``géométriques'', indexée par $K^*$ (plus précisément, un sous-torseur\dots). On vérifie que les classes de $\pi$-scindages obtenus par des indices $i$, $i'$ distincts sont distinctes, même en se ramenant aux scindages correspondants de l'extension de $\Gamma$ par $\pi_{ab}$ déduite de l'extension $E$ de $\Gamma$ par $\pi$ (je ne fais pas le détail des vérifications, via Mordell-weil\dots) et distincts des scindages associées aux points rationnels sur $K$ de $X \textbackslash S$.

Il faudrait corriger la conjecture bordélique dans le cas III, en énon\c{c}ant (sous toutes réserves, encore !) qu'il n'y a (peut-être) pas d'autres scindages de l'extension $E$ par $\pi$ que ceux-là\dots














%%%%%%%%%%%%%%%%%%%%%%%%%%%%%%%%%%%%%%%%%%%%%%%%%%%%%%%%%%%%%%%
\chapter*{\S \space 10. --- AJUSTEMENT DES HYPOTHÈSES (REMORDS)}\thispagestyle{empty}
\addcontentsline{toc}{section}{10. Ajustement des hypothèses (remords)}
\label{sec:10}
\section*{}

Je me rends compte que dans la définition des morphismes $(X', S') \xlongrightarrow{f} (X, S)$, l'hypothèse $f^{-1}(S)_{\text{red}} = S'$ est étriquée - il faut prendre des morphismes \emph{quelconques} $X' \textbackslash S' \xlongrightarrow{f} X \textbackslash S$ (se prolongeant bien sûr en $\hat{f}: X' \to X$). On aura donc $S' \supset \hat{f}^{-1}(S)_{\text{red}}$, mais $S'$ peut être strictement plus grand que $\hat{f}^{-1}(S)_{\text{red}}$.

Il faut ajuster en conséquence la description de la ``catégorie bordélique'' - les objets restent les diagrammes de groupoïdes $D: \Pi_{D^*} \to \Pi_U \to \Pi_e$ (plus donnée de Kummer), mais un morphisme d'un $D'$ dans un $D$ ne définit plus nécessairement un $\Pi'_{D^*} \to \Pi_{D^*}$ (en plus de $\Pi'_U \to \Pi_U$) il faut se donner une partie $I'_f$ de $I' = \pi_0(\Pi'_{D^*})$ et se donner seulement $\Pi'_{D^*, I'_f} \xlongrightarrow{f_{D^*}} \Pi_{D^*}$ (avec donnée de commutation $\alpha_{D^*, U}$ relative au carré avec $f_U$). Bien sûr, dans la description en termes de $E'$, $E$ etc, on exige que pour $i' \in I' \textbackslash I'_f$, $f_E: E' \to E$ est trivial sur $E'_{i'} \subset  E'$ - et $\tau$ est défini comme $I'_{f'} \to I$; les données relatives aux $i' \in I'_{f'}$ ($f_{i'}: E_{i'} \to E_{i = \tau(i')}$, les $\alpha_{i'} \in E$) sont pareilles que dans le cas envisagé précédemment.

N.B. Cela signifie en fait qu'on commence à ``boucher les trous'' (de $X'$) correspondants aux $i' \in I' \textbackslash I'_{f'}$, en rempla\c{c}ant $E'$ par le groupe quotient de $E'$ par le sous-groupe invariant engendré par les $L_{i'} = \kappa_{i'}(T(\overline{K})) (i' \in I' \textbackslash I'_{f'})$, et les $E_{i'} (i' \in I'_f)$ par leurs images dans le dit groupe quotient, et en oubliant $I' \textbackslash I'_{f'}$ i.e. rempla\c{c}ant $I'$ par $I'_f$.

[Pour bien faire, il faudrait exprimer ces opérations aussi au niveau des diagrammes de groupoïdes $\Pi'_{D^*} \to \Pi'_U \to \Pi_e$\dots].

Cela signifie donc qu'en fait, on s'est ramené à la situation envisagée au début, où $S' = \hat{f}^{-1}(S)$\dots Donc finalement la différence des deux points de vue n'est pas énorme, et celui adopté au début a l'avantage de l simplicité plus grande (tout est relatif !)














%%%%%%%%%%%%%%%%%%%%%%%%%%%%%%%%%%%%%%%%%%%%%%%%%%%%%%%%%%%%%%%
\chapter*{\S \space 11. --- CONDITIONS SUR LES SYSTÈMES DE GROUPOÏDES OBTENUS À PARTIR DE SITUATIONS GÉOMÉTRIQUES}\thispagestyle{empty}
\addcontentsline{toc}{section}{11. Conditions sur les systèmes de groupoïdes obtenus à partir de situations géométriques}
\label{sec:11}
\section*{}

On en a déjà cité au passage, par exemple que les groupes noyaux de $\Pi_{D^*} \to \Pi_e$ sont abéliens (avec l'isomorphisme de Kummer $\kappa$) et que les images sont des sous-groupes ouverts, et de même pour l'image des $\pi_1$ dans le cadre de l'interprétation en termes de
$$
T(\overline{K}) \xlongrightarrow{\kappa_i} E_i \xlongrightarrow{\rho_i} E \xlongrightarrow{p} \Gamma
$$
(cas de la ``dim $1$'' - on n'exclut pas le cas $I = \emptyset$ le cas ``dim $0$'' étant trivial\dots)

\begin{enumerate}
    \item[a)] L'image $\Sigma$ de $p$ est ouverte, plus précisément il existe un sous-groupe ouvert $\Gamma' \subset  \Gamma$ au dessus du quel $E$ ait une section [et même une ``section admisible'' en un sens qui sera défini par la suite - de sorte à exclure les sections ``triviales'' provenant des $E_i$\dots].
    \item[b)] L'image $\Sigma_i$ de $E_i$ dans $\Gamma$ est ouverte, et l'extension $E_i$ de $\Sigma_i$ par $L_i( \isom (\overline{K}))$ est triviale. [En fait, on a une classe privilégiée de scindages, dits ``algébriques'', formant un torseur sous $K^*_i \hookrightarrow \varprojlim(K^*_i)_n$, cf n$^\circ$ 9 -- mais on ne va pas considérer pour l'instant cet élément de structure supplémentaire\dots]
\end{enumerate}
Le reste des conditions concerne essentiellement la structure des groupes
$$
\pi = \Ker(p: E \to \Gamma)
$$
avec ses classes de conjugaison de sous-groupes (ou plutôt les homomorphismes extérieures $\kappa_i: T(= T(\overline{K}^*) \to \pi)$, et la fa\c{c}on dont $\Sigma$ opère extérieurement sur $\pi$. Ces conditions s'expriment de fa\c{c}on particulièrement simple lorsque les $\Sigma_i \subset  \Gamma$ sont égaux à $\Gamma$ (``les points de $S$ sont rationnels sur $K$'') i.e. $E_i \to \Gamma$ surjectif, a fortiori $\Sigma = \Gamma$ i.e. $E \to \Gamma$ surjectif. On va se borner à se cas. Le cas général s'en déduit par extension finie du corps   de base [chaque point de $S$ non rationnel sur $K$ i.e. chaque $\Sigma_i \neq \Gamma$ donne naissance à $n_i$ points, $n_i = [\Gamma : \Sigma_i]$ - et $X_{K'}$ se scinde en $n = (\Gamma : \Sigma)$ composantes connexes qui sont géométriquement connexes] - mais j'ai la flemme d'expliciter comme il faudrait, dans le contexte des groupoïdes ou des homomorphismes de groupes profinis, l'opération d'extension du corps de base\dots

Il est entendu que les conditions que je vais décrire seront invariantes par extension du corps de base.
\begin{enumerate}    
    \item[c)] $\forall i$, l'homomorphisme extérieur
    $$
    \kappa_i: T \to \pi
    $$
    est compatible avec l'action extérieure de $\Gamma$ (opérant sur $T$ par le caractère cyclotomique $\chi$, et sur $\pi$ grâce à l'extension $E$ de $\Gamma$ par $\pi$). En d'autres termes, pour tout $g \in E$, existe un $\alpha \in \pi$ (N.B. pas seulement $\alpha \in E$ !) tel que l'on ait :
    $$
    \text{int}(g)\kappa_i(\xi) = \text{int}(\alpha)\kappa_i(\chi(p(g))\xi) \quad \text{i.e.} \quad \text{int}(\alpha^{-1}g)\kappa_i(\xi) = \kappa_i(\chi(p(g))\xi)
    $$
\end{enumerate}
Ceci signifie que $1^\circ)$ $\alpha^{-1}g$ normalise $L_i = \kappa_i(T)$ [et ceci signifie même, probablement, que $\alpha^{-1}g \in E_i$ - et qu'on puisse trouver un tel $\alpha \in \pi$ (tel que $\alpha^{-1}g \in E_i$) provient de l'hypothèse $E_i \to \Gamma$ surjectif - on prend un $\beta (= \alpha^{-1}g) \in E_i$ ayant même image que $g$ dans $\Gamma$ et on prend $\alpha = g \beta^{-1}$] et que $2^\circ)$ l'action intérieure de $\beta = \alpha^{-1}$ sur $T$ n'est autre que par multiplication par $\chi(g) = \chi(\beta)$ - ce qui (pour $\beta \in E_i$) n'est autre que la condition déjà explicitée que l'homomorphisme de groupes $\kappa_i: T \to E_i$ est compatible avec l'action de $E_i$, opérant sur $T$ via $\chi \circ p|_{E_i}$, et sur lui même par automorphismes intérieures.

Donc la condition c) n'est pas vraiment nouvelle - je la réexplicite en termes un peu différents, à cause de son importance. Elle implique que l'opération de $\Gamma$ sur $\pi$ est \emph{très} non triviale (puisque $\chi: \Gamma \to \widehat{\mathbf{Z}}^*$ a une image ouverte !) - il n'était pas même évident, a priori (sans raisons arithmétiques profondes !) - compte tenu de la structure de $\pi$ qu'on va donner - qu'il existe de telles opérations de $\Gamma$ sur $\pi$ !

Cette condition sera complétée par une condition de non trivialité à la Weil.
\begin{enumerate}    
    \item[d)] $\exists \eta \in T^*$ (une base de $T$), et des $\alpha_i \in \pi$ (afin de conjuguer $\kappa_i$ en $\kappa'_i = \text{int}(\alpha_i) \circ \kappa_i$), enfin un entier $g \geq 0$ et des éléments $x_j$, $y_j \in \pi$ $(1 \leq j \leq g)$, tels que l'on ait 
    \begin{enumerate}    
    \item[$1^\circ)$] Les $\kappa'_i(\eta)$, et les $x_j$, $y_j$ engendrent le groupe profini $\pi$.
    \item[$2^\circ)$] Ils satisfont la relation 
    $$
    \boxed{[x_1, y_1].[x_2, y_2]\dots[x_g, xy_g]\kappa'_1(\eta)\kappa'_2(\eta)\dots\kappa'_{\nu}(\eta) = 1}
    $$
    \item[$3^\circ)$] Cette relation, avec les générateurs envisagés, décrit $\pi$ (en tant que groupe profini) par générateurs et relations\dots
\end{enumerate}
\end{enumerate}
 N.B. On sait que par ces conditions, $g$ est uniquement déterminé, par exemple par le fait que $\pi_{ab}/\Sigma \kappa_i(T)$ est un $\widehat{\mathbf{Z}}$-module libre de rang $2g$ - $\pi_{ab}$ étant libre de rang $2g + \nu -1$ où $\nu = \text{card}(I)$.
 
 Pour le choix de $\eta$, on voit que si $\eta$ convient, alors tout $\chi(\alpha)\eta$ aussi (où $\alpha \in \Gamma$) - quitte à prendre des conjugués. Donc les $\eta$ qui conviennent contiennent un sous-torseur de $T^*$ sous le sous-groupe ouvert $\chi(\Gamma)$ de $\widehat{\mathbf{Z}}^*$.
 
 En fait, comme la structure du groupe $\pi$ est indépendante de $K$, prenant $K = \mathbf{Q}$ (et en admettant qu'il existe une courbe lisse projective (géométriquement connexe de genre $g$ sur $\mathbf{Q}$, ayant $\nu$ points rationnels sur $\mathbf{Q}$ !), on trouve que si les $l_i$ ]$(1 \leq i \leq \nu)$ s'insèrent dans un système de générateurs privilégies (avec des $x_j$, $y_i$) alors pour tout $\rho \in \widehat{\mathbf{Z}}^*$, on peut trouver des conjugués $l'_i$ des $l^{\rho}_i$ qui s'insèrent de même. Même pour $\rho = -1$ ce n'est pas entièrement trivial\dots 
 
 Enfin, on va énoncer une condition draconienne de non trivialité de l'opération de $\Gamma$ sur $\pi$. Soient $E'$ un sous-groupe ouvert (donc d'indice fini) de $E$, $\pi' = \pi \cap E' = \Ker(E' \to \Gamma)$, $\Gamma'$ l'image de $E'$ dans $\Gamma$. On trouve donc une extension de $\Gamma' (= \Gal(\overline{K}'/K'))$ sur $\pi'$, donc aussi par $(\pi'_{ab})(\ell)$ ($\ell$ étant un nombre fourni), qui est (on le sait par $d$)) un $\mathbf{Z}_{\ell}$-module libre de type fini, sur lequel $\Gamma'$ opère.
 
 [Avec un peu de travail\footnote{\c{C}a se fait très élégamment dans le contexte $\Pi_{D^*} \to \Pi_{U} \to \Pi_{e}$.}, on doit pouvoir mettre sur $E' \to \Gamma$ une ``structure à lacets'' i.e. des $E'_{i'}$ comme pour $E$, et décrire dans $(\pi´_{ab}(\ell))$ la somme des images des $L'_{i'}$, sur lesquels $\Gamma'$ opère donc via le caractère $\chi$. On s'intéresse au quotient de $(\pi'_{ab})(l)$ par ce sous-module relativement trivial (la partie ``VA'' du module $\ell$-adique envisagé). Ceci posé, on exige que la représentation de $\Gamma'$ là dessus soir ``pure de poids 1'' - et que les polynômes caractéristiques des frobénius (qui sont à coefficients dans $\mathbf{Z}$, pas seulement dans $\mathbf{Z}_{\ell}$) soient \emph{indépendants} de $\ell$.












%%%%%%%%%%%%%%%%%%%%%%%%%%%%%%%%%%%%%%%%%%%%%%%%%%%%%%%%%%%%%%%
\chapter*{\S \space 12. --- L'ANALOGIE TOPOLOGIQUE}\thispagestyle{empty}
\addcontentsline{toc}{section}{12. L'analogie topologique}
\label{sec:12}
\section*{}

Soit $X$ une surface (topologique) compacte orientable, $S$ une partie finie de $X$. Si $X$ est connexe, on considère l'ensemble $\Omega$ de ces deux orientations, on l'utilise pour tordre le groupe $\mathbf{Z}$, d'où un groupe
$$
T = \mathbf{Z} \wedge_{\pm 1} \Omega
$$
isomorphe (non canoniquement) à $\mathbf{Z}$. Plus généralement, supposons donné un tel groupe $T$, i.e. un $\omega \in \Ob \Ens_2$, une $T$-orientation de $X$ sera par définition un élément de l'ensemble $\Or (X) \wedge_{\pm 1} \omega$ [dans le cas précédent, $X$ sera donc canoniquement $T$-orienté\dots]

Considérons le groupoïde fondamental de $X \textbackslash S = U$ soit $\Pi_U$, et le groupoïde fondamental $\Pi_{D^*}$ des germes d'espaces de $X$ autour de $S$, \emph{privé} de $S$ (groupoïde des germes de revêtements universels de $X \textbackslash S$ au voisinage de $S$). On a donc un foncteur canonique
$$
\Pi_{D^*} \to \Pi_U
$$
et d'autre part, si $X$ est $T$-orienté, on a une structure supplémentaire intéressante sur le groupoïde $\Pi_{D^*}$ : le système local de ses $\pi_1$ est canoniquement isomorphe à $T$.

Associant à tout $X$ $T$-orienté le système
$$
\Pi_{D^*} \to \Pi_X, \quad \kappa: T_{(\Pi_{D^*})} \isom \pi_1(\Pi_{D^*}/e),
$$
on trouve une 2-équivalence entre la 2-catégorie isotopique des couples $(X, S)$ d'une variété compacte $T$-orientée (pour les homéomorphismes à isotopie près\dots), et de la 2-catégorie des systèmes précédents (pour les équivalences) qui satisfont les conditions
\begin{enumerate}
    \item[a)] $\pi_0(\Pi_{D^*})$, $\pi_0(\Pi_X)$ finis
    \item[b)] Pour toute composante connexe de $\Pi_U$, soit $\Pi_{U_0}$, et la partie $\Pi_{D^*_0}$ au-dessus, explicitant la situation groupoïde par un groupe $\pi$, et une famille d'homomorphisme $T \xlongrightarrow{\kappa_i} \pi$ (N.B. le tout dépendant de choix, mais $\pi$ étant intrinsèque comme groupe extérieur, et les $\kappa_i$ comme homomorphismes extérieurs - définissant une ``$T$-structure à lacets sur $S$'' - ) on a ce qui suit :
    
    il existe générateur $t \in T$ (i.e. $t \in T^*$) et un ordre $i_1, \dots , i_\nu$ sur l'ensemble $I$, des conjugués $l_i$ des $\kappa_i (g)$, et $g \in \mathbf{N}$ et des $x_\alpha$, $y_\alpha$ $\in \pi (1 \leq \alpha \leq g)$ tels que l'on ait
    $$
    (\prod^g_{\alpha = 1} [ x_\alpha, y_\alpha ]) \prod^\nu_{i = 1} l_i = 1
    $$
    et que ceci soit une relation de définition de $\pi$.
\end{enumerate}
Il y a cependant un grain de sel pour $I = \emptyset$ (auquel cas $T$ ne sert à rien apparemment dans la description groupoïde de $(X, S) = (X, \emptyset) = X$) il faut alors, au lieu des données ``kummériennes'' $\kappa$, se donner un isomorphisme\footnote{Il faut introduire ceci comme donnée supplémentaire dans la définition des groupes à lacets. L'exclusion des cas $X_0 \isom \mathbb{S}^2$ (dans le cas $S = \emptyset$) est alors particulièrement convaincante.}
$$
\mathrm{H}^2 (\pi, \mathbf{Z}) \overset{\kappa}{\isom} T
$$
Enfin, il faut (même avec ce grain de sel) exclure le cas $X = \mathbb{S}^2$, $S = \emptyset$ (en tant que composante connexe) - i.e. du coté groupes, le cas d'une composante connexe de $\Pi_U$ avec $\pi = (1)$ et $I = \emptyset$. Si je me rappelle bien, il n'y a pas lieu d'exclure $(\mathbb{S}^2, \pt)$ (i.e. $X \textbackslash S = U \isom \mathbf{R}^2 \isom E^1_{\mathbf{C}}$ où pourtant on a $\pi = 0$. Mais sauf dans ces deux cas (correspondant au cas $g = 0$, $\nu = 1$) l'homomorphisme $\kappa_i: T \to \pi$ est injectif. Si $T = \mathbf{Z}$, la donnée des $\kappa_i$ équivaut à celle d'éléments $l_i \in \pi$, et on retrouve la définition usuelle des structures à lacets.

Ceci est explicité dans la thèse de Yves Ladegaillerie - ce qui y manque, est (entre autre) la considération de flèches entre $(X, S)$ autres que des homéomorphismes (modulo isotopies) ; par exemple des applications $X' \xlongrightarrow{f} X$ telles que $S' = f^{-1}(S)$ [et que $X'$ soit étale sur $X \textbackslash S$, si on y tient], ce qui se ramène au cas précédent - le cas plus général où on suppose seulement que $X'$ est un revêtement ramifié de $X$ (pouvant être ramifié aussi en dehors de $S'$) et $S' \supset f^{-1} (S)$ (mais $S'$ pouvant être plus grand) demanderait une étude soigneuse, avec une notion ad-hoc de l'isotopie\dots

Une autre direction importante (notamment pour l'étude du cas non orienté, non orientable) est l'introduction de groupes finis d'homéomorphismes, ne respectant pas nécessairement l'orientation. Pour traiter le cas du changement d'orientation, notons que dans la description groupoïdale $(\Pi_{D^*} \to \Pi_{U}, \kappa)$ d'une $(X, S)$ $T$-orientée, le passage à l'orientation opposée s'exprime en gardant tel que $\Pi_{D^*} \to \Pi_U$, et en rempla\c{c}ant $\kappa$ par $\overline{\kappa} = \kappa^{-1}$
$$
\overline{\kappa}(\xi) = \kappa(-\xi) = \kappa(\xi)^{-1}
$$
(si $T = \mathbf{Z}$, sur le système $(\pi, (l_i))$, cela revient à remplacer les $l_i$ par les $l^{-1}_i$\dots), et itou (si $I = \emptyset$) pour $\kappa: T \isom \mathrm{H}^2(\pi, \mathbf{Z})$ remplacé par $-\kappa$.

Soit donc $\Gamma$ un groupe (a priori pas nécessairement fini) qui opère sur $(X, S)$ donc sur $U = X \textbackslash S$ et sur $\Pi_U$, $\Pi_{D^*}$. On trouve alors des groupoïdes fondamentaux mixtes par la construction bien connue
$$
\Pi_{D^*, \Gamma} \to \Pi_{U, \Gamma} \to \Pi_{e, \Gamma}
$$
(où $\Pi_{e, \Gamma}$ est la catégorie des $\Gamma$-torseurs i.e. des objets 1-connexes dans $\Gamma-\Ens$) correspondant aux foncteurs en sens inverse
$$
\Gamma-\text{revêtement étale de}~D^* \longleftarrow \Gamma-\text{revêtement étale de}~U \longleftarrow \Gamma-\Ens.
$$
Les composantes connexes de $\Pi_{D^*, \Gamma}$ correspondent aux \emph{orbites de $\Gamma$ dans $S = \pi_0(\Pi_{D^*})$}, et même pour celles de $\Pi_{U, \Gamma}$. Le cas $\Pi_{U, \Gamma}$ connexe i.e. le topos $\hat{\Pi}_{U, \Gamma}$ connexe est celui où $\Gamma$ transitif sur $\pi_0(U) \isom \pi_0(X)$ - quitte à remplacer $X$ par une composante connexe $X_0$, et $\Gamma$ par le sous-groupe $\Sigma \subset  \Gamma$ qui le stabilise, on serait ramené (pour l'étude des topos $\hat{\Pi}_{U, \Gamma}$ et des morphismes de topos
$$
\hat{\Pi}_{D^*, \Gamma} \to \hat{\Pi}_{U, \Gamma} \to (\hat{\Pi}_{e, \Sigma} \to) \hat{\Pi}_{e, \Gamma}
$$
au cas de $(X_0, S_0, \Sigma)$. Mais l'analogie que j'ai en vue le cas ``arithmétique'' prend cette réduction inopportune dans le cas général (cas dans le cas arithmétique, on ne se borne pas non plus au cas où $\Sigma = \Gamma$ i.e. $E \to \Gamma$ surjectif, i.e. la composante connexe $X$ \emph{géométriquement} connexe sur $K$). Notons ici que $(X, S, \Gamma)$ se récupère à partir des $(X_0, S_0, \Gamma_0 = \Sigma)$ comme somme amalgamée $X = X_0 \wedge_\Sigma \Gamma$\dots

Quand on exprime (pour $\Pi_{U, \Gamma}$ connexe) la situation en termes de théorie de groupes, on trouve donc un groupe fondamental mixte
$$
E_{U, \Gamma} = \pi_1 (U, \Gamma)
$$
et un homomorphisme
$$
E_{U, \Gamma} \to \Gamma
$$
surjectif si et seulement si $U$ connexe (on a $\Gamma/\Sigma \isom \pi_0(U)$), enfin un ensemble d'indices $I$ $(\isom S/\Gamma \isom \pi_0(\Pi_{D^*, \Gamma}))$ et des groupes fondamentaux mixtes.
$$
E_i \isom \pi_1(D^*_i, \Gamma) \isom \pi_1(D^*_{i, 0}, \Sigma_i)
$$
où $D^*_{i, 0}$ est une composante connexe du multidisque troué $D^*_i$, (correspondant au choix d'un $s_{i, 0} \in S$) et où $\Sigma_i \subset  \Gamma$ est son stabilisateur (i.e. le stabilisateur de $s_{i, 0}$ dans $\Gamma$, qui est (si $\Gamma_i$ est fini et opère fidèlement au voisinages de $s_i$) un groupe cyclique ou dièdral\dots On trouve donc, si $X$ est $T$-orientée, une extension de $\Gamma_{i, 0}$ par $T$, l'homomorphisme $E_i \to \Sigma_i \hookrightarrow \Gamma_i$ étant induit bien sûr via $E_{U, \Gamma} \to \Gamma$ et $E_i \to E_{U, \Gamma}$.

Il faut encore lier l'action de $\Gamma$ sur $X$ à l'orientation de $X$ - pour ceci on suppose donné un caractère 
$$
\chi: \Gamma \to \mathbf{Z}^* = \{ \pm 1 \}
$$
et on exige que pour $g \in \Gamma$, $g_X$ conserve l'orientation si $\chi (g) = 1$, la renverse si $\chi (g) = -1$. Ceci implique que l'on a un isomorphisme
\[\begin{tikzcd}
	{\underbrace{\pi_1(\Pi_{D^*, \Gamma}/\Pi_{e, \Gamma})}_{\hspace*{-5mm}\text{système local des noyaux des}~\pi_1(\xi) \to \pi_1(\phi \rho (\xi))~\text{pour}~\xi \in \Ob \Pi_{D^*, \Gamma}\hspace*{-5mm}}} && T
	\arrow["\sim", from=1-3, to=1-1]
	\arrow["{\kappa_\Gamma}"', from=1-3, to=1-1]
\end{tikzcd}\]
$T$ étant considéré comme système local sur $\Pi_{\Gamma, e}$ i.e. comme $\Gamma$-groupe, grâce à l'action de $\Gamma$ via le caractère $\chi$. Il revient au même de dire que $\forall i \in I$, l'application
$$
\kappa_i: T \isommap L_i = \Ker (E_i \to \Gamma)
$$
en tant que homomorphisme de $T$ dans $E_i$, est compatible avec l'action de $E_i$ (opérant sur $T$ via $\chi p_i$ $(p_i: E_i \to \Gamma)$), et sur lui même par automorphisme intérieure\dots)

Si on exclut le cas où $(X_0, S_0)$ est isomorphe à $(\mathbb{S}^2, \emptyset)$ ou $(\mathbb{S}^2, 1 \pt)$ (i.e. le cas $\pi = 0$), les $\kappa_i: T \to \pi$ sont injectifs, donc aussi les $E_i \to E_{U, \Gamma}$, donc les $E_i$ peuvent être considérés comme des \emph{sous-groupes} de $E_{U, \Gamma}$.

Ici il serait particulièrement contre-indiqué (même si on suppose $\Sigma = \Gamma$ i.e. $X = X_0$ i.e. $X$ connexe) de supposer que les $\Gamma_i$ sont égaux à $\Gamma$ i.e. que les $s \in S$ sont fixés par $\Gamma$ !

Comme le centre de $\pi$ est réduit à 1 (si on excepte le cas $(X_0, S_0) \isom (\mathbb{S}^2$, deux points) i.e. $U_0 \isom \mathbf{C}^*$ - cas de la couronne - ), la donnée d'une extension de $\Sigma$ $(\subset  \Gamma)$ par $\pi$ revient (à isomorphisme unique près) à la donnée d'un homomorphisme
$$
\Sigma \to \Autext (\pi)
$$
$E$ se reconstitue comme image inverse de l'extension
$$
1 \to \pi \to \Aut (\pi) \to \Autext (\pi) \to 1.
$$
Mais ici les automorphismes extérieures relatifs aux $\alpha \in \Sigma$ \emph{respectent la structure à lacets} de $\pi$, modulo le signe $\chi (\alpha)$ - i.e. pour $g \in E$, et $s \in S$ $\exists \alpha \in pi$ et $s' \in S$ ($s'$ unique !) tels que
$$
\text{int}(\alpha^{-1} g) \kappa_s (\xi) = \text{int}(\alpha) \kappa_{s'} (\chi (p(g))\xi)
$$
$\forall \xi \in \Gamma$, i.e.
$$
\text{int}(\alpha^{-1} g) \kappa_s (\xi) = \kappa_i (\chi (p(g))\xi)
$$
Ainsi, l'opération de $\Sigma$ sur $S_0$ (donc de $\Gamma$ sur $S = S_0 \wedge_\Sigma \Gamma$) est connue, par l'opération extérieure de $\Sigma \subset  \Gamma$ sur $\pi$ muni de sa structure à ``$T$-lacets'' (i.e. les homomorphismes extérieures $\kappa_s: T \to \pi$) - donc aussi $I = S_0/\Sigma \isom S/\Gamma$. Peut-on reconstituer $E_i$ $(i \in I)$ à partir de la structure d'extension ? On voit, en vertu des choix faits, que si $i \in I$ (donc $i$ une orbite de $\Sigma$ dans $S_0$) $\exists s \in i$ tel que $(\Ker E_i \to \Gamma)$ ne soit autre que $L_s = \kappa_s (T)$, et on a donc $E_i \subset  \Norm_E (L_s)$, mais on a
$$
\Norm_E (L_s) \cap \pi = \Norm_\pi (L_s) = L_s = E_i \cap \pi,
$$
et d'autre part l'image de $\Norm_E (L_s)$ dans $\Sigma \subset  \Gamma$ est inclus dans $\Sigma_s =$ stabilisateurs de $s$ dans $\Sigma =$ Image de $E_i$ dans $\Sigma$, donc en résumé
$$
E_i = \Norm_E (L_s)
$$
Inversement, la donnée de $E \to \Gamma$ et des $E_i$, $\kappa_i: T \to E_i$ redonne la structure à lacets de $\pi$, en prenant les $\kappa_i$ et tous les conjugués extérieures distincts par les $\alpha \in \Sigma$ (modifiés par $\chi (\alpha)$\dots).

Donc la donnée de la situation $E \xlongleftarrow{p} \Gamma$, $E_i \hookrightarrow E$ (famille de sous-groupes, chacun défini modulo conjugaison \emph{dans} $E$) équivaut à la donnée de 
\begin{enumerate}
    \item[a)] $\pi = \Ker p$, avec sa structure à $T$-lacets (ensemble fini d'homomorphismes extérieurs de $T$ dans $\pi$)
    \item[b)] Un sous-groupe $\Sigma \subset  \Gamma$, et un homomorphisme
    \[\begin{tikzcd}
	{\begin{matrix} \Sigma \end{matrix}} & {\begin{matrix} \text{Teichmüller étendu de}~\pi \\ \text{(automorphismes extérieurs de}~\pi, \\ \text{respectant la structure à lacets modulo signe)} \end{matrix}}
	\arrow[from=1-1, to=1-2]
    \end{tikzcd}\]
    compatible avec le caractère $\chi | {}_\Sigma$ et le caractère ``signe'' sur Teichmüller étendu. En fait, $(X, S, \Gamma)$ où $(U, \Gamma)$ ne définit $\pi$ que comme groupe extérieur à $T$-lacets, sur lequel $\Gamma$ opère de fa\c{c}on compatible avec $\chi$.
\end{enumerate}
En résumé, on a un foncteur canonique
% \footnote{flèches : classes d'isotopie d'homéomorphismes respectant toutes les structures}
\[\begin{tikzcd}
	{\boxed{\begin{matrix} \text{Catégorie isotopique des}~(X, S, \Gamma), X~\text{surface compacte}~T\text{-orientée,}~S~\text{partie discrète,}~\Gamma \\ \text{opérant par}~\chi\text{-automorphisme (}\gamma \in \Gamma~\text{respectant l'orientation si}~\chi(\gamma) = + 1,~\text{la renversant} \\ \text{sinon),}~\Gamma~\text{transitif sur}~\pi_0(X),~\text{et si}~(X_0, S_0)~\text{est une composante connexe de}~(X, S),~\text{on veut} \\ \text{que si}~X_0 \isom \mathbb{S^2}~\text{on ait}~\card S \geq 3  \end{matrix}}} \\
	{\boxed{\begin{matrix} \text{Catégorie des groupes extérieurs à}~T-\text{lacets}~\pi, \pi~\text{non commutatif (i.e.}~\pi \neq 0, \mathbf{Z})~\text{sur} \\ \text{lesquels un sous-groupe}~\Sigma \subset  \Gamma~\text{de}~\Gamma~\text{opère.}\\ \text{(Cette description étant équivalente à une description en termes de système de groupoïdes} \\ \Pi_{D^*, \Gamma} \to \Pi_{U, \Gamma} \to \Pi_{e, \Gamma}~\text{et}~\kappa\dots,~\text{plus conceptuelle dans certains contextes).} \end{matrix}}}
	\arrow[from=1-1, to=2-1]
\end{tikzcd}\]
Je présume que la démonstration du fait que ce foncteur soit pleinement fidèle ne fasse pas de difficultés essentielles\footnote{Ca vaudrait drôlement le coup de le faire très soigneusement\dots}, en utilisant ce qui est connu pour $\Gamma = 1$. Mais le fait que, pour $\Gamma$ groupe fini donné, il soit essentellement surjectif est un problème ouvert sur lequel les gens sèchent. Bien sûr on peut supposer $\Sigma = \Gamma$, et $\Gamma \subset $ Teichmüller étendu et la question est si tout sous-groupe fini de Teichmüller étendu se réalise comme groupe opérant sur $(X_0, S_0)$, de fa\c{c}on essentiellement unique. Plus précisément, si $A =$ groupe des homéomorphismes (ou difféomorphismes, si $X_0$ est différentiable) de $X_0$, $A^\circ$ sa composante connexe [neutre] (N.B. $A^\circ$ est contractile dans le cas anabélien) donc $A/A^\circ = T_{g, \nu}$ (groupe de Teichmüller pour genre $g$ et $\nu$ trous), la question revient à ceci si pour tout homomorphisme d'un groupe fini $\Sigma$ dans $T_{g, \nu}$, (on peut supposer $\Sigma \subset  T_{g, \nu}$), celui-ci se relève en un homomorphisme dans $A$, et si deux tels relèvements sont conjugués par un $a \in A^\circ$ (isotopie au sens strict de deux relèvements\dots).

Ayant aboutit à une réinterpretation tellement simple de la 1-catégorie (``1-isotopique'') déduite de la 2-catégorie des systèmes
$$
(\Pi_{D^*, \Gamma} \to \Pi_{U, \Gamma} \to \Pi_{e, \Gamma}, \quad \kappa)
$$
en termes de groupes extérieurs à lacets $\pi$ [munis d'un ensemble d'homomorphismes extérieurs $\kappa_i: T \to \pi$, et à défaut d'un $\kappa: T \to \mathrm{H}^2(\pi, \mathbf{Z})$ (pour bien faire, il faudrait écrire $T^{(\otimes - 1)} \isom \mathrm{H}^2(\pi, \mathbf{Z})$, mais ici on a un isomorphisme canonique $T^{\otimes - 1} \isom T$ i.e. $T^{\otimes 2} \isom \mathbf{Z}$\dots)], la question se pose comment récupérer, (à équivalence définie à isomorphisme unique près) ce diagramme, en termes de $\kappa$ ; tout revient à la description des catégories $\Pi_{D^*, \Gamma}$ et $\Pi_{U, \Gamma}$ et des deux foncteurs $\Pi_{D^*, \Gamma} \to \Pi_{U, \Gamma}$, $\Pi_{U, \Gamma} \to \Pi_{e, \Gamma}$ ; ou ce qui revient au même, des topos (multigaloisiens) et morphismes de topos correspondants.
\begin{enumerate}
    \item[a)] Description de $\Pi_{U, \Gamma}$ et de $\Pi_{U, \Gamma} \to \Pi_{e, \Gamma}$.
\end{enumerate}
Soit plus généralement $\pi$ un groupe extérieur \emph{dont le centre soit trivial} (ceci correspond à l'hypothèse anabélienne !), montrons comment on lui associe une topos classifiant $\B_\pi$, qui (comme catégorie de faisceaux) sera $\Ens (\pi)$) de fa\c{c}on ``fonctorielle'' (pour les isomorphismes). Tout revit à voir comment, à une \emph{classe de conjugaison d'isomorphismes}
$$
\pi' \xlongrightarrow{u} \pi
$$
on associe un foncteur ``image inverse'' $\Ens(\pi) \to \Ens (\pi')$, défini à isomorphisme unique près. Considérons pour tout $u$ de la classe $\theta$, le foncteur ``$u$-restriction des opérations''
$$
\Ens (\pi) \xlongrightarrow{u^*} \Ens (\pi')
$$
[qui définit donc, [?] $(u^{-1})^*$, une équivalence de topos
\[\begin{tikzcd}
	{\B_{\pi'}} && {\B_\pi \quad ]}
	\arrow["\approx"', from=1-1, to=1-3]
	\arrow["{u_\bullet}", from=1-1, to=1-3]
\end{tikzcd}\]
On va, entre ces équivalences pour $u \in \theta$, définir un système transitif d'isomorphismes (ce qui permet donc de les identifier entre eux !).

Soit $u, u' \in \theta$, d'où $u^*$, $u^{'*}$ ; on a par hypothèse un $g \in \pi$ tel que $u' = \text{int}(t) \circ u$, de plus $g$ est déterminé module un élément de $\Centr_\pi u(\pi') = \Centr (\pi) = 1$, donc ici $g$ est unique. Mais $g$ peut servir à définir un isomorphisme fonctoriel
$$
i_{u', u}: u^* \isommap u^{'*}
$$
en prenant, pour $E \in \Ob \Ens (\pi)$,
$$
i_{u', u}(E): u^*(E) \to u^{'*}(E) \quad i_{u', u}(E) = g_E.
$$
N.B. Le raisonnement marche pour toute classe de conjugaison d'homomorphismes de groupes $\pi' \to \pi$ (i.e. tout homomorphisme extérieur) dont le centralisateur dans $\pi$ est réduit à 1 (ce qui pour un épimorphisme se réduit à l'hypothèse $\Centr (\pi) = 1$)\footnote{Marche pour les groupes à lacets anabéliens, et les homomorphismes de tels dont l'image soit d'indice fini\dots (car le centralisateur dans un tel groupe $\pi$ d'un sous-groupe d'indice fini est encore réduit à $e$)}.

Si maintenant un groupe $\Sigma$ opère sur le groupe extérieur $\pi$, alors par le résultat précédent on peut dire qu'il opère aussi sur le topos $\B_\pi$, d'où un topos $\B_{\pi, \Gamma}$. On peut dire aussi que (comme $\Centr (\pi) = 1$) l'opération de $\Sigma$ sur $\pi$ définit une extension $E$ de $\Sigma$ par $\pi$ d'où un topos $\B_{\pi, \Gamma} = \B_E$. On récupère bien sûr aussi $\B_{\pi, \Gamma} \to \B_\Gamma$. Pour se tranquilliser il faudrait s'assurer que si on a un homomorphisme de groupes extérieurs $\pi' \xlongrightarrow{\theta} \pi$ commutant à l'action d'un $\Sigma$, avec $\Centr_\pi (\theta) = (1)$, alors il existe un $\theta_\Gamma: \B_{\pi, \Gamma} \to \B_{\pi, \Gamma}$ défini à isomorphisme canonique près. 

Or tout $u \in \theta$ définit un homomorphisme d'extension $u_\Gamma$ (de $\Sigma$ par $\pi'$ resp. $\pi$) $E' \to E$ au-dessus de $u$, et si on passe de $u$ à $u' = \text{int}(g) \circ u$, on aura $u'_\Gamma = \text{int}_E (g) \circ u_\Gamma$, et on termine comme plus haut avec unicité de $g$ ; c'est maintenant un homomorphisme extérieur \emph{injectif}
$$
\Lambda = [ \kappa ]: T \to \pi \quad\quad\quad (T~\text{commutatif})
$$
($\Lambda$ comme initiale de ``lacets'')

L'injectivité dans le cas qui nous intéresse résulte de l'hypothèse anabélienne.

Supposons que (pour $\kappa \in [ \kappa ]$)
$$
\Centre_\pi \kappa = \kappa (T)
$$
(ce qui ne dépend pas de choix de $\kappa$ dans $[ \kappa ]$). Je vais alors définir un groupoïde abélien connexe $\Pi_\lambda$, et un isomorphisme de son $\pi_1$ avec $T$ (d'où un topos, qui joue le rôle de $\hat{\Pi}_D$). Un objet sera un $\kappa \in [ \kappa ]$. Un homomorphisme de $\kappa$ dans $\kappa'$ sera un élément $g \in \Transp_\pi (\kappa, \kappa')$. La composition des homomorphismes est évidente. On trouve bien un groupoïde connexe, dans lequel
$$
\Aut(\kappa) = \Centr_\pi (\kappa) \overset{\text{par hyp.}}{=} \kappa (T) \overset{\kappa}{\isom} T. \quad \text{OK.}
$$
Bien sûr, on a un homomorphisme
$$
\Pi_\Lambda \to \text{groupoïde ponctuel}
$$
d'où sur les topos classifiants définis par $\pi$
$$
\B_\Lambda \to \B_\pi
$$
Il faut voir le comportement de cette construction par homomorphisme extérieur. Soit donc
$$
\theta = f_\pi = [ u ]: \pi' \to \pi
$$
un homomorphisme extérieur tel que $\Centr_\pi (\theta) = 1$, et $\Lambda': T \to \pi'$ un homomorphisme extérieur injectif tel que $\kappa' \in \Lambda' \Rightarrow \Centr_{\pi'} (\kappa') = \text{Im} \kappa'$ et considérons le composé
$$
f_\pi \circ \Lambda': T \to \pi
$$
soit $d \in \mathbf{Z}$ et considérons
$$
\Lambda = f_\pi \circ \Lambda' \circ (d \id_T) 
$$
supposons que $\kappa \in \Lambda$ $(\Rightarrow \Centr_\pi (\kappa) = \text{Im} \kappa)$.

On va définir un homomorphisme de $\Pi_{\lambda'}$ dans $\Pi_{\lambda}$, d'où un homomorphisme de topos $\B_{\Lambda'} \to \B_{\Lambda}$, et une donnée de commutativité $\alpha$ du
\[\begin{tikzcd}
	{\B_{\Lambda'}} && {\B_{\Lambda}} \\
	{\B_{\pi'}} && {\B_\pi}
	\arrow[from=1-1, to=2-1]
	\arrow[""{name=0, anchor=center, inner sep=0}, "{\B_{f_\pi}}"', from=2-1, to=2-3]
	\arrow["{f_D}", from=1-1, to=1-3]
	\arrow[""{name=1, anchor=center, inner sep=0}, from=1-3, to=2-3]
	\arrow["\alpha", shorten <=6pt, shorten >=6pt, from=0, to=1]
\end{tikzcd}\leqno{(*)}
\]
Soit $\kappa \in T \to \pi'$ un objet de $\Pi_{\Lambda'}$ on veut définir (à isomorphisme unique près) un objet $\kappa$ de $\Pi_\Lambda$. Pour tout $u \in f_{\pi} = \theta$, on considère $u \circ \kappa \circ (d \id_T)$,il y a entre eux un système transitif d'isomorphismes pour $u$ variable dans $\theta$, on peut les identifier entre eux.

La fonctorialité de cet objet par rapport à $\kappa'$ variable est évidente : on peut dire que $u \in \theta$ définit un homomorphisme de groupoïdes $u_\Lambda: \Pi_\Lambda \to \Pi_{\Lambda'}$, et entre ceux-ci il y a un système transitif d'isomorphismes. En fait, pour $u$ fixé on a un diagramme commutatif d'homomorphismes de   groupoïdes
\[\begin{tikzcd}
	{\Pi_{\Lambda'}} && {\Pi_{\Lambda}} \\
	{(e, \pi')} && {(e, \pi)}
	\arrow[from=1-1, to=2-1]
	\arrow["u"', from=2-1, to=2-3]
	\arrow["{u_\Lambda}", from=1-1, to=1-3]
	\arrow[from=1-3, to=2-3]
\end{tikzcd}\]
et entre ces diagrammes il y a un système transitif d'isomorphismes, d'où le diagramme (*) et la donnée de commutation $\alpha$.

Quant on a une famille $\Lambda'$ (ou un ensemble) d'homomorphisme extérieurs $\Lambda'_{i'}$ $(i' \in I'): T \to \pi'$, et une famille $\Lambda (\Lambda_i)_{i \in I}$ d'homomorphismes extérieures $T \to \pi$, et un $\tau: I' \to I$, et $(d_{i'})_{i' \in I'}$ avec $d_{i'} \in \mathbf{Z}$, tels que (pour $\theta: \pi' \to \pi$ homomorphisme extérieur donné) $\forall i' \in I'$, posant $i = \tau (i')$, on ait $\Lambda_i = \Lambda_{i'} \circ \theta \circ (d_i \id)$

[N.B. Si $I \to \Homext (T, \pi)$ injectif, ces conditions montrent que $\tau$ est déterminé par $\theta$, et si de plus $\mathbf{Z}$ opère fidèlement sur $T$, par exemple si $T \isom \mathbf{Z}$, alors $(d_{i'})$ est également unique\dots] alors on construit $\Pi_{D^*, \Lambda'} =$ groupoïde somme des $\Pi_{\Lambda'_{i'}}$ et de même $\Pi_{D^*, \Lambda}$, et on trouve un diagramme essentiellement commutatif de topos
\[\begin{tikzcd}
	{\B_{D^*, \Lambda'}} && {\B_{D^*, \Lambda}} \\
	{\B_{\pi'}} && {\B_\pi}
	\arrow[from=1-1, to=2-1]
	\arrow[from=2-1, to=2-3]
	\arrow[from=1-1, to=1-3]
	\arrow[from=1-3, to=2-3]
\end{tikzcd}\]
Supposons maintenant (ouf !) que $\Sigma$ opère sur [un] groupe extérieur à lacets\dots Je déclare forfait - il est évident que tout marche bien !













%%%%%%%%%%%%%%%%%%%%%%%%%%%%%%%%%%%%%%%%%%%%%%%%%%%%%%%%%%%%%%%
\chapter*{\S \space 13. --- RETOUR AU CAS ARITHMÉTIQUE}\thispagestyle{empty}
\addcontentsline{toc}{section}{13. Retour au cas arithmétique; formulation ``galoisienne''}
\label{sec:13}
\section*{}


Retour au cas arithmétique, où on veut décrire en termes ``galoisiens'' les couples $(X, S)$ anabéliens connexes sur un corps $K$ de type fini sur $\mathbf{Q}$. [N. B. Si on prend un $K$ de type fini sur $\mathbf{F}_p$, il faudrait se borner aux groupes fondamentaux ``premiers à $p$'', à cela près nos développements pourraient se faire quand même\dots]

Il est devenu clair qu'en termes d'une clôture algébrique $\overline{K}$ de $K$, d'où un groupe de Galois profini $\Gamma = \Gal(\overline{K}/K)$, la description la plus simple est en termes de groupes extérieures à lacets et d'actions extérieures de sous-groupes ouverts $\Sigma$ de $\Gamma$ dessus.

De fa\c{c}on précise, on choisit une composante connexe $\overline{X}_0$ de $\overline{X}$ (ou ce qui revient au même, $\overline{U}_0$ de $\overline{U} = (X \textbackslash S)_{\overline{K}}$), soit $\Sigma$ son stabilisateur dans $\Gamma$ (il est remplacé par un conjugué, quand on change $\overline{U}_0$). Alors $\Sigma$ opère sur le schéma $\overline{U}_0$, donc opère extérieurement sur $\pi_1$ (considéré comme groupe extérieure). Or sur celui-ci il y a une $T(\overline{K})$-structure à lacets, avec comme ensemble d'indices $I = S_0(\overline{K}) \subset  S(\overline{K}) \isom S_0(K) \wedge_{\Sigma} \Gamma$ [vide si et seulement si $S \neq \emptyset$] et l'opération de $\Sigma$ sur $\pi$ est compatible avec cette structure à lacet, et le caractère cyclotomique $\chi: \Gamma \to \widehat{\mathbf{Z}}^*$ (plutôt, $\chi | _\Sigma$). Ainsi l'opération de $\Sigma$ sur $\pi$ implique son action sur $\Sigma$, d'où $S(\overline{K}) \isom S_0(\overline{K}) \wedge_{\Sigma} \Gamma$ en tant que $\sigma$-ensemble - on récupère donc le $K$-schéma étale $S$. Mais mieux, on récupère tout le diagramme
$$
\Pi_{\overline{D_0^*}} \to [\Pi_{\overline{U}_0} \to ] \Pi_{\overline{U}} \to \Pi_e
$$
et l'opération de $\Sigma$ dessus d'où le diagramme des topos classifiant - où si on préfère, le diagramme
$$
\Pi_{D^*} \to \Pi_U \to \Pi_e
$$
(avec les notations du début de ces notes, qui deviendraient ici
$$
\Pi_{D^*, \Gamma} \to \Pi_{U, \Gamma} \to \Pi_{e, \Gamma}
$$
plus bien sur $K$ [?]\dots

Les homomorphismes $(X', S') \xlongrightarrow{f} (X, S)$ [$S' \supset f^{-1}(S)$, $f$ dominant] se décrivent simplement (via le choix d'un $\overline{X}'_0$ au dessus d'un $\overline{X}_0$ par des homomorphismes extérieures $\pi (= \pi_1(\overline{X}_0)) \to \pi(= \pi_1(\overline{X}_0))$, compatibles avec les actions de $\Sigma$ ($\subset  \Sigma$) et de $\Sigma$, et avec les structures à lacets anabéliennes (ce qui s'exprime à l'aide d'une application $\tau: (I' \subset  \overline{S}'_0) \to I = \overline{S}_0$\footnote{N.B. $I' = \overline{S}'_0 \cap f^{-1}(\overline{S}_0)$} compatible avec $\sigma'$, et un système d'entiers naturels $(d_{i'})_{i' \in I}$).

Il faut cependant compléter, pour $I = \emptyset$, la définition de la structure à lacets, par la donnée d'un isomorphisme
$$
\kappa: T^{\otimes-1} \isommap \HH^2(\pi, \widehat{\mathbf{Z}})
$$
[N.B. en caractéristique $p \ge 0$, on doit se borner aux composantes $\ell$-adiques avec $\ell \neq p$] i.e.
$$
\widehat{\mathbf{Z}} \isommap \HH^2(\pi, T)
$$
compatible avec l'action de $\Sigma$ - et il faut exiger, dans l'interprétation ``galoisienne'' de $f: (X', S') \to (X, S)$, quand $S = S' = \emptyset$ que l'homomorphisme $\pi \to \pi'$ induit un diagramme commutatif
\[\begin{tikzcd}
	{\widehat{\mathbf{Z}}} & {\mathrm{H}^2(\pi, T)} \\
	{\widehat{\mathbf{Z}}} & {\mathrm{H}^2(\pi', T)}
	\arrow["{f^*}", from=1-2, to=2-2]
	\arrow["d"', from=1-1, to=2-1]
	\arrow["\sim"', from=1-1, to=1-2]
	\arrow["\kappa", from=1-1, to=1-2]
	\arrow["\sim"', from=2-1, to=2-2]
	\arrow["{\kappa'}", from=2-1, to=2-2]
\end{tikzcd}\]
où $d \in \mathbf{N}$ est le \emph{degré} (défini de fa\c{c}on unique par cette condition, comme l'ordre de $\Coker f^*$). Quand $S = \emptyset$, mais $S' \neq \emptyset$, il devrait y avoir encore une compatibilité pour les données kummériennes (de natures différentes sur $\pi'$, où il y a bel et bien ``des lacets'', et sur $\pi$, où elle est purement cohomologique). La question équivaut sans doute à celle de décrire une structure kummérienne ``cohomologique'' sur le groupe $\tilde{\pi}'$, déduit d'un $\pi'$ à lacets (avec $I' \neq \emptyset$) en divisant par les dits lacets\footnote{Paradigme du passage de $(X, S)$ à $X$ : ``bouchage de trous''\dots}. La question est la même, semble-t-il dans le cadre topologique, ou le cadre arithmétique, il me faudra revenir dessus. Il faudrait que pour tout homomorphisme $\pi' \to \pi$ de groupes à lacets, l'homomorphisme $\tilde{\pi}' \to \tilde{\pi}$ correspondant respecte aussi (pour un degré convenable) la structure à lacets. 

Quand on s'intéresse aux systèmes anabéliens $(\overline{X}, \overline{S})$ définis directement sur $\overline{K}$, avec $\overline{X}$ connexe disons, ceci s'exprime par un groupe extérieur à $T$-lacets, muni d'une action (non de $\Gamma$ mais des) \emph{noyaux} de groupes profinis définis par $\Gamma$. Si on considère les ``$\Gamma$-automorphismes'' d'un tel objet (formés d'un $\gamma \in \Gamma$ et d'un automorphisme extérieur $f$ de $\pi$ respectant la structure à lacets, $f$ et $\gamma$ étant compatible dans un sens évident\dots), on trouve un groupe (profini ???) (``discret'' ??) $G$ et un homomorphisme $G \to \Gamma$ (à image un sous-groupe ouvert de $\Gamma$, et à noyau $G_0$ le sou-groupe des automorphismes extérieures à lacets de $\pi$ qui commutent à l'action extérieure du ``  noyau'' (lequel $G_0$ est conjecturellement par la ``conjecture bordélique'', isomorphe au groupe fini $\Aut_{\overline{K}}(\overline{K}, \overline{S}\dots)$. En termes de cette suite exacte
$$
1 \to G_0 \to G \to \Gamma
$$
la ``restriction de $(\overline{X}, \overline{S})$ au corps $K$'' s'exprime donc (on l'espère, de fa\c{c}on pleinement fidèle, si la conjecture bordélique est valable) par un scindage $\Gamma \to G$ de $G \to \Gamma$\dots














%%%%%%%%%%%%%%%%%%%%%%%%%%%%%%%%%%%%%%%%%%%%%%%%%%%%%%%%%%%%%%%
\chapter*{\S \space 13 bis. --- RETOUR SUR LA NOTION DE GROUPE À LACETS}\thispagestyle{empty}
\addcontentsline{toc}{section}{13 bis. Retour sur la notion de groupe à lacets}
\label{sec:13bis}
\section*{}

Soit $\pi$ un groupe, $([L_i])_{i \in I}$ une famille de classes de conjugaison de sous-groupes de $\pi$. On dit que cela définit une ``structure à lacets'' sur $\pi$ (de type $(g, \nu)$) si $\exists g \in \mathbf{N}$, $\forall i \in I$ un $L_i \in [L_i]$, un générateur $l_i \in L_i$, des éléments $x_\alpha$, $y_\alpha$ $\in \pi$ $(i \leq \alpha \leq g)$ enfin un ordre $i_1, \dots, i_\nu$ sur $I$ ($\nu = \card (I)$) tels que (posant $l_\alpha = l_{i_\alpha}$ pour simplifier)
$$
[x_1, y_1][x_2, y_2] \dots [x_g, y_g]l_1 \dots l_\nu = 1
$$
soit une présentation du groupe $\pi$. On n'exclut pas a priori le cas $g = 0$, ni le cas $\nu = 0$, i.e. $I = \emptyset$.

On déduit de ceci : 
\begin{enumerate}
    \item[a)] Si $\nu \neq 0$, $\pi$ est libre (à $2g + \nu - 1$ générateurs) - donc libre \emph{non abélien} sauf si $g = 0$, $\nu \leq 2$. 
    \item[b)]\footnote{Dire que $\pi = 0$ si et seulement si $\nu = 0$ ou 1, et qu'en dehors de ces cas les $L_i$ sont $\isom \mathbf{Z}$} Si $\nu = 0$, le seul cas où $\pi$ abélien est celui où $g \leq 1$. [donc $\pi \isom \pi_{g, \nu}$ est abélien si et seulement si $g = 0$, $\nu \leq 3$ ou $g = 1$, $\nu = 0$]
\end{enumerate}
En tout cas\footnote{N. B. Sauf si $\pi = 0$}, on a une suite exacte canonique de $\mathbf{Z}$-modules libres de types finis
$$
0 \to T_\pi \to \prod_{i \in I} L_i \xlongrightarrow{i} \pi_{ab} \to \tilde{\pi}_{ab} \to 0
$$
où pour $I \neq \emptyset$, $\pi \neq \emptyset$ $T_\pi$ est $\isom \mathbf{Z}$ (défini comme $\Ker i$), et où les projections
$$
T_\pi \to L_i
$$
sont des isomorphismes. Dans le cas $I \neq \emptyset$, on appelle $T_\pi$ le \emph{$\mathbf{Z}$-module des orientations de $\pi$} (muni de la famille des $(L_i)$) - on définit, si $I = \emptyset$, $(\nu = 0)$ mais $g \neq 0$ (donc $\pi \neq 1$)
$$
T_\pi = \underbrace{\mathrm{H}^2(\pi, \mathbf{Z})}_{\hspace*{-5mm}\mathbf{Z}-\text{module libre de rang 1}\hspace*{-5mm}}{}^{\otimes - 1}
$$
[on établira plus loin une relation entre les deux définition de $T_\pi$]. Ainsi $\pi_{ab}$ est libre de rang $2g + \nu - 1$ si $\nu \neq 0$, $2g$ si $\nu = 0$ (donc $g$ est uniquement déterminé par $\pi$ et $\nu = \card \pi$).

Notons que [si $\nu \neq 0$, et] sauf les cas ``abéliens'' $g = 0$, $\nu = 1, 2$, les classes de conjugaison des $L_i$ $(i \in I)$ sont distinctes, donc la structure à lacets de $\pi$ peut se décrire comme la donnée d'un ensemble de $\nu$ classes de conjugaison de sous-groupes de $_p i$. De plus, on voit que tout $g \in \pi$ qui normalise un $L_i$ le centralise\footnote{voir à part le cas $\nu = 1$ : normalisateur du sous-groupe engendré par $\prod [x_\alpha, y_\alpha]$, dans le groupe libre engendré par les générateurs $x_\alpha$, $y_\alpha$,\dots} (c'est évident en tout cas pour $\nu \geq 2$, car alors $L_i \to \pi_{ab}$ est injectif), ce qui implique que les $L_i$ d'une même classe $[L_i]$ sont canoniquement isomorphes entre eux (ce qui donne un sens intrinsèque au terme $\prod L_i$ dans la suite exacte plus haut, et à l'isomorphisme canonique $T_\pi \to L_i$\dots

Si\footnote{On suppose dorénavant qu'on est dans le cas anabélien, ou du moins on exclut $g = 0$, $\nu \leq 2$} un groupe $\Sigma$ opère sur la structure à lacets $(\pi, ([L_i]))$ il opère sur le $\mathbf{Z}$-module inversible $T_\pi$, d'où un caractère
$$
\chi: \Sigma \to \mathbf{Z}^* = \{ \pm 1 \}
$$
inversement, si l'on a un caractère $\chi$ sur $\Sigma$ donné d'avance, on parlera d'une action de $\Sigma$ sur $(\pi, ([L_i]))$ \emph{compatible avec $\chi$}.

On a es variantes profinies (ou profinies premières à $p$, si $p$ est premier donné\dots) - mais il y a dès maintenant à signaler deux points à vérifier dans ce cas :
\begin{enumerate}
    \item[] $\Norm_\pi (L_i) = \Centr_\pi (L_i) = L_i$ dans le cas $\nu = 1$, sinon pas de problème.
    \item[] $\Centr(\pi) = 1$ (cas anabéliens) plus généralement, le centralisateur de tout sous-groupe discret ouvert de $\pi$ est réduit à 1\dots
\end{enumerate}
([ces deux points ne sont] \emph{démontrés} pour le moment dans \emph{aucun} cas anabélien profini !)
















%%%%%%%%%%%%%%%%%%%%%%%%%%%%%%%%%%%%%%%%%%%%%%%%%%%%%%%%%%%%%%%
\chapter*{\S \space 14. --- DIGRESSION COHOMOLOGIQUE (SUR LE ``BOUCHAGE DE TROUS'')}\thispagestyle{empty}
\addcontentsline{toc}{section}{14. Digression cohomologique (sur le ``bouchage de trous'')}
\label{sec:14}
\section*{}

Soit\footnote{Il vaudrait peut être mieux démarrer avec le cas purement topologique d'une surface\dots et faire le lien avec les groupes \emph{discrets}.} un schéma localement noethérien, régulier de dimension 1, $S$ un sous-schéma fermé réduit discret tel que $\forall s \in S$, dim$_s S = 1$ (donc $S$ défini par une partie fermée discrète de $X$), $U = X \textbackslash S$. On veut expliciter par voie galoisienne les faisceaux d'ensembles étales constructibles sur $X$ tels que $F |_U$ soit localement constant. Par le tapis d'Artin sur les ouverts du topos, ils correspondent aux triples
$$
(F_U, F_S, \varphi)
$$
où $F_U$ est un faisceau constructible localement constant sur $U$, $F_S$ un faisceau (nécessairement localement constant) et $\varphi$ un homomorphisme
$$
F_S \xlongrightarrow{\varphi} j^* i_* (F_U)
$$
(où $i: U \hookrightarrow X$ et $j: S \hookrightarrow X$ sont les inclusions). Toute condition de constructibilité etc mises à part, un faisceau étale $F$ sur $X$ correspond à un tel triple - on se restreint ici aux $F$ tels que $F_U$ provienne du topos fondamental $\B_{\Pi_1(U)}$ de $U$\dots (il n'y a pas lieu de supposer $F_U$ à fibres finies pour ce qui suit).

Soit pour tout $s$ dans $S$, $\underline{\mathcal{O}}_s$ un hensélisé de $\underline{\mathcal{O}}_{X, s}$, $D_s = \Spec \underline{\mathcal{O}}_s$ (``disque en $s$''), $D^*_s = D_s \textbackslash \{ s \} = \Spec K_s$ ($K_s$ corps des fractions de $\underline{\mathcal{O}}_s$) (``disque épointé'' en $s$), $D = \amalg_{s \in S} D_s$, $D^* = \amalg_{s \in S} D^*_s$. On a un diagramme commutatif
\[\begin{tikzcd}
	& {D^*} \\
	U && {D(} & {S)} \\
	& X
	\arrow["{j_0}"', from=2-4, to=2-3]
	\arrow["j", from=2-4, to=3-2]
	\arrow["k"', from=2-3, to=3-2]
	\arrow["i", from=2-1, to=3-2]
	\arrow["\sigma", from=1-2, to=2-3]
	\arrow["\rho"', from=1-2, to=2-1]
\end{tikzcd}\]
en on voit (par Artin) que ce diagramme permet d'exprimer les faisceaux étales sur $X$ comme des systèmes $(F_U, F_D, \varphi)$ avec $F_U$ faisceau étale sur $U$, $F_D$ faisceau étale ``essentiellement localement constant'' sur $D$ [N.B. l'inclusion $S \hookrightarrow D$ définit une équivalence entre la catégorie de ces faisceaux sur $D$, et celle des faisceaux étales sur $S$] et un homomorphisme de faisceaux 
$$
F_D \to \sigma_* \rho^* F_U
$$
ou encore
$$
\boxed{\sigma^*(F_D) \xlongrightarrow{\varphi} \rho^* (F_U)}
$$
Or, si on se borne aux $F_U$ tels que $F_U$ soit lui-même essentiellement localement constant, alors les données $F_U$, $F_D$, $\varphi$ ne font intervenir que des faisceaux essentiellement localement constants, i.e. des topos fondamentaux (multigaloisiens) associés aux schémas envisagés, donc se décrivent entièrement en termes des diagrammes
\[\begin{tikzcd}
	& {\B_{\Pi_1 D^*}} \\
	{\B_{\Pi_1 U}} && {\B_{\Pi_1 D}}
	\arrow["\sigma", from=1-2, to=2-3]
	\arrow["\rho"', from=1-2, to=2-1]
\end{tikzcd}\leqno{(*)}
\]
- d'ailleurs le cas où $\varphi$ est un \emph{isomorphisme} correspond justement au cas des faisceaux localement essentiellement constant sur $X$, i.e. de faisceaux sur $\B_{\Pi_1 X}$ - lequel topos fondamental apparaît donc comme somme amalgamée du diagramme de topos précédent, s'insérant dans le carré 
\[\begin{tikzcd}
	& {\B_{\Pi_1 D^*}} \\
	{\B_{\Pi_1 U}} && {\B_{\Pi_1 D}} \\
	& {\B_{\Pi_1 X}}
	\arrow[from=1-2, to=2-3]
	\arrow[from=1-2, to=2-1]
	\arrow[from=2-1, to=3-2]
	\arrow[from=2-3, to=3-2]
\end{tikzcd}\]
Supposons pour simplifier $U$ connexe, choisissons un revêtement universel de $\widetilde{U}$ de $U$, et un revêtement universel $\widetilde{D}^*_i$ de chaque $D^*_i$ - d'où un revêtement universel $\widetilde{D}_i$ de $D_i$ - et des isomorphismes (= ``classes de chemins'') entre $\rho_! (\widetilde{D}^*_i)$ et $\widetilde{U}$ - donc le diagramme de topos (*) - ou des groupoïdes fondamentaux - s'explicite en termes d'un diagramme
\[\begin{tikzcd}
	& {\Gal(\overline{K}_i/K_i)} \\
	& {E_i} \\
	E && {\Gamma_i} \\
	{\pi_1(U, \widetilde{U})} && {\Gal(\overline{k(s)})}
	\arrow["{\rho_i}"', from=2-2, to=3-1]
	\arrow["{\sigma_i}", from=2-2, to=3-3]
	\arrow[shift left=1, no head, from=1-2, to=2-2]
	\arrow[shift right=1, no head, from=1-2, to=2-2]
	\arrow[shift right=1, no head, from=3-1, to=4-1]
	\arrow[shift left=1, no head, from=3-1, to=4-1]
	\arrow[shift right=1, no head, from=3-3, to=4-3]
	\arrow[shift left=1, no head, from=3-3, to=4-3]
\end{tikzcd}\]
et la donnée d'un $F$ comme envisagé sur $X$ revient à la donnée d'un système $(E_U, (E_i)_{i \in I}, (\varphi_i)_{i \in I})$ où $E_U$ est un $E$-ensemble, $E_i$ un $\Gamma_i$-ensemble $(\forall i \in I)$, et $\phi_i$ un $E_i$-homomorphisme de $E_i$ dans $E_U$ (i.e. un $\Gamma_i$-homomorphisme
$$
\varphi_i: E_i \to E^{\pi_i}_U,
$$
où $\pi_i = \Ker (\sigma_i)$ s'insère dans la suite exacte
$$
1 \to \pi_i \to E_i \to \Gamma_i \to 1 \quad )
$$
Le cas où les $E_i \to E_U$ sont des isomorphismes, i.e. $E_i$ s'identifiant tous à $E_U$, avec action triviale de $\pi_i$ sur $E_U$, correspond au cas où $F$ est essentiellement localement constant sur $X$, d'où aussitôt
$$
\pi_1(X) = E = \pi_1(U) /~\text{sous-groupe invariant engendré par les}~\rho_i(\pi_i)
$$
Notre propos est celui d'un calcul galoisien (si possible) de la cohomologie de $X$ pour les $F$ envisagés.

Soit $\B_{X, U}$ le topos dont les faisceaux sont les triples $(F_U, F_D, \varphi)$ comme dessus, qui s'envoie donc dans le topos $\B_{\Pi_1 X}$, correspondant aux couples pour lesquels $\varphi$ est un isomorphisme, et re\c{c}oit le topos $X_{\text{ét}}$ :
$$
X_{\text{ét}} \xlongrightarrow{f} \B_{X, U} \xlongrightarrow{g} B_{\Pi_1 X}
$$
On se pose la question
\begin{enumerate}
    \item[a)] Calcul [?] ``explicite'', en termes de cohomologie des groupes profinis, de la cohomologie du topos abracadabra $\B_{X, U}$. (N.B. La cohomologie de $\B_{\Pi_1 X}$ n'est autre que la cohomologie galoisienne profinie de $\pi_1(X) = E/\dots$ calculée plus haut\dots).
    \item[b)] Vérifier si pour un faisceau de torsion $F$ sur $\B_{X, U}$ l'homomorphisme canonique
    $$
    \mathrm{H}^*(\B_{X, U}, F) \to \mathrm{H}^*(X_{\text{ét}}, f^* F)
    $$
    est un isomorphisme.
\end{enumerate}

N. B. Jusqu'à maintenant, l'hypothèse dim $X = 1$ n'a pas servi, ni l'hypothèse noethérienne - seulement le fait que $S$ soit partie fermée discrète, $X$ connexe\dots

Pour les calculs qui suivent, correspondants du cas où les $\underline{\cO}_s$ sont des jauges à corps résiduels $k(s)$ de caractéristique 0, on va supposer que
$$
\pi_i \isom \hat{\mathbf{Z}} \quad \text{(non canoniquement)}
$$
--- en fait, par la théorie Kummérienne on a un système locaux de Tate $T_E$ (sur $E$), $T_i$ (sur $\Gamma_i$), et des $E_i$-isomorphismes $T_E \isom T_i$ (i.e. un $T_X$ sur lequel opère $\tilde{E} = E \textbackslash$\dots) et on aura un isomorphisme canonique Kummérien
\[\begin{tikzcd}
	{T_i} & {\Pi_i} \\
	&& {}
	\arrow["\sim"', from=1-1, to=1-2]
	\arrow["{\kappa_i}", from=1-1, to=1-2]
\end{tikzcd}\]
On aimerait pouvoir paraphraser, sur $\B_{X, U}$, la suite exacte de cohomologie bien connue
$$
\to \mathrm{H}^i(X, F) \to \mathrm{H}^i(U, F) \to \mathrm{H}^{i + 1}_S(X, F) \to \dots
$$
relative à l'ouvert $U$ du topos $X$, et son ``complémentaire'' fermé $S$. Or, tout comme $X$ s'insère dans un diagramme de topos
\[\begin{tikzcd}
	& {D^*_{\text{ét}}} \\
	{U_{\text{ét}}} && {D_{\text{ét}}} & {S_{\text{ét}},} \\
	& {X_{\text{ét}}} && {}
	\arrow[from=1-2, to=2-1]
	\arrow[from=1-2, to=2-3]
	\arrow[from=2-1, to=3-2]
	\arrow[from=2-3, to=3-2]
	\arrow[from=2-4, to=2-3]
\end{tikzcd}\]
de même $\B_{X, U}$ s'insère dans
\[\begin{tikzcd}
	& {\B_{\Pi_1 D^*}} \\
	{\B_{\Pi_1 U}} && {\B_{\Pi_1 D}} & {\B_{\Pi_1 S},} \\
	& {\B_{X, U}} && {}
	\arrow[from=1-2, to=2-1]
	\arrow[from=1-2, to=2-3]
	\arrow[from=2-1, to=3-2]
	\arrow[from=2-3, to=3-2]
	\arrow["\approx"', from=2-4, to=2-3]
\end{tikzcd}\]

[N.B. Les $\varprojlim$ finies et les $\varinjlim$ quelconques dans $\B_{X, U}$ i.e. pour les $(F_U, F_D, \varphi)$, se calculent ``termes à termes'']

Je dis que $\B_{\Pi_1 U} \to \B_{X, U}$ s'identifie à un morphisme d'induction, relatif à l'objet (noté encore $U$ par abus de notation) de $\B_{X, U}$ défini par
$$
F_U =~\text{faisceau final}~e_U, \quad F_D =~\text{faisceau initial}~\emptyset_D
$$
(et $\varphi$ étant alors fixé !).

En effet, les objets de $\B_{X, U}$ au dessus de $U$ s'identifient aux $(F_U, F_D, \varphi)$ avec $F_D= \emptyset_D$ ([?] que fixe $\varphi$ ? [?]) donc ils forment une catégorie équivalente à celle des $F_U$, i.e. ` $\B_{\Pi_1 U}$.

On voit que le topos résiduel ('identifiant à la catégorie des $F = (F_U, F_D, \varphi)$ tels que $F_U = e_U$ faisceau final), s'identifie de même à la catégorie $\B_{\Pi_1 D}$ des $f_D$ - le foncteur canonique `` image inverse sur $\B_{\Pi_1 D}$ de l'image directe sur $\B_{\Pi_1 U}$'' n'étant autre [que [?]] $\sigma_* \rho^*$, de sorte que l'on retrouve la description typique d'Artin d'un topos déduit par ``recollement'' d'un ouvert et du fermé complémentaire.

On trouve donc une suite exacte
$$
\to \mathrm{H}^i(\B_{X, U}, F) \to \mathrm{H}^i(\B_{\Pi_1 U}, F_U) \to \mathrm{H}^{i+1}_{S (\text{ou}~D)}(\B_{X, U}, F) \to \dots
$$
s'envoyant dans la suite exacte analogue relative à $X_{\text{ét}}$, $U_{\text{ét}}$, $S_{\text{ét}}$. (On n'a toujours pas utilisé d'hypothèse spéciale sur $S$\dots).

Pour vérifier que l'on a des isomorphismes au niveau des $\mathrm{H}^i(\B_{X, U}, F) \to \mathrm{H}^i(X_{\text{ét}}, F)$, il suffit par le lemme des cinq de le prouver au niveau des $\mathrm{H}^i(\B_{\Pi_1, U}, F) \to \mathrm{H}^i(U_{\text{ét}}, F)$ [i.e. vérifions que la cohomologie de $U_{\text{ét}}$ ``se calcule galoisiennement''] \emph{et} au niveau des $\mathrm{H}^i_S$.

Regardons d'abord ces derniers - on a 

$\mathrm{H}^0_S (\B_{X, U}, F) =$ ensemble des sections de $F$ [i.e. des couples d'une section de $F_U$ et d'une de $F_D$ se correspondant par $\varphi$, i.e. d'un élément $x$ de $E_U$ invariant par $E$, et des $x_i \in E_i$ invariants par les $\Gamma_i$, tels que $\forall i$ $x = \varphi_i (x_i)$] tels que $X = 0$ $= \prod_i (\Ker(E_i \to E_U))$

On constate que ce foncteur se factorise par le foncteur $\mathrm{H}^0_S$ relatif à ``l'inclusion'' analogue
$$
\B_{\underbrace{\Pi_1 D}_{\hspace*{-5mm}\text{rempla\c{c}ant}~\Pi_1 U\hspace*{-5mm}}} \to \B_{D, D^*}
$$
et les $\mathrm{H}^i_S$ sur $D_{X, U}$ se calculent comme ceux dans la situation locale $\B_{D, D^*}$ - où on trouve le calcul habituel\footnote{que j'ai un peu oublié !} en termes de l'homomorphisme de groupes $E_i \to \Gamma_i$. Mais dans le cas actuel, l'hypothèse dim$_s X = 1$ aux points $s \in S$, implique la situation du topos [?] $D^*_{\text{ét}} \hookrightarrow D_{\text{ét}}$ est déjà entièrement définie en termes des topos fondamentaux i.e. le topos $\B_{D, D^*}$ est équivalente à $D_{\text{ét}}$ - or les $\mathrm{H}^i_S(X_{\text{ét}}, F)$ se calculent bien sur $D_{\text{ét}}$\dots

On trouve donc
\vskip .3cm
{
Proposition\footnote{Corollaire : {\it Dans ce cas la cohomologie de $X$ à coefficients dans un $F_X$ localement constant se calcule également ? Non, à cause du genre 0 !! Dans le cas de courbes projectives lisses de genre $\neq 0$ il faut un argument spécial.}} : \it  L'homomorphisme de suites exactes de cohomologie envisagé est un isomorphisme, pourvu que l'on sache que l'homomorphisme $\mathrm{H}^i(E, F_E) \to \mathrm{H}^i(U_{\text{ét}}, F_U)$ est un isomorphisme pour tout $i$ ($F_U$ faisceau étale sur $U_{\text{ét}}$ défini par un groupe $F_E$ sur lequel $E = \pi_1 (U)$ opère).
}
\vskip .3cm

{\bf Exemple} : OK si $U$ est une courbe affine sur $k$ algébriquement clos $F$ premier à car [$k$ [?]] (car OK pour $i = 0, 1$, et pour $i = 2$ les deux [membres [?]] sont nuls (la cohomologie galoisienne, car le groupe fondamental premier à $p$ est libre\dots) - d'où on déduit le cas analogue pour $U$ affine sur $k$ quelconque, puis même si $U$ n'est pas affine mais simplement quasi-projective ?)

Nous nous intéressons maintenant au cas où $X$ projective (connexe) sur $k$ algébriquement clos, on voit donc que :
\begin{enumerate}
    \item[a)] Si $S = \emptyset$ i.e. $X = U$, la cohomologie de $U$ (à coefficients dans des [?] locaux) est celle de $\pi$. Donc on a un isomorphisme canonique
    $$
    \mathrm{H}^2(\pi, T) \isom \mathrm{H}^2(X, T) \quad (\isom \hat{\mathbf{Z}})
    $$
    (avec [un] grain de sel en caractéristique $p > 0$)
    \item[b)] Si $S \neq \emptyset$, les $\mathrm{H}^i_!(U, -)$ se décrivent et [se] calculent par voie galoisienne, en termes de groupes à lacets, permettant de reconstituer la situation
    \[\begin{tikzcd}
	& {\B_{\Pi_1 D^*}} \\
	{\B_{\Pi_1 U}} && {\B_{\Pi_1 D}} & {~\text{(= topos discret défini par}~S = I)} \\
	& {\B_{\Pi_1 X}}
	\arrow[from=1-2, to=2-1]
	\arrow[from=1-2, to=2-3]
	\arrow[from=2-1, to=3-2]
	\arrow[from=2-3, to=3-2]
    \end{tikzcd}\]
	en notant que le composant $\B_{\Pi_1 D^*_i}$ n'est autre que $\B_{L_i}$ (avec les notations du numéro précédent). On trouve alors des isomorphismes canoniques (par [calculs] [?] locaux)
	$$
	\mathrm{H}^2_{s_i}(\B_{U, X}, L_i) \isom \hat{\mathbf{Z}} \quad \text{i.e.} \quad \mathrm{H}^2_{s_i}(\B_{U, X}, \hat{\mathbf{Z}}) \isom L_i^{\otimes -1}
	$$
	(si $g \neq 0)$
	$$
	\mathrm{H}^2_{s_i}(\B_{U, X}, \hat{\mathbf{Z}}) \isommap \mathrm{H}^2(X, \hat{\mathbf{Z}}) \isom \mathrm{H}^2(\pi, \hat{\mathbf{Z}})
	$$
	d'où en mettant ensemble, des isomorphismes canoniques
	$$
	L_i \isom \mathrm{H}^2(\pi, \hat{\mathbf{Z}})^{\otimes - 1}
	$$
	d'où en mettant ensemble\footnote{Mais dans tous les cas (même si $g = 0$, du moment qu'on n'a pas $\nu = 0$) on trouve $\mathrm{H}^2_!(\B_{\pi, U}, \hat{\mathbf{Z}}) \isom \hat{T}_{\pi}^{\otimes-1}$ canoniquement, d'où une description cohomologique des modules des orientations, commune au cas sans trous et avec trous\dots}, des isomorphismes canoniques
	$$
	L_i \isom \mathrm{H}^2(\pi, \hat{\mathbf{Z}})^{\otimes - 1}
	$$
	On constate que les composés
	$$
	T_\pi \isommap L_i \to \mathrm{H}^2(\pi, \hat{\mathbf{Z}})^{\otimes - 1}
	$$
	ne dépendent pas du choix de $i$, de sorte que le module d'orientation $T_\pi$ du groupe profini à lacets $\pi$ s'identifie au \emph{dual} de $\mathrm{H}(\pi, \hat{\mathbf{Z}})$ (si $g \neq 0$).
\end{enumerate}

Considérons maintenant un homomorphisme de groupes à lacets $\pi' \xlongrightarrow{f} \pi$, $I'_0 \xlongrightarrow{\tau} I$ associé à un isomorphisme $T_{\pi'} \isom T_\pi$ et une application degré $d: I'_0 \to \mathbf{N}$ (N.B. $i' \in I' \textbackslash I'_0 \to f(L'_{i'}) = (1)$).

On voudrait en déduire un diagramme de morphismes de topos
\[\begin{tikzcd}
	{\B_{\Pi_1 U'}} & {\B_{X', U'}} & {\B_{\Pi_1 X'}} \\
	{\B_{\Pi_1 U}} & {\B_{X, U}} & {\B_{\Pi_1 X}}
	\arrow[from=1-1, to=1-2]
	\arrow[from=1-2, to=1-3]
	\arrow[from=2-2, to=2-3]
	\arrow[from=2-1, to=2-2]
	\arrow[from=1-1, to=2-1]
	\arrow[from=1-2, to=2-2]
	\arrow[from=1-3, to=2-3]
\end{tikzcd}\]
et un homomorphisme \emph{trace} sur la cohomologie à supports propres de $U'$, $U$ (définie en termes de cohomologie sur $\B_{X', U'}$, $\B_{X, U}$ relativement) qui induise un isomorphisme 
$$
\mathrm{H}^2_! (U', \widehat{\mathbf{Z}}) \to \mathrm{H}^2_! (U, \widehat{\mathbf{Z}})
$$
qui soit justement (contragrédiant de) l'isomorphisme des modules d'orientations associé à $f$\dots
 














%%%%%%%%%%%%%%%%%%%%%%%%%%%%%%%%%%%%%%%%%%%%%%%%%%%%%%%%%%%%%%%
\chapter*{\S \space 14 bis. --- OÙ ON REVIENT SUR LES MORPHISMES \emph{MIXTES}}\thispagestyle{empty}
\addcontentsline{toc}{section}{14 bis. Où on revient sur les morphismes mixtes}
\label{sec:14bis}
\section*{}

(Correspondants, dans le cadre topologique, au cas de $f: (X', S') \to (X, S)$ (avec $X$, $X'$ $T$-orientés) tels que l'on ait [$f (X' \textbackslash S') \subset  X \textbackslash S$, i.e. $S' \supset f^{-1} (S)$, mais] $S' \neq f^{-1}(S)$, i.e. il y a des points de $S'$ qui sont envoyés dans $U = X \textbackslash S$).

Dans le paradigme toposique et groupoïdal, $(X, S)$ est décrit par un diagramme de groupoïdes
$$
\Pi_{D^*} \to \Pi_U \quad \text{(et }~\kappa: T_{\Pi_{D^*}} \isom~\text{système local des}~\pi_1~\text{sur}~\Pi_{D^*})
$$
ou de topos
$$
\B_{D^*} \to B_U
$$
et un isomorphisme $\kappa$ du système local constant $T$, qui permet de définir le topos discret $B_D (\isom \B_{\Pi_0 D^*})$ et le topos $B_{X, U}$, s'insérant dans le diagramme de topos
\[\begin{tikzcd}
	& {\B_{D^*}} \\
	{\B_U} && {\B_D} & {\text{(topos discret}~\isom \B_S)} \\
	& {\B_{X, U}}
	\arrow["\rho"', from=1-2, to=2-1]
	\arrow["\sigma", from=1-2, to=2-3]
	\arrow["\psi"', from=2-3, to=3-2]
	\arrow["\varphi", from=2-1, to=3-2]
\end{tikzcd}\]
où $\varphi$, $\psi$ sont des morphismes d'inductions de sous-topos, $\varphi$ ouvert $\psi$ fermé, complémentaire l'un de l'autre, et pour $(X', S')$, décrit par un diagramme analogue 
\[\begin{tikzcd}
	& {\B_{D'^*}} \\
	{\B_{U'}} && {\B_{D'} \isom \B'_S} & {} \\
	& {\B_{X', U'}}
	\arrow[from=1-2, to=2-1]
	\arrow[from=1-2, to=2-3]
	\arrow[from=2-3, to=3-2]
	\arrow[from=2-1, to=3-2]
\end{tikzcd}\]
Ceci dit, on veut absolument, dans une description de $f: (X', S') \to (X, S)$, que celle-ci permette de retrouver non seulement $\B_{U'} \to \B_U$ (ce qui sera acquis par la donnée d'un $f_\pi: \pi' \to \pi$), mais aussi $\B_{X', U'} \to \B_{X, U}$. On aura $S' = S'_0 (= f^{-1} (S)) \amalg S'_1$, donc $D' = D'_0 \amalg D'_1$ et en fait $f$ induit $S'_1 \xlongrightarrow{f} U$ (et non $S'_1 \to S$) qui doit s'expliciter, au niveau des topos multigaloisiens, par un morphisme
$$
B_{D'_1} (\isom \B_{S'_1}) \to \B_U
$$
i.e. la donnée d'une famille de revêtements universels de $U$, paramétrée par $S'_1 = I'_1$, ou (si un te revêtement universel est choisi, $U$ étant connexe, d'où un $\pi = \Aut (\widetilde{U})$) par une famille de torseurs sous $\pi$, $(P_{i'})_{i' \in I'_1}$. Ceci étant posé, on pourra décrire, en termes de 
\[\begin{tikzcd}
	{\B_{D'^*_0}} & {\B_{U'}} \\
	{\B_{D^*}} & {\B_U} & {}
	\arrow[from=1-1, to=1-2]
	\arrow[from=1-1, to=2-1]
	\arrow[""{name=0, anchor=center, inner sep=0}, from=1-2, to=2-2]
	\arrow[""{name=1, anchor=center, inner sep=0}, from=2-1, to=2-2]
	\arrow["\sim", shorten <=4pt, shorten >=4pt, from=1, to=0]
\end{tikzcd}\]
et de 
$$
B_{D'^*_1} \to \B_{D'_1} \to B_U
$$
l'homomorphisme de topos $\B_{X', U'} \to \B_{X, U}$\footnote{Il semble qu'on soit en train de faire la description des morphismes de topos $\B_{X', U'} \to B_{X, U}$ qui induisent $U' \to U$ et qui sur $D'^*_1 (= D'^* |_U)$, se factorisent par $D'^*_1 \to D'_1$\dots}

- Je passe sur le détail de la description.

Quand on se donne un groupe d'opérateurs $\Gamma$ sur la situation $(X', S') \to (X, S)$ donc sur la situation groupoïdale ou toposique, il faut en tenir compte dans la description ci-dessus.

Ainsi $I'_0$ [?] $I'_1$ sera stable par $\Gamma$, et le morphisme $\B_{D'_1} \to \B_U$ doit être stable par $\Gamma$. CE qui signifie aussi, sans doute, qu'on a un morphisme de topos de $\B_{D'_1, \Sigma} = \B_{I'_1, \Sigma}$ dans $\B_{U, \Gamma}$. Choisissant pour toute orbite de $\Gamma$ dans $I'_1$ un représentant $s' \in I'_1$, et considérant son stabilisateur $\Gamma_{s'} \subset  \Gamma$, il faut donc pour toute telle orbite i.e. tout $i'$ se donner un $\Gamma_{s'}$ objet de la catégorie des revêtements universels de $(U, \Gamma)$.

Décrivant $(U, \Gamma)$ en termes d'une extension $E$ de $\Sigma \subset  \Gamma$ par $\pi$ (en choisissant un revêtement universel $\widetilde{U}$ de $U$), la donnée de $\B_{\Gamma_{s'}} \to \B_{X, U}$ compatible avec tout revient sauf erreur à la donnée d'un torseur $P$ à droite sous $\pi$ (permettant de tordre le revêtement universel de référence $\widetilde{U}$ de $U$, à l'aide de $P$ don aussi de tordre $E$ par $P$), et d'un scindage de $E^P \to \Gamma$ au dessus de $\Gamma_{s'}$\dots)

Un automorphisme d'un tel couple $(P, \Gamma_{s'} \xlongrightarrow{q} E^P)$ correspond à un $\alpha \in \pi$ qui \emph{centralise} $\Gamma_{s'_i}$, i.e. qui soit fixé par $\Gamma_{s'_i}$ opérant sur $\pi$.

Si les $\Gamma_{s'_i}$ opèrent assez fortement sur $\pi$ pour que l'on sache que $\pi^{\Gamma_{s'_i}} = (1)$, l'objet $(P, q: \Gamma_{s_i} \to E^P)$ est défini à \emph{isomorphisme unique} près par la classe d'isomorphie (classe de scindage de $E$ sur $\Gamma_{s_i}$).

Il me semble probable que ceci soit toujours le cas dans le cas \emph{arithmétique}, où $\Gamma_{s'_i}$ est un sous-groupe ouvert du groupe de Galois $\Gamma$ (dont l'opération extérieure est alors draconienne !) ; ceci en direction de la conjecture qu'une classe de scindage de $E$ sur $\Gamma$ est ``aussi bonne'' qu'un point rationnel de $U$ sur $K$ --- et permet de paradigmer ce qu'un tel point rationnel permettrait d'obtenir\dots







%%%%%%%%%%%%%%%%%%%%%%%%%%%%%%%%%%%%%%%%%%%%%%%%%%%%%%%%%%%%%%%
\chapter*{\S \space 15. --- RETOUR SUR LE CAS TOPOLOGIQUE \\ Structure des $\Gamma$-orbites critiques en termes d'extensions}\thispagestyle{empty}
\addcontentsline{toc}{section}{{\bf 15.} Retour sur le cas topologique: orbites critiques des scindages d'extensions;}
\label{sec:15}
\section*{}

Soit $\Gamma$ un groupe fini opérant sur un groupe à lacets $\pi$ ; supposons que ceci provienne d'une situation topologique, $\Gamma$ opérant sur $(X, S)$. Il y a des points à vérifier (\emph{cas anabélien}).
\begin{enumerate}
    \item[a)] Opération de $\Gamma$ triviale $\Longleftrightarrow$ $\Gamma \xlongrightarrow[\text{trivial}]{} \Autext (\pi)$. 
\end{enumerate}
En d'autres termes : un \emph{automorphisme} d'ordre fini de $(S, X)$ ne peut être isotope à l'identité (ou même seulement homotope, cela revient au même d'ailleurs) que si il est trivial.

C'est même connu en Géométrie Algébrique ``abstraite'' du moment qu'on admet que $u$ conserve une structure complexe - il en est justement ainsi si on admet qu'il n'y a pas de sauvagerie\dots

Sans doute toute action d'un groupe fini $\Gamma$ laisse [une] structure conforme invariante, et même si on restreint à $\Gamma^\circ = \ker (\Gamma \xlongrightarrow{\chi} \{ \pm 1 \})$, [laisse une] structure complexe invariante.

Supposons dorénavant que $\Gamma$ opère fidèlement $(\Gamma \neq 1)$, le choix d'une structure complexe sur $Y = X / \Gamma^\circ$ en définit une sur $X$\footnote{(N. B. $\Gamma / \Gamma^\circ$ opère encore sur $X / \Gamma^\circ$, en fait si $\Gamma \neq \Gamma^\circ$ i.e. $\Gamma / \Gamma^\circ \isom \{ \pm 1 \}$, $X / \Gamma^\circ$ est muni d'une ``structure de courbe algébrique réelle''\dots)} stable par $\Gamma^\circ$ - et une structure conforme stable par $\Gamma$ si on choisit celle de $Y$ invariée par l'élément non trivial de $\Gamma / \Gamma^\circ$ (s'il y en a un)\footnote{(cela marche chaque fois qu'on a un revêtement ramifié de surface conforme).}.

Tout $x \in U^\Gamma$ définit une classe de $\pi$-conjugaison de scindages de l'extension $E$ de $\Gamma$ par $\pi$, comme on voit en prenant $x$ comme point base\footnote{(N.B. $U^\Gamma \neq \emptyset$ implique que $\Gamma$ est cyclique).}.

Notons que si $\Gamma^\circ \neq 1$ (i.e. $\Gamma$ n'est pas réduit à l'identité et une anti-involution). $U^\Gamma$ est un ensemble fini - on trouve une application $U^\Gamma \to$ ensemble des classes de $\pi$-conjugaison de scindages de $E \to \Gamma$, i.e. ensemble des relèvements de $\Gamma \to \Autext \pi$ ou $\Gamma \to \Aut (\pi)$ mod. $\pi$-conjugaison.

{\it Question}. --- Si $\Gamma^\circ \neq \{ 1 \}$, cette application est-elle bijective ? Si $\Gamma^\circ = \{ 1 \}$, $\Gamma / \Gamma^\circ \isom \{ \pm 1 \}$, alors $X^\Gamma$ est l'ensemble des points réels d'une courbe algébrique réelle et $U^\Gamma = X^\Gamma \textbackslash S^\Gamma$ est le complémentaire d'une partie finie dedans, on a : 
$$
\pi_0 (X^\Gamma \textbackslash S^\Gamma) \to \pi-\text{classe de scindages de}~E~\text{sur}~\Gamma
$$
et la question analogue de bijectivité se pose, pour l'extension de $\{ \pm 1 \}$ par $\pi$\dots

Pour l'injectivité de l'application dans le cas $\Gamma^\circ \neq \{ 1 \}$, on peut supposer $\Gamma = \Gamma^\circ \isom \mathbf{Z}/p\mathbf{Z}$, avec $p$ premier, i.e. $\Gamma$ engendré par un automorphisme complexe $u$ d'ordre $p$, qui définit (si $x, y \in U^\Gamma$, $x \neq y$) un automorphisme d'ordre $p$ dans $\pi_1 (U, x)$, $u_x$ donc une classe de $\pi$-conjugaison d'automorphisme de $\pi$ d'ordre $p$, et de même un automorphisme $u_y$ de $\pi_1 (U, y)$. Il faut prouver que $u_x$, $u_y$ ne sont pas conjugués sur $\pi$.

Soient $U$ un espace topologique connexe par arcs, $\Gamma$ un groupe fini opérant sur $U$, $\widetilde{U}$ un revêtement universel de $U$, d'où un groupe extension
$$
1 \to \pi \xlongrightarrow{i} E \xlongrightarrow{p} \Gamma \to 1
$$
opérant fidèlement sur $\widetilde{U}$ ($\pi = \Aut_U (\widetilde{U} \isom \pi_1 (U))$\dots). Pour tout point fixe $x \in U^\Gamma$, $\Gamma$ opère sur le revêtement universel ponctué sur $x$, soir $R_x$, en laissant fixe le point marqué $\tilde{x}$ dans $R_x$ au-dessus de $x$, d'où un scindage de l'extension relative
\[\begin{tikzcd}
	1 & {\pi_x} & {E_x} & \Gamma & 1
	\arrow[from=1-1, to=1-2]
	\arrow["{i_x}", from=1-2, to=1-3]
	\arrow[from=1-4, to=1-5]
	\arrow["{\sigma_x}"', shift right=2, tail reversed, no head, from=1-3, to=1-4]
	\arrow["{p_x}", shift left=2, from=1-3, to=1-4]
\end{tikzcd}\]
et pour tout isomorphisme $c$ (``chemin'') : $R_x \isom \widetilde{U}$, induisant un isomorphisme d'extension $R_x \isom E$ (défini de manière compatible avec l'automorphisme intérieur induit par un $\alpha \in \pi$\dots) on trouve par transport de structure un scindage $\sigma_{x, l}$ de $E \xlongleftarrow{p} \Gamma$ (qui, pour $l$ variable, est défini à automorphisme intérieur près par un $\alpha \in \mathbf{R}$).

On trouve ainsi une application
\[\begin{tikzcd}
	{U^{\Gamma}} & {\begin{matrix} \text{classes de}~\pi-\text{conjugaison des scindages de l'extension}~E ~\text{de}~\Gamma~\text{par}~\pi  \\ \text{(= classes de}~\pi-\text{conjugaison de sous-groupes}~\Gamma'~\text{de}~E \\ \text{[images de sections])} \end{matrix}}
	\arrow[from=1-1, to=1-2]
\end{tikzcd}\leqno{(*)}\]

L'image de cette application est donc formée des classes de conjugaison de sous-groupes sections $\Gamma' \subset  E$ tels que $\widetilde{\Gamma'} \neq \emptyset$, et pour un tel $\Gamma'$, l'ensemble des $x \in U^{\Gamma}$ qui donnent comme image cette classe de conjugaison est l'image de $U^{\Gamma}$ dans $U$\footnote{(N.B. $\widetilde{U}_x$ étant identifié à $\Isom_U (R_x, \widetilde{U})$, $\widetilde{U}_x^{\Gamma'}$ s'identifie à $\Isom_{U, \Gamma} (R_x, \widetilde{U})$. Les deux isomorphismes qui commutent à l'action de $\Gamma$, et $\pi^\Gamma = (1)$ signifie donc que cet isomorphisme est unique en terme de la classe de conjugaison des sections de [?])}.

Enfin, si $x$ est dans cette image, l'ensemble $\widetilde{U}^{\Gamma'}_x$ des $\tilde{x} \in \widetilde{U}^{\Gamma'}$ au-dessus de $x$ ($\neq \emptyset$ par hypothèse sur $x$) est un torseur sous $\pi^{\Gamma'}$ i.e. si $\tilde{x} \in (U^{\Gamma'})_x$, et si $\alpha \in \pi$, alors
$$
\alpha \tilde{x} \in \widetilde{U}^{\Gamma'} \Longleftrightarrow \alpha \in \pi^{\Gamma'}
$$
(vérification triviale, comme dans toutes les assertions précédentes).
\begin{enumerate}
    \item[a)] (*) est une bijection, et pour tout sous-groupe $\Gamma$ de $E$ section de l'extension, on a $\pi^{\Gamma'} = \{ 1 \}$.
    \item[b)] pour tout $\Gamma'$ comme dans a), $\Gamma'$ opérant sur $\widetilde{U}$ a un point fixe et un seul.
\end{enumerate}
Ceci posé, prouvons le
\vskip .3cm
{
Lemme fondamental. --- \it Soit $\Gamma$ un groupe fini, opérant fidèlement sur un espace $D \isom \mathbf{R}^2$, soit $\Gamma^\circ$ le sous-groupe de $\Gamma$ (d'indice 1 ou 2) formé des $g \in \Gamma$ tels que $g_D$ conserve l'orientation, et supposons $\Gamma^\circ \neq \{ 1 \}$ (i.e. $\Gamma$ n'est réduit ni au groupe unité, ni au groupe $\{ 1, \sigma \}$, où $\sigma$ est une anti-involution de $D$). Alors
\begin{enumerate}
    \item[a)] $\Gamma$ admet un point fixe et un seul dans $D$ i.e. card $D^\Gamma = 1$.
    \item[b)] $\Gamma^\circ$ est cyclique, et si $\Gamma \neq \Gamma^\circ$. $\Gamma$ est un groupe diédral.
\end{enumerate}
}
\vskip .3cm
Plus précisément, il existe un homéomorphisme $D \isom \mathbf{C}$ tel que le groupe d'homéomorphismes de $\mathbf{C}$ transformé de $\Gamma$ soit : soit le groupe des homothéties par $\mu_n (\mathbf{C})$ (si $\Gamma = \Gamma^\circ$ d'ordre $n$), soit le groupe diédral associé
$$
z \mapsto \xi \tau^\epsilon (z) \quad (\xi \in \mu_n (\mathbf{C}), \epsilon = \pm 1)
$$
où $\tau$ est la conjugaison complexe.

\emph{Démonstration du lemme fondamental}.

\begin{enumerate}
    \item[a)] Supposons d'abord qu'on puisse trouver une structure $C^2$ sur $D$ invariante par $\Gamma$ alors un argument standard montre qu'il existe une structure conforme invariante par $\Gamma$, or le théorème fondamental de la représentation conforme montre qu'alors
    
    ou bien $D \isom$ intérieur $\Delta$ du disque unité ou du demi-plan de Poincaré
    
    ou bien $D \isom \mathbf{C}$ (isomorphisme conforme).
\end{enumerate}
Dans le premier cas, les groupes des automorphismes conformes est
$$
\isom \Sl (2, \mathbf{R}) ~\tilde{} = \{ u \in \Sl (2, \mathbf{R}) / \det u = \pm 1 \}  (u = \begin{pmatrix}
a & b \\
c & d 
\end{pmatrix})~\text{opérant par}~\theta_u \tau^{\det u}~\text{où}~\tau
$$
est la conjugaison complexe et $\theta_u (z) = \frac{az + b}{cz + d}$ en laissant stable le demi plan de Poincaré, et tout sous-groupe compact est contenu dans un conjugué du sus-groupe compact maximal qui (en repassant au disque unité $\Delta$) s'identifie au groupe $O (2, \mathbf{R})$ des transformations du disque unité de la forme
$$
z \mapsto \chi \tau^\epsilon \quad \xi \in \mathbb{U} = \{ \xi \in \mathbf{C} / |\xi| = 1 \}
$$
$$
\epsilon \in \{ \pm 1 \}
$$
$$
\tau~\text{est la conjugaison complexe}.
$$
Tout sous-groupe fini de ce groupe est de l'un des types explicités plus haut.

On gagne, en utilisant un homéomorphisme $[ 0, 1 [ \to [0, +\infty [$ pour définir un homéomorphisme $D \isom \mathbf{C}$ commutant à l'action de $G = O (2, \mathbf{R})$.

Dans le cas $D \isom \mathbf{C}$, on voit que le groupe des endomorphismes conformes de $\mathbf{C}$ est le groupe des transformations $az + b$ ou $a \overline{z} + b$, dans lequel un sous-groupe compact maximal est le même groupe $O (2, \mathbf{R})$ que tantôt --- et tout sous-groupe compact (à fortiori tout sous-groupe fini) est contenu dans un conjugué de celui-ci.

On gagne encore.

Le reste du travail consiste essentiellement à montrer que l'hypothèse de non-sauvagerie est toujours satisfaite, du moins pour $\Gamma^\circ$. Supposons d'abord $\Gamma = \Gamma^\circ$.

On suppose que tout est prouvé pour les ordres < card $\Gamma$.

\begin{enumerate}
    \item[b)] $\Gamma$ admet un point fixe où $D^r \neq \emptyset$. 
\end{enumerate}
Sinon, les sous-groupes des orbites $\tilde{x}$ des $x \in D$ étant d'ordre < card $\Gamma$, par hypothèse de récurrence les $\Gamma_x$ ont la structure dite dans le théorème, donc $D^\Gamma = U$ est une surface topologique et $D \to U$ est un revêtement ramifié ; choisissons une structure conforme sur $U$, il y a une unique structure conforme sur $D$ telle que $D \to U$ soit ``conforme'' (holomorphe ou antiholomorphe), celle-ci est invariante par $\Gamma$ et, d'après a), $\Gamma$ admet un point fixe, contradiction.

\begin{enumerate}
    \item[c)] $\Gamma$ n'admet pas d'autre point fixe que 0. On fait opérer $\Gamma$ fidèlement sur $D^* = D \textbackslash \{ 0 \}$ et il faut prouver que $D^{* \Gamma} = \emptyset$.
\end{enumerate}

Soit donc $x \in D^{* \Gamma}$. On va alors aboutir à une contradiction. Considérons le revêtement universel $\widetilde{D}^*$ de $D^*$ ponctué en $x$, donc $\Gamma$ opère sur $\widetilde{D}^*$ avec point fixe $\tilde{x}$ au-dessus de $x$. Ici $\pi = \pi_1 (D^*) \isom \mathbf{Z}$, et $\Gamma$ y opère trivialement (car $\Gamma = \Gamma^\circ$) donc $\Gamma \times \mathbf{Z}$ opère sur $\widetilde{D}^*$.

On peut supposer $D = \mathbf{C}$, $O = 0$, $x = 1$, $D^* = \mathbf{C}^*$, $\widetilde{D}^* = \mathbf{C}$, $\tilde{x} = 0$, $\widetilde{D}^* \to \widetilde{D}$ donné par exp $(2 i \pi z)$, et $\mathbf{Z}$ opérant sur $\mathbf{C}$ par $\theta_n z = z + n$ ($n \in \mathbf{Z}$).

Il reste à prouver que si un groupe fini $\Gamma$ opère sur $\mathbf{C}$ en commutant à l'action de $\mathbf{Z}$ sur $\mathbf{C}$, et en laissant fixe le point 0, alors $\Gamma$ opère trivialement (ce qui contredit l'hypothèse de fidélité de l'opération).

On est ramené au

\vskip .3cm
{
Lemme. --- \it Soit $u$ un homéomorphisme d'ordre fini de $\mathbf{C}$ commutant à $z \mapsto z + 1$ et laissant invariant l'origine, alors $u \isom \id$.  
}
\vskip .3cm

On peut supposer qu'il existe un nombre premier $p$ tel que $u^p = \id$, i.e. que $u$ correspond à une opération de $\Gamma = \mathbf{Z}/p \mathbf{Z}$ sur $\mathbf{C}$. Tous les points de $\mathbf{Z} \subset  \mathbf{C}$ sont fixe par $\Gamma$. Passant à $\tilde{\mathbf{C}} \isom \mathbb{S}^2$, on trouve que $\infty$ est un point d'accumulation des points fixes sous $\Gamma$. Si $\Gamma$ n'opérait pas trivialement, ce serait décidément très sauvage ! On doit pouvoir terminer par la suite spectrale d'Adams\dots je n'entre pas dans ces dédales\dots

\begin{enumerate}
    \item[d)] La partie purement topologique étant ainsi supposée prouvée, on en conclut aussi, si $\Gamma \neq \Gamma^\circ$, $\Gamma^\circ \neq \{ 1 \}$, comme $D^{\Gamma^\circ}$ est invariant sous $\Gamma$, comme $D^{\Gamma^\circ}$ est réduit à un point, celui-ci est invariant sous $\Gamma$ tout entier, pas seulement $\Gamma^\circ$. D'autre part, on en sait assez maintenant pour savoir que si $\Gamma$ groupe fini opère sur une surface compacte $U$, les $\Gamma_x$ $(x \in U)$ respectant l'orientation, alors $U \textbackslash \Gamma \isom V$ est un surface $U \to V = U \textbackslash \Gamma$ est un revêtement ramifié, choisissons une structure conforme sur $V$, on trouve par image   inverse une structure conforme sur $U$ invariante par $\Gamma$. Pour le cas $U = D$, on termine par $a)$ pour le complément du lemme fondamental.
\end{enumerate}

[Mais pour bien faire, il faudrait prouver qu'il existe toujours une structure conforme invariante si $\Gamma$ est un groupe fini opérant sur une surface compacte - donc $U \textbackslash \Gamma$ est une surface à bord\dots Ici ce qui manque, c'est l'analyse de l'action d'une anti-involution d'une surface au voisinage d'un point fixe\dots]\footnote{Il faudrait prouver que si $\tau$ est un anti-automorphisme involutif de $D$, alors il existe un isomorphisme $D \isom \mathbf{C}$ tel que $\tau$ devienne $z \mapsto \overline{z}$ (donc $D^\tau \isom \mathbf{R}$ !) ce qui doit permettre de prouver, si $\Gamma = \mathbf{Z}/2\mathbf{Z}$ opère par anti-automorphisme sur $U$ ([orientable $U \neq S^2$]) que $\pi_0 (U^\Gamma) \to $ classes de $\pi$-conjugaison de sections de $E$ sur $\Gamma$ est bijectif.}

Conséquence du lemme fondamental :
\vskip .3cm
{
Théorème. --- \it Soit $U$ une surface topologique paracompacte 0-connexe, $\Gamma$ un groupe fini opérant fidèlement sur $U$, on suppose $\widetilde{U}$ non compacte (i.e. $U$ non homéomorphe à $\mathbb{S}^2$ ni au plan projectif réel) on suppose que de plus si $U$ est orientable, le sous-groupe $\Gamma^\circ$ de $\Gamma$ des $g \in \Gamma$ qui conservent une orientation soit $\neq \{ 1 \}$ [donc $\Gamma$ n'est ni réduit à 1, ni à 1 et une anti-involution], et si $U$ non orientable, que $\Gamma \neq \{ 1 \}$ i.e. card $\Gamma > 3$.

Ceci posé considérons l'extension $E$ de $\Gamma$ par $\pi = \pi_1 (U)$, et l'application
$$
U^\Gamma \to \text{classes de}~\pi-\text{conjugaison des scindages de}~E \to \Gamma
$$
on a ceci :
\begin{enumerate}
    \item[a)] Cette application est bijective 
    \item[b)] Pour tout sous-groupe section $\Gamma' \subset  E$ on a $\pi^{\Gamma'} = \{ 1 \}$.
    \item[c)] Si $U^\Gamma \neq \emptyset$, i.e. il existe un scindage, alors $\Gamma$ est cyclique ou diédral (et dans le cas $U$ ouvert, $\Gamma^\circ$ est cyclique)\footnote{N.B. Les hypothèses sur $U$ assurant que $\widetilde{U} \isom D$, et celles sur $\Gamma$ que $\Gamma$ opérant sur $D$ satisfait aux hypothèses du lemme fondamental.}.
\end{enumerate}
}
\vskip .3cm
{
Corollaire. --- \it Supposons $U$ orientée, $\Gamma$ conservant l'orientation. Soit $U^!$ l'ensemble de $x \in U$ tels que $\Gamma_x \neq \{ 1 \}$ (qui est donc une partie discrète dans $U$). A tout $x \in U^!$, associons la classe de $\pi$-conjugaison des sous-groupes de $E$ (sections partielles de $E$ sur $\Gamma_{x'}$) qui correspond à cet $x \in U^{\Gamma_{x'}}$.

Alors
\begin{enumerate}
    \item[a)] les sections partielles ainsi obtenues sont \emph{maximales} parmi celles qui sont $\neq \{ 1 \}$.
    \item[b)] l'application de $U^!$ vers l'une des classes de $\pi$-conjugaison des sections partielles $\neq (1)$ maximales est bijective\footnote{(N.B. Cette application commute aux actions naturelles de $\Gamma$ !).}.
    \item[c)] pour toutes telles sections partielles, on a $\pi^{\Gamma'} = \{ 1 \}$, i.e. les automorphismes du revêtement universel $R_x$ de $U$ qui commutent à l'action de $\Gamma_x$ sont triviaux. Il y a un isomorphisme unique commutant à l'action $\Gamma_x$ entre ce torseur et le torseur déduit de $\widetilde{U}$ en tordant par le $\pi$-torseur $P_x$ de $\Gamma''$ dans la classe [$\Gamma'$]\dots
\end{enumerate}
}
\vskip .3cm
Prouvons a). Soit $\Gamma' \subset  E$ section partielle sur $\Gamma_x$ déduite de $x \in U^!$. Soit $\Gamma'' \supset \Gamma'_x$ un autre sous-groupe tel que $\Gamma'' \cap \pi = \{ 1 \}$ i.e. $\Gamma'' \hookrightarrow \Gamma$. Soit $\Gamma_1 \supset \Gamma_x$ son image dans $\Gamma$. Par le théorème précédent, il est défini par un unique $y \in U^{\Gamma_1}$, et il est clair que cet $y$ ne change pas si on remplace l'action de $\Gamma_1$ sur $U$ par l'action d'un groupe plus petit $\neq \{ 1 \}$ (et la section induite) tel $\Gamma_x$, donc [?] $\Gamma_1$ fixe $x$ donc (par définition de $\Gamma_x$) $\Gamma_1 = \Gamma_x$ donc $\Gamma'' = \Gamma'$.

b) Soient $x$, $y$ donnant même image [$\Gamma'$], [$\Gamma''$], donc $\Gamma_x = \Gamma_y$, soit $\Gamma_1$, et appliquant le théorème à $\Gamma_1$ opérant sur $U$, on trouve $x = y$. Soit d'autre part $\Gamma'_0$ une section partielle $\neq 1$ maximale, $\Gamma_0 \subset  \Gamma$, son image ;  par le théorème appliqué à l'action de $\Gamma_0$ sur $U$, $\exists x \in U$ tel que $x \in U^{\Gamma'_0}$ i.e. $\Gamma'_0 \subset  \Gamma_x$ et que [$\Gamma'_0$] soit défini par $x$, mais si [$\Gamma'$] est défini par $x$ pour l'action de $\Gamma_x$ tout entier, on aura [$\Gamma'_0$] $\subset $ [$\Gamma'$], donc par le caractère maximal de [$\Gamma'_0$], on aura [$\Gamma'_0$] $=$ [$\Gamma'$], ce qui prouve b). D'autre part c) est clair. 

Revenons maintenant au cas où $U = X \textbackslash S$, $X$ surface $T$-orientée compacte avec $S$ partie finie, anabélienne. Donc on a une description ``pleinement fidèle'' de $U$ par un $\pi$ avec structure à lacets, et on voudrait se convaincre que l'opération extérieure d'un $\Gamma$ sur $\pi$, quand $\Gamma$ conserve l'orientation (pour simplifier) est également suffisante pour décrire pleinement l'objet $(M, \Gamma)$ dans le catégorie isotopique qui convient.

On récupère déjà une description de $U^!$ en terme de l'action de $\pi$, soit $J (\isom U^!)$ l'ensemble des classes de $\pi$-conjugaison de sections partielles $(\neq 1)$ maximales de $E$ sur $\Gamma$. Pour tout $j \in J$, $j$ est un $\pi$-torseur comme classe de $\pi$-conjugaison de sections [que ce soit un $\pi$-torseur résulte de $\pi^{\Gamma'} = \{ 1 \}$]. En fait l'ensemble de toutes ces sections partielles maximales $(\neq 1)$ est de fa\c{c}on naturelle un $E$-ensemble (par conjugaison) sur lequel $\pi$ opère librement et cet $E$-ensemble s'identifie canoniquement à $\widetilde{U}|_{U^!} = \widetilde{U}$ pour la structure de $E$-ensemble.

Soit $S' = S \cup U^!$, $U' = X \textbackslash S'$, il s'impose d'essayer de reconstituer (en terme de l'extension $E$ de $\Gamma$ par le groupe à lacets $\pi$) le groupe extérieur à lacets correspondant à $X'$, $S'$ i.e. à $U'$ et l'action extérieure de $\Gamma$ sur celui-ci. Mais il faudrait d'abord s'assurer du caractère intrinsèque de la définition de $J (\isom U^!)$ comme $\Gamma$-ensemble, en terme du groupe extérieur $\pi$, et de l'action extérieure de $\Gamma$ dessus. (Ceci est assez évident d'ailleurs : en termes justement de classes de $\pi$-conjugaison de relèvements partiels de $\Gamma \to \Autext_{\text{lac}}(\pi)$ (vers $\Aut_{\text{lac}}(\pi)$)). On aimerait cependant aussi une description intrinsèque de $\widetilde{U}|_{U^!}$, en un paradigme pour l'application de $\Gamma$-espace : $U^! \to U$ ;  on doit donc décrire un morphisme de topos avec opération de $\Gamma$ dessus
$$
\B_J \xlongrightarrow{\nu} \B_E
$$
qui correspond donc à un foncteur ``image inverse'' $\nu^*$ (compatible avec l'action de $\Gamma$)
\[\begin{tikzcd}
	{\B_E} && {\B_J} \\
	{\pi-\text{ensemble}} && {\text{Ensemble sur}~J}
	\arrow["\approx", from=1-3, to=2-3]
	\arrow["\approx", from=1-1, to=2-1]
	\arrow["{\nu^*}", from=1-1, to=1-3]
\end{tikzcd}\]
Ce n'est autre que le produit contracté sur $\pi$ avec l'ensemble des sections partielles $\neq 1$ invariantes de $E$ sur $\Gamma$.

Revenons `l'extension $E$ de $\Gamma$ par $\pi$ provenant d'une situation géométrique (laquelle extension dans le cas anabélien est définie déjà en terme d'une opération extérieure de $\Gamma$ sur $\pi$) on voit que celle-ci satisfait des conditions supplémentaires draconiennes. (Pour les formules, on va supposer l'action de $\Gamma$ fidèle).

Tout-sous-groupe $\Gamma' \subset  E$ tel que $\Gamma' \cap \pi = \{ 1 \}$ (``sections partielles'') est cyclique (si $\Gamma = \Gamma^\circ$) ou diédral, et l'ensemble des classes de $\pi$-conjugaison de tels sous-groupes est fini. Tout sous-groupe section $\Gamma' \neq 1$ est contenu dans un unique sous-groupe section \emph{maximal}. (On l'a établi tout au moins dans le cas $\Gamma = \Gamma^\circ$, il faudrait revenir sur le cas général, je pense que cela reste vrai tel quel, à vérifier\dots)

De plus on vit apparaître une \emph{structure supplémentaire} sur le groupe $E$ [qui dans le cas anabélien s'identifie à un sous-groupe de $\Aut (\pi)$], à savoir une application
$$
\mu: \{ \text{élément}~u~\text{d'ordre fini de}~E \} \to T \otimes \mathbf{Q}/\mathbf{Z}
$$
obtenu en notant que si $u \in E$ est d'ordre fini $n$ d'où $\mathbf{Z}/n\mathbf{Z} \hookrightarrow{i} E$, on a Im $i \cap \pi = \{ 1 \}$ ($\pi$ n'a pas d'élément de torsion) donc $\mathbf{Z}/n\mathbf{Z} \hookrightarrow \Gamma$ d'où un sous-groupe $\Gamma_1 \subset  \Gamma$ (i.e. l'image a le même ordre $n$) et le relèvement $\Gamma_1 \to E$ définit un $x \in U^{\Gamma_i}$ et $U_1$ opérant sur $U$ en laissant fixe $n$ correspond donc au voisinage de $n$ à un ``multiplicateur'', qui (pour une orientation locale choisie) est une racine primitive $n^{\text{ième}}$ de 1 i.e. un élément de $\mathbf{Z}/n \mathbf{Z}$ et qui pour l'orientation changeante s'interprète intrinsèquement comme élément de $T \otimes \mathbf{Z}/n\mathbf{Z} \subset  T \otimes \mathbf{Q}/\mathbf{Z}$\footnote{Pour un $u$ d'ordre fini de $E$, on doit avoir, si $u \notin E^\circ$, que $u$ est d'ordre 2 exactement (\dots) [?].}.

L'application $\mu$ satisfait les conditions évidentes :
$$
\begin{cases}
\mu (\alpha u \alpha^{-1}) = \mu (u) \quad \text{si} \quad \alpha \in \pi \\
\mu (u^n) = n \mu(u) \quad \text{si} \quad n \in \mathbf{Z}
\end{cases}
$$
J'ignore si cette application $\mu$ peut se définir intrinsèquement en terme de l'extension ou si au contraire il peut exister deux extensions $E$, $E'$ de $\Gamma$ par $\pi$, définies par des situations géométriques de $\Gamma$ opérant sur $U$ et $U'$ et un isomorphisme d'extension $E$, $E'$ qui ne soit compatible avec les fonctions $\mu$, $\mu'$. Le cas non trivial le plus simple à regarder est le cas abélien (où $\pi \isom \mathbf{Z}$ (card $I = 2$) ou $\pi \isom \mathbf{Z}^2$ $(I = \emptyset)$). Dans le premier cas $\pi = \mathbf{Z}$, on doit avoir (pour avoir une action fidèle de $\Gamma = \Gamma^\circ$), 

$\Gamma^\circ$ cyclique, $E \isom \mathbf{Z}$, il n'y a pas d'éléments d'ordre fini dans $E$ sauf 1 donc la question ne se pose pas.

Le cas $\pi \isom \mathbf{Z}^2$ est plus intéressant ;  si $\pi$ est un $\mathbf{Z}$-module libre de rang $n$, on aune suite exacte
$$
0 \to \pi \to \pi \otimes_{\mathbf{Z}} \mathbf{R} \to X_0 (\pi) \to 0
$$
et si $\Gamma$ opère sur $\pi$, la suite exacte de cohomologie donne :

(compte tenu de $\mathrm{H}^i (\Gamma, \pi \otimes_{\mathbf{Z}} \mathbf{R}) = 0$ pour $i > 0$)
\[\begin{tikzcd}
	{\mathrm{H}^2(\Gamma, \pi)} && {\mathrm{H}^1(\Gamma, X_0(\pi))} \\
	{\text{classes d'extension}} && {\text{classes de}~X_0(\pi)~\text{torseur} \\ && \text{avec opération de}~\Gamma~\text{dessus compatible avec} \\ && \text{son action sur}~X_0(\pi)}
	\arrow["\sim"', shorten <=34pt, shorten >=34pt, from=1-3, to=1-1]
\end{tikzcd}\]
donc la donnée d'une extension $E$ de $\Gamma$ par $\pi$ revient essentiellement à celle d'un $\Gamma$-$X_0(\pi)$-torseur $X$ ($n$-tore inhomogène intrinsèque sur lequel $\Gamma$ opère - donc il opère sur son groupe des translations $X_0$, donc sur $\pi = \pi_1 (X_0, 0)$\dots). S'il était vrai pour $n = 1$ que toute opération de $\Gamma$ sur une surface torique $(\isom S' \times S')$ est isomorphe à une telle action standard, alors le caractère intrinsèque de l'application $\mu$ dans ce cas serait établi - ce qui ne rendrait pas inintéressant pour autant le calcul de $\mu$, qui prend ses valeurs dans $T \otimes \mathbf{Q}/\mathbf{Z}$ où ici $T \isom \Lambda^2_{\mathbf{Z}} \pi$ (dim 2 dans $\mathrm{H}^2(\pi, \mathbf{Z}) \isom \Lambda^2 \mathrm{H}^2 (\pi, \mathbf{Z}) = \pi$). 

La question revient à ceci : on a une extension \emph{scindée} d'un groupe cyclique $\mathbf{Z}/n \mathbf{Z}$ de générateur $u$ par $\pi$ (décrite entièrement par un automorphisme $\theta$ de $\pi$ tel que $\theta^n = \id_\pi$), décrire en termes de ceci un élément de $T \otimes \mathbf{Z}/\mathbf{Z}$.

Réponse : la situation géométrique standard correspond à $X = X_0(\pi)$, avec 0 comme point fixe sous $\Gamma$. Si on renverse l'orientation il est d'ordre 1 ou 2, il n'y a pas de problème, sinon c'est dans $\pi \otimes_{\mathbf{Z}} \mathbf{R}$ une rotation autour de 0 (d'ordre 2, 3, 4 ou 6) qui se repère bien par un élément de $T \otimes \mathbf{Z}/ \nu \mathbf{Z}$ (si $\nu$ est l'ordre). C'est aussi (si on identifie $T \otimes \mathbf{Z}/\nu \mathbf{Z}$ à $\mu_\nu (\mathbf{C}^*)$), en posant $T \xlongleftarrow{\sim} \mathbf{Z}$ donné i.e. $\pi$ orienté i.e. $X_0(\pi) = X$ \emph{orienté}) \emph{une} de deux valeurs propres de $u \otimes_{\mathbf{Z}} \mathbf{C}$ (automorphisme du vectoriel sur $\mathbf{C}$ de dimension 2 $\pi \otimes_{\mathbf{Z}} \mathbf{C}$).

Ceci nous montre, dans ce cas de la géométrie algébrique sur un corps algébriquement clos $\Omega$, que si $X = X_0$ est une courbe elliptique, $u$ un automorphisme d'ordre fini, on a comme description paradigmatique non \emph{seulement} $\pi = \pi_1 (X, 0)$ ($\mathbf{Z}$-module libre de rang 2) et l'action de $u$ sur $\pi$, mais comme structure \emph{supplémentaire} l'une des deux solutions dans $\Omega$ de l'équation caractéristique de $u$ (à coefficients entiers)
$$
T^2 + a T + b = 0 \quad (a = - \Tr u, b = \det u)
$$
Il est clair que cette structure supplémentaire ne peut se déduire de la seule connaissance de l'action de $u$ sur $\pi$.

Mais il reste la question si ce $\mu (u) \in \mu_n (\Omega)$ peut se déduire de la connaissance au moins de l'action extérieure de $u$ sur le groupe ``avec un lacet'' correspondant à la situation géométrique --- i.e. un automorphisme extérieur d'ordre $n$ d'un tel groupe\footnote{Oui il le peut grâce à la considération des ``sous-groupes de ramification'' de $E$ qui définissent des structures d'extensions
$$
1 \to T \xlongrightarrow{n_i \id_T} E_I (\isom T) \to \Gamma_I \to 1
$$
d'où $\Gamma_i \isom T/n_i T$\dots
}

(Où si là encore il faut la considérer décidément comme une donnée supplémentaire).

Mais s'il en était bien ainsi, cela impliquerait d'autre part que la construction de $u$ appartenant au groupe à (1) lacet(s), avec l'opération de $\Gamma$ dessus ne peut se faire non plus à l'aide de la seule connaissance de l'action de $\Gamma (= \mathbf{Z}/n \mathbf{Z})$ sur $\pi$.

C'est cette question de ``forage de trous'' qu'il faut donc en fin de compte, à la fin du fin, attaquer !

Quand à la question de savoir si dans le cas \emph{anabélien}, l'application $\mu: {}_\infty E \to T \otimes \mathbf{Q}/\mathbf{Z}$ (${}_\infty E$ : ensemble des éléments d'ordre fini de $E$) est déduisible de l'action de $\Gamma$ sur $\pi$ [si elle est \emph{fidèle}, $E$ s'identifie donc à un sous-groupe de $\Autext_{\text{lac}}(\pi) = A$, et on peut se demander si $\mu$ n'est pas alors définissable sur ${}_\infty E$ tout entier], ou si c'est une donné supplémentaire dont il faut disposer pour reconstruire la situation géométrique. La question reste entière\footnote{N.B. Cela semble bien ainsi, compte tenu que pour $\Gamma$ \emph{résoluble} (a fortiori pour $\Gamma$ cyclique) sauf erreur, on sait que toute action de $\Gamma$ sur $\pi$ se réalise géométriquement.}











%%%%%%%%%%%%%%%%%%%%%%%%%%%%%%%%%%%%%%%%%%%%%%%%%%%%%%%%%%%%%%%
\chapter*{\S \space 16. --- BOUCHAGE ET FORAGE DE TROUS : PRÉLIMINAIRES TOPOLOGIQUES GÉNÉRAUX}\thispagestyle{empty}
\addcontentsline{toc}{section}{16. Bouchage et forage de trous: préliminaires topologiques
généraux}
\label{sec:16}
\section*{}

Considérons une situation\footnote{(N.B. Une telle situation topossique (de topos multigaloisiens) décrit 2-fidèlement la situation topologique $(X, S)$ ou $U$, pourvu qu'aucune composante irréductible de $U$ ne soit $\isom \mathbb{S}^2$, et du fait qu'elle reste très proche du langage et de l'intuition topologique, elle est supérieure au point de vue ``groupe à lacets'', qui correspond plutôt à l'approche calculatoire.)}
$$
\B_{D^*} \to B_U
$$
Je m'aper\c{c}ois qu'il me fut revenir sur les notations des divers topos associes à une telle donnée. Mais je vais me guider sur la situation des n$^\circ$\dots où on a un schéma régulier $X$ de dimension 1, un sous-schéma fermé $S$ de dimension 0, d'où $U = X \textbackslash S$ --- dans ce cas $X_{\text{ét}}$ \emph{ne} peut se reconstituer à partir des $\B_{D^*}$ comme $\B_I$ avec $I = \pi_0 (\B_{D^*})$, il faut tenir compte des corps résiduels $k(s)$ ($s \in S$) i.e. des groupes de Galois $\Gal (\overline{k(s)}/k(s))$.

Donc il y a lieu de revenir à \emph{une} situation de départ (qui est adaptée au cas arithmético-géométrique) de morphismes de topos \emph{multigaloisiens}
\[\begin{tikzcd}
	& {\B_{D^*}} \\
	{\B_U} && {\B_S}
	\arrow["\rho"', from=1-2, to=2-1]
	\arrow["\sigma", from=1-2, to=2-3]
\end{tikzcd}\leqno{(*)}\]
(attention, on écrit $\B_S$, non $\B_D$, qui aura un autre sens), \emph{où $\sigma$ induit un isomorphisme sur les $\pi_0$} (et est surjectif sur les $\Hom$).

Pour l'instant, on ne va faire aucune hypothèse particulière sur cette situation, qui pour $\B_U$ connexe, et en termes des choix (de revêtements universels $\widetilde{D}^*_i$ des composantes connexes $D^*_i$ de $D^*$, d'un revêtement universel $\widetilde{U}$ de $U$, et d'isomorphismes entre les $\rho_! (\widetilde{D}^*_i)$ et $\widetilde{U}$ s'explicite par la donné des groupes $E$ (ou $\pi$) ($= \Aut (\widetilde{U})$) et $E_i$ (ou $\pi_i$ ($= \Aut (\widetilde{D}^*_i)$), $\Gamma_i (i \in I)$), et des homomorphismes de groupes 
\[\begin{tikzcd}
	& {E_i} \\
	E && {\Gamma_i}
	\arrow["{\rho_i}"', from=1-2, to=2-1]
	\arrow["{\sigma_i}", from=1-2, to=2-3]
\end{tikzcd}\]
avec les $\sigma_i$ surjectifs (quand il y a un corps de base $K$ pour la situation géométrique alors dans la description toposique, posons $e = \Spec K$ et désignons par $E_e$ le topos étale $e_{\text{ét}}$ de $e$, i.e. $\B_\Gamma$ si $\Gamma = \Gal (\overline{K}/K)$, le diagramme (*) s'insère dans
\[\begin{tikzcd}
	& {\B_{D^*}} \\
	{\B_U} && {\B_S} \\
	& {\B_e}
	\arrow["\varphi", from=2-1, to=3-2]
	\arrow["\rho"', from=1-2, to=2-1]
	\arrow["\psi"', from=2-3, to=3-2]
	\arrow["\sigma", from=1-2, to=2-3]
\end{tikzcd}\]
avec donnée de commutativité pour le carré envisagé\dots Comme au début de ces notes).

En termes de (*), on construit par ``recollement'' de $\B_U$ et de $\B_S$ (via le foncteur de recollement $\sigma_* \rho^*$) un topos mixte, qui n'est pas en général multigaloisien, noté précédemment $\B_{X, U}$, et que je préfère maintenant noté $\B_{X, S}$ [pour rappeler qu'il s'agit de faisceaux sur $X$, mais n'ayant de singularités que sur $S$].

Il s'insère dans un diagramme de topos
\[\begin{tikzcd}
	& {\B_{D^*}} \\
	{\B_U} && {\B_S} \\
	& {\B_e}
	\arrow[""{name=0, anchor=center, inner sep=0}, from=2-1, to=3-2]
	\arrow[from=1-2, to=2-1]
	\arrow[from=1-2, to=2-3]
	\arrow[""{name=1, anchor=center, inner sep=0}, from=2-3, to=3-2]
	\arrow["{\alpha_X}"', shorten <=6pt, shorten >=6pt, from=1, to=0]
\end{tikzcd}\]
avec une flèche ``de commutation'' $\alpha_X$ qui n'est telle que par abus de langage --- ce n'est pas un isomorphisme mais un morphisme de foncteurs sans plus
$$
\sigma^* \psi^* \xlongrightarrow{\alpha_X} \rho^* \varphi^*
$$

De la même fa\c{c}on, recollant $D^*$ et $S$ via $\sigma_*$ (i.e. rempla\c{c}ant $\B_U$ par $\B_{D^*}$ dans la construction précédente de $\B_{X, S}$), on trouve un topos, pas multigaloisien en général, noté $\B_{D, S}$. Dans le modèle géométrique avec un $X$, $S$ comme dessus, $\B_X$ correspond aux faisceaux sur $X$ qui sont essentiellement localement constants sur $U$ (et sur $S$, où ils n'ont pas de mérite) i.e. sur $U$ provenant de l'image inverse par $U$ $U \to \B_{\Pi, U}$ d'un faisceau sur $\B_{\Pi, U}$ et de même $\B_{D, S}$ correspond aux faisceaux sur $D = \amalg_i \Spec (\mathcal{O}_i =$ hensélisé de $\underline{\mathcal{O}}_{X, S}$) qui sont localement constants sur $D^* = D \textbackslash S$ (et sur $S$, sans mérite !) mais avec l'hypothèse de dimension faite on a en fait
$$
\B_{D, S} \isom D_{\text{ét}}
$$
($D_i$ n'a que 2 points, $D^*_i = \{ \eta_i \}$\dots).

Ainsi $\B_{D, S}$ ($\isom D_{\text{ét}}$ quand on part de $X$, $S$) s'insère dans un triangle de morphisme de 
\[\begin{tikzcd}
	& {\B_{D^*}} \\
	{\B_{D^*}} && {\B_S} \\
	& {\B_{D, S}}
	\arrow["{\varphi_D}", from=2-1, to=3-2]
	\arrow[shift left=1, from=1-2, to=2-1]
	\arrow[""{name=0, anchor=center, inner sep=0}, "\sigma", from=1-2, to=2-3]
	\arrow["{\psi_D}"', from=2-3, to=3-2]
	\arrow[""{name=1, anchor=center, inner sep=0}, shorten <=6pt, shorten >=6pt, no head, from=1-2, to=2-1]
	\arrow["{\alpha_D}", shorten <=7pt, shorten >=7pt, from=0, to=1]
\end{tikzcd}\]
où $\alpha_D$, comme $\alpha_X$ ci-dessus, n'est qu'un vulgaire homomorphisme (pas isomorphisme en général)\footnote{(N.B. dans le cas géométrique, les trois topos de ce diagramme sont des topos étales $(D^*_{\text{ét}}, D_{\text{ét}}, S_{\text{ét}})$ et les flèches $\varphi_D$ et $\psi_D$ correspondent à des morphismes de topos, mais non $\sigma$).}.

Quand on parle du morphisme canonique de $\B_D$ dans $\B_{D, S}$ c'est de $\varphi_D$ (et non $\psi_D \sigma$) qu0il s'agit --- dans le cas géométrique on a $\B_{D^*} \isom D^*_{\text{ét}}$ et $\varphi_D$ correspond à l'inclusion de schémas
$$
D^* \supset D \textbackslash S \to D
$$
et $\psi_D : \B_S \to B_{D, S}$ à l'inclusion de schémas $S \to D$ alors que $\sigma$ ni $\psi_D \sigma$ ne correspondent en général à des morphismes de schémas.

On obtient ainsi un diagramme de topos
\[\begin{tikzcd}
	&&& {\underline{\B}_S} \\
	{\underline{\B}_{D^*}} && {\B_{D, S}} \\
	&&& {\underline{\B}_S} \\
	{\underline{\B}_U} && {\B_{X, S}}
	\arrow[from=2-1, to=2-3]
	\arrow[from=4-1, to=4-3]
	\arrow[from=2-3, to=4-3]
	\arrow[from=2-1, to=4-1]
	\arrow[from=3-4, to=4-3]
	\arrow[from=1-4, to=2-3]
	\arrow[shift left=1, from=1-4, to=3-4]
	\arrow[shorten <=10pt, shorten >=13pt, no head, from=1-4, to=3-4]
	\arrow[from=2-1, to=1-4]
\end{tikzcd}\leqno{(**)}\]
(Attention le triangle n'est pas essentiellement commutatif mais on a une pseudo-commutativité\dots)

où les trois topos soulignés $\underline{\B}_U$, $\underline{\B}_S$, et $\underline{\B}_{D^*}$ sont multigaloisiens, et $\B_{D, S}$ et $\B_{X, S}$ sont composites (obtenus par recollement de deux topos multigaloisiens).

Sauf erreur les deux carrés sont [non seulement essentiellement commutatifs, mais aussi] 2-cartésiens (dans la 2-catégorie des topos).

Passant aux $\B_{\Pi_1}$ ? des topos envisagés (correspondant aux objets localement constants sur ce topos) on trouve un topos $\B_X = \B_{\pi_1 \B_{X, S}}$ comme somme amalgamée de topos dans le diagramme 
\[\begin{tikzcd}
	& {\B_{D^*}} \\
	{\B_U} && {\B_S} \\
	& {\B_X}
	\arrow[from=1-2, to=2-1]
	\arrow[""{name=0, anchor=center, inner sep=0}, from=2-1, to=3-2]
	\arrow[""{name=1, anchor=center, inner sep=0}, from=2-3, to=3-2]
	\arrow[from=1-2, to=2-3]
	\arrow["\sim"', shorten <=7pt, shorten >=7pt, from=1, to=0]
\end{tikzcd}\]
(ce carré est bel et bien essentiellement commutatif) et de même (en rempl\c{c}ant $\B_U$ par $\B_{D^*}$) un $\B_D$, qui est cependant (par $\B_S \to \B_D$) isomorphe (plutôt équivalent) à $\B_S$. Le diagramme (**) devient alors
\[\begin{tikzcd}
	&&& {\underline{\B}_S} \\
	{\underline{\B}_{D^*}} && {\B_D} \\
	&&& {\underline{\B}_S} \\
	{\underline{\B}_U} && {\B_X}
	\arrow[from=2-1, to=2-3]
	\arrow[from=4-1, to=4-3]
	\arrow[from=2-3, to=4-3]
	\arrow[from=2-1, to=4-1]
	\arrow[from=3-4, to=4-3]
	\arrow["\sim", from=1-4, to=2-3]
	\arrow[shift left=1, from=1-4, to=3-4]
	\arrow[shorten <=10pt, shorten >=13pt, no head, from=1-4, to=3-4]
	\arrow["\sigma"', from=2-1, to=1-4]
\end{tikzcd}\leqno{(***)}\]
où cette fois-ci le triangle supérieur est bien essentiellement surjectif (c'est bien comme \c{c}a que l'on a défini $\sigma$, dans le cas géométrique !) et on a un morphisme de (**) dans (***), qui par les flèches qui sont par essence des identités [à] savoir $\B_{D, S} \to \B_D$ et $\B_{X, S} \to \B_X$, ont une nette tendance à être ``acyclique'' ou à induire des isomorphismes sur la cohomologie (il faudrait vérifier ce point). Enfin, dans le cas géométrique, on a un diagramme analogue de morphismes de topos étales
\[\begin{tikzcd}
	&&& {S_{\text{ét}}} \\
	{D^*_{\text{ét}}} && {D_{\text{ét}}} \\
	&&& {S_{\text{ét}}} \\
	{U_{\text{ét}}} && {X_{\text{ét}}}
	\arrow[from=2-1, to=2-3]
	\arrow[from=2-1, to=4-1]
	\arrow[from=4-3, to=4-1]
	\arrow[from=2-3, to=4-3]
	\arrow[from=3-4, to=4-3]
	\arrow[from=1-4, to=2-3]
	\arrow[from=2-1, to=1-4]
	\arrow[shift left=1, from=1-4, to=3-4]
	\arrow[shorten <=13pt, shorten >=13pt, no head, from=1-4, to=3-4]
\end{tikzcd}\leqno{(****)}\]
(triangle supérieur pas essentiellement commutatif)

où toutes les flèches sauf $D^*_{\text{ét}} \to S_{\text{ét}}$ sont induites par des morphismes de topos, le (****) s'insère dans le diagramme homologue (**), en induisant des isomorphismes de topos pour $D^*$, $D$, $S$ et, pour $U_{\text{ét}} \to \B_U$ et $X_{\text{ét}} \to X$, induisant des morphismes qui ont moins tendance à être acyclique, mais qui le sont quand même dans des cas importants, rappelés dans une section antérieure\dots

Les constructions de (**) et (***) en termes du diagramme de départ
\[\begin{tikzcd}
	& {\B_{D^*}} \\
	{\B_U} && {\B_S}
	\arrow[from=1-2, to=2-1]
	\arrow[from=1-2, to=2-3]
\end{tikzcd}\]
sont purement formelles, et indépendantes de toutes hypothèses. La construction d'un $\B_X$ multigaloisien, comme somme amalgamée, peut être interprété comme la traduction (au niveau des groupoïdes fondamentaux) d'une opération de ``bouchage de trous''.

Dans le contexte calculatoire (avec choix de $\widetilde{U}$, $\widetilde{U}_i$, $\rho_! (\widetilde{U}_i) \isom \widetilde{U}$) avec
\[\begin{tikzcd}
	& {E_i} \\
	\pi && {\Gamma_i}
	\arrow["{\rho_i}"', from=1-2, to=2-1]
	\arrow["{\sigma_i}", from=1-2, to=2-3]
\end{tikzcd}\]
posant $\widetilde{X} =$ image de $\widetilde{U}$ par $i$ ! $(i: \B_U \to \B_{\pi_1 X})$, $\B_U$ est décrit en termes de ce $\widetilde{U}$ comme le classifiant $\B_{\pi X}$, où $\pi_X = \pi_1 (X, \widetilde{X})$ se calcule comme quotient de $\pi$ par le sous-groupe invariant engendré par les $\rho_i (L_i)$, où $L_i = \ker \sigma_i \supset E_i$.

Si on se donne une sommande directe $\B_{D^*_0}$ dans $\B_{D^*}$ (correspondant à une partie de $I_0$ de $I = \pi_0 (\B_{D^*})$) on trouve de même une somme amalgamée de $\B_U$ et de $\B_{S_D}$ sous $\B_{D^*_0}$ notée $\B_{U \cup S_1}$ qui se visualise comme un bouchage partiel de trous, interprété au niveau des groupoïdes fondamentaux. Dans le cas géométrique, si on pose $I = I_0 \amalg I_1$, i.e. $S = S_0 \amalg S_1$, on peut interpréter ce topos comme $\B_{\pi_1 U_1}$, où $U = X \textbackslash S_1$. 

Bien sûr, on a un homomorphisme de diagrammes cartésiens de topos relatifs à 
\[\begin{tikzcd}
	& {\B_{D^*_0}} \\
	{\B_U} && {\B_{S_0}}
	\arrow[from=1-2, to=2-3]
	\arrow[from=1-2, to=2-1]
\end{tikzcd}\]
celui relatif à $\B_{D^*}$ s'envoyant dans $\B_U$, et $\B_S$.

\[\begin{tikzcd}
	&& {\B_{D^*_1}} &&& {\B_{D^*}} \\
	{[\B_{D^*_1}]} &&& {\B_{S_0}} &&& {\B_S} \\
	& {\B_U} &&& {\B_U} \\
	&& {\B_{U \cup S_0}} &&& {\B_{U \cup S}}
	\arrow[from=2-4, to=4-3]
	\arrow[from=1-3, to=3-2]
	\arrow[from=1-3, to=2-4]
	\arrow[from=3-2, to=4-3]
	\arrow[from=2-4, to=2-7]
	\arrow[shift right=1, from=3-2, to=3-5]
	\arrow[shorten <=34pt, shorten >=34pt, no head, from=3-2, to=3-5]
	\arrow[from=4-3, to=4-6]
	\arrow[from=3-5, to=4-6]
	\arrow[from=1-6, to=2-7]
	\arrow[from=2-7, to=4-6]
	\arrow[from=1-6, to=3-5]
	\arrow[from=1-3, to=1-6]
	\arrow[dashed, from=2-1, to=3-2]
\end{tikzcd}\]
et d'autre part on a un composé
$$
\B_{D^*_1} \to \B_U \to B_{U_1}
$$
qui avec $\B_{D_1} \to \B_{S_1}$, donne un diagramme de topos
\[\begin{tikzcd}
	& {\B_{D^*_1}} \\
	{\B_{U_1}} && {\B_{S_1}}
	\arrow[from=1-2, to=2-1]
	\arrow[from=1-2, to=2-3]
\end{tikzcd}\]
du même type qu'au début, qu'on peut utiliser pour construire encore une somme amalgamée. Et il est évident que celle-ci est canoniquement équivalente à $\B_X$, la somme amalgamée correspondant à cette situation du départ\dots

Toutes ces opérations sont essentiellement triviales et sans mystère, et indépendantes de toutes hypothèses spéciales du type ``groupe à lacets''. Un intérêt particulier s'attache au cas où $\B_S$ est un topos discret : $e$ s'identifie $B_I$ où $I = \pi_0 (\B_{D^*})$ de sorte qu'on part simplement d'un morphisme de topos multigaloisiens 
$$
\B_{D^*} \to \B_U
$$
mais où de plus on a un groupe $\Gamma$ (discret, disons ou profini dans le contexte arithmétique) qui opère sur $\B_{D^*}$, $\B_U$ ; le morphisme précédent commutant à l'action de $\Gamma$.

Notons que la donnée d'une action de $\Gamma$ sur un topos $\B$ permet de construire un topos $(\B, \Gamma)$, et un morphisme $(\B, \Gamma) \to \B_\Gamma$ (topos classifiant de $\Gamma$) \emph{i.e.} un $\Gamma$-torseur dans $(\B, \Gamma)$ et $\B$ s'identifie au topos induit par $X = (\B/\Gamma)$ sur ce $\Gamma$ torseur.

Inversement, la donnée d'un topos $X$ et d'un $\Gamma$-torseur $U$ dans $X$ et d'un isomorphisme de $\B$ avec le topos induit $X/U$ (identifié à $U$, ou à $\B$) permet de récupérer des opérations de $\Gamma$ sur $\B$, via les opérations sur $U$. Donc la \emph{donnée} d'une opération de $\Gamma$ sur un topos $\B$ revient à celle de la donnée de $\B$ comme revêtement galoisien de groupe $\Gamma$ d'un autre topos (essentiellement unique, noté alors $(\B, \Gamma)$\dots). Ainsi, faire opérer $\Gamma$ sur $\B_{D^*} \to \B_U$, c'est la même chose que d'insérer cette flèche dans un diagramme commutatif
\[\begin{tikzcd}
	{\B_{D^*}} && {\B_U} \\
	{\B_{D^*, \Gamma}} && {\B_{U, \Gamma}}
	\arrow[from=1-1, to=1-3]
	\arrow[from=1-1, to=2-1]
	\arrow[from=1-3, to=2-3]
	\arrow[from=2-1, to=2-3]
\end{tikzcd}\]
où les flèches verticales sont $\Gamma$-galoisiennes, et le carré est 2-cartésien, ou encore (indépendamment de la donné préalable de $\b_{D^*}$, $\B_U$) c'est se donner un triangle essentiellement commutatif de morphismes de topos
\[\begin{tikzcd}
	{\B_{D^*, \Gamma}} && {\B_{U, \Gamma}} \\
	& {\B_\Gamma}
	\arrow[from=1-1, to=2-2]
	\arrow[from=1-3, to=2-2]
\end{tikzcd}\]
Si $\Gamma$ opère sur un topos multigaloisien, on veut que $(\B, \Gamma)$ soit aussi multigaloisien, et la situation d'un topos multigaloisien $\B_U$ et d'une opération de $\Gamma$ dessus revient à celle d'un topos multigaloisien $\B_{U, \Gamma}$ et d'un morphisme $\B_{U, \Gamma} \to \B_\Gamma$.

Donc la donnée d'une situation $\B_{D^*} \to \B_U$ de topos multigaloisiens et d'une opération de $\Gamma$ dessus revient exactement à celle d'homomorphismes de topos multigaloisiens
$$
\B_{D^*, \Gamma} \to \B_{U, \Gamma} \to \B_\Gamma
$$
$\B_{U, \Gamma}$ est 0-connexe si et seulement si card $\pi_0 (\B_U)/\Gamma = 1$, i.e. $\B_U$ non vide est $\Gamma$-transitif sur $\pi_0 (\B_U)$. Notons que la situation envisagée au début, avec $(X, S)$ sur un corps $K$, d'où $\B_{D^*} \to \B_U \to \B_\Gamma$ ($\Gamma = \Gal (\overline{K}/K)$), peut être interprétée comme déduite de la situation ``géométrique'' $(\overline{K}, \overline{S})$ sur $\overline{K}$, $\B_{\overline{D}^*_E} \to \B_{\overline{U}}$, en tenant compte des opérations de $\Gamma$ dessus. Il se trouve que pour beaucoup de questions, c'est cette interprétation ``géométrique'' (au sens strict, i.e. $\overline{K}$ algébriquement clos) avec opérations d'un groupe de Galois $\Gamma$, qui est la plus commode.

Si on regarde une opération de $\Gamma$ sur un topos ($\B_{D^*}$ disons), il opère sur le topos discret $\B_I$ ($I = \pi_0 (\B_{D^*})$), et $\B_{D^*} \to B_I$ est compatible aux actions de $\Gamma$.

Mais le topos $\B_{(I, \Gamma)}$ est aussi celui des $\Gamma$-ensembles au dessus de $I$ (sur lequel $\Gamma$ opère) un topos induit dans $\B_\Gamma$. Ses composantes connexes correspondent aux orbites de $\Gamma$ sur $I$. Si $\Gamma$ est transitif sur $I$ non vide (ou si on regarde \emph{une} telle orbite\dots), choisissant $i \in I$, le topos en question s'identifie à $\B_{\Gamma_i}$ où $\Gamma_i$ est le stabilisateur de $i$ dans $\Gamma$.

N.B. Si on donne une opération de $\Gamma$ sur un topos \emph{discret} $\B (= \B_I)$, quand on l'interprète en tant que morphisme d'un topos multigaloisien $\B' = (\B, \Gamma) \to \B_\Gamma$, est caractérisé par le fait que le morphisme du groupoïde qui la décrit soit \emph{injectif} sur les flèches, i.e. en termes d'un système d'homomorphismes de groupe $\Gamma_j \to \Gamma$ $i \in J (\isom I\textbackslash\Gamma)$, par la condition que ces homomorphismes soient injectifs. $\Gamma$ est donc $I$ se reconstitue comme la somme directe des $\Gamma\textbackslash \Gamma_j$\dots]

Ainsi un diagramme $\B_{D^*} \to \B_U$ \emph{avec action de $\Gamma$} équivaut à la donnée d'un diagramme 
$$
\B_{D^*, \Gamma} \to \B_{U, \Gamma} \to \B_\Gamma
$$
et celui-ci se complète (en utilisant l'action de $\Gamma$ sur $\pi_0 (\B_{D^*})$ --- $\B$ étant lui-même déduit de $\B_{D^* \Gamma} \to \B_\Gamma$ comme l'image inverse du torseur universel) en
$$
\B_{D^*, \Gamma} \to \B_{I, \Gamma} \to \B_\Gamma
$$
i.e. il s'agit de la factorisation canonique d'un homomorphisme de groupoïdes en homomorphisme bijectif (pour les objets) épimorphique (pour les $\Hom$) suivi d'un homomorphisme épimorphique (sur les $\Hom$).

On trouve ainsi un carré essentiellement commutatif
\[\begin{tikzcd}
	& {\B_{D^*,\Gamma}} \\
	{\B_{U,\Gamma}} && {\B_{I,\Gamma}} \\
	& {\B_{\Gamma}}
	\arrow[from=1-2, to=2-1]
	\arrow[from=1-2, to=2-3]
	\arrow[from=2-3, to=3-2]
	\arrow[from=2-1, to=3-2]
\end{tikzcd}\]
qui correspond au diagramme de groupoïdes au début des notes (\S 7)
\[\begin{tikzcd}
	& {\Pi_{D^*}} \\
	{\Pi_U} && {\Pi_D} \\
	& {\Pi_e}
	\arrow[from=1-2, to=2-1]
	\arrow[from=1-2, to=2-3]
	\arrow[from=2-3, to=3-2]
	\arrow[from=2-1, to=3-2]
\end{tikzcd}\]

J'ai l'impression d'avoir à peu près compris le mécanisme des actions des groupes sur des topos multigaloisiens, et comment l'opération de passage d'un topos $\B$ avec opération de $\Gamma$ au topos ``quotient'' $(\B, \Gamma) = ``\B/\Gamma$'', commute aux opérations du type passage de $\B_{D^*} \to \B_U$ à un $\B_{X,S}$, via un $\B_X$ (``Bouchage des trous''). Le temps semble donc mûr enfin pour s'expliquer avec l'opération inverse hypothétique de ``forage des trous''.
 












%%%%%%%%%%%%%%%%%%%%%%%%%%%%%%%%%%%%%%%%%%%%%%%%%%%%%%%%%%%%%%%
\chapter*{\S \space 17. --- COMPLÉMENTS SUR LES OPÉRATIONS DE GROUPES FINIS SUR LES SURFACES \\ (COMPLÉMENT AU \S 15)}\thispagestyle{empty}
\addcontentsline{toc}{section}{17. Complément au \S 15 ; sous-groupes de groupes à lacets}
\label{sec:17}
\section*{}

\vskip .3cm
{
Théorème. --- \it Soit $U$ surface paracompacte connexe telle que $\pi_1 (U) \neq (1)$, i.e. $U \not\simeq \mathbb{S}^2$, $\mathbf{R}^2$. On dit que $U$ est ``anabélienne'' si $\pi = \pi_1 (U)$ non abélien (auquel cas $\Centre (\pi) = 1$) et si $U \not\simeq \mathbf{C}^*$, $\mathbb{S}^1 \times \mathbb{S}^1$.

Soit $\Gamma$ un groupe fini opérant sur $U$\footnote{N. B. : Il est prudent de supposer que $\Gamma$ opère en conservant l'orientation de $U$ (supposée orientable) sinon on a des ennuis par exemple avec $z \mapsto \overline{z}^{-1}$ de $\mathbf{C}^* \to \mathbf{C}^*$ (Cela doit être le sel contre-exemple dans le cas où $\Gamma$ ne conserve pas l'orientation\dots) En tout cas un contre-exemple doit être tel que (si $\Gamma$ fidèle) $\Gamma = \isom \{ \pm 1 \}$, opère par anti-involutions\dots Il faudrait tirer au clair le cas de la situation générale\dots}, on a les conditions équivalentes :

\begin{enumerate}
    \item[a)] (cas anabélien) $\Gamma$ opère trivialement ou (cas abélien) structure de groupe topologique sur $U$ (donc $U \isom \mathbb{S}^*$ de $\mathbb{S}^1 \times \mathbb{S}^1$) de fa\c{c}on que $\Gamma$ opère par translations,
    \item[b)] $\forall g \in \Gamma$, $g_U$ est isotope à l'identité,
    \item[b')] $\forall g \in \Gamma$, $g_U$ est homotope à identité,
    \item[c)] l'opération extérieure de $\Gamma$ sur $\pi_1 (U)$ est triviale,
    \item[d)] (cas anabélien) l'extension $E$ de $\Gamma$ par $\pi$ est isomorphe à une extension produit, ou (cas abélien) elle est centrale
\end{enumerate}
}
\vskip .3cm

{\emph Démonstration}. a) $\Rightarrow$ b) $\Rightarrow$ b') $\Rightarrow$ c) trivial.

c) $\Rightarrow$ d). Dans [le] cas anabélien, cela provient du fait que $\Centre (\pi) = 1$ une extension de $\Gamma$ par $\pi$ est définie déjà par l'action extérieure, comme image inverse de l'extension 
$$
1 \to \pi \to \Aut \pi \to \Autext \pi \to 1. 
$$
Dans le cas abélien c'est trivial.

d) $\Rightarrow$ a) est la partie pas évidente. OPS que $\Gamma$ opère fidèlement.

\emph{Cas anabélien} : Si on avait $\Gamma = 1$, pour un scindage de l'extension de $\Gamma$ par $\pi$, on doit avoir par le théorème du n$^\circ$ 15 $\pi^\Gamma = 1$, or le scindage canonique de $\pi \times \Gamma$ sur $\Gamma$ donne $\pi^\Gamma = \pi$, absurde.

\emph{Cas abélien}. $U \isom \mathbf{C}^*$ (plus intrinsèquement $U$ est un torseur sous $U_0 = \pi \otimes_{\mathbf{Z}} \mathbf{C}/\pi$) ou $U \isom \mathbb{S}^1 \times \mathbb{S}^1$ (plus intrinsèquement $U$ est un torseur sous $U_0 \isom \mathbb{S}^1 \times \mathbb{S}^1$).

Je dis qu'une action de $\Gamma$ sur $U$ est (à homéomorphisme près) défini par une action de $\Gamma$ sur le torseur\footnote{pas prouvé !} ; la classe d'isomorphisme d'un tel torseur s'identifie par ailleurs par la suite exacte de cohomologie associée à la suite exacte
\[\begin{tikzcd}
	0 & \pi & {\pi \otimes_{\mathbf{Z}} \mathbf{C}} & {U_0} & 0 \\
	&& {(\text{ou}~\pi \otimes_{\mathbf{Z}} \mathbf{R})}
	\arrow[from=1-1, to=1-2]
	\arrow[from=1-2, to=1-3]
	\arrow[from=1-3, to=1-4]
	\arrow[from=1-4, to=1-5]
\end{tikzcd}\]
à une classe d'extension de $\Gamma$ par $\pi$.

Mais dire que l'action de $\Gamma$ sur $\pi$ est triviale, signifie que l'action de $\Gamma$ sur le torseur $U$ sous $U_0$ se fait par translations.

\vskip .3cm
{
Corollaire --- Scholie. --- \it Le cas ``abélien'' n'est pas tout à fait démontré faute d'avoir établi la classification topologique des opérations d'un groupe fini sur $\mathbf{C}^*$ ou sur $\mathbb{S}^1 \times \mathbb{S}^1$. Cependant, si dans le cas abélien on suppose d'avance que $U^{\Gamma} = \emptyset$ \emph{alors} il est encore vrai que l'opération triviale de $\Gamma$ sur $\pi$ équivaut à la trivialité de l'action de $\Gamma$ sur $U$. Car on est ramené au cas où $\Gamma$ opère fidèlement et à prouver dans ce cas que si l'opération de $\Gamma$ sur $\pi$ est triviale on a $\Gamma = 1$. Et on fait comme plus haut dans le cas anabélien.
}









%%%%%%%%%%%%%%%%%%%%%%%%%%%%%%%%%%%%%%%%%%%%%%%%%%%%%%%%%%%%%%%
\chapter*{\S \space 18. --- FORAGE DE TROUS ; APPLICATION AUX ACTIONS EXTÉRIEURES DE GROUPES FINIS}\thispagestyle{empty}
\addcontentsline{toc}{section}{{\bf 18.} Forage de trous ; applications aux sous-groupes finis} % de ${\rm Autext}_{\rm lac}(\pi)$}
\label{sec:18}
\section*{}

Soit $\pi'$ un groupe à lacets de type $(g, \nu + 1)$ non abélien (i.e. si $g = 0$ on a $\nu \geq 2$).

Si $T$ son module des orientations et $I'$ (card $I' = \nu + 1$) l'ensemble de ses classes de conjugaison de sous-groupes de lacets.

Fixons nous un $i \in I'$ et soit $L'_i \subset  \pi'$ dans la classe $i$.

Quand on se donne seulement un groupe \emph{extérieur} à lacets $[\pi']$ (ce qui équivaut à la donnée de $\B_{D'_*} \to \B_U$), la donnée d'un $i \in I'$ équivaut à celle d'une composante connexe de $\B_{D'_*}$, et celle d'une réalisation du groupe extérieur (i.e. d'un couple d'un groupe $\pi'$ et d'un isomorphisme extérieur $\pi' \to [\pi']$) équivaut à isomorphisme près à celle d'un objet de $\pi_1 \B_{U'}$ (revêtement universel $\widetilde{U}'$ de $U'$ pour $\pi' = \Aut(\widetilde{U}')$).

Enfin la donnée d'un couple $(\pi', L'_i)$ ($L'_i$ dans la classe de conjugaison) équivaut (à isomorphisme unique près) à la donnée d'un objet $\widetilde{D}^{'*}_i$ dans $\pi_1 \B_{D^{'*}_i}$ en prenant l'image $\widetilde{U}'$ de $\widetilde{D}^{'*}$ dans $\pi_1 \B_{U'}$ et $\pi' = \Aut (\widetilde{U}')$, $L'i = \Im (\Aut(\widetilde{D}^{'*}_i))$ dans $\Aut(\widetilde{U}')$.

Quand on se donne un objet $U'_0$ de $\pi_1 \widetilde{\B_{U'}}$, d'où une réalisation $\pi' = \Aut (\widetilde{U}'_0)$. (On va laisser tomber provisoirement les primer) alors la donnée d'un $L_i \subset  \pi$ dans la classe $i$ équivaut à la donnée d'un couple $(\widetilde{D}_i, \lambda)$ d'un $\widetilde{D}_i \in \Ob \pi_1 \B_{D_i}$ et d'un isomorphisme de $\rho_! (\widetilde{D}_i) = \widetilde{U}$ avec $\widetilde{U}_0$.

Quand la situation topossique est réalisée à partir d'un situation topologique $(X, S)$ et qu'on définit $\widetilde{U}_0$ à l'aide d'un point de base $a \in U$, alors la fa\c{c}on standard de définir un $L_i \subset  \pi = \Aut (\widetilde{U}_0) = \pi_1(U, a)$ est de choisir une petite rondelle $\Delta_i$ autour de $s_i \in X$, un point $b_i$ sur le bord et une classe d'homotopie de chemin dans $U - (\Delta^\circ_i - \{ s_i \}) = V_i$ de $a$ vers $b_i$ et de prendre le groupe $L_i$ engendré par l'un quelconque des deux lacets correspondants autour de $s_i$ (qui donnent des lacets opposés dans $\pi$).

%%%%%%%%%%%%%

On voit que l'on trouve ainsi une application surjective de $\Isom_{\pi_1 V_i}(a, b_i) \isom \isom_{\pi_1 U}(a, b_i)$ sur l'ensemble des $L_i \subset  \pi$ dans la classe $i$, application compatible avec l'action naturelle de $\pi$ opérant sur $\Isom_{\pi_1 U}(a, b_i)$ de la fa\c{c}on évidente par composition et sur l'ensemble des $L_i$ par automorphisme intérieur

Ici le lien avec la description ``abstraite'' topossique s'établit ainsi : le choix d'un $b_i$ peut servir de point de base pour définir un revêtement universel de $\Delta_i \textbackslash \{ s_i \} \isom D^*_i$, d'où un objet $\rho_! (\widetilde{D^*_i}(b_i))$ et les $L_i \subset  \pi$ correspondant (d'après la description abstraite) aux isomorphismes de $\widetilde{U}_0 = \widetilde{U}(a_i)$ avec $D^*_i (b_i)$ modulo composition avec un automorphisme de $D^*_i (b_i)$ provenant d'un automorphisme de $D^*_i(b'_i)$ mais les isomorphismes $\widetilde{U}(a_i) \isom \rho_! (\widetilde{D}^*_i(b_i))$ correspondent justement aux classes de chemins de $a$ vers $b_i$.

(On revient aux notations $\pi'$, $u'$,\dots)

Considérons un objet de $\pi_1 \B_{D^{'*}_i}$, i.e. un couple $(\pi', L'_i)$. Soit $\pi$ le groupe quotient de $\pi'$ par le sous-groupe invariant de $\pi'$ engendré par $L'_i$. \emph{Je dis qu'à isomorphisme près (isomorphisme effectif de groupes, par seulement extérieur !) il ne dépend pas du choix de l'objet $\widetilde{D}^*_i$ dans $\pi_1 \B_{D^{'*}_i}$}. En effet $\pi (\widetilde{\B}^*_i)$ dépend fonctoriellement de  $\widetilde{D}^*_i$ et tout revient à voir que ce foncteur est \emph{constant}, i.e., que l'opération de $\Aut(\widetilde{D}^*_i) \isom T \isom L_i$ sur $\pi = \pi (\widetilde{D}^*_i)$ est triviale. Or soit $u \in L_i$, l'automorphisme de $\pi$ qu'il défini est défini par l'automorphisme intérieur int$(u)$, pas passage au quotient donc (comme $u$ devient 1 dans $\pi$) il est trivial.

On trouve ainsi un \emph{foncteur} : ``bouchage du trou $i$''
\[\begin{tikzcd}
	{\begin{pmatrix} \text{Groupes extérieurs à lacets}~\pi' \\ \text{de type}~(g, \nu + 1)~\text{munis d'une} \\ \text{classe de lacets}~i \in I(\pi') \end{pmatrix}} & {\begin{pmatrix} \text{groupes (réalisés)} \\ \text{à lacets de type}~(g, \nu) \end{pmatrix}} \\
	{(\pi', 1)} & {\beta(\pi', i)}
	\arrow[from=1-1, to=1-2]
	\arrow[shorten <=18pt, shorten >=18pt, maps to, from=2-1, to=2-2]
\end{tikzcd}\]
qui s'exprime par un homomorphisme de groupes 
\[\begin{tikzcd}
	{\begin{matrix} \Autext (\pi', i) \\ \text{(= ensemble des automorphismes extérieures} \\ \text{de}~\pi'~\text{respectant la structure à lacets)} \\ \text{et fixant la classe de lacets}~i \end{matrix}} & {\begin{matrix} \Aut \pi \\ (\text{où}~\pi=\beta(\pi', i)) \end{matrix}}
	\arrow[from=1-1, to=1-2]
\end{tikzcd}\]
D'un point de vue géométrique ce fait (existence d'un foncteur) ne fait qu'exprimer le fait qu'après ``bouchage du trou'' $i$ on a un $U = U' \cup \{ s_i \}$ \emph{muni} d'un point $s_i$, que l'on peut utiliser comme \emph{point de base canonique} pour calculer $\pi_1(U)$. On peut dire aussi que le choix de $i$ permet de construire la somme amalgamée partielle (bouchage partiel de trous) $\B_U$.
\[\begin{tikzcd}
	& {\B_{D^{'*}_i}} \\
	{\B_{U'}} && {\B_{s_i}} \\
	& {\B_U}
	\arrow[from=1-2, to=2-1]
	\arrow[from=1-2, to=2-3]
	\arrow[from=2-1, to=3-2]
	\arrow[from=2-3, to=3-2]
\end{tikzcd}\]
et $\B_{s_i} \to \B_{U^*}$ fournit un point géométrique dans $\B_{U^*}$ qui permet de décrire un objet canonique de $\pi_1 \B_{U^*}$ (revêtement universel relatif à ce point) d'où canoniquement un groupe $\pi$, qui bien sûr est un groupe à lacets de type $(g, \nu - 1)$.
\vskip .3cm
{
Théorème. --- \it Supposons $g, \nu$ tel que non seulement les groupes $\pi'$ de type $(g, \nu)$, mais aussi le groupe $\pi$ (de type $(g, \nu - 1)$) soient anabélien, i.e. $2 g + \nu \geq 4$. Alors le foncteur précédent $(\pi', i) \mapsto \pi$ est une équivalence de catégorie. En d'autres termes (comme il s'agit de groupoïdes 0-connexes)
$$
\Autext (\pi', i) \to \Aut (\pi)
$$
est un isomorphisme.
}
\vskip .3cm
Notons pour ceci que l'on a une suite exacte
$$
1 \to \pi \to \Aut \pi \to \Autext \pi \to 1
$$
(car centre $\pi = 1$ par hypothèse anabélienne sur $\pi$) or on va définir une suite exacte
$$
1 \to \pi \to \Autext (\pi', i) \to \Autext \pi \to 1
$$
et un homomorphisme d'extension de celle-ci dans la précédente (qui sera nécessairement un isomorphisme).
O.P.S. $T = \mathbf{Z}$, on considère les groupes à lacets standard $\pi_{g, \nu + 1}$, $\pi_{g, \nu}$. Posons
\[\begin{tikzcd}
	{\begin{matrix} \Autext (\pi_{g, \nu})  \\ \text{(induisant l'identité sur} \\ I(\pi_{g, \nu}~\text{et sur}~T(\pi_{g, \nu})) \end{matrix}} & {=} & {\begin{matrix} T^{\circ\circ}_{g, \nu} \\ \text{(groupe de Teichmüller} \\ \text{d'indice}~(g, \nu)) \end{matrix}}
\end{tikzcd}\]
N.B. Le revêtement universel universel de $M_{g, \nu}$ est contractile par Teichmüller.

On sait que (pour $2g + \nu \geq 3$) $T_{g, \nu}$ est le groupe fondamental du topos modulaire complexe $M_{g, \nu}$ des courbes complexes (projectives non singulières connexes de genre $g$, avec un système de $\nu$ point $s_1 \dots s_{\nu}$ distincts données). Or le topos modulaire $M_{g, \nu + 1}$ n'est autre que la ``courbe complexe universelle de genre $g$ à $\nu$ tous numérotés'' sur $M_{g, \nu}$ [car se donner une courbe $U'$ de genre $g$ avec $\nu + 1$ trous $x_1 \dots x_{\nu + 1}$ \emph{plus} un point de $U$] d'où une suite exacte d'homotopie  
\[\begin{tikzcd}
	{\pi_2(M_{g, \nu})} & {\pi_1(M)} & {\pi_1(M_{g, \nu + 1})} & {\pi_1(M_{g, \nu})} & {\pi_0(U)} \\
	0 & {\pi = \pi_{g, \nu}} & {T^{\circ\circ}_{g, \nu + 1}} & {T^{\circ\circ}_{g, \nu}} & 1
	\arrow[from=1-1, to=1-2]
	\arrow[from=1-2, to=1-3]
	\arrow[from=2-1, to=2-2]
	\arrow[from=2-2, to=2-3]
	\arrow[shift left=1, no head, from=1-1, to=2-1]
	\arrow[no head, from=1-1, to=2-1]
	\arrow[shift left=1, no head, from=1-2, to=2-2]
	\arrow[no head, from=1-2, to=2-2]
	\arrow[shift left=1, no head, from=1-3, to=2-3]
	\arrow[no head, from=1-3, to=2-3]
	\arrow[from=1-3, to=1-4]
	\arrow[from=2-3, to=2-4]
	\arrow[from=1-4, to=1-5]
	\arrow[from=2-4, to=2-5]
	\arrow["\sim"', no head, from=1-4, to=2-4]
	\arrow[shift left=1, no head, from=1-5, to=2-5]
	\arrow[no head, from=1-5, to=2-5]
\end{tikzcd}\]
d'où une structure d'extension
$$
1 \to \pi_{g, \nu} \to T^{\circ\circ}_{g, \nu + 1} \to T^{\circ\circ}_{g, \nu} \to 1
$$
qui est (à passage à un sous-groupe d'indice $2 \nu$ ! près) la structure d'extension annoncée. Les vérification de compatibilités sont laissées\dots 

On a donc un foncteur quasi-inverse (défini à isomorphisme unique près)
\[\begin{tikzcd}
	{\begin{pmatrix} \text{groupes à lacets de type} \\ g, \nu \quad 2g + \nu \geq 3 \end{pmatrix}} & {\begin{pmatrix} \text{groupes extérieurs à lacets de type}~(g, \nu + 1) \\ \text{avec une ``classe de lacets'' distinguée} \\ \text{(couples}~(\pi', i \in I(\pi'))) \end{pmatrix}}
	\arrow[from=1-1, to=1-2]
\end{tikzcd}\]
Au niveau topossique, quand on a un système $\B_{D^*} \to \B_U$ de type $(g, \nu)$ $(2g + \nu \geq 3)$ et un ``point'' de $\B_U$ i.e. un $\widetilde{U} \in \Ob \pi_1 \B_U$, alors on peut de fa\c{c}on canonique trouver un $\B_{D^*_i}$ ($T$-groupoïde connexe, où $T$ est le module d'orientation) et un système
$$
\B_{D^*} \amalg \B_{D^*_i} \to \B_{U'}
$$
de type $g$, $\nu + 1$ de fa\c{c}on que $(\B_{D^*}, \B_U, \widetilde{U})$ s'en déduise  par l'opération de ``bouchage du trou $D^*_i$''.

Ces constructions sont si fonctorielles qu'elles commutent aux actions de groupe $\Gamma$. Si un groupe $\Gamma$ opère sur un groupe à lacet $\pi$ de type $(g, \nu)$ (pas seulement extérieurement), alors on en déduit une opération \emph{extérieure} de $\Gamma$ sur un $\pi'$ à lacets de type $(g, \nu + 1)$ qui fixe une classe de lacets privilégiée $i \in I$.

Mais alors, dans l'extension correspondante
$$
1 \to \pi' \to E' \to \Gamma \to 1,
$$
choisissons un $L'_i \subset  \pi'$ est soit $E'_i$ le normalisateur dans $L_i$ de $E'$, on trouve
$$
1 \to L'_i (\isom T) \to E'_i \to \Gamma \to 1
$$
(ceci marche sans hypothèse de finitude sur $\Gamma$, le cas universel étant celui où $\Gamma$ est le groupe de Teichmüller de $\pi'$ fixant $i$ ; i.e. $\Autext_{\text{lac}}(\pi', i) \isommap \Aut_{\text{lac}}(\pi)$).

J dit que si $\Gamma$ opère \emph{fidèlement sur $\pi$}, i.e. si son opération extérieure sur $\pi'$ est fidèle, et $\Gamma$ est \emph{fini} alors $\Gamma$ est nécessairement cyclique (cas $\Gamma = \Gamma^\circ$) ou diédral ($\Gamma = \Gamma^\circ$) et que si de plus $\Gamma = \Gamma^\circ$ alors $E_i \isom \mathbf{Z}$ i.e. $\exists !$ isomorphisme $T \isommap E'_i$ tel que $T \xlongrightarrow{\kappa_i} E'_i$ s'identifie à $n \id_T$ (de sorte qu'on trouve $\Gamma = E'_i / nE'_i \isom T \otimes_{\mathbf{Z}} \mathbf{Z}/n\mathbf{Z}$ si $n = $card $\Gamma$\dots)

Changeant de notations, ceci revient au
\vskip .3cm
{
Théorème\footnote{Il y a équivalence si dans le théorème on suppose $(g, \nu)$ anabélien sinon le théorème est un peu plus général.}. --- \it Soit $\Gamma$ un groupe fini opérant fidèlement sur un groupe extérieur à lacet $\pi$ de type $(g, \nu + 1)$ $(\nu \geq 0)$, en laissant fixé un $i \in I(\pi)$. Alors $\Gamma$ est cyclique (si $\Gamma = \Gamma^\circ$) ou diédral (si $\Gamma = \Gamma^\circ$) et dans l'extension correspondante $E$ de $\Gamma$ par $\pi$, si $E_i$ est le normalisateur d'un $L_i$ dans $E$ [de sorte que l'on a une suite exacte $1 \to L_i \to E_i \to \Gamma \to 1$] l'image inverse $E^\circ_i$ de $\Gamma^\circ$ est $\isom \mathbf{Z}$.

Donc si $n = \card \Gamma^\circ = [E^\circ_i : L_i] x \mapsto x^n$ est un isomorphisme $E^\circ_i \isom L_i$, qui compte tenu de $\kappa_i$ donne un isomorphisme $E^\circ_i \isom T$, dont le composé avec $\begin{tikzcd}
	T & {L_i \subset  E^\circ_i}
	\arrow["{\kappa_i}", from=1-1, to=1-2]
	\arrow["\sim"', from=1-1, to=1-2]
\end{tikzcd}$ est $n \id_T$ de sorte que l'on a un isomorphisme canonique 
$$
\Gamma^\circ \isom E^\circ_i / L_i \isom T/nT
$$
(évidemment indépendant du choix de $L_i$\dots).
}
\vskip .3cm
{\bf Démonstration}. Considérons l'extension $E^\circ$ de $\Gamma^\circ$ par $L_i \isom T$ (isomorphe non canoniquement à $\mathbf{Z}$), comme $\Gamma^\circ$ opère trivialement sur $T$ (sans torsion) cette extension (par la suite exacte de cohomologie associée à
$$
0 \to T \to T \otimes_{\mathbf{Z}} \mathbf{Q} \to T \otimes_{\mathbf{Z}} \mathbf{Q}/\mathbf{Z} \to 0)
$$
est canoniquement isomorphe à l'extension définie par un homomorphisme $\Gamma^\circ \to T_n = T/nT$, comme image de l'extension $o \to T \xlongrightarrow{n} T \to T_n \to 0$ de $T_n$ par $T$. Je dis que cet homomorphisme est un isomorphisme (d'où résulteront les autres assertions), ou ce qui revient au même puisque $\Gamma^\circ$ et $T_n$ sont tous deux d'ordre $n$, qu'il est injectif. Rempla\c{c}ant $\Gamma^\circ$ par le noyau de $\Gamma^\circ \to \mathbf{Z}/n\mathbf{Z}$, OPS que l'homomorphisme en question est nul et il faut prouver que cela implique que l'action de $\Gamma^\circ$ est triviale (ce qui, puisque par hypothèse l'action de $\Gamma$ est fidèle, implique $\Gamma^\circ = \{ 1 \}$, OK). Donc on est ramené au
\vskip .3cm
{
Lemme fondamental. --- \it Tout automorphisme direct extérieur $u$ d'ordre fini $n$ d'un groupe à lacets $\pi$, qui fixe une classe de lacets $i$ et est tel que l'extension de $\mathbf{Z}/n\mathbf{Z}$ par $L_i \in i$ définie par $u$ soit triviale est trivial.   
}
\vskip .3cm
L'hypothèse signifie que $u$ se remonte en un automorphisme $u_0$ de $\pi$ qui normalise $L_i$, et qui soit aussi d'ordre $n$ (ou d'ordre fini, cela revient au même compte tenu que $T$ est sans élément d'ordre fini) alors $u_0$ est trivial ; i.e. cela équivaut au corollaire :
\vskip .3cm
{
Corollaire\footnote{N.B. Dans le lemme ou son corollaire, le cas $g = 0$, $\nu + 1 = 1$ ou $\nu + 1 = 2$ est trivial, le cas $\nu + 1 = 3$ ($(g, \nu) = (0, 2)$ abélien !) n'est pas trivial par contre, ni le cas $g = 1$, $\nu + 1 = 1$ (i.e. $(g, \nu) = (1, 0)$ abélien). Pourtant le résultat doit être valable aussi dans ce cas.}. --- \it Tout automorphisme direct d'ordre fini d'un groupe à lacet $\pi$ qui normalise un sous-groupe à lacet $L_i$ (i.e. qui centralise $L_i$) est réduit à l'identité.
}
\vskip .3cm
Pour le démontrer on est ramené aussitôt au cas où $u_0$ est tel que $u^p_0 = 1$ avec $p$ premier, i.e. $u_0$ correspond à une opération au sens strict (pas seulement extérieure) de $\mathbf{Z}/p\mathbf{Z}$ sur $\pi$. Mais (que l'on puisse ou non trouver un tel $p$) considérons le cas où l'on sait que l'opération \emph{extérieure} de $\Gamma = \mathbf{Z}/n\mathbf{Z}$ ($n$ = ordre de $u$) sur $\pi$ se réalise géométriquement par une opération (fidèle) de $\Gamma$ sur $U$ de type $(g, \nu + 1)$ $U = X \textbackslash S$, $S \isom I$, $X$ compacte de genre $g$, le point $s_i$ de $S$ correspondant à un point fixe de $\Gamma$ opérant sur $U$. On exprime l'hypothèse de l'opération [\dots ?] de $\Gamma$ sur $\pi$, centralisant un $L_i$, en disant que l'opération extérieure donne une extension de $\Gamma$ par $L_i = T$ triviale. Mais on sait par ailleurs dans le cas géométrique (et opération fidèle) qu'elle \emph{n'est pas} triviale !

Il suffirait donc pour prouver le lemme fondamental de savoir que toute action extérieure (fidèle) d'un groupe cyclique sur un groupe à lacets de type $(g, \nu + 1)$ est réalisable, et il suffit même de le savoir pour un groupe cyclique d'ordre premier. Or sauf erreur, ce résultat est connu (même pour les groupes résolubles) (comme théorème d'existence de point fixe d'un tel groupe opérant sur l'espace de Teichmüller\dots) de sorte que le lemme fondamental semble démontré. J'ai seulement un doute s'il est démontré dans le cas général d'un $(g, \nu)$, ou seulement pour $g \geq 2$, $\nu = 0$. Mais s'il en est bien ainsi, je pense que (pour $g \geq 2$ tout au moins) on n'en tire par dévissage pour le cas $\nu$ quelconque et le cas $g = 0, 1$ demanderait aussi un traitement à part. Je reviendrai là-dessus par la suite et préfère pour l'instant admettre le ``lemme fondamental'', et examiner des conséquences et corollaires de celui-ci.

Pour une action extérieure fidèle d'un $\Gamma$ fini sur un groupe à lacets \emph{anabélien} $\pi$, correspondant à une extension $E$ de $\Gamma$ par $\pi$ on a donc établi\footnote{i.e. on a établi l'existence d'une application canonique de l'ensemble des éléments d'ordre fini de $\Aut_{\text{lac}}(\pi)$ dans $T \otimes \mathbf{Q}/T$, satisfaisant les conditions examinées précédemment.}.
\begin{enumerate}
    \item[a)] Que les scindages partiels de celle-ci ne peuvent se faire que sur des sous-groupes $\Gamma'$ de $\Gamma$ tels que $\Gamma{'\circ}$ soit cyclique et $\Gamma'$ dihédral si $\Gamma \neq \Gamma^{'\circ}$.
    \item[b)] Pour toute classe de $\pi$-conjugaison de tels scindages partiels, on a un isomorphisme canonique correspondant $\Gamma' \isom T_n( = T/nT)$ où $n =$ card $T'$. Ce sont là des résultats que l'on avait précédemment obtenus pour le cas d'un opération \emph{réalisable}. 
\end{enumerate}
Il n'y avait pas lieu d'ailleurs de se borner au cas anabélien, du moment que l'on suppose $\pi \neq 0$ ) (cas essentiellement vide !) ce qui inclut les cas abéliens $g = 0$, $\nu = 2$ et $g = 1$, $\nu = 0$ pour lesquels un traitement direct est possible, et a déjà été donné essentiellement, ces cas là où l'on part d'une \emph{extension} (pas d'une extension ``extérieure'' i.e. ici d'une action tout court de $\Gamma$ sur $\pi \isom \mathbf{Z}$ ou $\mathbf{Z}^2$) étant toujours réalisables. (N.B. Dans ce cas, l'hypothèse d'une action fidèle est remplacée par celle que l'extension n'est une extension \emph{produit} sous aucun sous-groupe $\Gamma' \subset  \Gamma$ $\Gamma' \neq 1$\dots).

Il reste cependant d'autres résultats [de ?] cas réalisable qu'il faudrait examiner dans le cas général :

[O.P.S. $\Gamma = \Gamma^\circ$ donc $\Gamma$ engendré par un automorphisme direct $u$ ou $\Gamma$ engendré par un anti-isomorphisme d'ordre 2].
\begin{enumerate}
    \item[c)] Si $\Gamma$ est un sous-groupe fini de $\Aut(\pi)$ i.e. un groupe fini opérant \emph{fidèlement} sur $\pi$ a-t-on $\pi^\Gamma = \{ 1 \}$ ?
    \item[d)] Si $\Gamma'$ et $\Gamma''$ sont des sous-groupes finis de $E$ (extension du groupe fini $\Gamma$ par $\pi$ correspondant à une opération extérieure fidèle, respectant l'orientation) tels que $\Gamma' \cap \Gamma'' = \{ 1 \}$, alors $\Gamma'$ et $\Gamma''$ sont-ils contenus dans le $\pi$-conjugué d'un sous groupe fini $\Gamma'''$ de $E$ ?
    
    [i.e. tout sous-groupe section $= 1$ est contenu dans un unique sous-groupe section maximal modulo conjugaison].
    
    N.B. Deviendrait faux en se pla\c{c}ant dans le groupe $\Aut_{\text{lac}}(\pi)$ [\dots?].
    \item[e)] Pour tout sous-groupe $\Gamma'$ de $\Gamma$ l'ensemble des classes de $\pi$-conjugaison de relèvements de $\Gamma'$ sur $E$ est-il fini ?
    
    [OPS $\Gamma' = \Gamma$ cyclique (et $\Gamma = \Gamma^\circ$) ou dihédral sinon].
\end{enumerate}
Dans le cas c), OPS $\Gamma$ cyclique d'ordre premier et c'est OK s'il est acquis qu'une opération extérieure d'un tel groupe sur un $\pi_{g, \nu}$ est réalisable. De même e) est établi si l'on admet que les opérations extérieures de groupes cycliques sur un $\pi \isom \pi_{g, \nu}$ sont réalisables.

Démontrons d). Nous identifions $\Gamma'$ et $\Gamma''$ à des sous-groupes de $\Gamma$, et posons $\Gamma_0 = \Gamma' \cap \Gamma''$. Soit $E^!$ le sous-groupe de $E$ formé des $g \in E$ tel que int$(g) \Gamma_0$ soit $\pi$-conjugué de $\Gamma_0$ et dont l'image dans $\Gamma$ centralise $\Gamma_0$.

On a $E^! \supset \pi$ et $E'$ est donc l'image inverse d'un sous-groupe de $\Gamma^!$ de $\Gamma$ qui centralise $\Gamma_0$ et qui contient $\Gamma'$ et $\Gamma''$ (car $E^!$ contient le centralisateur de $\Gamma_0$ dans $E$, donc $\Gamma'$ et $\Gamma''$). Quitte à remplacer $\Gamma$ par $\Gamma^!$, $E$ par $E^!$, OPS $E = E^!$, $\Gamma = \Gamma^!$ i.e. que $\Gamma_0 \subset $ Centre de $\Gamma$ et que $\Gamma_0 \hookrightarrow E$ invariant modulo $\pi$-conjugaison par $\Gamma$.

On va construire une section de $E$ sur $\Gamma$ tout entier, ainsi. Soit $\widetilde{\Gamma} \subset  E$ le centralisateur de $\Gamma_0$ dans $E$, on a $\widetilde{\Gamma} \cap \pi = (1)$ (car cela signifie $\pi^{\Gamma_0} = (1)$) donc $\widetilde{\Gamma} \to \Gamma$ est injectif, je dis qu'il est surjectif. En effet, soit $\gamma \in \Gamma$, $g \in E$ au-dessus de $\Gamma$, par hypothèse $\exists \alpha \in \pi$ tel que int$(g) \Gamma_0 =$ int$(\alpha)\Gamma_0$.

i.e. OPS int$(g)\Gamma_0 = \Gamma_0$, i.e. $g$ normalise $\Gamma_0$, mais comme $\Gamma_0$ est central dans $\Gamma$ cela signifie que $g$ centralise $\Gamma_0$ i.e. $g \in \widetilde{\Gamma}$. Ainsi $\widetilde{\Gamma} \isom \Gamma$ est un sous-groupe \emph{fini} de $\Aut(\pi)$ contenant $\Gamma'$ et $\Gamma''$ \emph{c.q.f.d.}.

N.B. Si on n'avait pas au début supposé $\Gamma$ fini il serait vrai encore que $\Gamma'$ et $\Gamma''$ engendre un sous-groupe $\widetilde{\Gamma} \subset  \Aut(\pi)$ tel que $\widetilde{\Gamma} \cap \pi = (1)$ mais cela nous fait une belle jambe.















%%%%%%%%%%%%%%%%%%%%%%%%%%%%%%%%%%%%%%%%%%%%%%%%%%%%%%%%%%%%%%%
\chapter*{\S \space 19. --- TOUR DE TEICHMÜLLER}\thispagestyle{empty}
\addcontentsline{toc}{section}{{\bf 19}. Tour de Teichmüller}
\label{sec:19}
\section*{}

Soit $g \in \mathbf{N}$ et $X_g$ une surface compacte connexe orientable de genre $g$. Soit $(a_{g, i})_{i \in \mathbf{N}}$ une suite de points distincts de $X_g$. On pose pour $\nu \in \mathbf{N}$
$$
S_{g, \nu} = \{ a_{g, i} | 0 \leq i \leq \nu - 1 \} \quad (\nu = \card S_{g, \nu})
\leqno{(1)}
$$
$$
U_{g, \nu} = X_g \textbackslash S_{g, \nu} \quad \text{N.B. on a}~a_{g, \nu} \in U_{g, \nu}.
\leqno{(2)}
$$
On a donc $S_{g, 0} = \emptyset$ et $U_{g, 0} = X_g$.

Les $S_{g, \nu}$ forment une suite strictement croissante de parties finies de $X_g$ et les $U_{g, \nu}$ une partie strictement décroissante d'ouverts de $X_g$. On prendra par la suite $U_{g, \nu}$ comme surface orientable type, de type $(g, \nu)$.

(3) Soit $A_g = \Aut (X_g)$ le groupe des automorphismes de $X_g$ muni de la topologie de la convergence uniforme de $u$ et de son inverse. On pose
$$
A_{g, \nu} = \{ u \in A_g | u(S_{g, \nu}) = S_{g, \nu} \}
\leqno{(4)}
$$
on a un isomorphisme canonique (de restriction)\footnote{N.B. C'est sans doute un isomorphisme topologique quand $A_{g, \nu}$ est muni de la topologie induite par $A_g$ et $\Aut (U_{g, \nu})$ de la topologie habituelle de la convergence compacte de $u$ et de son inverse.}
$$
A_{g, \nu} \isommap \Aut(U_{g, \nu})
\leqno{(5)}
$$
on a aussi un morphisme canonique surjectif
$$
A_{g, \nu} \to \Aut(S_{g, \nu}) \isom \gS_\nu
\leqno{(6)}
$$
dont le noyau est noté $A^!_{g, \nu}$
$$
A^!_{g, \nu} = \{ u \in A_g | u(a_{g, i}) = a_{g, i} \forall i \in \{ 0, \dots, \nu-1 \} \}
\leqno{(7)}
$$
d'où la suite exacte :
$$
1 \to A^!_{g, \nu} \to A_{g, \nu} \to \gS\nu \to 1
\leqno{(8)}
$$
Soit $A^\circ_{g, \nu}$ la composante neutre du groupe $A_{g, \nu}$ on a donc :
$$
A^\circ_{g, \nu} (= A^{! \circ}_{g, \nu}) \subset  A^!_{g, \nu}
\leqno{(9)}
$$
Posons
$$
\Gamma_{g, \nu} = A_{g, \nu}/A^\circ_{g, \nu} = \pi_0(A_{g, \nu}) \quad \text{groupe de Teichmüller de type}~g, \nu)
\leqno{(10)}
$$
On pose aussi $\Gamma_g = \Gamma_{g, 0} (= \Gamma^!_{g, 0})$
$$
\Gamma^!_{g, \nu} = A^!_{g, \nu} / A^\circ_{g, \nu}
\leqno{(11)}
$$
la suite exacte (8) donne donc une suite exacte.
$$
1 \to \Gamma^!_{g, \nu} \to \Gamma_{g, \nu} \to \gS_\nu \to 1
\leqno{(12)}
$$
On a des homéomorphismes canoniques :

(13) $A_g/A_{g, \nu} \isom$~ouvert $\Sym^\nu (X_g)^*$ du produit symétrique $(\Sym^\nu (X_g))$ formé des parties finies de $\card \nu (= \textfrak{P}_\nu (X_g))$.

(14) $A_g/A^!_{g, \nu} \isom$~ouvert $(X^\nu_g)^*$ des $\nu$-uples de points distincts $\isom \Mon (I_\nu, X_2)$ (ou $I_\nu = \{ 0, 1, \dots, \nu \}$).

(Cet homéomorphisme respectant les actions naturelles de $\gG_\nu = A_{g, \nu} / A^!_{g, \nu}$).

Les $A^!_{g, \nu}$ pour $\nu$ variable forment une suite décroissante de sous-groupes de $A_g$.
$$
A_g = A^!_{g, 0} \supset A^!_{g, 1} \supset A^!_{g, 2} \supset \dots \supset A^!_{g, \nu} \supset \dots  
\leqno{(15)}
$$
et les homomorphismes correspondants entre espaces homogènes de $A_g$ s'insèrent dans le diagramme commutatif :
\[\begin{tikzcd}
	{A_g/A^!_{g, \nu}} && {\Mon(I_\nu, X)} \\
	&&& {0 \leq \nu' \leq \nu} \\
	{A_g/A^!_{g, \nu}} && {\Mon(I_{\nu'}, X)}
	\arrow[from=1-1, to=3-1]
	\arrow[from=1-1, to=1-3]
	\arrow[from=1-3, to=3-3]
	\arrow[from=3-1, to=3-3]
\end{tikzcd} \leqno{(16)}\]
pour $\nu = 1$ on a $A^!_{g, 1} = A_{g, 1}$ et l'isomorphisme (13) (ou au choix (14)) s'écrit
$$
A_g/A_{g, 1} \isom X_g
\leqno{(17)}
$$
(homéomorphisme compatible avec les actions de $A_g$).

D'ailleurs si à tout $x \in X_g$ on associe son stabilisateur $A_{g, x}$ dans $A_g$ on trouve une application évidemment surjective

(18) $X_g \to$ ensemble des conjugués du sous groupe $A_{g, 1}$ de $A_g (\isom A_g/\Norm_{A_g}(A_{g, 1}))$ 

qui s'identifie via (17) à l'application canonique sur les espaces homogènes
$$
A_g/A_{g, 1} \to A_g/\Norm_{A_g}(A_{g, 1})
$$
déduite de l'inclusion $A_{g, 1} \subset  \Norm_{A_g}(A_{g, 1})$.

On voit de suite que (18) est bijective i.e. que 
$$
A_{g, 1} = \Norm_{A_g}(A_{g, 1})
\leqno{(19)}
$$
Plus généralement pour tout $\nu$ on a 
$$
A_{g, \nu} = \Norm_{A_g} (A_{g, \nu}) = \Norm_{A_g}(A^!_{g, \nu})
\leqno{(20)}
$$
Ce qui signifie que les applications canoniques de $A_g$-ensembles homogènes :
\[\begin{tikzcd}
	& {\textfrak{P}_\nu(X_g)} & {\text{ensemble des conjugués de}~A_{g, \nu}} \\
	{(21)} & S & {\text{stabilisateur}~A_{g, S}~\text{de}~S} \\
	{(22)} & S & {A^!_{g, S}}
	\arrow[from=1-2, to=1-3]
	\arrow[shorten <=10pt, shorten >=10pt, maps to, from=2-2, to=2-3]
	\arrow[shorten <=8pt, shorten >=40pt, maps to, from=3-2, to=3-3]
\end{tikzcd}\]
sont non seulement surjectives mais même \emph{bijectives}. Cela provient du fait que l'on retrouve $S$ en termes de $A_{g, S}$ (ou de $A^!_{g, S}$) :
\[\begin{tikzcd}
	{(23)} & S & {= \{ x \in X_g | u(x) = x \quad \forall u \in A^!_{g, S} \} = X^{(A'_{g, S})}} \\
	{(24)} && {= \{ x \in X_g | A_{g, S} x~\text{fini} \}} \\
	&& {= \{ x \in X_g | A_{g, S} x \neq X_g \}}
\end{tikzcd}\]
On a aussi une application canonique d'espaces homogènes sous $A_g$
\[\begin{tikzcd}
	& {\textfrak{P}_\nu (X_g)} & {\text{Ensemble des conjuguées de}~A^\circ_{g, \nu}~\text{dans}~A_g (\isom A_g/\Norm_{A_g}(A^\circ_{g, \nu}))} \\
	{(25)} & S & {A^\circ_{g, S}}
	\arrow[shorten <=14pt, shorten >=71pt, maps to, from=2-2, to=2-3]
	\arrow[from=1-2, to=1-3]
\end{tikzcd}\]
qui est bijective car on a la relation suivante qui renforce (23) et (24) : $\forall S \in \textfrak{P}_\nu (X_g)$
\[\begin{tikzcd}
	{(26)} & S & {= \{ x \in X_g | A^\circ_{g, S} = \{ x \} \} = X^{A^\circ_{g, S}}_g} \\
	&& {= \{ x \in X_g | A^\circ_{g, S} x~\text{fini} \}} \\
	&& {= \{ x \in X_g | A^\circ_{g, S} x \neq X_g \}}
\end{tikzcd}\]
ainsi pour tout $\nu$
$$
A_{g, \nu} = \Norm_{A_g} (A_{}g, \nu) = \Norm_{A_g}(A^!_{g, \nu}) = \Norm_{A_g}(A^\circ_{g, \nu})
\leqno{(27)}
$$

Soi $G$ un groupe topologique, muni d'une classe de conjugaison $X$ de sous-groupes ; soit $G_1$ dans cette classe. On dit que $(G, X)$ est un couple de Teichmüller de type $g$, s'il existe un isomorphisme de groupes topologiques $G \isom A_g$, transformant $X$ en la classe de conjugaison de $A_{g, 1}$

Il revient au même de dire que $X$ avec sa topologie d'espace homogène sous $G$ $(\isom G/G_1)$ est une surface compacte connexe orientable de genre $g$, et que l'application naturelle
$$
G \to \Aut(X)
\leqno{(28)}
$$
est un homéomorphisme de groupes topologiques.

On voit alors que $(G, X) \mapsto X$ de la catégorie des couples de Teichmüller de type $g$, vers la catégorie des surfaces compactes orientables de genre $g$ est une équivalence de catégorie. Il en résulte que pour un automorphisme $u$ du groupe topologique $G$, $u$ est intérieur si et seulement si $u$ conserve la classe $X$ i.e. si et seulement si $U(G_1)$ est conjugué de $G_2$.

D'ailleurs le centre de $G$ est trivial.

On peut donner une description analogue pour la catégorie des surfaces orientées de genre $g$ à $\nu$ trous (i.e. homéomorphismes à $U_{g, \nu}$) en termes d'un groupe topologique $G_v$ $(\isom A_{g, \nu})$ et d'une classe de conjugaison $U_\nu$ de sous-groupes $H_\nu$ de celui-ci $(\isom A_{g, \nu} \cap A_{g, \nu + 1})$ avec les conditions que $(G_\nu, \{ H_\nu \})$ soit isomorphe à $(A_{g, \nu}, \{ A_{g, \nu} \cap A_{g, \nu + 1} \})$ ou encore que l'homomorphisme continu
$$
G \to \Aut (U_\nu)
\leqno{(29)}
$$
soit un homéomorphise de groupes topologiques. On trouve alors une équivalence entre la catégorie des couples de Teichmüller $(G_\nu, \{ H_\nu \})$ de type $(g, \nu)$, et celle des surfaces orientables de type $(g, \nu)$. On trouve encore que les automorphismes d'un tel $G_\nu$ $(\isom A_{g, \nu})$ qui fixent la classe $U_\nu$ (i.e. qui transforme $H_\nu$ en un conjugué) sont intérieurs et que le centre de $H_\nu$ est $\{ 1 \}$.

En fait $U_\nu$ peut être interprété aussi comme espace homogène sous $G^\circ_\nu$ et pas seulement sous $G_\nu$ ; plus généralement on a que $G^\circ_\nu$ est transitif sur $U_\nu$ et même sur $\textfrak{P}_{\nu'}(U_\nu)$ pour tout $\nu' \in \mathbf{N}^*$.

Revenant à la situation type avec $U_{g, \nu}$, ou trouve :
\[\begin{tikzcd}
	{(30)} && {U_{g, \nu}} & {\isom A_{g, \nu} / A_{g, (\nu, \nu + 1)}~\text{(avec}~A_{g, (\nu, \nu + 1)} = A_{g, \nu} \cap A_{g, \nu + 1})} \\
	&&& {\isom A^\circ_{g, \nu}/B_{g, (\nu, \nu + 1)}}
\end{tikzcd}\]
(avec $B_{g, (\nu, \nu + 1)} = A_{g, \nu + 1} \cap A^\circ_{g, \nu} = \{ u \in A^\circ_{g, \nu}~\text{tel que}~u(a_{g, \nu}) = a_{g, \nu} \}$).

On a ainsi un diagramme d'inclusion de sous-groupes de $A_{g, \nu}$ :
(en posant $B_{g, \nu} = A^\circ_{g, \nu} \cap A^!_{g, \nu + 1}$ et en remarquant que $A^\circ_{g, \nu + 1} = B^\circ_{g, \nu} = (A^!_{g, \nu + 1})^\circ$)
\[\begin{tikzcd}
	{A^\circ_{g, \nu + 1}} & {B_{g, \nu}} & {A^!_{g, \nu + 1}} & {A_{g, (\nu, \nu + 1)}~[} & {A_{g, \nu + 1}]} \\
	& {A^\circ_{g, \nu}} & {A^!_{g, \nu}} & {A_{g, \nu}}
	\arrow[from=1-4, to=2-4]
	\arrow[from=1-3, to=2-3]
	\arrow[from=1-2, to=2-2]
	\arrow["{\Gamma^!_{g, \nu}}"', hook, from=2-2, to=2-3]
	\arrow["{\text{inv.}}", from=2-2, to=2-3]
	\arrow["{\gS_\nu}"', hook, from=2-3, to=2-4]
	\arrow["{\text{inv.}}", from=2-3, to=2-4]
	\arrow[hook, from=1-4, to=1-5]
	\arrow["{\pi_{g, \nu}}"', hook, from=1-1, to=1-2]
	\arrow["{\Gamma^!_{g, \nu}}"', hook, from=1-2, to=1-3]
	\arrow["{\text{inv.}}", from=1-2, to=1-3]
	\arrow["{\gS_\nu}"', hook, from=1-3, to=1-4]
	\arrow["{\text{inv.}}", from=1-3, to=1-4]
\end{tikzcd}\leqno{(31)}\]
où les 2 carrés sont cartésiens et les inclusions horizontales (sauf celle entre crochets) sont des inclusions de sous groupes invariants (invariant même dans le groupe le plus grand $A_{g, \nu}$ (resp. $A_{g, \nu + 1}$)).

(L'égalité $B^\circ_{g, \nu} = (A^!_{g, \nu + 1})^\circ$, (qui précise l'inclusion triviale de $B^\circ_{g, \nu}$ dans $(A^!_{g, \nu + 1})^\circ$ par l'inclusion inverse équivalente à $(A^!_{g, \nu + 1})^\circ \subset  B^\circ_{g, \nu}$) provient de l'inclusion
$$
(A^!_{g, \nu + 1})^\circ \subset  (A^!_{g, \nu})^\circ = A^\circ_{g, \nu}, \quad \text{d'où} \quad A^\circ_{g, \nu + 1} \subset  A_{g, \nu + 1} \cap A^\circ_{g, \nu} = B_{g, \nu}).
$$
On notera dorénavant $A_{g, (\nu, \nu + 1)}$ par $A^\bullet_{g, \nu}$ (comme $A_{g, \nu}$ ponctué par $a_{g, \nu}$).

Les trois inclusions verticales définissent trois espaces homogènes et les homomorphismes d'inclusions entre ceux-ci qui sont \emph{bijectifs}
\[\begin{tikzcd}
	{(A_{g, \nu} \isom) \quad A_{g, \nu}/A^\bullet_{g, \nu}} & {A^!_{g, \nu}/A^!_{g, \nu + 1}} & {A^\circ_{g, \nu}/B_{g, \nu}}
	\arrow["\sim"', from=1-2, to=1-1]
	\arrow["\sim"', from=1-3, to=1-2]
\end{tikzcd}\]
et de même les groupes quotient $\gS_\nu$ et $\Gamma^!_{g, \nu}$ définis par les inclusions de la première ligne son isomorphes par les inclusions verticales aux quotients correspondants dans la deuxième ligne. Ainsi l'extension de groupe 
$$
1 \to \Gamma^!_{g, \nu} \to \Gamma_{g, \nu} \to \gS_\nu \to 0
\leqno{(12)}
$$
peut se déduire indifféremment de la 1$^{\text{ère}}$ ligne ou de la 2$^{\text{ème}}$ ligne de (31) en particulier
$$
\gS_\nu \isom A^\bullet_{g, \nu}/A^!_{g, \nu + 1}
$$
$$
\Gamma_{g, \nu} \isom A^\bullet_{g, \nu} / B_{g, \nu}
\leqno{(32)}
$$
$$
\Gamma^!_{g, \nu} \isom A^!_{g, \nu + 1}/B_{g, \nu}
$$
Ainsi le torseur canonique e groupe $A^\bullet_{g, \nu}$ sur l'espace homogène $U_{g, \nu}$ et celui de groupe $A^!_{g, \nu + 1} (\subset  A·^\bullet_{g, \nu})$ qui lui donne naissance sont l'un et l'autre déduit par extension du groupe structural d'un torseur de groupe $B_{g, \nu}$ (dont la fibre en $x \in U_{g, \nu}$ est l'ensemble des $u \in A^\circ_{g, \nu}$ tel que $u(a_{g, \nu}) = x$). Le revêtement galoisien associé de groupe $B_{g, \nu}/B^\circ_{g, \nu}$ est donc aussi l'espace homogène $A^\circ_{g, \nu} / A^\circ_{g, \nu + 1}$ qui est évidemment connexe et ponctué au dessus de $a_{g, \nu} \in U_{g, \nu}$.

Posons pour $(g, \nu) \neq (1, 0)$ et $(g, \nu) \neq (0, 2)$.
$$
\widetilde{U}_{g, \nu} = A^\circ_{g, \nu} / A^\circ_{g, \nu + 1}
\leqno{(13)}
$$

On a le théorème :
\vskip .3cm
{
Théorème. --- \it $\widetilde{U}_{g, \nu}$ est simplement connexe (et même contractile si $(g, \nu) \neq (0, 0)$) et s'identifie donc au revêtement universel de $U_{g, \nu}$ ponctué en $a_{g, \nu}$.
}
\vskip .3cm
{
Corollaire. --- \it On a un isomorphisme canonique
$$
B_{g, \nu}/B^\circ_{g, \nu} \isom \pi_1(U_{g, \nu} ; a_{g, \nu}) \defeq \pi_{g, \nu}
\leqno{(34)}
$$
}
\vskip .3cm
Démonstration du théorème en termes de choses ``bien connues''.

La suite exacte d'homotopie pour la fibration de $A^\circ_{g, \nu}$ sur $A^\circ_{g, \nu + 1}/A^\circ_{g, \nu + 1} \isom \widetilde{U}_{g, \nu}$ est
\[\begin{tikzcd}
	\dots & {{\pi_{i + 1}(\widetilde{U}_{g, \nu})}} & {{\pi_i(A^\circ_{g, \nu + 1})}} & {{\pi_i(A^\circ_{g, \nu})}} & {{\pi_i(\widetilde{U}_{g, \nu})}} & {{\pi_{i - 1}(A^\circ_{g, \nu + 1})}\dots} \\
	&& {\dots{\pi_1(A^\circ_{g, \nu + 1})}} & {{\pi_1(A^\circ_{g, \nu})}} & {{\pi_1(\widetilde{U}_{g, \nu})}} & 1
	\arrow[from=1-1, to=1-2]
	\arrow[from=1-2, to=1-3]
	\arrow[from=1-3, to=1-4]
	\arrow[from=1-4, to=1-5]
	\arrow[from=1-5, to=1-6]
	\arrow[from=2-3, to=2-4]
	\arrow[from=2-4, to=2-5]
	\arrow[from=2-5, to=2-6]
\end{tikzcd}\leqno{(35)}\]
Elle montre que $\pi_1(\widetilde{U}_{g, \nu})$ est isomorphe au conoyau de $\pi_1(A^\circ_{g, \nu + 1}) \to \pi_1(A^\circ_{g, \nu})$ dont le noyau est un quotient de $\pi_2(\widetilde{U}^\circ_{g, \nu})$\dots 

Si $(g, \nu) \neq (0, 0)$ on sait que $\pi_i(\widetilde{U}_{g, \nu}) = 0$ $\forall i \geq 2$ d'où si $(g, \nu) \neq (0, 0)$ on a 
$$
\pi_i(A^\circ_{g, \nu + 1}) \to \pi_i (A^\circ_{g, \nu})
\leqno{(36)}
$$
est bijectif si $i \geq 2$, injectif à image invariante si $i = 1$ et l'on veut prouver que c'est bijectif pour $i = 1$.

La chose à retenir (?) est celle ci :
\vskip .3cm
{
Théorème (bien connu). --- \it Conditions équivalentes sur le couple $(g, \nu) \in \mathbf{N} \times \mathbf{N}$:
\begin{enumerate}
    \item[a)] $2g + \nu \geq 3$ (i.e. on n'est pas dans les cas : $(1, 0), (0, 0), (0, 1), (0 2)$).
    \item[b)] $\pi_1(U_{g, \nu})$ est non abélien.
    \item[c)] $A^\circ_{g, \nu}$ est simplement connexe.
    \item[d)] $A^\circ_{g, \nu}$ est contractile et $(g, \nu) \neq (0, 1)$.
\end{enumerate}
En tous cas, que ces condition soient ou non vérifies, on a $\pi_i(A^\circ_{g, \nu}) = 0$ si $i \geq 2$, et $(g, \nu) \neq (0, 0)$.
}
\vskip .3cm
(i.e. les $A^\circ_{g, \nu}$ sont des espaces classifiants de groupes discrets, avec la seule exception de $A^\circ_{0, 0} \isom \Aut^\circ \mathbb{S}^2$).

Compte tenu de (36) ce théorème équivaut aux relations :
$$
\pi_i (A^\circ_{g, 0}) = 0 \quad \text{si}~g\geq 2, i \geq 1~\text{(i.e. pour}~g \geq 2 A_{g, 0}~\text{est cotractile)}
\leqno{(37)}
$$
$$
\pi_i (A^\circ_{g, 0}) = 0 \quad \text{pour}~i \geq 1~\text{(i.e.}~A^\circ_{1, 1}~\text{est contractile)}
$$
$$
\pi_i (A^\circ_{0, 1}) = 0 \quad \text{pour}~i \geq 1~\text{(i.e.}~A^\circ_{0, 1}~\text{est contractile)}
$$
$$
\pi_i (A^\circ_{1, 0}) = 0~\text{et}~\pi_i(A^\circ_{0, 1}) = 0~\text{pour}~i \geq 2
\leqno{(38)}
$$
$$
\text{i.e}~A^\circ_{1, 0}~A^\circ_{0, 1}~\text{sont des}~K(\pi, 1)
$$
et elles ont comme conséquences que $\widetilde{U}_{g, \nu}$ est simplement connexe dans les cas anabéliens $(2g + \nu \geq 3)$. Pour savoir ce qu'il en est dans le cas abélien, il faut préciser la structure topologique de $A^\circ_{g, \nu}$ dans les 4 cas ``abéliens'' $2 g + \nu \leq 2$.

Or on a le
\vskip .3cm
{
Corollaire. --- \it Pour que $U = U_{g, \nu}$ soit à $\pi_i$ abélien (i.e. ne satisfasse pas aux conditions équivalentes du théorème) il faut et il suffit que $U$ puisse :
\begin{enumerate}
    \item[] Soit être muni d'une structure de groupe topologique (qui sera nécessairement isomorphe à $\mathbb{U} \times \mathbb{U}$ ($\mathbb{U} = \{ z/|z| = 1 \}$), (cas $(1, 0)$) et $\mathbf{C}^*$ (cas $(0, 2)$ ou $\mathbf{C}$ (cas $(0, 1)$))).
    \item[] Soit d'une structure d'espace homogène sous un groupe topologique (on peut prendre $\SO (3, \mathbf{R})$ ou $\Gl(1, \mathbf{C})$ pour le cas de $\mathbb{S}^2$) par un sous groupe connexe. Dans tous les cas le groupe topologique en question peut se décrire e la fa\c{c}on suivante : on choisit une structure complexe sur $X_g = \widehat{U}_{g, \nu} = \widehat{U}$ (le compactifié pur de $U$) et on prend la structure complexe induite sur $U$ (i.e. on choisit une structure de courbe algébrique (sur $\mathbf{C}$) sur $U$\dots) et on prend $G = \Aut^\circ_{\mathbf{C}} U$ composante neutre du groupe des automorphismes complexes de $U$ i.e. des automorphismes complexes qui invarient $S_\nu = \widehat{U}\textbackslash U$.
    
Ceci posé, l'inclusion $G \hookrightarrow \Aut^\circ U \isom A^\circ_{g, \nu}$ est une équivalence d'homotopie.\footnote{N.B. Le fait que si $U$ est un espace homogène de groupe topologique par un sous groupe connexe alors $\pi_1$ est abélien provient de la suite exacte d'homotopie et du fait que le $\pi_1$ d'un groupe topologique est commutatif. Le réciproque dans le cas des surfaces topologiques est assez remarquable !}\footnote{On a un résultat plus précis : si $k$ est le plus petit entier tel que $(g, \nu + k)$ soit un couple anabélien, alors $\dim G = k$ et $G$ est simplement transitif sur l'ensemble des $k$-uples $(u_1, \dots, u_k)$ de points distincts de $U_{g, \nu}$ d'où il résulte aussitôt que tout élément de $A_{g, \nu}$ s'écrit de fa\c{c}on unique comme un produit $gu$ avec $g \in G$ et $u \in A_{g, \nu, \nu + k}$ ; donc $A_{g, \nu}$ est homéomorphe à $A_{g, (\nu, \nu + k)} \times G$. Comme $G$ est connexe on en conclut $\Gamma_{g, \nu} \isom \Gamma_{g, (\nu, \nu + k)} \subset  \Gamma_{g, \nu + k}$, et $A^\circ_{g, \nu} \isom G \times A^\circ_{g, \nu + k}$ et comme $A^\circ_{g, \nu + k}$ est $\infty$-connexe on e conclut un homotopisme $G \xlongrightarrow{\approx} A^\circ_{g, \nu}$.}    
\end{enumerate}
}
\vskip .3cm
Le corollaire nous donne :
\[\begin{tikzcd}
	{G = \mathbb{U} \times \mathbb{U}} & {\pi_1(A^\circ_{1, 0}) = \mathbf{Z} \times \mathbf{Z}} & {\text{i.e.}} & {A^\circ_{1, 0} \isom K(\mathbf{Z} \times \mathbf{Z}, 1)} \\
	{G = \mathbf{C}^* = \Gl(1, \mathbf{C})} & {\pi_1(A^\circ_{0, 2}) \isom \mathbf{Z}} & {\text{i.e.}} & {A^\circ_{0, 2} \isom K(\mathbf{Z}, 1)} \\
	{G = \Aff(1, \mathbf{C})} & {\pi_1(A^\circ_{0, 1}) \isom \mathbf{Z}} & {\text{i.e.}} & {A^\circ_{0, 1} \isom K(\mathbf{Z}, 1)}
\end{tikzcd}\leqno{(39)}\]
($\Aff(1, \mathbf{C})$ homotope à $\mathbf{C}^*$ par l'inclusion $\Gl(\mathbb{S}^2) \subset  \Aff(1, \mathbf{C})$)
$$
A^\circ_{0, 0} = \Aut (\mathbb{S}^2) \isom \Aut (\mathbb{P}^1_{\mathbf{C}})
$$
est homotope par l'inclusion au sous groupe $\GP (1, \mathbf{C})$ (donc à son sous groupe compacte maximal $\SO (3, \mathbf{R})/\{ \pm 1 \}$) qui \emph{n'est pas} un $K(\pi, 1)$ et dont le $\pi_1$ est $\isom \mathbf{Z}/2\mathbf{Z}$.

On va utiliser ce corollaire pour déterminer la nature de $\widetilde{U}_{g, \nu}$ (comme revêtement de $U_{g, \nu}$) pour les cas ``abéliens''.

Notons pour ceci que l'inclusion $G \hookrightarrow A^\circ_{g, \nu}$ induit une inclusion $G_0 \hookrightarrow B_{g, \nu}$ où
$$
G_0 = G \cap B_{g, \nu} =~\text{stabilisateur de}~a_{g, \nu}~\text{dans}~G
\leqno{(40)}
$$
et comme $G/G_0 \isom A^\circ_{g, \nu}/B_{g, \nu} \isom U_{g, \nu}$ ($G$ étant transitif sur $U_{g, \nu}$), on trouve que le fait que $G \to A^\circ_{g, \nu}$ soit une équivalence d'homotopie équivaut à celle que $G_0 \hookrightarrow B_{g, \nu}$ en soit une. Or dans tous les cas envisagés, $G_0$ est déjà connexe : il est réduit à 1 dans les cas $(1, 0)$ et $(0, 2)$ (donc dans ce cas le corollaire équivaut à la \emph{contractibilité} de $B_{g, \nu}$) et c'est $\isom \mathbf{C}^*$ dans le cas $(0, 1)$, enfin c'est $\Aff (1, \mathbf{C})$ dans le cas $(0, 0)$. Donc on trouve le corollaire :
\vskip .3cm
{
Corollaire. --- \it Dans les cas abéliens $2g + \nu < 3$ (et dans ceux-là seulement bien sûr) $B_{g, \nu}$ est connexe (i.e. $= A^\circ_{g, \nu + 1}$) i.e. $A^\circ_{g, \nu}/A^\circ_{g, \nu + 1} \to U_{g, \nu}$ est un homéomorphisme. Donc $A^\circ/A^\circ_{g, \nu + 1} \to U_{g, \nu}$ fait du premier membre un revêtement universel de $U_{g, \nu}$ si et seulement si $(g, \nu) \neq (0, 2)$ et $\neq (1, 0)$. 
}
\vskip .3cm
Supposons donc $(g, \nu)$ différent de $(0, 1)$ et $(0, 2)$ (moralement on travaille dans le cas anabélien, les cas abéliens inclus $(0, 1)$ et $(0, 0)$ étant sans intérêt pour ce qui va suivre il me semble).

Reprenons le diagramme (31) où dans la 1$^{\text{ère}}$ ligne le groupe quotient $B_{g, \nu}/B^\circ_{g, \nu}$ s'identifie donc à $\pi_{g, \nu} = \pi_1(U_{g, \nu}, a_{g, \nu})$.

On trouve donc :
\vskip .3cm
{
Théorème. --- \it Supposons $(g, \nu) \neq (1, 0)$ et $(0, 2)$, le sous groupe $A^\bullet_{g, \nu}/A^\circ_{g, \nu + 1}$ du groupe de Teichmüller $\Gamma_{g, \nu + 1} = A_{g, \nu + 1}/A^\circ_{g, \nu + 1}$ admet une suite de composition de longueur 3 dont les facteurs successifs sont $\gS_\nu$, $\Gamma^!_{g, \nu}$ et $\pi_{g, \nu}$, déduite d'une structure d'extension sur $\Gamma^!_{g, \nu + 1} = A^!_{g, \nu + 1}/A^\circ_{g, \nu + 1}$
$$
1 \to \pi_{g, \nu} \to \Gamma^!_{g, \nu + 1} \to \Gamma^!_{g, \nu} \to 1
\leqno{(41)}
$$
}
\vskip .3cm
Les opérations extérieures de $\Gamma^!_{g, \nu}$ sur $\pi_{g, \nu}$ sont celles déduites par passage au quotient de celles de $A^!_{g, \nu} \subset  \Aut(U_{g, \nu})$ opérant extérieurement sur $\pi_{g, \nu}$. 
Une autre fa\c{c}on de dire les choses est celle-ci : distinguons dans $\Gamma_{g, \nu + 1}$ opérant sur $S_{\nu + 1}$ le stabilisateur du dernier élément $a_{g, \nu}$, i.e. de son complémentaire ; soit  $\Gamma'_{g, \nu + 1} \isom A^\bullet_{g, \nu} /A^\circ_{g, \nu + 1}$ le sous-groupe d'indice $\nu + 1$ image inverse de $\gS_\nu$ dans $\gG_{\nu + 1}$ par $\Gamma_{g, \nu + 1} \to \gS_{\nu + 1}$. Ceci posé on a un homomorphism évident de groupes discrets, déduit de l'inclusion $A^\circ_{g, \nu} \hookrightarrow A_{g, \nu + 1}$\footnote{N.B. c'est un isomorphisme dans le cas abéliens $2g + \nu \leq 2$.}
$$
\Gamma'_{g, \nu + 1} \to \Gamma_{g, \nu}
\leqno{(42)}
$$
et cet homomorphisme est surjectif (sans condition sur $(g, \nu)$) et pour $(g, \nu) \neq (1, 0)$ et $(0, 1)$ son noyau est canoniquement isomorphe à $\pi_{g, \nu}$ de sorte que l'on a une extension
$$
1 \to \pi_{g, \nu} \to \Gamma'_{g, \nu + 1} \to \Gamma_{g, \nu} \to 1
\leqno{(43)}
$$
où les opérations extérieurs correspondants de $\Gamma_{g, \nu}$ sur $\pi_{g, \nu}$ sont celles déduites par passage au quotient de celles de $A_{g, \nu} \isom \Aut(U_{g, \nu})$ sur $\pi_1(U_{g, \nu}, a_{g, \nu}) = \pi_{g, \nu}$. La structure d'extension (41) est déduite de (43) par image inverse par l'inclusion $\Gamma^!_{g, \nu} \to \Gamma_{g, \nu}$.

Supposons que l'on soit dans le cas anabélien $2 g + \nu \geq 3$. Comme pour toute structure d'extension par un noyau de centre trivial, il y a un homomorphisme canonique dans la structure d'extension canonique associée à $\pi_{g, \nu}$
\[\begin{tikzcd}
	1 & {\pi_{g, \nu}} & {\Gamma'_{g, \nu + 1}} & {\Gamma_{g, \nu}} & 1 \\
	1 & {\pi_{g, \nu}} & {\Aut(\pi_{g, \nu})} & {\Autext(\pi_{g, \nu})} & 1
	\arrow[from=1-1, to=1-2]
	\arrow[from=2-1, to=2-2]
	\arrow[from=1-2, to=1-3]
	\arrow[from=2-2, to=2-3]
	\arrow[from=2-3, to=2-4]
	\arrow[from=1-3, to=1-4]
	\arrow[from=2-4, to=2-5]
	\arrow[from=1-4, to=1-5]
	\arrow[from=1-4, to=2-4]
	\arrow[from=1-3, to=2-3]
	\arrow[from=1-2, to=2-2]
	\arrow[shift right=1, shorten <=3pt, shorten >=2pt, no head, from=1-2, to=2-2]
\end{tikzcd}\leqno{(44)}\]
où la flèche verticale centrale s'obtient en associant à $g \in \Gamma'_{g, \nu + 1}$ la restriction à $\pi_{g, \nu} \subset  \Gamma'_{g, \nu + 1}$ de l'automorphisme intérieur $\text{int}(g)$.
\vskip.3cm
{
Théorème (bien connu). --- \it Dans le cas anabélien $(2g + \nu \geq 3)$\footnote{N.B. En fait ceci reste vrai pour le couple $(1, 0)$ (cf. plus bas) et même dans le cas $(0, 2)$ si on définit ad-hoc la notion de groupe à lacets de type $(0, 2)$.}
$$
\Gamma_{g, \nu} \isommap \Autext_{\text{lac}} (\pi_{g, \nu})
$$
ou ce qui revient au même par (44) :
$$
\Gamma'_{g, \nu + 1} \isommap \Aut_{\text{lac}}(\pi_{g, \nu})
$$
L'image de $\Gamma_{g, \nu} \to \Autext(\pi_{g, \nu})$ est formé des automorphismes extérieurs qui respectent la structure à lacets de $\Gamma_{g, \nu}$ (condition vide si $\nu = 0$ d'ailleurs\dots)
}
\vskip.3cm
{
Corollaire. --- \it Dans le cas anabélien le foncteur $X \mapsto \pi_1(X)$ de la catégorie isotopique (les fèlches dans la catégorie isotopique sont les classes d'isotopie d'homéomorphisme) des surfaces de type $(g, \nu)$ vers la catégorie des groupes \emph{extérieurs} à lacets de type $(g, \nu)$ est une équivalence de catégorie, ainsi que le foncteur $(X, s) \mapsto \pi_1(X, s)$ de la catégorie isotopique des surfaces \emph{ponctuées} de type $(g, \nu)$ vers la catégorie des groupes à lacets de type $(g, \nu)$\footnote{N.B. C'est même une équivalence au niveau des catégories $\infty$-\emph{isotopiques} ce qui exprime seulement le fait que $A^\circ_{g, \nu}$ et $A^\circ_{g, \nu + 1}$ sont contractiles.}.
}
\vskip.3cm
Il reste à examiner dans quelle mesure on peut adapter ces résultats au cas ``abélien''. Rappelons que dans ce cas on a
$$
\Gamma'_{g, \nu + 1} \isommap \Gamma_{g, \nu} \quad \text{si} \quad 2g + \nu \leq 2 \quad \text{(cas ``abélien''))}.
\leqno{(45)}
$$
i.e. $\Gamma_{g, \nu}$ s'identifie au sous-groupe de $\Gamma_{g, \nu + 1}$ formé des éléments qui fixent $a_{g, \nu}$.
\begin{enumerate}
    \item[(1))] Cas $g = 1$, $\nu = 0$ donc $\Gamma_{1, 0} \isom \Gamma'_{1,1} = \Gamma_{1, 1}$. 
\end{enumerate}
Considérons l'homomorphisme canonique
$$
\Gamma_1 = \Gamma_{1, 0} \to \Autext(\pi_{1, 0}) = \Aut(\pi_{1, 0})
\leqno{(46)}
$$
( $\isom \Gl(2, \mathbf{Z})$ quand on a choisi une base de $\pi_{0, 1} \isom \mathbf{Z}^2$), qui correspond à l'homomorphisme
$$
\Gamma_{1, 1} \to \Aut \pi_1 (U_{1, 0}, a_{1, 0}) \quad (\isom \Gl (2, \mathbf{Z}))
\leqno{(46^{\text{bis}})}
$$
déduits l'un et l'autre par passage au quotient à partir des opérations évidentes de $A_1 = A_{1, 0} = \Aut (X_1)$ et de $A_{1, 1} = \Aut (X_1, a_{1, 0})$. Il est immédiat que ce homomorphisme est surjectif mais moins évident que ce soit un isomorphisme i.e. que tout homéomorphisme de $X_1$ qui induit l'identité sur $\pi_1$ soit isotope à l'identité ; c'est pourtant un résultat vrai (et connu).

Dans le cas actuel où $X_1$ s'identifie à l'espace topologique sous-jacent à un groupe topologique $H$ (ce qui est le cas dans tous les cas abéliens sauf celui de $g = 0$, $\nu = 0$ de la 2-sphère), par translation on a un homomorphisme naturel 
$$
H \to A_{g, \nu} \isom \Aut_{\text{top}} H
\leqno{(47)}
$$
permettant d'identifier $H$ à un sous-groupe topologique de $A_{g, \nu}$ et on trove que l'application 
\[\begin{tikzcd}
	{H \times A_{g, \nu + 1}} & {A_{g, \nu}} \\
	{(g, u)} & gu
	\arrow[from=1-1, to=1-2]
	\arrow[shorten <=2pt, shorten >=2pt, maps to, from=2-1, to=2-2]
\end{tikzcd}\leqno{(48)}\]
est un homéomorphisme.

Ceci redonne (45) (puisque $H$ est connexe) et le précise considérablement par
$$
\pi_i (A_{g, \nu + 1}) \isom \pi_i (A_{g, \nu}) \times \pi_i (H)
\leqno{(49)}
$$
Cas abéliens non sphériques i.e. un des trois cas $(1, 0), (0, 1), (0, 2)$.

Dans le cas ``limites'' $(1, 0)$ et $(0, 2)$ (quand $(g, \nu)$ est abélien et $(g, \nu + 1)$ anabélien) comme $A^\circ_{g, \nu} \isom H \times A^\circ_{g, \nu + 1}$ contractile, on retrouve que $H \to A^\circ_{g, \nu}$ est un homotopisme. Notons que dans les cas anvisagés $(1, 0), (0, 2), (0, 1)$, si on choisit une structure complexe sur le compactifié pur $\widehat{X}$ de $X = U_{g, \nu}$ et qu'on choisit $a_{g, \nu}$ comme origine, il y a une structure de groupe $\mathbf{C}$-algébrique unique sur $X$ admettant $a_{g, \nu}$ comme unité et la composante neutre du groupe des automorphismes de la variété algébrique $X$ n'est autre justement que le groupe des translations dans les cas limites $(1, 0)$ et $(0, 2)$.
\begin{enumerate}
    \item[2)] Cas $g = 0$, $\nu = 2$ donc $H = \mathbf{C}^*$, $A_{g, \nu} \isom H \times A^\bullet_{g, \nu + 1}$ i.e. $A_{0, 2} \isom \mathbf{C}^* \times A^\bullet_{0, 3}$ donc
    $$
    \Gamma_{0, 2} \isom \Gamma'_{0, 3}
    $$
    \[\begin{tikzcd}
	{A^\circ_{0, 2}} & {\mathbf{C}^* \quad \text{(équivalence d'homotopie)}}
	\arrow["\approx"', from=1-2, to=1-1]
    \end{tikzcd}\leqno{(50)}\]
\end{enumerate}
Ici $\pi_{0, 2} = \pi_1(U_{0, 2}) \isom \mathbf{Z}$, considérons
$$
\Gamma_{0, 2} \xlongrightarrow{\rho} \Aut(\pi_{0, 2}) \isom \{ \pm 1 \} = \gS_2
\leqno{(51)}
$$
s'identifiant à
$$
\Gamma'_{0, 3} \to \Aut (\pi_{0, 2}) \isom \{ \pm 1 \}
\leqno{(52)}
$$
Cette fis-ci l'homomorphisme (51) \emph{n'est pas bijectif}, il s'identifie à l'homomorphisme canonique.
$$
\Gamma_{0, 2} \xlongrightarrow{\rho} \gS_2
\leqno{(51^{\text{bis}})}
$$
de noyau $\Gamma^!_{0, 1}$ et $\Gamma^!_{0, 2}$ n'est pas réduit à 1, par exemple (si on prend $U_{0, 2} = \mathbf{C}^*$) $z \to \overline{z}$ définit un élément de $\Gamma^!_{0, 2}$ qui n'est pas égal à 1. Notons maintenant l'homomorphisme canonique\footnote{Cet homomorphisme est toujours surjectif. Nous noterons son noyau $\Gamma^+_{g, \nu}$ (et non plus $\Gamma^\circ_{g, \nu}$ !)} (défini sans restriction sur $(g, \nu)$)
$$
\Gamma_{g, \nu} \xlongrightarrow{\chi} \{ \pm 1 \}
\leqno{(53)}
$$
via l'action de $\Gamma_{g, \nu}$ sur le moule d'orientation $T_g = T$ de $U_{g, \nu}$ (qui se définit pour $(g, \nu) \neq (0, 0), (0, 1)$ et $(0, 2)$ en termes de la structure à lacets de $\pi_{g, \nu}$). Dans le cas actuel mettant ensemble $\rho$ et $\chi$ on trouve un homomorphisme
$$
g \mapsto (\rho (g), \chi (g)): \Gamma_{0, 2} \to \gS_2 \times \{ \pm 1 \} \isom \{ \pm 1 \} \times \{ \pm 1 \}
\leqno{(54)}
$$
qui est évidemment surjectif (si $g$ correspond à $z \to z^{-1}$ son image est $(-1, 1)$, s'il correspond à $z \to \overline{z}$, son image est $(1, -1)$). Je dis qu'il est \emph{bijectif}
\[\begin{tikzcd}
	{\Gamma_{0, 2}} & {\gS \times \{ \pm 1 \}}
	\arrow["\sim", from=1-1, to=1-2]
	\arrow["{\rho, \chi}"', from=1-1, to=1-2]
\end{tikzcd}\]
ou ce qui vient au même, que la restriction de $\rho$ (51$^{\text{bis}}$) au noyau $\Gamma^+_{0, 2}$ de $\chi: \Gamma_{0, 2} \mapsto \{ \pm 1 \}$ est un isomorphisme ou ce qui revient au même
\vskip .3cm
{
Théorème (bien connu !). --- \it $\Gamma^{!+}_{0, 2} = 1$, ou encore $\rho^+_{0, 2}: \Gamma^+_{0, 2} \to \gS_2$ est un isomorphisme (ou $\chi^!_{0, 1}: \Gamma^!_{0, 2} \isommap \{ \pm 1 \}$)\footnote{N.B. Ceci suggère que pour une description isotopique de la catégorie des surfaces de types $(0, 2)$ il faut utiliser le couple de $(\pi_1 (X), T)$ où $T$ est le module d'orientation ; itou plus bas pour le cas du type $(0, 1)$ mais alors $\pi_1 = 0$ et il suffit de $T$.}.
}
\vskip .3cm
[N. B. Si on veut un énoncé commun aux deux cas ``abéliens limites'' $(1, 0)$ et $(0, 2)$ on dira que dans ce cas $\Gamma^+_{g, \nu} \to \Aut (\pi)$ est injectif, et a comme image le groupe des automorphismes du $\mathbf{Z}$-module libre $\pi$ de rang 2 ou 1 qui sont de déterminant égal à 1].

Ceci équivaut, modulo l'isomorphisme $\Gamma'_{0, 3} \isom \Gamma_{0, 2}$ au
\vskip .3cm
{
Corollaire. --- \it
    $$
    \begin{cases}
    \Gamma^+_{0, 3} \xlongrightarrow{\rho^+_{0, 3}} \gS \quad \text{est un isomorphisme et} \\
    \Gamma_{0, 3} \xlongrightarrow{(\rho_{0, 3}, \chi_{0, 3})} \gS_3 \times \{ \pm 1 \} \quad \text{aussi}.
    \end{cases}\leqno{(56)}
    $$
}
\vskip .3cm
\begin{enumerate}
    \item[3)] Cas $g = 0$, $\nu = 1$ donc $H \isom \mathbf{C}$ donc 
    \[\begin{tikzcd}
	{A_{0, 1}} & {\mathbf{C} \times A_{0, 2} = \mathbf{C} \times \mathbf{C}^* \times A_{0, 3}}
	\arrow["{\text{homéo}}", from=1-1, to=1-2]
    \end{tikzcd}\leqno{(57)}\]
\end{enumerate}
(avec$\mathbf{C} \times \mathbf{C}^* = \Aff (1, \mathbf{C})$ (qui est simplement transitif sur les couples d'éléments distincts)) en prenant sur $A_{0, 3}$ une structure de groupe algébrique complexe $\isom \mathbf{C}$\footnote{N. B. Comme $A^\circ_{0, 3}$ est contractile cela redonne bien que l'inclusion de $G = \Aff(1, \mathbf{C}) = \Aut_{\mathbf{C}}(X)^\circ$ dans $A_{g, 1}$ est un homotopisme.}.

On a ici $\Gamma_{g, 1} \isom \Gamma'_{g, 2} = \Gamma^!_{g, 1} \isom \{ \pm 1 \}$ i.e. :

\vskip .3cm
{
Corollaire. --- \it On a par la signature un isomorphisme :
\[\begin{tikzcd}
	{\Gamma_{0, 1}} & {\{ \pm 1 \}}
	\arrow["\chi", from=1-1, to=1-2]
	\arrow["\sim"', from=1-1, to=1-2]
\end{tikzcd}\leqno{(58)}\]
}
\vskip .3cm
\begin{enumerate}
    \item[4)] Cas $g = 0$, $\nu = 0$. Ici il n'y a pas de $H$, i.e. de structure de groupe topologique sur $X \isom X_0 = U_{0, 0} \isom \mathbb{S}^2$ mais prenant une structure complexe sur $X$ (d'où $X \isom \mathbb{P}^1_{\mathbf{C}}$) on trouve un groupe $G$ de $\mathbf{C}$-automorphismes, $G = \GP (1, \mathbf{C})$, et
    $$
    G \hookrightarrow A_0 = A_{0, 0}
    \leqno{(59)}
    $$
\end{enumerate}
Prouvons à nouveau que c'est un homotopisme. Comme $G$ est simplement transitif sur les triples de 3 points distincts de $X_0$, on trouve encore un homéomorphisme
\[\begin{tikzcd}
	{G \times A^!_{0, 3}} & {A_0} \\
	{(g, u)} & gu
	\arrow["\sim", from=1-1, to=1-2]
	\arrow[shorten <=2pt, shorten >=2pt, maps to, from=2-1, to=2-2]
\end{tikzcd}\leqno{(60)}\]
d'où
$$
\Gamma_0 \isom \Gamma^!_{0, 3} \isom \{ \pm 1 \}
$$
et compte tenu que la composante neutre $A^\circ_{0, 3}$ de $A^!_{0, 3}$ est contractile on trouve bien encore que (59) est un homotopisme.

{\bf Conclusion commune à tous les cas}.

Il convient d'inclure dans la notion de ``groupe à lacets'' également les quatre cas abéliens, et on le fait de la manière suivante :
\begin{enumerate}
    \item[1)] Type $(1, 0)$ : on n'a pas à compléter la définition générale qui revient à dire ici que $\pi$ est un groupe abélien, $\mathbf{Z}$-module libre de rang 2. Son module d'orientation $T$ peut se définir alors comme $\mathrm{H}_2(\pi, \mathbf{Z})$ ou comme $\bigwedge^2_{\mathbf{Z}}\pi$. La signature d'un automorphisme de $\pi$ est donné par son action su $T$, c'est aussi son déterminant.
    \item[2)] Type $(0, 2)$ : $\pi \isom \mathbf{Z}$. Ici il faut se donner \emph{en plus} de $\pi$, un $\mathbf{Z}$-module inversible $T$ et la structure à lacets est définie par l'ensemble des deux isomorphismes $T \isom \mathbf{Z}$. Un automorphisme de la structure à lacets est donc défini par n'importe quel couple $(u_\pi, u_T)$ d'un automorphisme de $\pi$ et d'un de $T$.  
    \item[3)] Types $(0, 1)$ et $(0, 0)$ qui sont ceux où $\pi_1 = \{ 1 \}$. Par définition, une structure de groupe à lacets de type $(g, \nu)$ est définie ici par la donné d'un ``module d'orientation'' sans plus, qui est un $\mathbf{Z}$-module inversible $T$ ; et il faut donner de plus le type i.e. (sous entendu $g = 0$) le $\nu \in \{ 0, 1 \}$ ; les automorphismes sont ceux de $T$.
\end{enumerate}
Avec ces définitions et pour $g, \nu$ fixés on a le
\vskip .3cm
{
Théorème. --- \it Soit $(g, \nu) \in \mathbf{N} \times \mathbf{N}$, le foncteur naturel de la catégorie isotopique des surfaces de type $(g, \nu)$ vers la catégorie des groupes extérieures à lacets de type $(g, \nu)$ est une équivalence de catégorie. (N.B. Comme flèches on prend les classes d'isomorphisme d'un coté comme de l'autre.)
}
\vskip .3cm

Mais ce théorème n'est pas sûr pleinement satisfaisant dans le cas abélien par exemple. La donnée d'un objet de la catégorie isotopique (explicité par son $\pi_1$ extérieur à lacets) dans le cas d'une action d'un groupe ne permet pas même de récupérer l'extension de ce groupe par le dit $\pi_1$ ! Ce qui cloche, on le sent bien, est le fait que cette équivalence de catégorie soit isotopique (i.e. tient compte des $\pi_0$ des espaces d'homéomorphismes) mais néglige la structure topologique interne des espaces topologiques $\Homeo(X, X')$, en négligeant la structure homotopique des composantes connexes qui, en tant que torseurs sous des groupes topologiques $\isom A^\circ_{g, \nu}$, sont homéomorphes à $A^\circ_{g, \nu}$. C'est justement dans le cas anabélien et dans celui là seulement que ce groupe est $\infty$-connexe. Dans le cas abélien, l'expérience prouve que la description précédente doit être remplacée par celle de Ladegaillerie en termes des
$$
\B_{D^*} \to \B_U \quad \text{ou} \quad \pi_{D^*} \to T_U
$$
elle devient alors (dans tous les cas sauf $(0, 0)$) $\infty$-fidèle (i.e. tient compte des $\pi_i(A^\circ_{g, \nu})$\footnote{N.B. Il n'y a pas à se donner un $k$ ici, $T$ se décrit intrinsèquement à partir de la structure de topos ou de groupoïde.})

Le seul cas entièrement réfractaire (d'importance il faut bien dire !) est celui du type $(0, 0)$ i.e. des surfaces homéomorphes à $\mathbb{S}^2$ le type d'homotopie de 
$$
A^\circ_0 = A^+_0 \isom \GP (1, \mathbf{C}) \equiv \mathbb{S}^3 / \pm 1 \quad \text{(quaternions)}
$$
et en particulier ses $\pi_i$ n'étant pas tous bien connus.

Il faudrait pour commencer expliciter la $\Gr$-catégorie $G$, d'invariant $\pi_0 \isom \{ \pm 1 \}$ et $\pi_1 \isom \pi_1 (A^\circ_0) \isom \{ \pm 1 \}$, déduit de $A_0$ en tuant les groupes d'homotopie supérieurs $\pi_i (i \geq 2)$ de $A_0$ ou, si l'on préfère, du groupe des automorphismes conformes de $\mathbb{P}^1_{\mathbf{C}}$ (extension scindée de $\mathbf{Z}/2\mathbf{Z}$ par $\GP (1, \mathbf{C})$). La 2-catégorie 2-isotopique des surfaces homómorphes à $\mathbb{P}^1_{\mathbf{C}}$ est alors décrite par le 2-groupoïde des 1-torseurs sous la $\Gr$-catégorie précédente\footnote{On peut supposer que $\mathcal{G}$ n'a que deux objets correspondant à l'identité et à la conjugaison complexe de la sphère de Riemann $\widehat{\mathbf{C}}$ ; en fait elle est scindable et même canoniquement scindée\dots}.


















%%%%%%%%%%%%%%%%%%%%%%%%%%%%%%%%%%%%%%%%%%%%%%%%%%%%%%%%%%%%%%%
\chapter*{\S \space 20. --- DIGRESSION : DESCRIPTION 2-ISOTOPIQUE DE LA CATÉGORIE DES SPHÈRES TOPOLOGIQUES}\thispagestyle{empty}
\addcontentsline{toc}{section}{20. Digression: description 2-isotopique de la catégorie des isomorphismes topologiques}
\label{sec:20}
\section*{}

Voici une fa\c{c}on de trouver une description 1-isotopique (et même 2-isotopique, il se trouve) en utilisant un groupe revêtement canonique $\widetilde{A}_0$ de $A_0$, s'insérant dans la suite exacte
$$
1 \to \mu_2 (\mathbf{C}) \to \widetilde{A}_0 \to A_0 \to 1,
\leqno{(62)}
$$
qui contient la suite exacte correspondante de sous-groupes
$$
1 \to \mu_2 \to \widetilde{\Sl}(2, \mathbf{C}) \to \widetilde{\GP}(1, \mathbf{C}) \to 1
\leqno{(63)}
$$
où $\widetilde{\Sl}(2, \mathbf{C})$ est formé des automorphismes semi-linéaires de $\mathbf{C}^2$ (pour un $\mathbf{R}$-automorphisme $\rho$ non précisé, $\id$ ou la conjugaison complexe $\tau$), tels que l'automorphisme $\rho$-linéaire correspondant de $\bigwedge^2 \mathbf{C}^2 \isom \mathbf{C}$ soit l'identité si $\rho = \id$ et soit $\tau: z \mapsto \overline{z}$ si $\rho = \tau$ i.e. indépendamment des cas, on considère un vectoriel unimodulaire $V$ sur $\mathbf{R}$, d'où $V_{\mathbf{C}}$ sur $\mathbf{C}$, avec une restriction de $\bigwedge^2 V_{\mathbf{C}}$ à $\mathbf{R}$, d'où une conjugaison complexe sur $\bigwedge^2 V_{\mathbf{C}}$ et une base de $\bigwedge^2 V_{\mathbf{C}}$ invariante par celle-ci - et on s'intéresse aux automorphismes $\rho$-linéaires ($\rho \in \Aut_{\mathbf{R}} \mathbf{C}$ de $V_{\mathbf{C}}$ qui sur $\bigwedge^2 V_{\mathbf{C}}$ soient $\id$ ou la conjugaison complexe\dots)

$\widetilde{\GP}(1, \mathbf{C})$\footnote{N.B. $\widetilde{\GP}(1, \mathbf{C})$ est aussi le grupe des automorphismes $\mathbf{R}$-linéaires de la $\mathbf{R}$-algèbre $M_2(\mathbf{C})$ (et la classification $\infty$-isotopique des 2-sphères équivaut donc à celle des algèbres simples de rang 4 sur une extension quadratique non précisée de $\mathbf{C}$\dots)} s'identifie au groupe des automorphismes conformes de la sphère de Riemann $P^1_{\mathbf{C}} \isom \mathbb{P}^1(\mathbf{C}^2)$, i.e. des automorphismes de $P^1_{\mathbf{C}}$ comme $\mathbf{R}$-schéma.

On choisit ici $X_0 = P^1_{\mathbf{C}}$, de sorte qu'on a une inclusion canonique 
$$
\widetilde{\GP}(1, \mathbf{C}) \hookrightarrow A_0
$$
qui est (par ce qui précède (59)) une équivalence d'homotopie.

Sauf erreur il en résulte que la classification des extensions du groupe topologique $A_0$ par un groupe discret disons $\mu$ est équivalente (par le foncteur restriction, à la catégorie des extensions correspondantes pour $\GP (1, \mathbf{C})$, ce qui permet de construire $\widetilde{A}_0$, à isomorphisme unique près).

Ceci posé la donnée d'une surface compacte orientée $X$ de genre 0, qui équivaut à celle d'un torseur [$\Isom (\mathbb{P}^1_{\mathbf{C}}, X)$] sous $A_0$, définit 
\begin{enumerate}
    \item[1$^\circ$)] le torseur sous $\{ \pm 1 \}$ qui s'en déduit par $\chi: A_0 \to \{ \pm 1 \}$ (qui est surjectif de noyau $A^\circ_0$) et
    \item[2$^\circ$)] le groupoïde des relèvements de ce $A_0$-torseur en un $\widetilde{A}_0$-torseur, qui est un groupoïde connexe (= gerbe) lié par le lien abélien $\underbrace{\mu_2 (\mathbf{C}) = \{ \pm 1 \}}_{\mu}$, et sur lequel par suite $\Tors(\mu) \isom \Ens_2$ opère (c'est un 1-torseur sous la $\Gr$-catégorie $\Tors(\mu)$).
\end{enumerate}
Associant ainsi à tout $X$ le couple $(\omega, R)$ du $\mu$-torseur associé $\omega$ (i.e. l'ensemble à 2 éléments des deux orientations de $X$, ou ce qui revient au même, le module des orientations de $X$), plus le $\mu$-groupoïde $R$, on trouve un 2-foncteur de la 2-catégorie 2-isotopique des 2-sphères topologiques dans la catégorie des couples $(\omega, R)$, et celui-ci est une équivalence de 2-catégorie.

De ce point de vue on a envie de dire qu'elle est 3-fidèle, mais comme la surjectivité essentielle sur les objets est triviale, il vaut mieux l'appeler 2-fidèle, et même (comme pour la décrire on a fait attention de respecter les $\pi_1$ (en plus des $\pi_0$) des composantes connexes du $A_0$-torseur $\Isom(X_0, X)$, qui sont des $A^\circ_0$-torseurs donc à $\pi_1$ isomorphe à $\mu = \{ \pm 1 \}$, il vaut encore mieux l'appeler 1-fidèle).

Mais en fait elle est même 2-fidèle (en le sens correspondant) grâce au fait que
\[\begin{tikzcd}
	{\pi_2(A^\circ_0)} & {\pi_2(\GP(1, \mathbf{C}))} & {\pi_2(S^3/ \pm 1) \isom \pi_2 S^3 = 0}
	\arrow["\sim"', from=1-3, to=1-2]
	\arrow["\sim"', from=1-2, to=1-1]
\end{tikzcd}\leqno{(65)}\]
Par contre elle n'est pas 3-fidèle, car
\[\begin{tikzcd}
	{\pi_i(A^\circ_0)} & {\pi_i(S^3)} & {\text{pour}~i \geq 2}
	\arrow["\sim"', from=1-2, to=1-1]
\end{tikzcd}\leqno{(66)}\]
et $\pi_0 (S^3) \isom \mathbf{Z} (\neq 0)$ donc $\pi_3 (A^\circ_0) \isom \mathbf{Z} \neq 0$.

On cherche une description $\infty$-isotopique, qui tienne compte des groupes d'homotopie de tous ordres (à déterminer !) de $A^\circ_0$ i.e. de $S^3$ (ou encore $\Sl (2, \mathbf{C})$, ou de $\GP (1, \mathbf{C})$). Au concours !

Mais déjà pour la modeste description proposée à prétention 1 ou 2-isotopique, faute d'avoir écrit les choses avec soin je ne suis pas trop sûr si la description donnée est bien correcte - je suis un peu inquiet du fait que je n'ai pas imposé de relations entre le $\mu$-torseur $w$, et le $\mu$-groupoïde $R$.

Soit plus généralement un groupe topologique $G$ ($\widetilde{A}_0$, ou $\widetilde{\GP}(1, \mathbf{C})$) tel que $G^\circ$ soit simplement connexe : 
$$
\pi_1 (G^\circ) = 0
\leqno{(67)}
$$
Soit
$$
\mu = \Centre (G^\circ), \quad \Gamma = G/G^\circ
\leqno{(68)}
$$
On suppose $\mu$ discret (donc $G^\circ$ s'identifie au groupe revêtement universel de $G^\circ/\mu = H^\circ$ si $H = G/\mu$).

Alors l'extension de groupes topologiques $G$ de $\Gamma$ par $G^\circ$ définit
$$
\Gamma \to \Autext(G^\circ) (\to \Aut(\mu))
\leqno{(69)}
$$
et l'ensemble des classes d'extensions correspondants à une opération extérieure donnée de $\Gamma$ sur $G^\circ$ est de fa\c{c}on naturelle un torseur sous $\mathrm{H}^2(\Gamma, \mu)$ (s'il n'est vide, ce qu'on a exclu par l'hypothèse de départ, en parlant de $G$\dots)

Cette catégorie d'extension est d'ailleurs équivalente à celle des scindages d'une certaine $\Gr$-catégorie définie par Sinh via (69) dont les $\pi_0$ et $\pi_1$ sont respectivement $\Gamma$ est $\mu$ - laquelle est donc ici scindée par la donnée de l'extension $G$. Dans le cas qui nous intéresse, $\Gamma \isom \mu \isom \{ \pm 1 \}$, et cette extension est même scindée, et correspond à une opération d'ordre 2 de $\Gamma$ sur $G^\circ$, dont je doute fort que ce soit un automorphisme intérieur.

En fait, j'ai l'impression que dans les deux cas qui nous occupent ($\widetilde{\GP}(1, \mathbf{C})$ et $\widetilde{A}_0$) que (69) est un \emph{isomorphisme} : $\Gamma = \{ \pm 1 \} \isom \Autext (G^\circ)$.

Ceci posé, la donné d'un torseur sous $H = G/\mu$ définit\footnote{on suppose que $\Gamma$ opère \emph{trivialement} que $\mu$ i.e. $\mu \subset  \Centre (G)$}
\begin{enumerate}
    \item[1$^\circ$)] un torseur sous $\Gamma$, grâce à $H \to \Gamma \isom H/H^\circ$
    \item[2$^\circ$)] le groupoïde des relèvements des relèvements de ce torseur    est un torseur sous $G$, qui est un $\mu$-groupoïde connexe (sur lequel $\mu$ opère).
\end{enumerate}

Si on reprend en termes de fibrés sur un espace de base $S$, on trouve encore sur $S$ (pour tout $H_S$-torseur topologique) un couple $(\omega, R)$ d'un $\Gamma$-torseur et d'une $\mu$-gerbe sur $S$, qui sont décrits, (à isomorphisme et à équivalence près) par $\mathrm{H}^1(S, \Gamma)$ et $\mathrm{H}^2(S, \mu)$ respectivement\footnote{N.B. $\omega$ ne dépend pas du choix de l'extension $G$ de $H$ par ? $\isom \pi_1(H^\circ)$ ; par contre $R$ en dépend (et il faudrait voir comment).}.

On voudrait dégager des conditions sur $G$ et sur $S$ pour que l'application
$$
\mathrm{H}^1(S, H) \to \mathrm{H}^1(S, \Gamma) \times \mathrm{H}^2(S, \mu)
\leqno{(70)}
$$
soit bijective.

Injectivité : si l'image d'un $\xi \in \mathrm{H}^1(S, H)$ dans $\mathrm{H}^2(S, \mu)$ est nulle, $\xi$ se relève en un élément $\tilde{\xi}$ dans $\mathrm{H}^1(S, \widetilde{H} = G)$, dont l'image dans $\mathrm{H}^1(S, \Gamma)$ est la même que celle de $\xi$. Donc si elle est triviale, on voit que $\tilde{\xi}$ provient d'un $\tilde{\xi}_0 \in \mathrm{H}^2(S, G^\circ)$.

Si on sait que $\mathrm{H}^2(S, G^\circ) = \{ 1 \}$, on gagne. Par les marteaux-pilons d'homotopie, \c{c}a marche si $\exists n \in \mathbf{N}$ avec :
$$
\pi_i (G^\circ) = 0~\text{si}~i \leq n~\text{(donc}~\pi_i(B^\circ_G) = 0~\text{si}~i \leq n + 1)
$$
(dans le cas qui nous occupe on peut prendre $n = 2$) et $S$ un CW-complexe de dimension $\leq n + 1$.

Par la surjectivité notons que pour un élément dans $\mathrm{H}^2(S, \mu)$, l'obstraction à ce qu'il se relève en un élément $\xi$ dans $\mathrm{H}^1(S, H)$ est dans $\mathrm{H}^2(S, G)$ [par la suite exacte de cohomologie associée à $1 \to \mu_S \to G_S \to H_S \to 1$].

Il faut exprimer qu'une certaine gerbe liée par $g_s$ est neutre - brr ! Mais partons plutôt de l'élément $\omega$ de $\mathrm{H}^1(S, \Gamma)$, si l'extension de $\Gamma$ par $H^\circ$ est scindée, alors on peut trouver un $\xi_0 \in \mathrm{H}^1(S, H_S)$ qui donne naissance à $\omega$.

Elle a une certaine obstruction $\rho_0$ dans $\usepackage{H}^2(S, \mu)$, et il s'agit de corriger $\rho_0$ en $\xi$, de telle fa\c{c}on que l'obstruction devienne $\rho \in \mathrm{H}^2(S, \mu)$ donnée.

Utilisant $\rho_0$ pour tordre $H_S$ en $H'_S$, et (via l'opération de $H_S = G_S/\mu_S$ sur $G_S$, compte tenu que $\mu_S \subset  \Centre G_S$) pour tordre aussi $G_S$ et $G'_S$, d'où
$$
1 \mu \to G'_S \to H'_S \to 1
$$
on trouve que les $\xi$ ayant même image dans $\mathrm{H}^1(s, \Gamma)$ que $\xi_0$ correspond bijectivemment aux $\xi' \in \mathrm{H}^1(S, H'_S)$.

Pour un tel $\xi'$, soit $S'(\xi') \in \mathrm{H}^2(S, \mu)$ l'obstruction à le relever dans $\mathrm{H}^1(S, G'_S)$ et $S(\xi') = S(\xi)$ l'obstruction à le relever dans $\mathrm{H}^2(S, G_S)$.

Sauf erreur on a 
$$
\delta'(\xi') = \delta (\xi') - \rho_0
$$
i.e. $\delta(\xi') = \deta(\xi) = \delta' (\xi') + \rho_0$ et on veut $\delta(\xi) = \rho$ i.e. $\delta'(\xi') = \rho - \rho_0$ et la question revient encore à la surjectivité de 
$$
\mathrm{H}^1(S, H'_S) \to \mathrm{H}^2(S, \mu)
$$
- je ne m'en tire pas. Il faudrait consulter des gens compétents, comme Giraud ou Larry Breen. On sent qu'il faudrait travailler avec un $\Gr$-champ $\mathbf{H}$ de coefficients, d'invariants, $\underline{\pi}_0 = \Gamma$ et $\underline{\pi}_1 = \mu$, (mais pas nécessairement un champ de Picard !) et $\mathrm{H}^1(S, \mathbf{H}) \isom$ Classes d'applications de $S$ dans $B_{\mathbf{H}}$, qui est un espace connexe avec $\pi_1(B_{\mathbf{H}}) \isom \Gamma$ et $\pi_2(B_{\mathbf{H}}) = \mu$. 

Ici la classe de Postnikov dans $\mathrm{H}^3(\Gamma, \mu)$ est nulle. [mais peut-être n'y a-t-il pas lieu trop sa raccrocher à cette hypothèse, correspondant à l'existence d'une extension $G$ de $H$ par $\pi_1(H^\circ)$ qui redonne l'extension universelle de $H^\circ$ par $\pi_1(H^\circ)$].

On a un homomorphisme $B_H \to B_{\mathbf{H}}$ qui induit un isomorphisme sur les $\pi_i$ pour $i \leq 2$, et pour $i \leq n$ (où $n \geq 2$ est donné) si et seulement si $\pi_i(B_H) = 0$ pour $2 < i \leq n$, i.e. $\pi_i(H) = 0$ pour $2 \leq i \leq n - 1$ (dans le cas qui nous intéresse $H = A_0$, on peut prendre $n = 3$).

Ceci implique que pour tout CW-complexe $S$ de dimension $\leq n$, on a 
$$
\Hot (S, B_H) \isommap \Hot (S, B_{\mathbf{H}})
$$
($\Hot$ désignant les morphismes dans la catégorie homotopique non ponctuée) i.e.
$$
\mathrm{H}^1(S, H) \isommap \mathrm{H}^1(S, \mathbf{H})
$$
Mais si l'invariant de Postnikov-Sinh $k \in \mathrm{H}^3(\Gamma, \mu)$ est nul, (ainsi qu l'action de $\Gamma$ sur $\mu$) alors sauf erreur $B_{\mathbf{H}}$ s'identifie à un produit $K(\Gamma, 1) \times K(\pi, 2)$ (cette ``idetification'' dépendant justement du choix des $\mathrm{H}^2(\pi_1, \pi_2) \isom \mathrm{H}^2(\Gamma, \mu)$ !) et par suite
$$
\mathrm{H}^1(S, \mathbf{H}) \isom \mathrm{H}^1(S, \Gamma) \times \mathrm{H}^2(S, \mu) 
$$
pour tout espace $S$ donc pour $S$ un CW-complexe de dimension $\leq n$
$$
\mathrm{H}^1(S, H) \isommap \mathrm{H}^1(S, \Gamma) \times \mathrm{H}^2(S, \mu)
\leqno{(71)}
$$
Donc la classification des fibrés en sphères topologiques sur un espace topologique $S$, pour un CW-complexe $S$ de dimension $\leq 3$, marche bel et bien.

Bien entendu, le fait qu'on soit obligé ici à se borner à $S$ de dim $\leq 3$ tient au fait que nous n'avons trouvé (via $\mathbf{H}$) qu'une description 2-isotopique (et non $\infty$-isotopique) de la catégorie des 2-sphères topologiques.

Je voudrais reprendre la classification pour un CW-complexe $S$ quelconque, en utilisant le fait que l'inclusion
$$
\widetilde{\GP} (1, \mathbf{C}) \hookrightarrow A_0
$$
est une équivalence d'homotopie, et induit donc une bijection
$$
\mathrm{H}^1(S, \widetilde{\GP}(1, \mathbf{C})) \to \mathrm{H}^1(S, A_0)
$$
et même une $\infty$-équivalence des $\infty$-catégories des torseurs sur $S$ de groupe $\widetilde{\GP}$ (correspondant aux fibrés en sphères conformes, en droites projectives sur une $\mathrm{R}$-algèbre non précisée isomorphe à $\mathbf{C}$) et de groupe $A_0$ - correspondant aux fibrés en sphères sur $S$.

Le premier invariant d'un fibré de groupe $\widetilde{\GP}$ est un torseur $\eta$ sous $(\mathbf{Z}/2\mathbf{Z})_S$, défini (à isomorphisme \emph{non unique près}) par un $\chi \in \mathrm{H}^1(S, \mathbf{Z}/2\usepackage{Z})$ i.e. un revêtement de degré 2. La donnée de $\eta$ revient à la donnée d'une système $T$ d'entiers tordus sur $S$.

Ce torseur servira à tordre $\mathbf{C}$ (via l'opération fidèle de $\mathbf{Z}/2\mathbf{Z}$ sur la $\mathbf{R}$-extension $\mathbf{C}$), d'où un fibré localement constant en extension quadratiques de $\mathbf{R}$, soit $C$, et on [?] $S$ par le faisceau $\underline{C}_S$ (ou $\underline{C}$) des sections continues de $C$ (ce qui est une fa\c{c}on de tordre $\GP(1, \mathbf{C})_S$ ou $\GP(1, C)_S = \GP(1, \underline{C})$) et il s'agit de décrire de fa\c{c}on compréhensible - en passant au besoin aux $n$-isotopiques, pour $n = \dots$ - la catégorie (dépendant du choix de $\chi$ via $\underline{C}$) des algèbres d'Azumaya de rang 4 sur $\underline{C}$, ou encore des fibrés en droites projectives sur $\underline{C}$, ou des torseurs sous $\GP(1, \underline{C})$.

Or utilisant la suite exacte
$$
1 \to \mu_{2 S} \to \Sl(2, \underline{C}_S) \to \GP (1, \underline{C}_S) \to 1
$$
on associe à un objet de la catégorie - disons un torseur sous $\GP (1, \underline{C}_S)$ - la catégorie fibrée (sur des ouverts variables de $S$) de ces relèvements à $\Sl(2, \underline{C}_S)$, qui est une \emph{gerbe} $G$ liée par $\mu_2$. C'est cette gerbe qui est le deuxième invariant complet - plus fin que sa classe d'équivalence qui s'identifie à un $\xi \in \mathrm{H}^2(S, \mu_2) = \mathrm{H}^2(S, \{ \pm 1 \})$ (jouant le rôle d'un groupe de Brauer).

Un isomorphisme de fibrés, d'invariants $(T, G) \isom (T', G')$, définira un isomorphisme $T \isom T'$ (d'où un isomorphisme $\underline{C}_T \isom \underline{C}_{T'}$ d'où un isomorphisme $\GP(1, \underline{C}_T) \isom \GP(1, \underline{C}_{T'})$) et une équivalence de gerbes.

Il faudrait expliciter que pour deux isomorphismes $f$ et $g$ des torseurs, d'où
\[\begin{tikzcd}
	{f_T, g_T: T} & {T'} && {f_G, g_G: G} & {G'}
	\arrow["\sim", shift left=1, from=1-1, to=1-2]
	\arrow["\sim"', shift right=1, from=1-1, to=1-2]
	\arrow["\sim", shift left=1, from=1-4, to=1-5]
	\arrow["\sim"', shift right=1, from=1-4, to=1-5]
\end{tikzcd}\]
et pour toute homotopie $h_t$ $(0 \leq t \leq 1)$ de $f$ à $g$ on trouve $f_T = g_T$ et un isomorphisme $h_*: f_G \isommap g_G$ d'équivalence des gerbes, qui ne dépend que de la classe d'homotopie de cette homotopie.

On trouve ainsi un 2-foncteur de la 2-catégorie des torseurs sur $S$ de groupe $\widetilde{\GP}(1, \mathbf{C})$ vers la 2-catégorie formée des couples $(T, G)$ d'un système d'entiers tordus ($\Leftrightarrow$ d'un torseur sur $S$ de groupe $\mathbf{Z}/2\mathbf{Z}$) et d'une $\mu_2(\mathbf{C})$-gerbe $G$ sur $S$.

Ce 2-foncteur, sauf erreur, est 2-fidèle sous les conditions explicitées plus haut ($\dim S \leq 4$) et est 3-fidèle (i.e. l'application injective
$$
\mathrm{H}^1(S, \widetilde{\GP}) \to \mathrm{H}^1(S, \mathbf{Z}/2\mathbf{Z}) \times \mathrm{H}^2(S, \mu_2)
$$
est surjective) si on a même $\dim S \leq 3$.

On s'attend qu'elle soi 1-fidèle dès que $\dim S \leq 5$, 0-fidèle dès que $\dim S \leq 6$ (??\dots).

N.B. : l'assertion ``0-fidèle'' signifie que si ci-dessus on a deux homotopies $h$, $h'$, de $f$ à $g$, telles que $h_* = h'_*$ alors $h$ et $h'$ sont homotopes.

La condition ``1-fidèle'' signifie 0-fidèle et que tout isomorphisme de $f_G$ à $g_G$ est de la forme $h_*$ (avec $h$ bien déterminé à homotopie près, par la condition précédente de 0-fidélité).

La condition 2-fidèle signifies de plus que pour toute $\phi: T \isommap T'$, $\psi: G \xlongrightarrow{\approx} G'$, il existe un isomorphisme de fibrés $f$ tel que $\phi = f_T$, et un isomorphisme $\psi \isommap f_G$.

Enfin 3-fidèle signifie que de plus, pour tout couple $T$, $G$, $\exists T', G'$ provenant d'un fibré, un isomorphisme $T \isommap T'$ et une équivalence $G \isommap G'$.

Ces notions catégoriques doivent interpréter simplement que, si on regarde $H = \widetilde{\GP}(1, \mathbf{C}) \to \mathbf{H}$ - ou $\mathbf{H}$ est déduite de $H$ en ``tuant les groupes d'homotopie $\pi_i$ pour $i \geq 2$'', de sorte que $\alpha$ induit un isomorphisme des $\pi_i$ pour $i \leq 1$ (et même pour $i \leq n$ si $\pi_i (H) = 0$ pour $1 < i \leq n$) d'où $B_\alpha: B_H \to B_{\mathbf{H}}$\footnote{N.B. $B_{\mathbf{H}}$ est déduite de $B_H$ en tuant les $\pi_i$ pour $i \geq 3$} (induisant un isomorphisme des $\pi_i$ pour $i \leq n + 1$), alors prenant pour un espace donné $S$, l'application correspondante des espaces d'applications continues
$$
\underline{\Cont} (S, B_H) \to \underline{\Cont}(S, B_{\mathbf{H}})
$$
celle-ci induit un isomorphisme pour les $\pi_i$ $(0 \leq i \leq 2)$.

Il semblerait que 0-fidèle est une condition d'injectivité pour les $\pi_2$, 1-fidèle la bijectivité pou les $\pi_2$, l'injectivité pour les $\pi_1$, 2-fidèle la bijectivité $\pi_2$ et $\pi_3$ et l'injectivié pour $\pi_0$, 3-fidèle la bijectivité pour $\pi_0$, $\pi_1$, $\pi_2$.

Comme $\pi_i (\Cont (S, B)) \isom \pi_0 (S \times S^i, B)$, il semblerait qu'on est conduit, pour la 0-fidèlité, de faire l'hypothèse draconienne que $S \times S^2$ de dim $\leq 4$ pour la 1-fidèlite que $S \times S^2$ de dim $\leq 3$ i.e. dim $S \leq 1$ ?, (qui impliquerait alors la 3-fidélité\dots)

Prenons l'analogue arithmétique d'une description (plus ou moins ``fidèle'') de nature ``profinie'' des droites projectives (éventuellement tordues) définies sur un corps de type fini $K$ sur $\mathbf{Q}$ (plus généralement sur un schéma $S$ quelconque), celle-ci forment à priori une catégorie sans plus - un groupoïde (pas nécessairement connexe) - je ne sais pas en faire une 2-catégorie raisonnablement, pour deux isomorphismes $f, g$ ($= f \circ u$) de tels torseurs, définir les homotopismes de $f$ à $g$, i.e. pour une ``forme'' $G$ de $\GP (1)_S$ et une section $u$ de $G$, définir les ``homotopies'' de $u$ à l'identité, ou plutôt une notion qui remplace la notion de classe d'homotopie de telles homotopies. Et serait sans doute un relèvement de $u$ en une section du revêtement simplement connexe $\widetilde{G}$ de $G$ !

Je tiens là quelque chose d'assez amusant, mais que je ne vais pas poursuivre - de toute fa\c{c}on il est clair que la ``description'' des droites projectives (sur un corps de type fini disons) à laquelle on aboutit, n'a rien de fidèle - par même 0-fidèle !

Ainsi tous les automorphismes de $\mathbb{P}^1_K$ provenant de $\Sl (2, K)$ seraient identifiés à l'identité - c'est un peu brutal ! Mais je me rends compte que le travail conceptuel autour du thème ``Brauer'' n'est pas terminé - qu'il y a à comprendre des choses pour l'étude de la catégorie (qui devrait être une {\bf 2}-catégorie !) des Algèbres d'Azumaya de degré \emph{fixé} (ici 4)\dots 




















%%%%%%%%%%%%%%%%%%%%%%%%%%%%%%%%%%%%%%%%%%%%%%%%%%%%%%%%%%%%%%%
\chapter*{\S \space 21. --- LIEN AVEC LES ESPACES DE TEICHMÜLLER}\thispagestyle{empty}
\addcontentsline{toc}{section}{21. Les espaces de Teichmüller}
\label{sec:21}
\section*{}

D'abord un complément lié au diagramme d'inclusion (31) du n$^\circ 19$. Considérons l'inclusion de sous-groupes de $A_g$ :
$$
A^\circ_{g, \nu} \supset B_{g, \nu} \supset A^\circ_{g, \nu + 1} (= B^\circ_{g, \nu})
\leqno{(72)}
$$
donne en divisant par $A^\circ_{g, \nu + 1}$ et dans le cas anabélien la filtration
$$
(A^\circ_{g, \nu}/A^\circ_{g, \nu + 1} =) \widetilde{U}_{g, \nu} \to U_{g, \nu}
\leqno{(73)}
$$
On cherche le plus grand sous-groupe de $A_g$ qui normalise les triples (72), et qui opère donc sur la fibration ci-dessus. Comme les normalisateurs de $A^\circ_{g, \nu}$ et $A^\circ_{g, \nu + 1}$ sont respectivement $A_{g, \nu}$ et $A_{g, \nu + 1}$, le groupe en question doit être contenu dans leur intersection $A^\bullet_{g, \nu}$, lequel normalise également $\B_{g, \nu} = A^\bullet_{g, \nu} \cap A^\circ_{g, \nu}$. C'est donc ce groupe qu'il y a lieu de faire opérer sur la fibration (73). Le sous-groupe formé des éléments de $A^\bullet_{g, \nu}$ dont l'opération sur $U_{g, \nu}$ est isotope à l'identité étant justement $\B_{g, \nu}$, donc c'est le groupe $\Gamma = A^\bullet_{g, \nu}/\B_{g, \nu} \isom \Gamma_{g, \nu}$ qui comme de juste opère à isotopie près. On voudrait décrire le groupe de tous les automorphismes de la fibration topologique $\widetilde{U}_{g, \nu}$ sur $U_{g, \nu}$ qui sera a priori une extension du groupe $A^\circ_{g, \nu} = \Aut (U_{g, \nu})$ par
$$
\pi_{g, \nu} = \pi_1 (U_{g, \nu}, a_{g, \nu}) = \Aut_{U_{g, \nu}}(\widetilde{U}_{g, \nu})
$$
Cette extension est scindé sur $A^\bullet_{g, \nu}$ stabilisateur du point de base $a_{g, \nu}$, et on retrouve ainsi l'opération de $A^\bullet_{g, \nu}$ sur la fibration (73). Or on a déjà une extension $\Gamma'_{g, \nu + 1}$ de $\Gamma_{g, \nu}$ par $\pi_{g, \nu}$ d'où par image inverse par $A_{g, \nu} \to \Gamma_{g, \nu}$ une extension (que je vais notes $\widetilde{A}_{g, \nu}$) de $A_{g, \nu}$ par $\pi_{g, \nu}$ :
$$
\widetilde{A}_{g, \nu} = A_{g, \nu} \times_{\Gamma_{g, \nu}} \Gamma^i_{g, \nu + 1} =
\begin{cases}
\text{groupe quotient du sous-groupe}~A^\natural_{g, \nu}~\text{de} \\
A_{g, \nu} \times A^\bullet_{g, \nu}~\text{formé des couples} \\
(u, v)~\text{tel que}~u \equiv v~\text{mod}.~A^\circ_{g, \nu}, \\
\text{par le sous-groupe}~1 \times A^\circ_{g, \nu + 1}
\end{cases}
$$
donc on a deux (et même trois) structures d'extension sur $\widetilde{A}_{g, \nu}$
\[\begin{tikzcd}
	1 & {\pi_{g, \nu}} & {\widetilde{A}_{g, \nu}} & {A_{g, \nu}} & 1 \\
	1 & {A^\circ_{g, \nu}} & {\widetilde{A}_{g, \nu}} & {\Gamma'_{g, \nu + 1}} & 1 \\
	1 & {A^\circ_{g, \nu} \times \pi_{g, \nu}} & {\widetilde{A}_{g, \nu}} & {\Gamma_{g, \nu}} & 1
	\arrow[from=1-1, to=1-2]
	\arrow[from=1-2, to=1-3]
	\arrow["p", from=1-3, to=1-4]
	\arrow[from=1-4, to=1-5]
	\arrow[from=2-1, to=2-2]
	\arrow[from=3-1, to=3-2]
	\arrow[from=3-2, to=3-3]
	\arrow[from=2-2, to=2-3]
	\arrow[from=2-3, to=2-4]
	\arrow[from=3-3, to=3-4]
	\arrow[from=2-4, to=2-5]
	\arrow[from=3-4, to=3-5]
\end{tikzcd}\leqno{(75)}\]
Je voudrais faire opérer $\widetilde{A}_{g, \nu}$ sur $\widetilde{U}_{g, \nu}$, i.e. faire opérer $H = \widetilde{A}^\natural_{g, \nu}$ avec opération triviale de $A^\circ_{g, \nu + 1}$, et ceci en respectant les conditions suivantes :
\begin{enumerate}
    \item[a)] (Compatibilité avec $\pi_{g, \nu} \to \widetilde{A}_{g, \nu}$). Le couple $(u, v) \in H$ opère sur $\widetilde{U}_{g, \nu}$ par un automorphisme compatible avec l'automorphisme $u$ de $U_{g, \nu}$ (on dira que c'est un $u$-automorphisme). 
    \item[b)] (Compatibilité avec $\pi_{g, \nu} \to \widetilde{A}_{g, \nu}$). Si $u = 1$, [donc $v \in A^\circ_{g, \nu}$ donc (comme $v \in A^\bullet_{g, \nu}$) $u \in \B_{g, \nu}$] alors l'opération de $(u, v) = (1, v)$ sur $\widetilde{U}_{g, \nu}$ n'est autre que celle définie par $v$ mod$\B^\circ_{g, \nu} = A^\circ_{g, \nu + 1}$ qui est un élément de $\pi_{g, \nu} \isom \Aut \widetilde{U}_{g, \nu}/U_{g, \nu}$ ou encore celle définie par translation à droite par $v^{-1}$.
    \item[c)] Compatibilité avec l'opération déjà obtenue plus haut de $A^\bullet_{g, \nu}$ sur $\widetilde{U}_{g, \nu}$ : si $(u, v) \in H$ est tel que $u = v$ (donc $u = v \in A^\bullet_{g, \nu}$) alors $(u, v) = (u, u)$ opère via l'automorphisme intérieur défini par $u$.
    \item[d)] Compatibilité avec $A^\circ_{g, \nu} \to \widetilde{A}_{g, \nu}$ ; l'opération de $A^\circ_{g, \nu}$ est continue (ce qui, joint à a) détermine une opération de fa\c{c}on unique et l'existence à priori d'une telle opération résulte de $\pi_1(A^\circ_{g, \nu}) = 0$. Or tout élément $(u, v)$ de $H$ s'écrit de manière unique comme produit d'un élément $(v, v)$ dans $\delta (A^\bullet_{g, \nu})$, par un élément $(v^{-1}u, 1)$ de $A^\circ_{g, \nu} \times 1$ (le 1$^{\text{er}}$ groupe normalisant le 2$^{\text{ème}}$)\footnote{N.B. L'opération de $A^\circ_{g, \nu} \times 1$ se décrit le plus simplement par \emph{translation} de $A^\circ_{g, \nu}$ sur l'espace homogène $\widetilde{U}_{g, \nu}$ de $A^\circ_{g, \nu}$} et une opération de ce produit semi-direct est donnée bel et bien par la donnée d'opérations des deux groupes facteurs, satisfaisant une condition de compatibilité, qui est vérifiée ici par transport de structure. On a bien défini une opération de $H = \widetilde{A}^\natural_{g, \nu}$ sur la fibration (73) satisfaisant aux contions c) et d) par construction même ; il reste à vérifier a) et b). Or pour a), il suffit de vérifier séparément pour des éléments de $H$ dans $\delta (A^\bullet_{g, \nu})$ et de $A^\circ_{g, \nu} \times 1$, où c'est trivial par construction dans les deux cas. Reste à vérifier b) et à expliciter l'opération d'un élément $(1, v)$, $v \in \B_{g, \nu}$ qu'on écrit comme $(1, v) = (v, v)(v^{-1}, 1)$ d'où $\rho (1, v) = \rho(v, v) \rho(v^{-1}, 1)$ or $\rho(v, v)$ est induit par int$(v)$ et $\rho(v^{-1}, 1)$ par translation à guache $x \to v^{-1}x$ donc le composé opère par $x \mapsto xv^{-1}$.
\end{enumerate}
On suppose maintenant que $X_g$ est muni d'une structure $C^\infty$ et l'on remplace dans les considérations précédentes les groupes d'homéomorphismes par des groupes de difféomorphismes.
\[
\text{Soit}~E_g =\  
\begin{array}{l}
\text{Ensemble des structures conformes sur}~X_g\\
\text{compatibles avec sa structure}~C^\infty
\end{array}
\leqno{(76)}
\]
On voit de suite que $E_g$ est un espace topologique $\infty$-connexe, comme quotient de l'espace $\infty$-connexe (et même convexe) des structures riemaniennes par le groupe $\infty$-connexe des applications $C^\infty$ de $X_g$ dans $\mathbf{R}^{+*}$. Sur $E_g$ le groupe $A_g$ opère mais bien sûr $E_g$ n'est plus un espace homogène. Notons tout de suite
\[
E_g / A_g \isom \  
\begin{array}{l}
\text{Ensemble des classes d'isomorphie de surfaces}\\
\text{conformes compactes orientables de genre}~g.
\end{array}
\leqno{(77)}
\]
Si on choisit une des deux orientations de $X_g$ de sorte que $E_g$ s'identifie à l'ensemble des structures complexes sur $X_g$ (compatible avec sa structure $C^\infty$ et son orientation) alors $E_g/A_g$ s'identifie à l'espace des classes d'isomorphie de courbes complexes (non singulières) connexes compactes de genre $g$, \emph{modulo} passage à la complexe conjuguée. D'autre part  
\[
E_g / A^+_g \isom \  
\begin{array}{l}
\text{Ensemble des classes d'isomorphie de courbes}\\
\mathbf{C}~\text{algébriques lisses connexes de genre}~g.
\end{array}
\leqno{(78)}
\]
plus généralement
\[
E_g / A^\bullet_g \isom \  
\begin{array}{l}
\text{Ensemble des classes d'isomorphie de surfaces compactes conformes}\\
\text{connexes multiponctuées de type}~(g, \nu).
\end{array}
\leqno{(79)}
\]
Si $X_g$ est orientée :
\[
E_g / A_g \isom \  
\begin{array}{l}
\text{Ensemble des classes d'isomorphie de courbes algébriques}\\
\text{[lisses projectives connexes de genre}~g] \\
\text{munies d'une multiponctuation de type}~(g, \nu)
\end{array}
\leqno{(80)}
\]
Ce sont là les ``espaces modulaires grossiers'' (``Coarse moduli'' de Mumford) qu'on peut noter $M^\natural_g$ et $M^\natural_{g, \nu}$ et qui sont justement trop grossiers pour les usages géométriques les plus importants. 

Beaucoup plus intéressant est le quotient
$$
E_g/A^\circ_{g, \nu} = \widetilde{M}_{g, \nu} \quad (\widetilde{M}_g = \widetilde{M}_{g, 0} = E_g/A^\circ_{g})
\leqno{(81)}
$$
sur lequel opère le groupe 
$$
\Gamma_{g, \nu} = A_{g, \nu}/A^\circ_{g, \nu}
$$
par passage au quotient de l'opération de $A_{g, \nu}$.

L'espace $\widetilde{M}_{g, \nu}$ (avec l'opération de $\Gamma_{g, \nu}$ est appelé \emph{l'espace de Teichmüller} de type $g, \nu$). Bien sûr on retrouve $M^\natural_{g, \nu}$ à partir de $\widetilde{M}_{g, \nu}$ et de l'opération de $\Gamma_{g, \nu}$ dessus par
$$
M^\natural_{g, \nu} = \widetilde{M}_{g, \nu}/\Gamma_{g, \nu}
$$
\vskip .3cm
{
Théorème (Teichmüller). --- \it L'espace de Teichmüller $\widetilde{M}_{g, \nu}$ est homéomorphe à $\mathbf{C}^\mu$ où $\mu = 3g - 3 + \nu$ dans le cas anabélien $2g + \nu \geq 3$ et $\mu = 3g - 3 + \nu + \delta$ avec $\delta = $dim${}_{\mathbf{C}} G$ dans le cas général $G$ étant le groupe des automorphismes algébriques d'une $U_{g, \nu}$ complexe (donc $\delta = 1$ dans le cas ``limites'' $(1, 0)$ et $(0, 2)$, et plus généralement $\delta$ augmente de 1 chaque fois pour $g$ fixé et $(g, \nu-1)$ abélien quand on passe de $\nu$ à $\nu - 1$), donc
\begin{enumerate}
    \item[] $\mu = 1$ dans le cas $(1, 0)$
    \item[] $\mu = 0$ dans le cas $(0, 2)$, $(0, 1)$, $(0, 0)$, i.e. $\widetilde{M}_{g, \nu}$ est réduit à 1 point i.e. l'action de $A^\circ_{g, \nu}$ sur $E_g = E_0$ est transitive et ce sont avec le cas anabélien $(0, 3)$ les seuls 4 cas où il en est ainsi.
\end{enumerate}
}
\vskip .3cm
A partis de $\widetilde{M}_{0, 3}$ ou de $\widetilde{M}_{1, 1}$ ou de $\widetilde{M}_{g, 0} = \widetilde{M}_g$ avec $g \geq 1$ quand $\nu$ augmente la ``dimension complexe'' des $\widetilde{M}_{g, \nu}$ augmente d'autant (par contre $\widetilde{M}_{1, 1} \isommap \widetilde{M}_{1, 0}$ est un homéomorphisme). On peut préciser le théorème de Teichmüller ainsi :
\vskip .3cm
{
Corollaire. --- \it Dans le cas \emph{anabélien}, $A^\circ_{g, \nu}$ opère librement sur $E_g$, de sorte que $E_g$ devient un torseur sur $\widetilde{M}_{g, \nu}$, de groupe structural $A^\circ_{g, \nu}$.
}
\vskip .3cm
On en déduit que $\widetilde{M}_{g, \nu}$ joue le rôle \emph{d'espace classifiant} pour $A^\circ_{g, \nu}$ et que
$$
\pi_i (\widetilde{M}_{g, \nu}) \isommap \pi_{i - 1} (A^\circ_{g, \nu})
\leqno{(83)}
$$
et le fait que $\widetilde{M}_{g, \nu}$ soit $\infty$-connexe (contenu dans le théorème de Teichmüller) équivaut à celui que $A^\circ_{g, \nu}$ le soit ce qui est un ``théorème bien connu'' rappelé au n$^\circ$ 19.

Je n'insiste pas ici sur l'interprétation de points de $\widetilde{M}_{g, \nu}$ comme des classes d'isomorphisme de courbes complexes, munies d'une ``rigidification de Teichmüller'' convenable et le point de vue espaces modulaires rigidifiés, qui permet de vérifier à priori que $\widetilde{M}_{g, \nu}$ est muni d'une structure de variété complexe non singulière ; mais déjà le fait que $\widetilde{M}_{g, \nu}$ soit \emph{simplement connexe} (ce qu'on peut exprimer en interprétant $\widetilde{M}_{g, \nu}$ comme revêtement universel d'un \emph{topos modulaire} $U_{g, \nu}$) est un résultat profond qui ne semble pas pouvoir rentrer dans le cadre de la topologie (ou de la topologie différentielle $(C^\infty)$) sans plus\dots 

N.B. Pour prouver que les $\widetilde{M}_{g, \nu}$ sont $\infty$-connexe on est réduit facilement au cas de $M_{g, 0}$ (si $g \geq 2$) ou de $M_{1, 1}$ (cas elliptique ponctué) en utilisant $A^\circ_{g, \nu}/A^\circ_{g, \nu + 1} \isom \widetilde{U}_{g, \nu}$ (qui est $\infty$-connexe) le cas $g = 1$ est d'ailleurs facile et bien compris\dots 

Notons que les inclusions des sous-groupes $A^\circ_{g, \nu + 1} \hookrightarrow A^\circ_{g, \nu} \dots \subset  A^\circ_{g, 0} = A^\circ_g$, définissent une tour d'applications continues :
$$
\dots \to \widetilde{M}_{g, \nu + 1} \to \widetilde{M}_{g, \nu} \to \dots \widetilde{M}_{g, \nu} = \widetilde{M}_g
\leqno{(84)}
$$
$$
\text{où}~\widetilde{M}_{g, \nu + 1} \to \widetilde{M}_{g, \nu}~\text{est pour}~(g, \nu)~\text{anabélien une fibration en fibre}~A^\circ_{g, \nu} \isom \widetilde{U}_{g, \nu}
\leqno{(85)}
$$
[il est donc à fibre contractile et l'$\infty$-connexité de $\widetilde{M}_{g, \nu}$ équivaut à celle de $\widetilde{M}_{g, \nu}$ ; ce qui nous ramène au cas $\widetilde{M}_g$ si $g \geq 2$ de $M_{1, 1}$ et de $M_{0, 3}$ si $g = 1$ ou $0$].

Dans le cas $(g, \nu)$ abélien on trouve
$$
\widetilde{M}_{g, \nu + 1} \isommap \widetilde{M}_{g, \nu}
$$
ce qui signifie que dans ce cas si on a deux structures complexes $\alpha$, $\beta$ sur $X$ qui sont congrues par $u \in A^\circ_{g, \nu}$ (i.e. $\beta = u \alpha$) elles sont mêmes congrues par $A^\circ_{g, \nu + 1}$. En effet si $G$ est la composante neutre du groupe des automorphismes complexes de $U_{g, \nu}$ pour $\beta$ on peut écrire $u = gv$ avec $g \in G$, $v \in A^\circ_{g, \nu + 1}$ donc $\beta = gu \alpha$ donc (comme $g^{-1}\beta = \beta$) $\beta = u \alpha$ c.q.f.d.

Donc $\widetilde{M}_{1, 1} \isom \widetilde{M}_{1, 0}$ et il est immédiat que celui-ci est isomorphe au demi plan de Poincaré. De même $\widetilde{M}_{3, 0} \isom \widetilde{M}_{1, 0} \isom \widetilde{M}_{0, 0}$ et comme $A^\circ_0$ est simplement formé des automorphismes de $X_0 = S^2$ qui conservent l'orientation, on voit que deux structures complexes sur $\mathbb{S}^2$ sont isotopes (puisqu'elles sont isomorphes et qu'un isomorphisme conserve l'orientation). Cela prouve que les $\widetilde{M}_{0, i}$ $(i \leq 3)$, sont réduits à des points ! Ainsi le théorème de Teichmüller est assez évident si $g = 0$ ou 1, c'est le cas $g \geq 2$ qui est profond\dots 

L'espace de Teichmüller (plus généralement tout espace $E$ $\infty$-connexe sur lequel $A_g$ opère de fa\c{c}on que $A^\circ_{g, \nu}$ opère librement) va permettre d'interpréter l'extension canonique $\Gamma'_{g, \nu + 1}$ de $\Gamma_{g, \nu}$ par $\pi_{g, \nu}$ (cas $(g, \nu)$ anabélien) comme groupe fondamental mixte d'un espace (homotope à $U_{g, \nu}$) sur lequel $\Gamma_{g, \nu}$ opère.

(On était ennuyé précédemment, car $\Gamma_{g, \nu}$ n'opérait pas lui même sur $U_{g, \nu}$ mais seulement le groupe $A_{g, \nu}$ dont $\Gamma_{g, \nu}$ est quotient, on avait l'impression que le passage au quotient par $A^\circ_{g, \nu}$ était pourtant inessentiel, car $A^\circ_{g, \nu}$, car $A^\circ_{g, \nu}$ est $\infty$-connexe et d'ailleurs on avait trouvé une extension $\widetilde{A}_{g, \nu}$ de $\Gamma'_{g, \nu + 1}$ par ce même groupe $\infty$-connexe $A^\circ_{g, \nu}$ qui opère sur le revêtement universel, et un homomorphisme surjectif $\widetilde{A}_{g, \nu} \to A_{g, \nu}$ de noyau $\pi_{g, \nu}$ compatible avec cette opération).

Or si $\widetilde{M} = E/A^\circ_{g, \nu}$, $E$ est un $A^\circ_{g, \nu}$ - torseur de base $\widetilde{M}$, qu'on peut utiliser pour tordre $U_{g, \nu}$ sur lequel $A^\circ_{g, \nu}$ opère continuement, on trouve donc un fibré $H$ $(\isom U_{g, \nu} \times E/A^\bullet_{g, \nu}$ opérant diagonalement) sur $\widetilde{M}$, de fibre $U_{g, \nu}$ (c'est pour $E = E_g$ le fibré universel en courbes complexes de type $(g, \nu)$ avec rigidification de Teichmüller\dots). Comme $E$ et $A^\circ_{g, \nu}$ sont $\infty$-connexes, $\widetilde{M}$ aussi, donc l'inclusion d'une fibre dans le fibré est un homotopisme. Or maintenant $\Gamma_{g, \nu}$ opère sur $X$ (de fa\c{c}on compatible avec son opération sur $\widetilde{M}$), d'où la construction d'une extension de $\Gamma_{g, \nu}$ par $\pi_1(H) = \pi_1(U_{g, \nu}) = \pi_{g, \nu}$.

On a fait tout ce qu'il fallait pour prouver que c'est bien essentiellement $\Gamma'_{g, \nu}$\dots








%%%%%%%%%%%%%%%%%%%%%%%%%%%%%%%%%%%%%%%%%%%%%%%%%%%%%%%%%%%%%%%
\chapter*{\S \space 23. --- RETOUR SUR LES SURFACES À GROUPES D'OPÉRATORS}\thispagestyle{empty}
\addcontentsline{toc}{section}{23. Retour sur les surfaces à groupes (finis) d'opérateurs (``mise en équations'' du problème)}
\label{sec:23}
\section*{}

Notre point de vue sera non celui des groupes extérieurs à lacets mais celui des topos multigaloisiens et morphismes entre ceux-ci, plus souple, on l'a vu. Avant de faire intervenir des opérations de groupes, introduisons la catégorie des surfaces $U$ orientables (NB pas orientées !) \emph{admissibles} (i.e. de la forme $X \textbackslash S$ où $X$ est compacte, $S$ fini et toute composante connexe de $X$ de genre zéro rencontre $S$) comme dans Ladegaillerie (en prenant comme 2-morphismes entre homéomorphismes $\begin{tikzcd}
	{f,g: U} & {U'}
	\arrow["\sim", shift left=2, from=1-1, to=1-2]
	\arrow["\sim"', shift right=2, from=1-1, to=1-2]
\end{tikzcd}$ les classes d'homotopie de chemins de $f$ à $g$ dans l'espace $\Isom(U, U')$) (qui, avec les hypothèses faites a comme composantes connexes des torseurs sous de produits de groupes $A^\circ_{g, \nu}$ (avec $(g, \nu) = (0, 0)$)) donc ce sont des $K(\pi, 1)$.

On trouve un 2-foncteur de la 2-catégorie précédente dans celle des morphismes de topos multigaloisiens (ou, si on préfère, dans celle des morphismes de groupoïdes) notés
$$
\B_{D^*} \xlongrightarrow{\rho} \B_U,
$$
où pour deux objets de la 2-catégorie $\B_{D^*} \to \B_U$ et $\B_{D^{'*}} \to \B_{U'}$, la catégorie des morphismes de l'une dans l'autre est formée des diagrammes essentiellement commutatifs de morphismes de topos (ou de morphismes de groupoïde).
\[\begin{tikzcd}
	{\B_{D^{'*}}} && {\B_{U'}} \\
	\\
	{\B_{D^*}} && {\B_U}
	\arrow["{\rho'}", from=1-1, to=1-3]
	\arrow["{f^{D^*}}"', from=1-1, to=3-1]
	\arrow[""{name=0, anchor=center, inner sep=0}, "{f_U}", from=1-3, to=3-3]
	\arrow[""{name=1, anchor=center, inner sep=0}, "\rho"', from=3-1, to=3-3]
	\arrow["\alpha", shorten <=7pt, shorten >=7pt, from=1, to=0]
	\arrow["\sim"', shorten <=7pt, shorten >=7pt, from=1, to=0]
\end{tikzcd}\]
où $f^{D^*}$ et $f_U$ sont des morphismes et $\alpha: \rho \circ f_{D^*} \isommap f_U \circ \rho'$ une donnée de commutativité. Les morphismes entre un $h$ et un $g$ étant définis ad hoc\dots

Il peut être utile de considérer les objets de la 2-catégorie comme des topos cofibrés (ou des groupoïdes cofibrés) sur la catégorie ``flèche'' $D^* \to U$ ayant deux objets $D^*$ et $U$ et une seule flèche non identique $D^* \to U$. En termes d'un foncteur entre groupoïdes, $\Pi_{D^*} \to \Pi_U$ la catégorie cofibrée (``en groupoïdes'') associée $\Pi$ est définie par : $\Ob \Pi = \Ob \Pi_{D^*} \amalg \Ob \Pi_U$.

Les flèches entre deux objets de $\Pi_{D^*}$, ou deux objets de $\Pi_U$, étant celles de $\Pi_{D^*}$, resp. de $\Pi_U$ et les flèches de $\widetilde{D}^* \in \Ob \Pi_{D^*}$ dans $\widetilde{U} \in \Ob \Pi_U$ étant les isomorphismes $\rho_!(\widetilde{D}) \isom \widetilde{U}$ dans $\Pi_U$.

On a un foncteur canonique $\Pi \to \Delta_1$ qui est ``cofibrant'' et pour lequel toute flèche de $\Pi$ est cocartésienne. [Quand on préfère travailler avec les topos et qu'on rapère les morphismes $\rho$ de topos par les foncteurs images inverses $\rho^*$ (N.B. on a une suite de trois foncteurs adjoints $\rho_! \rho^* \rho_*$) alors $\B_U \xlongrightarrow{\rho^*} \B_{D^*}$ est décrit par une catégorie \emph{fibrée} en topos sur $\Delta$ i.e. une catégorie fibrée $\B$ telle que les catégories fibres soient des topos, et le foncteur de changement de base soit exact à guache et commute aux limites inductives quelconques.]

Soit $\mathscr{S}$ la 2-catégorie des surfaces admissibles, $\mathscr{M}$ celle des morphismes de topos multigaloisiens (sans condition). On a un 2-foncteur de 2-catégories
$$
\mathscr{S} \to \mathscr{M}
$$
et on sait décrire l'image 2-essentielle par la condition ``structure à lacets''\footnote{N.B. On exclut par exemple $\B_{D^*} \isom$ ``topos vide'' $\B_U \isom$ ``topos ponctuel'' i.e. $\pi_{D^*} = \emptyset$ et $\pi_U \isom$ catégorie ponctuelle.} qui définit une sous 2-catégorie pleine de $\mathscr{M}$ soit $\mathscr{M}_{\text{lac}}$. Par ailleurs on sait que le foncteur est 2-fidèle, donc induit une 2-équivalence de 2-catégories\footnote{Il n'y a pas à se donner une structure supplémentaire dans le cas $\B_{D^*} \isom$ ``topos vide'' sur $\B_U$ - à savoir un isomorphisme $T \isom \mathrm{H}^2(M, \mathbf{Z})$ pour toute composante connexe - car on ne suppose pas que l'on travailler avec des structures orientées ! (Dans l'analogue arithmétique il n'en sera plus de   même bien sûr\dots)}
$$
\mathscr{S} \to \mathscr{M}_{\text{lac}}
$$
Si maintenant $\Gamma$ est un groupe et si on considère des surfaces admissibles avec action de $\Gamma$, elles définissent des topos (ou groupoïdes) $\B_{D^*}$, $\B_U$ avec opération de $\Gamma$ dessus [ou des topos cofibrés sur le groupoïde [$\Gamma$] défini par $\Gamma$] et des morphismes entre ceux-ci commutant à $\Gamma$. On peut considérer une telle donnée comme celle d'un topos multigaloisien (ou groupoïde) cofibré sur $\Delta \times [\Gamma]$. Mais la donnée d'un tel morphisme de topos multigaloisiens avec opération de $\Gamma$, équivaut à celle de morphismes de topos multigaloisiens :
$$
\B_{D^*, \Gamma} \to \B_{U, \Gamma} \to \B_{\Gamma}
$$
ceci semble un point de vue conceptuellement commode, notamment quand on fait varier $\Gamma$.

On peut remplacer $\Gamma$ par un groupoïde $\Pi$ (jouant le rôle d'un groupoïde fondamental) et se proposer de décrire la 2-catégorie $\mathscr{R}\mathscr{S}$ des foncteurs de $\Pi$ dans la catégorie des surfaces admissibles, [ou encore la 2-catégorie des ``surfaces fibrées admissibles'' sur le topos multigaloisiens $\widehat{\Pi}^\circ$ correspondant à $\Pi$]. On la décrit par la 2-catégorie $\mathscr{R}\textfrak{M}$ des diagrammes de topos multigaloisiens (ou de groupoïdes)
$$
\B_{D^*, \Pi} \to \B_{U, \Pi} \to \B_{\Pi}
\leqno{(1)}
$$
donc finalement on a un 2-foncteur entre deux 2-catégories : celle des représentations de groupoïdes dans la catégorie des surfaces admissibles et celle des diagrammes de topos multigaloisiens (ou de groupoïdes) (1).

Prenant pour toute composante connexe $\B_{\Pi_i}$ de $\B_{\Pi}$ un revêtement universel $\widetilde{B}_{\Pi_i}$, et prenant les produits fibrés, on récupère comme de juste un diagramme
$$
\B_{D^*_i} \xlongrightarrow{\rho_i} \B_{U_i} \to \widetilde{B}_{\Pi} \isom~\text{topos ponctuel}
\leqno{(2)}
$$
et une action de $\pi_i = \Aut_{\B_{\Pi_i}}\widetilde{B}_{\Pi_i}$ dessus ; et la famille de ceux-ci pour $i$ variable permet de récupérer la situation complète\dots 

On dira que le diagramme (1) est ``admissible'', si les $\rho_i$ dans (2) sont admissibles ; d'où une 2-catégorie $\mathscr{R}\mathscr{M}_{\text{lac}}$, et un 2-foncteur
$$
\mathscr{R}\mathscr{S} \to \mathscr{R}\mathscr{M}_{\text{lac}}
$$
On regarde la sous 2-catégorie pleine obtenue en se limitant aux $\Pi$ dont les groupes fondamentaux sont finis, d'où un 2-foncteur induit
$$
\mathscr{R}_f\mathscr{S} \to \mathscr{R}_f\mathscr{M}_{\text{lac}}
$$
et on se propose d'étudier ses propriétés de fidélité.

Je conjecture que c'est une équivalence de 2-catégorie i.e. qu'il est 3-fidèle.

Bien entendu on est ramené quand même au cas de groupoïdes connexes $\Pi_f$ définis par un groupe (fini s'il le faut) et on est aussi ramené par des arguments essentiellement triviaux à regarder le cas de diagrammes $\B_{D^*} \to \B_U$ avec $\B_U$ connexe.

Du côté géométrique la situation serait donnée par un $U = X \textbackslash S$ de type $(g, \nu)$, $(g, \nu) = (0, 0)$ et une opération de $\Gamma$ dessus. OPS $U = U_{g, \nu}$ donc on donne $\Gamma \to A_{g, \nu}$. Pour la question de $i$-fidélité avec $i \leq 2$ OPS qu'il s'agit du même groupe $\Gamma$ qui opère\dots
\begin{enumerate}
    \item[a)] {\bf 0-fidélité}. Soit $U$, $U'$ avec opérations de $\Gamma$ dessus $k, g:U \isommap U'$ commutant à $\Gamma$ et $\alpha$, $\beta$ deux homotopies de $f$ à $g$ i.e. : deux chemins de $f$ à $g$ dans $\Isom_\Gamma (U, U')$. On suppose que dans la description topossique
    \[\begin{tikzcd}
	{\B_{D^*, \Gamma}} && {\B_{U, \Gamma}} && {\B_{\Gamma}} \\
	\\
	{\B_{D^{'*}, \Gamma}} && {\B_{U', \Gamma}} && {\B_{\Gamma}}
	\arrow["{=}", from=1-5, to=3-5]
	\arrow[from=3-3, to=3-5]
	\arrow[from=1-3, to=1-5]
	\arrow[from=1-1, to=1-3]
	\arrow[from=3-1, to=3-3]
	\arrow["{g_{U, \Gamma}}", shift left=2, from=1-3, to=3-3]
	\arrow["{f_{U, \Gamma}}"', from=1-3, to=3-3]
	\arrow["{g_{D^*, \Gamma}}", shift left=1, from=1-1, to=3-1]
	\arrow["{f_{D^*, \Gamma}}"', shift right=1, from=1-1, to=3-1]
    \end{tikzcd}\]
    $\alpha_*^{D^*} = \beta_*^{D^*}: f_{D^*} \to g_{D^*}$ et $\alpha_*^U: \beta_*^U: f_U \to g_U$.
\end{enumerate}  
A prouver que $\alpha$ et $\beta$ sont homotopes. OPS $g = \Id$, $U = U_{g, \nu}$, donc $f \in A^{\Gamma}_{g, \nu}$, et on a deux chemins $\alpha, \beta$ de 1 à $f$ dans l'espace $A^\Gamma_{g, \nu}$. On suppose que les deux isomorphismes correspondants entre identité de $\Pi_{D^*}$ et $f_{D^*, \Gamma}: \Pi_{D^*_{S_\nu}, \Gamma} \to \Pi_{D^*_{S_\nu}, \Gamma}$ d'une part entre identité de $\pi_{U, \Gamma}$ et $f_{U, \Gamma}$ d'autre part sont les mêmes. On veut prouver que $\alpha$ et $\beta$ sont homotopes. Bien sur l'hypothèse sur $\alpha$, $\beta$ relative à l'action de $\Gamma$ est vérifiée a fortiori en se restreignant à un groupe plus petit, par exemple, $\Gamma' = 1$, et en fait on voit qu'elle est équivalente pour $\Gamma$ et pour son sous-groupe 1. Le résultat déjà connu (Ladegaillerie) pour $\Gamma = 1$, montre que ceci signifie que si deux chemins dans $A^\Gamma = (A^\Gamma_{g, \nu})$ de 1 à $f$ sont homotopes dans $A$ ils le sont dans $A^\Gamma$ ou encore que
$$
\boxed{\pi_1(A^\Gamma) \to \pi_1(A) \quad \text{est injectif.}}
$$
Dans le cas $\pi_1(A) = 0$ (cas $(g, \nu)$ anabélien) cela revient donc à prouver que $\pi_1(A^\Gamma) = 0$.

\begin{enumerate}
    \item[b)] {\bf 1 fidélité}. Cela signifie (en plus de la 0-fidélité) que tout isomorphisme entre $\B_f$ et $\B_g$ provient d'un chemin de $f$ à $g$. Avec la réduction précédente OPS $g = \id$ donc on a $f \in A^\Gamma$ d'où $f: (\B_{D^*} \to \b_U) \to (\B_{D^*} \to \B_U)$ (respectant $\Gamma$) et on a un isomorphisme avec l'identité.
\end{enumerate}
Soit $A$ un groupe topologique, d'où deux invariants :
$$
\pi_0 = \pi_0 (A), \quad \pi_1 = \pi_1(A)
$$
le premier est un groupe (pas nécessairement commutatif) le deuxième est un groupe commutatif sur lequel $\pi_0$ opère. On définit une $\Gr$-catégorie $\underline{A}$ d'invariant $\pi_0$, $\pi_1$ et correspondant à cette opération de $\pi_0$ sur $\pi_1$ en prenant comme catégorie sous jacente le groupoïde fondamental (naïf) de $A$ et comme foncteur de composition celui de $A \times A \to A$ (l'associativité de $\underline{A} = \Pi_1 A$ est stricte\dots).

Ceci posé, tout torseur sur $A$ définit un 1-torseur sous la $\Gr$-catégorie $\underline{A}$. Plus généralement pour tout espace topologique $S$ (ou tout topos qui est localement un espace topologique) tout torseur sur $S$ de groupe $A_S$ définit un $\underline{A}_S$-champ sur $S$ qui est un champ en $\underline{A}_S$-torseurs. Si $\pi_i (A) = 0$ pour $i \geq 2$, alors on trouve ainsi pour tout $n \in \mathbf{N}$, $n \geq 2$ une \emph{$n$-équivalence} entre la $n$-catégorie des torseurs sur $S$ de groupe $A_S$, et la $n$-catégorie déduite de la 2-catégorie des $\underline{A}_S$-torseurs en la prolongeant de fa\c{c}on discrète\dots 

Mais considérons un groupe $\Gamma$, et considérons la classification des $A_{\B_\Gamma}$-torseurs sur le topos classifiant - i.e. celle des $A$-torseurs avec une opération de $\Gamma$ dessus (commutant à l'action de $A$). Ces objets forment une 2-catégorie dont les composantes connexes correspondent aux classes de conjugaison d'homomorphismes de $\Gamma$ dans $A$. Si on a un homomorphisme $\Gamma \to A$, il définit une action sur le torseur trivial $1_A$. Si $u$, $v$ sont deux homomorphismes, les isomorphismes de $(1_A, u)$ avec $(1_A, v)$ correspondent aux éléments de l'ensemble :
$$
\Transp (u, v) = \{ g \in A | v = \text{int}(g) \circ u \}
$$
Mais on fera une catégorie 0-isotopique en rempla\c{c}ant par son groupoïde fondamental $\Pi_1$ $\Transp (u, v)$. 

Donc un isomorphisme de $u$ et $v$ est encore un point de $\Transp (u, v)$ ; mai si on a deux tels isomorphismes $f, g$ les isomorphismes $f \isom g$ sont les classes de chemins dans $\Transp (u, v)$ de $f$ à $g$. Ainsi la catégorie $\underline{\Aut}(u)$ est équivalente à
$$
\Pi_1 \Transp (u, v) \quad (\Transp (u, u) = A^u) \quad \underline{\Aut}(u) = \Pi_1 \Transp (u)
$$
et la catégorie $\Isom (u, v)$ est soit vide (si $u$, $v$ ne sont pas conjugués) soit un torseur sous $\underline{\Aut}(u)$.

Mais pour tout torseur $P$ sous $A$ sur lequel $\Gamma$ opère le 1-torseur $\Pi_1 P$ sous $\Pi_1 A = \underline{U}$ est muni d'opérations de $\Gamma$ d'où un $(\Gamma, \underline{A})$ torseur. On trouve ainsi un foncteur de 2-catégorie :
\[\begin{tikzcd}
	{\begin{pmatrix} \text{2-catégorie des}~(\Gamma, A)-\text{torseurs} \\ \text{[ou encore des homomorphismes}~\Gamma \to A] \end{pmatrix}} && {\begin{pmatrix} \text{2-catégorie des 1-torseurs sous} \\ \underline{A}~\text{avec opération de}~\Gamma~\text{dessus} \end{pmatrix}}
	\arrow[from=1-1, to=1-3]
\end{tikzcd}\]
 
 En somme, on vient de répéter sur le topos classifiant $\B_\Gamma$ la construction faite plus haut pour un espace topologique $S$. On aimerait encore exprimer des conditions pour que ce 2-foncteur soient une 2-équivalence.

Pour ceci, il conviendrait d'abord d'avoir une compréhension de la classification des classes d'équivalence d'objets de la deuxième catégorie qui sont eux de nature purement algébrique, en terme de la $\Gr$-catégorie $\underline{A}$ et de $\Gamma$. Quitte à se borner à des torseurs triviaux sous $\underline{A}$ (ce qui est licite) il faudrait expliciter ce qui signifie que $\Gamma$ opère sur le torseur trivial. On constate que cea signifie qu'on a un homomorphisme de $\Gr$-catégorie de la $\Gr$-catégorie discrète définie par $\Gamma$ dans $\underline{A}$. Donc la 2-catégorie envisagée est celle dont les objets sont les homomorphismes de $\Gr$-catégories 
$$
\Gamma \xlongrightarrow{u} \underline{A}
$$
et pour deux tels morphismes $u$ et $v$ il faut définir (non seulement un ensemble $\Hom(u, v)$ mais) une catégorie $\underline{\Hom}(u, v)$ comme la sous-catégorie $\underline{\Transp}(u, v)$ de $A$, à définir ad-hoc.

Considérons pour simplifier le cas où $\pi_1 = 0$, d'où $\underline{A}$ se réduit à $\pi_0$ et la catégorie des homomorphismes $\Gamma \to \underline{A}$ à celle des homomorphismes $\Gamma \to \pi_0$. Le fait que le foncteur de 2-catégorie plus haut soit une équivalence de 2-catégorie se traduit alors pas à pas ainsi :
\begin{enumerate}
    \item[1)] {\bf 0-fidèle} signifie que pour tout hom $u: \Gamma \to A$, on a $\pi_1(A^u) = 0$.
    
    \underline{N.B.} $A^u$ et $A^{\circ u}$ ont même composante neutre donc la condition s'écrit
    $$
    \boxed{\pi_1((A^\circ)^u) = 0}
    $$
    \item[2)] {\bf 1-fidèle} signifie que de plus, si $u$, $v$ sont des homomorphismes de $\Gamma \rightrightarrows A$, et $\alpha, \beta \in \Transp_A (u, v)$ ont même image dans $\Transp_{\pi_0}(\underline{u}, \underline{v})$ alors $\alpha$, $\beta$ sont dans une même composante connexe de $\Transp_A(u, v)$.
    
    Pour le voir OPS $u = v$ et on est ramené à exprimer que l'application $\pi_0 (A^u) \to \pi_0$ est injective i.e. que son noyau est réduit à 1, i.e. que le sous-groupe ouvert $A^u \cap A^\circ = (A^\circ)^u$ de $A^u$ est connexe, i.e.
    $$
    \boxed{\pi_0 ((A^\circ)^u) = 0}
    $$
    \item[3)] {\bf 2-fidèle} signifie que en plus des conditions précédentes, qui assurent que pour $u$, $v$, fixés, le foncteur $\underline{\Hom}(u, v) \to \underline{\Hom}(\underline{u}, \underline{v})$ est pleinement fidèle, que celui-ci est essentiellement surjectif, i.e. que si $u$, $v$ sont tels que $\underline{u}, \underline{v}: \Gamma \to \pi_0$ sont conjugués (si on veut égaux) alors $u$ et $v$ sont déjà conjugués par un élément de $\pi_0$.
    
    Il faut le dire de fa\c{c}on plus forte : $\Transp(u, v) \to \Transp(\underline{u}, \underline{v})$ surjectif : cela équivaut à dire que si $\underline{u} = \underline{v}$ alors $u$ et $v$ sont conjugués par un \emph{élément} de $A^\circ$.
    
    En d'autres termes : pour tout homomorphisme $\underline{u}: \Gamma \to \pi_0 = A/A^\circ$ s'il existe un relèvement de $u$ en $\Gamma \to A$, celui-ci est unique à conjugaison près.
    \item[4)] {\bf 3-fidèle} signifie qu'en plus tout hom $\underline{u}: \Gamma \to \pi_0$ se relève en $u: \Gamma \to A$. En résumé, si $\pi_1(A^\circ) = 0$ la 3-fidélité signifie que pour tout homomorphisme $\underline{u}: \Gamma \to \pi_0 = A/A^\circ$, il existe un relèvement $u: \Gamma \to A$, unique à conjugaison près, et qu'alors $(A^\circ)^u$ est connexe et simplement connexe. Je présume que dans le cas général où on ne suppose pas $\pi_1(A^\circ) = 0$ il faut remplacer la condition $\pi_1(A^{\circ u}) = 0$ par $\pi_1(A^{\circ u} \to \pi_1(A^\circ))$ est un isomorphisme.  
\end{enumerate}
Trop brutal ! la condition en question disant $\pi_1 (A^{\circ \Gamma}) \isommap \pi_1(A^\circ)^\Gamma$ mais il faut reprendre avec soin l'ensemble des conditions et les modifier ad-hoc\dots Cf feuille intercalaire.

(Le plus agréable serait que ce soit un homotopisme - c'est cela sans doute qui exprimerait qu'on a une $\infty$-équivalence\dots)

J'ai envie de prouver d'abord 1), 2), 3) dans le cas $A = A_{g, \nu}$, en me limitant au besoin au cas anabélien (sa doute le plus dur en fait ! mais moins touffu conceptuellement) et bien sûr au cas où $\Gamma$ est fini. Le premier travail sera bien sûr celui de déterminer $A^{\circ u}$ et sa structure topologique pour essayer de prouver que $A^{\circ u}$ est contractile dans ce cas.







\newpage

[Intercalaire]

\vskip 2cm

(On ne suppose plus $\pi(A^\circ) =0$).

\begin{enumerate}
    \item[1)] 0-fidèle. $\pi_1 (A^\circ \Gamma) \to \pi_1(A^\Gamma) = \mathrm{H}^\circ(\Gamma, \pi_1(A))$ injectif. 
    \item[2)] 1-fidèle. $\pi_1 (A^{\circ \Gamma}) \to \pi_1(A^\Gamma)$ bijectif et $\pi_0(A^{\circ \Gamma}) \to \mathrm{H}^1(\Gamma, \pi_1)$ injectif.
    \item[3)] 2-fidèle. $\pi_1(A^{\circ \Gamma}) \to \mathrm{H}^\circ (\Gamma, \pi_1(A))$ et $\pi_0 (A^{\circ \Gamma}) \to \mathrm{H}^1(\Gamma, \pi_1(A))$ bijectif et $\pi_0 (A^\Gamma) \to \Ker (\pi_0(A)^\Gamma) \to \mathrm{H}^2(\Gamma, \pi_1)$ (qui est déjà injectif par la condition précédente) est surjectif i.e. tout élément de $\pi_0(A)$ qui centralise \emph{strictement} $\Gamma$ provient d'un élément de $A$ qui centralise $\Gamma$. Il faut un peu plus, quand on a deux homomorphismes $u, v: \Gamma \to A$ qui coïncident \emph{strictement} dans $\underline{A}$, on veut qu'il soient conjugués par un élément de $A^\circ$. 
    \item[4)] 3-fidèle. En plus des conditions précédentes, on veut que tout homomorphismes $\Gamma \to \underline{A}$ (défini par $\Gamma \to \pi_0 (A)$ et un scindage de la $\Gr$-catégorie de Sinh-Postnikov extension de $\Gamma$ par $\pi_1(A)$) provienne d'un homomorphisme $\Gamma \to A$.
\end{enumerate}











%%%%%%%%%%%%%%%%%%%%%%%%%%%%%%%%%%%%%%%%%%%%%%%%%%%%%%%%%%%%%%%
\chapter*{\S \space 24. --- ESSAI DE DÉTERMINATION DE $A^{0\Gamma}$ ; LIEN AVEC LES RELATIONS ${\pi^\Gamma_{g,(\nu,\nu+n-1)}} =\{1\}$, ET PROGRAMME DE TRAVAIL}\thispagestyle{empty}
\addcontentsline{toc}{section}{24. Essai de détermination de $A^{0\Gamma}$ ;
lien avec les relations ${\pi_{g,(\nu,\nu+n-1)}}^\Gamma =\{1\}$, programme de travail}
\label{sec:24}
\section*{}

Considérons une opération (fidèle si on y tient) du groupe fini $\Gamma$ sur la surface $U \subset  X = \widehat{U}$ de type $(g, \nu)$, $U = X \textbackslash S$.

Soit $A = \Aut(U)$, donc $\Gamma \subset  A$ et $A^u = A^\Gamma$ est le centralisateur de $\Gamma$. Supposons d'abord $\Gamma = \Gamma^+$ ; évidemment $f \in A^\Gamma$ implique que $f$ induit un automorphisme $g$ de $X/\Gamma = X'$ dont $X$ est un revêtement galoisien de groupe $\Gamma$.

Soit $S'$ l'image de $S$ dans $X'$ (donc $S = X|_{S'}$) et soit $S^!$ l'ensemble des points de ramification de $\Gamma$ dans $U$, $S^{'!}$ son image dans $U' = X' \textbackslash S'$ (NB $U = X|_{U'}$).

Ainsi $g$ est un automorphisme de $(X', S')$, appliquant $S^{'!}$ sur lui même, et induisant donc un automorphisme de $V' = U' \textbackslash S^{'!} = X' \textbackslash (S' \cup S^{'!})$.

L'image inverse $V = X | V'$ est étale sur $V'$, c'est même un $\Gamma_{V'}$-torseur sur $V'$ et l'image inverse de celui-ci par $g |_{V'}$ est (via $f$) isomorphe à $V$. La donnée de $f$ équivaut à la donnée d'un automorphisme du $\Gamma$-revêtement $V$ de $V'$, induisant un isomorphisme de $V'$ qui, prolongé à $X'$, envoie $S'$ dans lui même. Cela donne une interprétation de $A^\Gamma$ en termes de constructions sur $X'$.

Considérons le groupe fondamental $\pi'$ de $V'$\footnote{On a choisit un point $x \in V$ et son image $x' \in V'$ comme points de base, pour expliciter $\pi$.}, avec sa structure à lacets indexée par $S' \sqcup S^{'!}$, on a donc un homomorphisme surjective
$$
\pi' \xlongrightarrow{\phi} \Gamma
$$
qui sur chacun des sous-groupes à lacets d'indice $s' \in S^{'!}$ n'est pas trivial, et un automorphisme $g$ de $V'$ définit un automorphisme extérieur $\overline{g}$ de $\pi'$. La condition à mettre dessus est que son composé avec $\phi$ est conjugué de $\phi$, et que $\overline{g}$ applique $S^{'!}$ dans lui même.

Ceci posé, $g$ est défini par $\overline{g}$ à isotopie près (si on excepte les cas $(g', \nu')$ dégénérés, à savoir $(0, \nu')$ avec $\nu' = 0, 1, 2$ qu'il faudra examiner à part\dots) et $g$ étant choisi, $f$ est défini modulo multiplication par un élément du centre de $\Gamma$ mais on sait déjà que $A^\circ \cap \Gamma = \{ 1 \}$, donc si on se borne à examiner des éléments $f$ dans $A^{^\circ \Gamma}$, alors $f$ sera déterminé par $g$ de fa\c{c}on unique.
 
On est ainsi amené au problème suivant :

Soit $V' = X' \textbackslash (S' \sqcup S^{'!})$ une surface admissible de type $(g', \nu')$ avec $X'$ compacte, et $\nu' = \card S' + \card S^{'!}$, muni d'un sous ensemble $S^{'!}$ de l'ensemble des points à l'infini et d'un point de base $x'$, d'où $\pi' = \pi_1 (V', x')$.

On se donne un revêtement galoisien connexe $V$ de $V'$ de groupe fini $\Gamma$, ponctué au dessus de $x'$, donc défini par un homomorphisme surjectif
$$
\pi' \xlongrightarrow{\phi} \Gamma
$$
et on suppose $V'$ ramifié en chacun des $s' \in S^{'!}$ i.e. (pour un choix du groupe à lacets $L_{s'}$ correspondant à $s'$) que $L_{s'} \to \Gamma$ n'est pas trivial. 

On considère le compactifié pour $X$ de $V$, et l'ensemble $S$ (resp. $S'$) des points de $X$ sur $S'$ (resp. $S^{'!}$).

Soit $g$ un automorphisme de $V$, définissant un automorphisme extérieur $\overline{g}$ de $\pi'$\footnote{OPS que $g$ fixe $x'$ donc $\overline{g}$ est un automorphisme bien défini de $\pi'$.}, on suppose que 
\[\begin{tikzcd}
	{g^*(V'|_{V})} && {V'}
	\arrow["\sim", from=1-1, to=1-3]
	\arrow["{V-\text{iso}}"', from=1-1, to=1-3]
\end{tikzcd}\]
i.e. que $\phi \circ \overline{g}$ est conjugué à $\phi$ (ce qui exprime que $g$ provient par passage au quotient d'un automorphisme $f$ de $V$, défini modulo multiplication par un élément $z \in \Centre \Gamma$).

Soit $U = V \cup S^! = X \textbackslash S$ (revêtement ramifié sur $U' = V' \cup S^{'!} = X' \textbackslash S'$), on veut exprimer que parmi les $f$ qui remonte $g$, il y en a un (nécessairement unique !) qui est isotope à 1 dans $\underline{\Aut}(U)$ - i.e. qui induise l'automorphisme extérieur trivial de $\pi_1(U)$. On aimerait prouver que se sont exactement ceux qui sont isotopes à 1 dans $\underline{\Aut}(V')$ - ou encore que pour un tel $f$, $f$ est nécessairement isotope à 1 dans $\underline{\Aut} (V)$ - i.e. est dans $\underline{\Aut} (V)^\circ$ et pas seulement dans $\underline{\Aut} (U)^\circ$.

Notons, lorsque $\underline{\Aut} (V')^\circ$ est connexe, que $\underline{\Aut} (V')^\circ$ se relève (en vertu de principes généraux) en 
$$
\underline{\Aut} (V')^\circ \to \underline{\Aut} (V)^{\circ \Gamma} \subset  \underline{\Aut} (U)^{\circ \Gamma} 
$$
Donc la condition énoncée sur $g$ d'isotopie à 1 est certainement \emph{suffisante}. Notons d'ailleurs que les résultats déjà obtenus impliquent que $\underline{\Aut} (U)^\Gamma \cap \underline{\Aut} (U)^\circ = \underline{\Aut} (U)^{\circ \Gamma}$ induit \emph{l'identité} sur $S^{'!}$ ($= U$ avec une notation antérieure).

Or soit $B \subset  \underline{\Aut} (U)^\circ \cap \Aut (U, S^!)$ le sous-groupe des automorphismes qui fixent les $s \in S^!$ de sorte que 
$$
\underline{\Aut} (U)^\circ / B \isom \underline{\Mon} (S^!, U) \quad \text{et} \quad B^\circ = \underline{\Aut} (V)^\circ 
$$
un argument connu nous montre que
$$
B/B^\circ \isom \pi_1 (\underline{\Mon}(S^!, U))
$$
Comme
$$
\underline{\Aut} (U)^{\circ \Gamma} \subset  B \quad \text{i.e.} \quad \underline{\Aut} (U)^{\circ \Gamma} = B^\Gamma
$$
on déduit donc de $1 \to B^\circ \to B \to B/B^\circ \to 1$
$$
1 \to B^{\circ \Gamma} \to B^\Gamma \to (B/B^\circ)^\Gamma \to \mathrm{H}(\Gamma, B^\circ) (= 1 ?)
$$
Donc on trouve que la composante neutre de $\underline{\Aut} (U)^{\circ \Gamma}$ est isomorphe à $\underline{\Aut} (V)^\circ$ (donc est $\infty$-connexe si $V$ anabélienne), et son $\pi_0$ est inclus dans $(B/B^\circ)^\Gamma \isom \pi_1 (\Mon (S^!, U))^\Gamma$. On voudrait prouver que le sous-groupe des invariants sous $\Gamma$ est réduit à $\{ 1 \}$
$$
\mathrm{H}^\circ (\Gamma, \pi_1(\Mon (S^!, U))) = 0
$$
On aimerait que ceci soit vrai même indépendemment d'hypothèse de réalisabilité, quand on se donne un homomorphisme de $\Gamma$ dans le groupe de Teichmüller d'un $V$, et qu'on fait les hypothèses adéquates\dots  
\vskip .3cm
{
Conjecture. --- \it Soit $\pi$ un groupe extérieur à lacets de type $(g, \nu)$, $I$ l'ensemble d'indices des classes des lacets, $\Gamma$ un groupe fini opérant extérieurement sur $\pi$. Alors il existe un groupe extérieur à lacets $\pi'$, d'ensemble $I'$ des classes de lacets, une opération extérieure de $\Gamma$ sur $\pi'$, une partie $I^{'!}$ de $I'$ stable par $\Gamma$, un homomorphisme extérieur de ``bouchage des trous de $I'$''
$$
\pi' \to \pi 
$$
compatible avec l'action de $\Gamma$, tels que :
\begin{enumerate}
    \item[$1^\circ$)] Le stabilisateur dans $\Gamma$ de tout élément de $I^{'!}$ soit réduit à 1.
    \item[$2^\circ$)] L'extension de $\Gamma$ par $\pi'$ déduite de l'action extérieure de $\Gamma$ n'a pas d'éléments d'ordre fini $\neq 1$ (i.e. pour aucun élément de $\Gamma \neq 1$, l'action extérieure sur $\pi'$ ne se réalise par un automorphisme de $\pi'$ d'ordre fini).
    
    De plus le $\Gamma$-groupe extérieur à lacets $\pi'$, muni du morphisme $\pi' \to \pi$, est déterminé à isomorphisme près.
\end{enumerate}
}
\vskip .3cm

{\bf Commentaire}. L'existence de $\pi'$, $\pi' \to \pi$ est évidente dans le cas ``réalisable''. L'unicité à isomorphisme unique près, même dans le cas réalisable n'est pas évidente, ni même prouvée. Le noyau de
$$
\Autext_{\text{lac}} (\pi', \pi) = \Autext_{\text{lac}} (\pi', I^{'!}) \to \Autext_{\text{lac}} (\pi)
$$
est justement le groupe $\pi_1$ de tantôt\footnote{Plutôt une extension de $\gS_{I'}$ par ce $\pi_1$}, et pour montrer que le foncteur des couples $(\pi', I^{'!})$ d'un $\Gamma$-groupe extérieur $\pi'$ et d'une partie $I^{'!}$ de $I(\pi')$ stable par $\Gamma$, telle que l'extension correspondante de $\Gamma$ par $\pi'$ soit ``sas torsion'' et que le stabilisateur de tout $i' \in I^{'!}$ dans $\Gamma$ soit non trivial, vers les $\Gamma$-groupes extérieurs, soit (non seulement essentiellement surjectif mais) pleinement fidèle, est déjà problématique.

La fidélité signifie justement que $\pi^{\Gamma}_1 = \{ 1 \}$, la pleine fidélité est plus forte que le fait que
$$
\Autext_{\text{lac}} (\pi', \pi)^\Gamma \to \Autext_{\text{lac}} (\pi)^\Gamma 
$$
est surjectif (dans le cas réalisable, cette surjectivité serait conséquence du fait que toute classe d'homéomorphisme commutant à $\Gamma$ contient un homéomorphisme commutant à $\Gamma$) - il faut ajouter à ceci que tout autre relèvement de $\Gamma$ en une opération extérieure sur $\pi'$ [i.e. un $\Gamma \to \Autext_{\text{lac}} (\pi', \pi)$], ayant la même action de $\Gamma$ sur $I^{'!}$, est conjugué du précédent par un élément du noyau $\pi_1$ de $\Autext_{\text{lac}}(\pi', I^{'!}) \to \Autext_{\text{lac}}(\pi) \times (\gS_{I^{'!}})$\dots 

On va ré-énoncer la conjecture précédente sous une forme un peu plus générale. Introduisons une notation :

Si $\Gamma$ est un groupe fini opérant extérieurement de fa\c{c}on directe sur un groupe à lacets $\pi$ anabélien, donnant lieu à une extension $E$ de $\pi$ par $\Gamma$, on va désigner par $\Phi = \Phi (\pi, \Gamma)$ (``points fixés'') l'ensemble des classes de $\pi$-conjugaison des sections partielles $\neq \{ 1 \}$ maximales de l'extension. (Le caractère intrinsèque de $\Phi(\pi, \Gamma)$, indépendemment des choix particuliers de $E$ a été déjà noté). Il est clair que $\Phi$ est un $\Gamma$-ensemble, on sait aussi qu'il est fini.

Soit maintenant $(\pi', \Gamma)$ un $\Gamma$-groupe extérieur à lacets avec $\Gamma$ opérant de fa\c{c}on directe $(\Gamma = \Gamma^+)$ et fidèle (pour simplifier), fixons nous une partie $I^{'!}$ de $I' = I(\pi')$, stable par $\Gamma$, telle $\forall i' \in I^{'!}$, on ait $\Gamma_{i'} \neq \{ 1 \}$, et considérons le groupe extérieur quotient, défini par bouchage de $I^{'!}$, soit $\pi$ ; il est clair que $\Gamma$ opère encore dessus.

On veut d'abord définir une bijection
$$
\Phi (\pi, \Gamma) \isom \Phi (\pi', \Gamma) \sqcup I^{'!}
\leqno{(*)}
$$
en supposant $(\pi, \Gamma)$ également anabélien pour être plus sur !
\begin{enumerate}
    \item[a)] Application $I' \to \Phi(\pi, \Gamma)$. 
    
    Soit $L_{i'} \subset  \pi'$ de classe $i' \in I'$, $Z_{i'}$ son centralisateur dans l'extension $E'$ de $\Gamma$ par $\pi'$ de sorte $L_{i'}$ est une extension de $\Gamma_{i'}$ par $L_{i'}$. LE passage au quotient $E' \to E$ définit une section partielle $\Gamma_{i'} \hookrightarrow E$, contenue dans une unique section partielle maximale (sans doute déjà maximale elle même\dots\footnote{oui car pour le voir on peut supposer $\Gamma$ cyclique et donc la situation est réalisable\dots! (?)}). On trouve ainsi une application $I' \to \Phi(\pi, \Gamma)$, évidemment compatible avec les actions de $\Gamma$.
    \item[b)] Application $\Phi (\pi', \Gamma) \to \Phi (\pi, \Gamma)$
    
    Toute section partielle $\neq 1$ de $E'$ sur $\Gamma$ en définit évidemment une de $E$ sur $\Gamma$ en passant au quotient ; on a comme ci dessus qu'une section maximale donne une section maximale, en se ramenant au cas $\Gamma$ cyclique. \footnote{pas clair, car l'hypothèse que les $\Gamma_{i'} \neq 0$, risque de ne pas être réalisée.}
    \item[c)] Bijectivité de (*)
    
    Elle n'est pas prouvée (sauf si la situation de départ est réalisable, ou si $I^{'!}$ réduit à un seul élément mais alors $\Gamma$ est cyclique et la situation est réalisable\dots)
\end{enumerate}
Ceci admis on a défini un foncteur
$$
(\pi', \Gamma, I^{'!}) \mapsto (\pi, \Gamma, I^{'!})
$$
allant de la catégorie des $\Gamma$-groupes extérieurs à lacets $\pi'$ munis d'une partie $I^{'!}$ de $I' = I(\pi')$ stable par $\Gamma$, telle que $s' \in I' \Rightarrow \Gamma_{s'} \neq \{ 1 \}$, dans la catégorie des $\Gamma$-groupes extérieures à lacets, munis d'une partie $I^{'!}$ de $\Phi (\pi, \Gamma)$.

La conjecture est que le foncteur (lui même un peu conjectural, sauf si on se réduit aux situations de départ réalisables, d'où situations d'arrivée également réalisables) est une équivalence de catégories i.e. i) pleinement fidèle et ii) essentiellement surjectif.

Le point ii) est clair, quand on se borne de part et d'autre à des situations réalisables. Mais même dans ce cas, le point i) n'est pas prouvé. Je pense que si $I'$ est \emph{réduit à un élément}, alors les propriétés du foncteur ``forage d'un trou'' permettront de l'établir aisément. On voit aussi que pour l'établir, on est ramené à établir la bijectivité de (*), et le théorème d'équivalence dans le cas où $\Gamma$ est transitif sur $I^{'!}$.  

Pour résumer, le programme d'attaque du problème de réalisabilité d'un $\Gamma$-groupe extérieur à lacets serait le suivant :
\begin{enumerate}
    \item[I)] Établir la bijectivité de (*) dans le cas général. 
    \item[II)] Établir la pleine fidélité du foncteur précédent (en se ramenant au besoin au cas où $I^{'!}$ est \emph{une} orbite (singulière) de $\Gamma$ dans $I' = I(\pi')$).
    \item[III)] Dans le cas où $\Phi(\pi, \Gamma) = \emptyset$ i.e. l'extension $E$ n'a pas d'élément d'ordre fini $\neq 1$, prouver que $E$, muni de l'ensemble des $E$-classes de conjugaison des centralisateurs des $L_i$ $(i \in I (\pi))$ dans $E$, est un groupe à lacets.
\end{enumerate}

D'ailleurs, pour disposer des propriétés préliminaires indispensables des ensembles   $\Phi (\pi, \Gamma)$, il faudrait commencer par réaliser ce programme pour les groupes cycliques (opérants de fa\c{c}on directe), et procéder dans ce cas par dévissage.

Si le théorème de classification complet (comme équivalence des 2-catégories - des opérations topologiques et des opérations isotopiques -) était vrai, les points I, II, III de ce programme devrait l'être aussi, et ce serait donc un bon programme d'approche. Les points I et II devraient être encore valables, dès que le 2-foncteur entre 2-catégories serait 2-fidèle (pas nécessairement 3-fidèle), du moins pour les situations réalisables. On dirait alors qu'une action extérieure de $\Gamma$ est \emph{admissible}, si
$$
\Autext (\pi', \Gamma, I^{'!}) \isom \Aut (\pi, \Gamma, I^{'!})
$$
et si de plus toute autre action de $\Gamma$ sur $\pi'$, induisant la même action sur $\pi$ et la même application $I^{'!} \to \Phi (\pi, \Gamma)$, est conjugué de l'action originelle.

Ceci posé, la condition nécessaire et suffisante de réalisabilité d'une action extérieure fidèle de $\Gamma$ sur $\pi$ serait alors que cette action se remonte (par ``forage de trous'' pour $I^{'!} = \Phi(\pi, \Gamma)$) en une action \emph{admissible} de $\Gamma$ sur un $(\pi', I^{'!})$, (par définition même le relèvement serait alors unique) et que de plus l'extension correspondante $E'$ de $\Gamma$ par $\pi'$ soit un groupe à lacets.











%%%%%%%%%%%%%%%%%%%%%%%%%%%%%%%%%%%%%%%%%%%%%%%%%%%%%%%%%%%%%%%
\chapter*{\S \space 25. --- GROUPES DE TEICHMÜLLER SPÉCIAUX}\thispagestyle{empty}
\addcontentsline{toc}{section}{25. Groupes de Teichmüller ``spéciaux''}
\label{sec:25}
\section*{}

Revenons aux notations du $n^\circ 19$, on va définir un sous-groupe $S A_{g, \nu}$ de $A^!_{g, \nu}$
$$
S A_{g, \nu} = \{ u \in A_g~|~u~\text{induise l'identité sur un voisinage de}~S_{g, \nu} \}
$$
i.e. ensemble des automorphismes de $U_{g, \nu}$ qui induisent l'identité dans le complémentaire d'un compact.

Contrairement aux autre sous-groupes de $A_g$ considérés jusqu'à présent, celui-ci n'est pas un sous-groupe fermé. N.B. Dans le cas où on travaille avec des surfaces compactes à bord au lieu de surfaces compactes (sans bord) ``trouées'', il y aurait lieu de prendre le groupe des automorphismes qui induisent l'identité sur le bord. 

$S A_{g, \nu}$ est un sous-groupe invariant de $A_{g, \nu}$, le quotient $A_{g, \nu}/SA_{g, \nu}$ étant isomorphe au groupe des \emph{germes} d'automorphismes de $X_g$ au voisinage de $S_{g, \nu}$, ou encore le groupe des germes à l'infini d'automorphismes de $U_{g, \nu}$ ([?] les complémentaires de compacts\dots)

On aura évidemment, puisque $SA_{g, \nu} \subset  A_{g, \nu}$
$$
(SA_{g, \nu})^\circ \subset  A^\circ_{g, \nu}
$$
Posons
$$
S A^!_{g, \nu} = SA_{g, \nu}/SA^\circ_{g, \nu}
$$
on aura un homomorphisme canonique
$$
SA^!_{g, \nu} \to \Gamma_{g, \nu}^{!+}
$$
qu'on va interpréter de fa\c{c}on algébrique, en termes de l'interprétation de $\Gamma^!_{g, \nu}$ comme le groupe des automophismes extérieures du groupe à lacets $\pi_{g, \nu}$, induisant l'identité sur $S_{g, \nu} = I(\pi_{g, \nu})$.

Pour tout $i \in S_{g, \nu}$, choisissons un $L_i$ de classe $i$ dans $\pi_{g, \nu}$ - ce qui revient à se donner un ``point'' de $B_{D^*_i}$ ($\isom$ un revêtement universel $\widetilde{D^*_i}$) et un isomorphisme entre son image dans $B_{U = U_{g, \nu}}$ avec le ``point'' $s = s_{g, \nu}$ de référence, qui servait à définir $\pi_{g, \nu}$ comme $\Aut_{B_U} \isom \pi_1(B_U, s)$.

Ceci dit, si $u$ est un automorphisme de $\pi = \pi_{g, \nu}$ le fait qu'il respecte (strictement, en induisant l'identité sur $S_{g, \nu}$) la structure à lacets, s'exprime par l'existence d'une famille d'éléments $g_i \in \pi$, tels que
$$
u(l) = \text{int}g^{-1}_i(l^\alpha) \quad (i \in I, l \in L_i, \alpha = \chi(n))
$$
-lesquels $g_i$ sont déterminés par $u$ modulo multiplication à droite par des $\lambda_i \in L_i$. Si on a de même $v$, $(h_i)$, alors : pour $l \in L_i$ on a (si $\alpha = \chi(u)$, $\beta = \chi(v)$)
$$
(uv)(l) = v(u(l)) = v(\text{int}(g^{-1}_i))v(l^\alpha) = \text{int}(v(g^{-1}_i) h^{-1}_i)(l^{\alpha \beta}) = \text{int}(h_i u(g_i)^{-1})l^{\alpha \beta}
$$
donc $vu$ est compatible avec le système des $h_i v(g_i)$. Posons 
$$
(v, (h_i))(u, (g_i)) = (vu, h_iv(g_i))
$$
On trouve alors une structure de groupe sur l'ensemble $S E^!$ des $(u, (g_i))$, sous-groupes du produit semi-direct de $E^! = \Autext_{\text{lac}}(\pi, \id$ sur $I)$ par $\pi^I$, sur lequel $E^!$ opère de fa\c{c}on évidente. L'homomorphisme naturel
$$
SE^! \to E^!
$$
est surjectif, et son noyau est essentiellement isomorphe à $\prod_{i \in I} L_i \isom T^I$, où $T$ est le module des orientations. D'où une structure d'extension où $E$ opère sur $T^I$ via son action sur $T$
$$
1 \to T^I \to SE^! \to E^! \to 1
$$
Et je voudrais interpréter cette extension comme l'image inverse d'une extension canonique de $\Gamma^!$ par $T^I$ (canonique en tous cas, une fois choisi les $\widetilde{D^*_i}$).

Pour ceci, notons que si $\dot{u}$ est une auto-équivalence de la situation $\B_{D^*} \to \B_U$, induisant l'identité sur $I$, on peut considérer les $\tilde{\dot{u}}$ au dessus de $\dot{u}$, à savoir les systèmes $(\dot{u}, \gamma_i)$, où pour tout $i \in I$, $\gamma_i: \widetilde{D^*_i} \to \dot{u}(\widetilde{D^*_i})$ [déterminé mod élément de $T$].

Le groupe des classes d'isomorphie d'automorphismes de $(\B_{D^*} \to \B_U, (\widetilde{D^*_i}))$, ou si on préfère, des classes d'isomorphie d'automorphismes de $(\B_I \to \B_{D^*} \to \B_U)$ induisant l'identité sur $I$, est donc une extension de $\Gamma^!$ par $T^I$, soit $ST^!$. 

On va définir un homomorphisme de suites exactes 
\[\begin{tikzcd}
	1 & {T^I} & {SE^!} & {E^!} & 1 \\
	1 & {T^I} & {S \Gamma^!} & {\Gamma^!} & 1
	\arrow[from=1-1, to=1-2]
	\arrow[from=1-2, to=1-3]
	\arrow[from=1-3, to=1-4]
	\arrow[from=1-4, to=1-5]
	\arrow[from=2-4, to=2-5]
	\arrow[from=2-3, to=2-4]
	\arrow[from=2-2, to=2-3]
	\arrow[from=2-1, to=2-2]
	\arrow[shift left=2, shorten <=1pt, shorten >=2pt, no head, from=1-2, to=2-2]
	\arrow[shift right=1, shorten <=1pt, shorten >=2pt, no head, from=1-2, to=2-2]
	\arrow[from=1-3, to=2-3]
	\arrow[from=1-4, to=2-4]
\end{tikzcd}\]
qui prouve que $SE^!$ est bien l'image de l'extension $S \Gamma^!$ par $T^I$. Pour ceci, on note que $E^!$ est le groupe des classes d'isomorphie d'auto-équivalences de
\[\begin{tikzcd}
	{\B_{D^*}} \\
	{\B_U} & {\pt . s}
	\arrow[from=2-2, to=2-1]
	\arrow[from=1-1, to=2-1]
\end{tikzcd}\]
et $SE^!$ celui des classes d'isomorphie d'auto-équivalences de 
\[\begin{tikzcd}
	{\B_{D^*}} && {\B_I} \\
	&&& {(\star~\text{est un isomorphisme de commutation})} \\
	{\B_U} && {\pt . s}
	\arrow[from=1-3, to=1-1]
	\arrow[from=1-3, to=3-3]
	\arrow[""{name=0, anchor=center, inner sep=0}, from=3-3, to=3-1]
	\arrow[""{name=1, anchor=center, inner sep=0}, from=1-1, to=3-1]
	\arrow["\star"', shorten <=7pt, shorten >=7pt, from=0, to=1]
\end{tikzcd}\]
qui s'envoie e fa\c{c}on naturelle dans celui des classes d'isomorphie d'auto-équivalences des diagrammes $\B_I \to \B_{D^*} \to \B_U$.

En fait, dans tout ceci il n'y avait aucunement lieu de se borner aux automorphismes de $\pi$ (extérieurs ou totaux) induisant l'identité sur $I$. On trouve de toutes fa\c{c}ons des extensions $SE$ de $E$ par $T^I$, $S \Gamma$ de $\Gamma$ par $T_I$, et un homomorphisme d'extension
\[\begin{tikzcd}
	&& 1 & 1 \\
	&& \pi & \pi \\
	1 & {T^I} & SE & E & 1 \\
	1 & {T^I} & S\Gamma & \Gamma & 1 \\
	&& 1 & 1
	\arrow[from=1-3, to=2-3]
	\arrow[from=1-4, to=2-4]
	\arrow[shift right=2, from=2-3, to=2-4]
	\arrow[shorten <=5pt, shorten >=5pt, no head, from=2-3, to=2-4]
	\arrow[from=2-4, to=3-4]
	\arrow[from=2-3, to=3-3]
	\arrow[from=3-4, to=4-4]
	\arrow[from=3-3, to=4-3]
	\arrow[from=3-3, to=3-4]
	\arrow[from=4-3, to=4-4]
	\arrow[from=4-3, to=5-3]
	\arrow[from=4-4, to=5-4]
	\arrow[from=4-4, to=4-5]
	\arrow[from=3-4, to=3-5]
	\arrow[from=3-2, to=3-3]
	\arrow[from=4-2, to=4-3]
	\arrow[from=4-1, to=4-2]
	\arrow[from=3-1, to=3-2]
	\arrow[shift left=2, shorten <=2pt, shorten >=2pt, no head, from=3-2, to=4-2]
	\arrow[shift right=1, shorten <=2pt, shorten >=2pt, no head, from=3-2, to=4-2]
\end{tikzcd}\]
de fa\c{c}on que l'on peut considérer $SE$ comme extension de $\Gamma$ par $T^I \times \pi$
$$
1 \to T^I \times \pi \to SE \to \Gamma \to 1
$$
et l'extension $E$ resp. $S\Gamma$ se déduit en divisant par $T^I$ resp. par $\pi$.

Revenant maintenant au cas type $U_{g, \nu} = X_{g, \nu} \textbackslash S_{g, \nu}$, on prends des notations
$$
E_{g, \nu} = \Aut_{\text{lac}}(\pi_{g, \nu}) \quad (\isom \Gamma_{g, \nu + 1}~\text{si}~(g, \nu) \neq (0, 0), (0, 1)~\text{i.e.}~\pi_{g, \nu} \neq 0)
$$
et 
$$
S \Gamma_{g, \nu} = S \Autext_{\text{lac}}(\pi_{g, \nu}) \quad \text{(extension de}~\Gamma_{g, \nu}~\text{par}~T^I)
$$
et
$$
SE_{g, \nu} \quad (\isom S \Gamma'_{g, \nu + 1}/\Gamma_{g, \nu}),
$$
(où $S \Gamma'_{g, \nu + 1}$ désigne le sous-groupe de $S \Gamma_{g, \nu + 1}$ [?] des éléments qui fixent le dernier élément $s_\nu$\dots)

On désigne par $S\Gamma^!_{g, \nu}$ et $S E^!_{g, \nu}$ les sous-groupes des précédents $S \Gamma_{g, \nu}$, $SE_{g, \nu}$ qui induisent l'identité sur $I = S_{g, \nu}$.

Et revenons enfin aux relations avec $S A_{g, \nu}$, on va définir
$$
SA_{g, \nu} \to S \Gamma^{!+}_{g, \nu} 
$$
en notant que si un $u \in A_{g, \nu}$ induit l'identité sur un voisinage de $S_{g, \nu}$, alors $u_{\bullet}(\widetilde{D^*_i}) = \widetilde{D^*_i}$ et on pourra définir un élément de $S\Gamma^!_{g, \nu}$ en prenant comme $\Gamma_i$ les identités. Le fait que l'homomorphisme obtenu soit trivial sur $(S A_{g, \nu})^\circ$ est sans doute trivial, d'où
$$
SA_{g, \nu} / (SA_{g, \nu})^\circ \to S \Gamma^{!+}_{g, \nu}
$$
Dire que c'est surjectif signifie que tout automorphisme dans $A^{!+}_{g, \nu}$ (i.e. tout automorphisme de $U_{g, \nu}$ induisant l'identité sur $S_{g, \nu}$ et respectant l'orientation) est isotope (par une isotopie, si on veut, qui reste l'identité dans l'extérieur d'un petit voisinage de $S_{g, \nu}$) à un automorphisme qui soit l'identité sur un voisinage, c'est facile. L'injectivité [est] peut-être plus délicate, elle revient essentiellement à déterminer le noyau de
$$
SA_{g, \nu} / (SA_{g, \nu})^\circ \to \Gamma^!_{g, \nu}, \quad \text{i.e.} \quad SA_{g, \nu} \cap A^\circ_{g, \nu}/(SA_{g, \nu})^\circ
$$
comme $T^I = T^{S_{g, \nu}}$. On peut sans doute se ramener au contexte des surfaces compactes à bord (notons par un $'$ les groupes topologiques correspondants), on a
$$
1 \to SA^{'+}_{g, \nu} \to A^{'!+}_{g, \nu} \to \prod_{i \in S_{g, \nu}} \Aut^+ (C_i) \to 1
$$
où $\Aut^+(C_i)$ homotope au groupe circulaire standard $\mathbb{U}$ tordu par $T$, et le $+$ indique les automorphismes conservant l'orientation et les $C_i$ sont les composantes connexes du bord. On en conclut la suite exacte d'homotopie
\[\begin{tikzcd}
	{\prod_i \pi_2 (\Aut^+(C_i)) = 0} \\
	& {\pi_1(A^{' +}_{g, \nu})} & {T^I} \\
	{\pi_1(SA^{' \circ}_{g, \nu})} & {\pi_1(A^{' ! +}_{g, \nu})} & {\prod_i \pi_1(\Aut^+(C_i))} \\
	& {\pi_0(SA^{' ! +}_{g, \nu})} & {\pi_0(A^{' ! +}_{g, \nu})} & {\prod_i \pi_0(\Aut^+(C_i)) = 1} \\
	& {\text{(à calculer)}} & {\Gamma'_{g, \nu}}
	\arrow["\sim"', from=3-3, to=2-3]
	\arrow[shift right=1, shorten <=2pt, shorten >=2pt, no head, from=2-2, to=3-2]
	\arrow[shift left=1, shorten <=2pt, shorten >=2pt, no head, from=2-2, to=3-2]
	\arrow[from=5-2, to=4-2]
	\arrow[from=4-2, to=4-3]
	\arrow[from=4-3, to=4-4]
	\arrow[from=3-3, to=4-2]
	\arrow[from=3-2, to=3-3]
	\arrow[shift right=1, shorten <=2pt, shorten >=2pt, no head, from=4-3, to=5-3]
	\arrow[shorten <=2pt, shorten >=2pt, no head, from=4-3, to=5-3]
	\arrow[from=3-1, to=3-2]
	\arrow[from=1-1, to=3-1]
\end{tikzcd}\]
[et $\pi_i(SA^{'\circ}_{g, \nu}) \isom \pi_i(A^{'\circ}_{g, \nu})$ si $i \geq 2$]
i.e.
$$
\begin{cases}
0 \to \pi_1(SA^{'+}_{g, \nu}) \to \pi_1(A^{'\circ}_{g, \nu}) \to T^I \to \pi_0(A^{'!+}_{g, \nu}) \to \Gamma^{'!+}_{g, \nu} \to 1 \\
\pi_i(SA^{'+}_{g, \nu}) \isommap \pi_i(A^{'\circ}_{g, \nu}) \quad \forall i \geq 2
\end{cases}
$$
On a un homomorphisme évidente (par ``recollement de disques'') $A^{'\circ}_{g, \nu} \to A_{g, \nu}$ induisant $SA'_{g, \nu} \to SA_{g, \nu}$, et ce sont là sûrement des équivalences d'homotopie donc la suite exacte précédente doit pouvoir s'interpréter comme suite exacte
$$
\begin{cases}
0 \to \pi_1(SA_{g, \nu})^\circ \to \pi_1(A^\circ_{g, \nu}) \to T^I \to \pi_0 (SA^{!+}_{g, \nu}) \to \Gamma^{!+}_{g, \nu} \to 1 \\
\pi_i(SA_{g, \nu})^\circ \isommap \pi_i(A_{g, \nu})^\circ \quad \forall i \geq 2
\end{cases}
$$
Dans le cas anabélien, on trouve bien, puisque $\pi_1(A^\circ_{g, \nu}) = 0$, une structure d'extension
$$
\boxed{1 \to T^{S_{g, \nu}} \to \pi_0(SA_{g, \nu}) \to \Gamma^{!+}_{g, \nu} \to 1}
$$
et de plus 
$$
\pi_i((SA_{g, \nu})^\circ) = 0 \quad \text{pour} \quad i \geq 2
$$
Dans le cas abélien, on doit expliciter
$$
\pi_1(A^\circ_{g, \nu}) \to T^{S_{g, \nu}} 
$$
et on va distinguer les deux cas abéliens (sous-entendant que l'on ait $I \neq \emptyset$ !) - on a toujours $g = 0$, $\nu = 1$ ou 2.

En tous cas, introduisant une structure analytique complexe et le groupe $G$, composante neutre du groupe des automorphismes complexes, on a 
$$
G \xlongrightarrow{\approx} A^\circ_{g, \nu} 
$$
\begin{enumerate}
    \item[a)] Cas $g = 0$, $\nu = 2$ $G \isom \mathbf{C}^*$, on voit que si la structure à lacets de $\pi$ est définie par les deux isomorphismes $\kappa_i: T \to \pi$ $(i \in I = \{ s_{0, 0}, s_{0, 1} \})$ alors $\pi_1(G) \to T^I$ s'identifie, à $T \to T^I$ dont les composantes sont ces deux $\kappa_i$ (qui sont symétriques, donc c'est un homomorphisme injectif dont l'image est le noyau de l'application somme $T^I \to T$, donc ici le noyau de $\pi_0(SA_{g, \nu}) \to \Gamma^{!+}_{g, \nu}$ (dont $\pi_0(SA_{g, \nu})$) lui même) est isomorphe, non à $T^I$, mais à son quotient $T$.
    
    Quant à $(SA'_{g, \nu})^\circ$, tous ses $\pi_i$ $(i \geq 1)$ sont nuls - il est encore $\infty$-connexe.
    \item[b)] Cas $g = 0$, $\nu = 1$. Alors $G \isom \Aff(1, \mathbf{C})$, $\pi_1(G) \isom T$ et $\pi_1(G) \to T^I = T$ est un isomorphisme. Ici, le noyau de l'homomorphisme $\pi_0(SA_{g, \nu}) \to \Gamma^{!+}_{g, \nu}$ est nul i.e. $\pi_0(SA_{g, \nu}) = 0$.
\end{enumerate}
Dans ce cas encore, on trouve que $\pi_i(SA^\circ_{g, \nu}) = 0$ pour tout $i \geq 1$. On trouve donc
\vskip .3cm
{
Théorème. --- \it Supposons $\nu > 0$. Si $(\gamma, \nu)$ est anabélien, alors $\pi_0(SA_{g, \nu})$ est canoniquement isomorphe à $S\Gamma^{!+}_{g, \nu}$. Dans tous les cas $(S A_{g, \nu})^\circ$ est $\infty$-connexe.
}
\vskip .3cm

Nous allons expliciter une relation de compatibilité pour $S \Gamma$ relatif à un $\pi$ à lacets, en fixant un $i \in I = I(\pi)$, d'où un stabilisateur $\Gamma_i \subset  \Gamma$ dont l'image inverse $(S \Gamma)_i$ dans $S \Gamma$ est donc une extension de $\Gamma_i$ par $T^I$.

Utilisant la projection $T^I \xlongrightarrow{\pr_i} T$, on trouve une extension de $\Gamma_i$ par $T$, qu'on va décrire d'une autre fa\c{c}on.

L'extension $S \Gamma$ est définie intrinsèquement par la donnée de $I$ réalisation $(\pi)_{i \in I}$ du groupe extérieur $\pi$ par des groupes, avec dans chaque $\pi_i$ un $L_i \subset  \pi_i$ de la classe $i$ (c'est cela, la donnée d'un système de $(\widetilde{D^*_i})$ !).

Pour un automorphisme extérieur à lacets $u \in \Gamma$ de $\pi$, pour tout $i$ il est possible de le réaliser pour $u_i \in \Aut_{\text{lac}}(\pi_i)$, avec $u_i(L_i) \subset  L_i$ - cet $u_i$ est déterminé modulo multiplication à droite par un int$(\kappa_i(\alpha))$, où $\alpha \in T$.

Si on regarde la sous-extension obtenue par restriction à $\Gamma_i$, on note que la projection $T^I \xlongrightarrow{\pr_i} T$ est stable par $\Gamma_i$, donc on en déduit  une extension de $\Gamma_i$ par $T$, qui n'est autre que le groupe des automorphismes à lacets de $\pi_i$ qui normalisent $L_i$ - qui est bien une extension de $\Gamma_i$ par $L_i$ (son intersection avec $\pi_i \subset  \Aut_{\text{lac}}(\pi_i)$ étant réduite à $L_i$).

Ceci nous montre que l'extension de $\Gamma$ par $T^I$ a une nette tendance à ne pas être triviale, car il en est ainsi (pour $i \in I$ fixé) de l'extension de $\Gamma_i$ par $T$ à qui elle donne naissance. Si par exemple on a un sous-groupe \emph{fini} $G \subset  \Gamma^+_i$, l'extension induite n'est jamais triviale si $G \neq 1$, on l'a vu. En ait, $G$ doit être cyclique et son image inverse dans l'extension en question est isomorphe à $T$\dots













%%%%%%%%%%%%%%%%%%%%%%%%%%%%%%%%%%%%%%%%%%%%%%%%%%%%%%%%%%%%%%%
\chapter*{\S \space 25 bis. --- CAS DES DEUX GROUPES. RETOUR SUR LES NOTATIONS}\thispagestyle{empty}
\addcontentsline{toc}{section}{25 bis. ``Cas des deux groupes'' d'opérateurs; retour sur les notations}
\label{sec:25bis}
\section*{}

On se place d'abord pour fixer les idées dans le cas topologique et discret, mais la motivation est le cas d'un courbe algébrique $U$ sur un corps de type fini $K$, où on a à la fois le groupe $G_K = \Aut_K(U)$\footnote{Cas anabélien donc $G$ fini.} et $\Gamma = \Gal(\overline{K}/K)$ qui opérant extérieurement sur le $\pi_1(U_{\overline{K}})$.

Dans ce cas, $G$ et $\Gamma$ commutent, mais on peut regarder plus généralement le cas du groupe (plus gros que $G_K$) $G_{\overline{K}} = \Aut_{\overline{K}}(U)$, sur lequel $\Gamma$ opère (de fa\c{c}on pas nécessairement triviale - cette opération décrit un groupe algébrique étale fini sur $K$).

Supposons donc qu'on ait une surface $U$ (orientable, $U = X \textbackslash S$, $X$ compacte connexe, $S$ finie) sur laquelle opère deux groupes $G$, $\Gamma$, l'action de $\Gamma$ normalisant celle de $G$ - donc on a un groupe $\mathcal{G} = \Gamma G$ (produit semi-direct, pour une certaine action de $\Gamma$ sur $G$) qui opère sur $U$. On suppose $G$ fini, mais pas nécessairement $\Gamma$ fini.

On suppose choisi un revêtement universel $\widetilde{U}$ de $U$, d'où un groupe à lacets $\pi = \Aut(\widetilde{U})$, sur lequel $\mathcal{G}$ opère extérieurement, d'où l'extension
$$
1 \to \pi \to E \to \mathcal{G} \to 1 \leqno{(1)}
$$
Si l'action de $\mathcal{G}$ sur $U$ est fidèle, alors $\mathcal{G} \hookrightarrow \Autext(\pi)$, et l'extension précédente est l'image inverse de l'extension de Teichmüller de $\pi$
$$
1 \to \pi \to \Aut_{\text{lac}}(\pi) \to \Autext_{\text{lac}}(\pi) \to 1
$$
On aura à regarder d'autres revêtements universels que $\widetilde{U}$, et leurs isomorphismes avec $\widetilde{U}$. Quand $\widetilde{U} = \widetilde{U}(P)$ est le revêtement universel basé en un certain $P \in U$, alors pour les revêtements universels $U(Q)$ basés en un point, les $U$-isomorphismes $\widetilde{U}(P) \isommap \widetilde{U}(Q)$ correspondent donc aux classes de chemins de $P$ à $Q$.

Soit $Q \in U$ tel que son sous-groupe d'isotropie $G_Q$ dans $G$ soit tel que $G^+_Q \neq 1$. (Donc $G^+_Q$ est cyclique). Choisissant une classe de chemins de $P$ à $Q$, on trouve une opération de $G_Q$ sur $\widetilde{U}(P)$ i.e. un relèvement $G_Q \xlongrightarrow{r_{G_Q}} E$ dans l'extension (1) - le changement de classe de chemins de $\lambda$ en $\lambda'$ donnera un relèvement $r'_{G_Q}$ qui sera conjugué de $r$ par un unique élément de $\pi = \pi_1(U, P)$ (l'unicité provient de $\pi^{G_Q} = {1}$), savoir celui qui fait passer d'un chemin à l'autre.

Considérons le stabilisateur $\Gamma_Q$ de $Q$ dans $\Gamma$, qui opère bien sur $\widetilde{U}(Q)$ tout comme $G_Q$ (en fait c'est $\mathcal{G}_Q$ qui opère d'où $r: \mathcal{G}_Q \to E$\dots), donc via $\lambda$ on a aussi un relèvement $r_{\Gamma_Q}: \Gamma_Q \to E$, qui a la même propriété de normaliser $r_{G_Q}$ (avec opération de $\Gamma_Q$ dessus, qui est celle provenant de l'opération de $\Gamma$ sur $G$)\footnote{N. B. Comme l'ensemble des points $Q$ est fini, et que $\mathcal{G}$ opère dessus, l'orbite de $Q$ sous $\mathcal{G}$ est finie, i.e. $\mathcal{G}_Q$ est d'indice fini dans $\mathcal{G}$ et de même $\Gamma_Q$ sous $\Gamma$.}. Pour simplifier, supposons quand même que $\Gamma$ opère trivialement sur $G$ i.e. $\mathcal{G} = \Gamma \times G$, alors $r_{\Gamma_Q}(\Gamma_Q) \subset  E$est contenu dans le \emph{centralisateur} de $r$ (ou de $r(G_Q)$) et comme $\pi^{r(G_Q)} = 1$, donc l'homomorphisme
$$
\Centr(r(G_Q)) \to \mathcal{G}
$$
est injectif\footnote{N. B. Indépendamment de toute hypothèse que $\Gamma$ centralise $G$, le normalisateur de $r(G_Q)$ dans $\pi$, égal à son centralisateur, est réduit à 1, donc $\Norm(r(G_Q)) \to \mathcal{G}$ est injectif.}, le relèvement en question $r_{\Gamma_Q}$ est uniquement déterminé par la condition précédente.

En ait, l'image de $\Centr r(G_Q)$ dans $\mathcal{G}$ contient $\mathcal{G}_Q$, et on [] de même le relèvement $\mathcal{G}_Q \xlongrightarrow{r_{G_Q}} E$. La chose intéressante, c'est que le choix d'un relèvement du (petit) groupe $\mathcal{G}_Q$, impose déjà le choix d'un relèvement du (grand) groupe $\Gamma_Q$, ou $\mathcal{G}_Q$.

Je dis   que l'image dans $\mathcal{G}$ du centralisateur (et même du normalisateur) de $r(G_Q)$ est $\mathcal{G}_Q$ lui même (a priori il le contient).

Revenant à $\widetilde{U}(Q)$ lui-même, cela signifie que si $g \in \mathcal{G}$ est tel qu'il existe un automorphisme $\widetilde{g}$ de $\widetilde{U}(Q)$ qui relève $g$, en normalisant l'action de $G_Q$, alors $g \in \mathcal{G}_Q$ et $\widetilde{g}$ est le relèvement évident). En effet, si $\widetilde{g}$ normalise l'action de $G_Q$, il invarie l'ensemble des points fixes de $G_Q$ dans $\widetilde{U}(Q)$, qui est réduit au point $\overline{Q}$.

L'ensemble $U^!$ des points $Q \in U$ tels que $G^+_Q \neq (1)$ est stable par l'action de $\mathcal{G}$, et s'identifie (avec cette action) à l'ensemble des relèvements maximaux modulo $\pi$ de sous-groupes (cycliques) $\neq 1$ de $G^+$.

Quand on connaît, pour un relèvement partiel $r$ d'un $G_Q$ dans $E$, i.e. le relèvement correspondant de $\mathcal{G}_Q$, alors de même pour les conjugués de $r$ par n'importe quel élément $g$ (non seulement de $\pi_1$ mais même de $E$), par simple conjugaison. Donc les cas à déterminer correspondent pratiquement aux orbites de $\mathcal{G}$ dans $U^!$. 

On peut s'intéresser à décrire $\mathcal{G} \to \Gamma$ en tant que sous-groupes de $\Autext_{\text{lac}}(\pi) = \Gamma'$ donnant lieu à l'extension $E'$ de $\Gamma'$ par $\pi$ (donc $E \subset  E'$). Mais pour tout relèvement $r$ d'un $G_Q$, considérons $\Centr_{E'}(r)$, on a encore $\Centr_{E'} \cap \pi = (1)$ i.e. on trouve une section au dessus de l'image de ce centralisateur dans $\Gamma'$, soit $\Gamma'(Q)$. Cette image ne dépend que de $Q$ i.e. de la classe de $\pi$-conjugaison de $r$ ou de $r(G_Q)$, et est remplacée par un $G$-conjugué quand $Q$ est remplacé par un $G$-conjugué. Ceci dit, l'intersection $\Gamma'^\natural$ des $\Gamma'(Q)$, pour $Q \in U^!$, est un sous-groupe de $\Gamma'$ qui contient l'intersection $\mathcal{G}^\natural$ des $\mathcal{G}_Q$, et le centralisateur de $G$ dans $\Gamma'\natural$ contient de même l'intersection $\Gamma^\natural$ des $\Gamma_Q$, qui est un sous-groupe invariant d'indice fini de $\Gamma$. Et on peut alors se proposer de voir s'il est posssible de caractériser au moins le sous-groupe fini $\Gamma^\natural$ de $\Gamma$ comme $\Centr_{\Gamma'^\natural}(G)$, et de récupérer peut être $\Gamma$ comme le normalisateur de $\Gamma^\natural$ dans $\Centr_G(\Gamma)$.

Je m'intéresse plus particulièrement à la variante profini de ceci, dans le cas où $U = \mathbb{P}^1_{\mathbf{Q}} \textbackslash {0, 1, \infty}$, $G = \gS_3$, $\Gamma =$ groupe de Galois sur $\mathbf{Q}$ de la clôture algébrique $\overline{\mathbf{Q}}$ de $\mathbf{Q}$ dans $\mathbf{C}$ et $p = \text{exp}(2i\pi/6)$.

Je n'ai pas vérifier que $\Gamma \to \Autext_{\text{lac}}(\widehat{\pi})$ soit injectif, cela m'empêche de faire des calculs dans $\Aut_{\text{lac}}(\widehat{\pi})$, fussent-ils heuristiques pour le moment.

Je bute sur des ennuis de notations - trop de groupes sont désignés par la lettre $\Gamma$ (avec éventuellement des primes, indices, exposants\dots) Il y a trois types de groupes qui interviennent dans mes réflexions :
\begin{enumerate}
    \item[a)] Les groupes de Teichmüller et ses variantes, qui jouent le rôle de groupes ``universels'' opérant (éventuellement modulo isotopie) sur des surfaces, ou sur des groupes extérieurs à lacets. Ces groupes ont tendance à être infinis.
    
    Des groupes (le plus souvent finis) opérant sur des surfaces topologiques, ou sur des courbes algébriques (sans, dans ce cas là, bouger le corps de base).
    
    \item[c)] Des groupes de Galois profinis (donc infinis), [] de corps de type fini sur $\mathbf{Q}$, opérant ``arithmétiquement'' sur des surfaces et leurs $\widehat{\pi}_1$-géométriques\footnote{Les cas b) et c) se mélangent parfois (dans un groupe $\mathcal{G}$ extension d'un groupe de Galois $\Gamma$ par un groupe fini $G$) dans le cas de la Géométrie Algébrique.}.
\end{enumerate}

C'est à cause des analogies profondes entre les cas b) et c), et leurs relations étroites avec le cas a), que j'avais été induit à adopter des notations communes, mais qui à la longue finissent par aboutir à des collisions. Il y a donc lieu de revoir les notations. Je vais réserver la lettre $G$ et variantes pour des actions géométriques (cas b)) de groupes, le plus souvent finis, la lettre $\Gamma$ et variantes pour des groupes de Galois, la lettre $\mathcal{G}$ pour des groupes mixtes.

Quant aux groupes ``universels'' de type Teichmüller, comme ceux notés $\Gamma_{g, \nu}$ précédemment, je vais plutôt les noter $\gT$, $\gT_{g, \nu}$ (initiale de ``Teichmüller'', alors que $\Gamma$, $G$ sont l'initiale de Galois).

Le groupe de Galois sur $\mathbf{Q}$ de la clôture $\overline{\mathbf{Q}}$ de $\mathbf{Q}$ dans $\mathbf{C}$ mérite une lettre spéciale, je le noterai $\GG$. Le quotient $\Norm_{\widehat{\gT}_{g, \nu}}(\widehat{\gT}_{g, \nu})/(\widehat{\gT}_{g, \nu})$, qui s'apparente plus à un groupe de Galois qu'à un group de Teichmüller, sera noté $\GG_{g, \nu}$ (lettre grasse !). Dans le cas $(g, \nu) = (0, 3)$ qui m'occupe plus particulièrement, $\GG_{0, 3}$\footnote{$\GG$ est ici produit semi-direct de $\gS_3 = \gT^+$ par $\gT^!$.} s'identifie au centralisateur de $G = \gS_3 = \gT^+_{0, 3}$ dans $\hat{\hat\gT}_{0, 3}$ il est contenu dans $\hat{\hat\gT}^!_{0, 3}$, et peut-être égal. On a un homomorphisme canonique $\GG \to \GG_{0, 3}$ (plus généralement $\Gamma \to \Gamma_{g, \nu}$) dont j'ignore pour l'instant s'il est injectif, et encore plus s'il est surjectif. Les réflexions qui précèdent suggèrent des conditions sur l'image, qui sont surtout intéressantes si on admet les relations
$$
\widehat{\pi}^\rho S = \widehat{\pi^{\sigma_0}} = (1)
$$
On a désigné par $E_{g, \nu}$ l'extension canonique de $\gT_{g, \nu}$ par $\pi_{g, \nu}$ (qui pour $(g, \nu)$ anabélien s'identifie à $\gT'_{g, \nu + 1}$). L'opération extérieure d'un groupe $G$, $\Gamma$, $\mathcal{G}$ définit aussi une extension par $\pi$ (ou par $\widehat{\pi}$), qu'on a également désigné par la lettre $E$ (initiale d'extension) - il y a à nouveau collisions de notations.

Je vais prendre la lettre $\gS$ (qui fait penser à $\gT$) pour ces extensions dans les cas universels à la Teichmüller, (en écrivant $\gS_{g, \nu}$ au lieu de $\Gamma'_{g, \nu+1}$, puisque l'optique est différente\dots), et en gardant la lettre $E$ dans le cas précédent. Donc $E$ a tendance à être une sous-extension d'un $\gS$.

Admettant que $\GG \to \GG_{0, 3}$ est injectif, on aurait donc
\[\begin{tikzcd}
	1 & {\widehat{\pi}_{0, 3}} & {\widehat{\widehat{\gS}}_{0, 3}} & {\widehat{\widehat{\gT}}_{0, 3}} & 1 \\
	1 & {\widehat{\pi}_{0, 3}} & {\widehat{\widehat{\gS'}}_{0, 3}} & {\gT^+_{0, 3} \times \GG_{0, 3}} & 1 \\
	1 & {\widehat{\pi}_{0, 3}} & E & {\gS \times \GG} & 1 \\
	1 & {\widehat{\pi}_{0, 3}} & {\gS_{0, 3}} & {\gS_3 \times \mathbf{Z}/2} & 1
	\arrow[from=1-1, to=1-2]
	\arrow[from=1-2, to=1-3]
	\arrow[from=1-3, to=1-4]
	\arrow[from=1-4, to=1-5]
	\arrow[from=2-4, to=2-5]
	\arrow[from=2-3, to=2-4]
	\arrow[from=2-2, to=2-3]
	\arrow[from=2-1, to=2-2]
	\arrow[hook, from=4-2, to=3-2]
	\arrow["\isom", from=2-2, to=1-2]
	\arrow["\isom", from=3-2, to=2-2]
	\arrow[hook, from=2-3, to=1-3]
	\arrow[hook, from=2-4, to=1-4]
	\arrow[hook, from=3-4, to=2-4]
	\arrow[from=3-4, to=3-5]
	\arrow[from=3-3, to=3-4]
	\arrow[from=3-2, to=3-3]
	\arrow[hook, from=3-3, to=2-3]
	\arrow[from=4-2, to=4-3]
	\arrow[from=3-1, to=3-2]
	\arrow[from=4-1, to=4-2]
	\arrow[from=4-3, to=4-4]
	\arrow[from=4-4, to=4-5]
	\arrow[hook, from=4-4, to=3-4]
	\arrow[hook, from=4-3, to=3-3]
\end{tikzcd}\]
N. B. On note $\widehat{\widehat{\gS}}'_{g, \nu}$ le normalisateur de $\widehat{\gS}^+_{g, \nu}$ dans $\widehat{\widehat{\gS}}_{g, \nu}$, extension de $\GG_{g, \nu}$ par $\widehat{\gS}^+_{g, \nu}$.

Si $\gamma \in \GG_{0, 3}$, pour qu'il soit dans l'image de $\GG$ il faut qu'il admette un relèvement $u$ qui commute à $\rho$, et un relèvement qui commute à $\sigma_0$ (ce qui, dès que $\gamma \in \widehat{\widehat{\gT}}_{0, 3}$, implique déjà que $\gamma$ dans $\widehat{\widehat{\gT}}_{0, 3}$ commute à $\gT^+_{0, 3} = \gS_3$, i.e. qu'ils est dans $\GG_{0, 3}$). Il se pourrait que tout élément de $\GG_{0, 3}$ ait déjà cette propriété, donc que cette condition ne pose pas de restriction sur l'image de $\GG$ dans $\GG_{0, 3}$. 













%%%%%%%%%%%%%%%%%%%%%%%%%%%%%%%%%%%%%%%%%%%%%%%%%%%%%%%%%%%%%%%
\chapter*{\S \space 26. --- GROUPES DE TEICHMÜLLER PROFINIS (DISCRÉTIFICATION ET PRÉDISCRÉTIFICATION)}\thispagestyle{empty}
\addcontentsline{toc}{section}{{\bf 26.} Groupes de Teichmüller profinis (Discrétification et prédiscrétification). Lien avec topos modulaires de Teichmüller. Conjecture hâtive}
\label{sec:26}
\section*{}

Soit $\pi$ un groupe profini à lacets de type $g,\nu$, $T$ le
$\hat{\mathbf{Z}}$-module inversible de ses orientations. 
On suppose qu'on est dans 
le cas anabélien, et on admettra qu'alors
$$
{\rm Centre}(\pi)=\{1\},\leqno{(1)}
$$
plus généralement que le centralisateur dans $\pi$ de tout
sous-groupe ouvert de $\pi$ est réduit à $\{1\}$.
On aura donc encore une suite exacte canonique de groupes profinis
$$
1 \to \pi \to {\rm Aut}_{\rm lac}(\pi)
\to {\rm Autext}_{\rm lac}(\pi) \to 1.\leqno{(2)}
$$
On posera aussi
$$
{\rm Autext}_{\rm lac}(\pi)=\hat{\hat\gT}(\pi),\leqno{(3)}
$$
et on l'appellera le {\it groupe de Teichmüller étendu}
de $\pi$. On posera aussi $\hat{\hat\gS}(\pi)=
{\rm Aut}_{\rm lac}(\pi)$, de sorte qu'on peut écrire (2) comme
$$
1 \to \pi \to \hat{\hat\gS}(\pi) \to 
\hat{\hat\gT}(\pi) \to 1.\leqno{(2')}
$$
Appelons ``base'' de $\pi$ un ensemble d'éléments de $\pi$:
$(x_i,y_i)_{1\le i\le g}$, $(l_j)_{1\le j\le \nu}$, les
$l_j$ engendrant les différents groupes à lacets, satisfaisant
$$
l_\nu l_{\nu-1}Dots l_1[x_g,y_g]Dots[x_1,y_1]=1,\leqno{(4)}
$$
et tels que ceci soit une relation génératrice. Si on choisit dans
$\pi_{g,\nu}$ une base (discrète) (définition correspondante)\footnote{ou encore 
on définit $\pi_{g,\nu}$ comme le groupe discret
de générateurs les $x_i,y_i,l_j$ et de relation de définition (4)},
d'où une base de $\hat\pi_{g,\nu}$: on aura une bijection évidente
$$
{\rm Bases}(\pi){\buildrel\sim\over\leftarrow} {\rm Isom}_{\rm lac}
(\hat \pi_{g,\nu},\pi).\leqno{(5)}
$$
L'ensemble des bases de $\pi$ est un ensemble homogène
sous $\hat{\hat\gS}(\pi)={\rm Aut}_{\rm lac}(\pi)$.  Si
la base correspond à un isomorphisme $u:\hat\pi_{g,\nu}
\to\pi$, le groupe $\pi_0$ engendré par les
$x_i$, $y_i$, $l_j$ n'est autre que $u(\pi_{g,\nu})$.
Les sous-groupes de $\pi$ qui peuvent s'obtenir ainsi
sont appelés les {\it discrétifications} de $\pi$.
Celles-ci forment un ensemble homogène sous
$\hat{\hat\gS}(\pi)={\rm Aut}_{\rm lac}(\pi)$, canoniquement isomorphe
au quotient du $\hat{\hat\gS}_{g,\nu}$-ensemble à
droite ${\rm Isom}_{\rm lac}(\hat\pi_{g,\nu},\pi)$
par $\gS_{g,\nu}$ : %${}^1$:
$$
{\hbox{Discrét}}(\pi){\buildrel\sim\over\leftarrow}
{\rm Isom}_{\rm lac}(\hat\pi_{g,\nu},\pi)/\gS_{g,\nu}.\leqno{(6)}
$$
Les automorphismes (profinis à lacets) de $\pi$ fixant une
discrétification $\pi_0$ s'identifient aux automorphismes
du groupe discret à lacets $\pi_0$:
$$
{\rm Aut}_{\rm lac}(\pi,\pi_0) \isom {\rm Aut}_{\rm lac}(\pi_0).\leqno(7)
$$
La bijection (5) met sur l'ensemble des bases de $\pi$ une structure
de bitorseur sous $\hat{\hat\gS}(\pi)$, $\hat{\hat\gS}_{g,\nu}$, et la
topologie correspondante en fait un ensemble profini.
L'ensemble des discrétifications de $\pi$, qui est un
quotient de l'ensemble précédent, hérite d'une topologie
quotient, qui n'est autre que la topologie quotient du deuxième
membre de (6). Cet espace n'est pas séparé
($\gS_{g,\nu}$ est toujours infini), car le groupe $\gS_{g,\nu}
\subset \hat{\hat\gS}_{g,\nu}$ n'est pas fermé.  On désigne
par ${\hbox{Discrét}}'(\pi)$ l'espace topologique séparé 
associé, s'identifiant à ${\rm Isom}_{\rm lac}(\hat\pi_{g,\nu},
\pi)/\overline\gS_{g,\nu}$, où $\overline\gS_{g,\nu}$
désigne l'adhérence de $\gS_{g,\nu}$ dans $\hat{\hat\gS}
_{g,\nu}$.  On a d'ailleurs un homomorphisme 
évident $\hat\gS_{g,\nu}\to\hat{\hat\gS}_{g,\nu}$
dont l'image est $\overline \gS_{g,\nu}$, nous admettrons qu'il est injectif 
et identifierons $\overline\gS_{g,\nu}$ à $\hat\gS_{g,\nu}$.  Ainsi
$$
{\hbox{Discrét}}'(\pi) \isom {\rm Isom}_{\rm lac}(\hat\pi
_{g,\nu},\pi)/\hat\gS_{g,\nu}.\leqno{(8)}
$$
Un élément de ${\hbox{Discrét}}'(\pi)$ s'appelle une
classe de dis\-crétifi\-ca\-tions de $\pi$ (ou pré\-discréti\-fica\-tion
de $\pi$). Si $\pi_0$ est une discrétification, on
désigne sa classe par $\pi_0^\natural$.  L'ensemble des classes de
discrétifications de $\pi$ est un espace homogène
sous $\hat{\hat\gS}(\pi)$, le stabilisateur de $\pi_0^\natural$
s'identifiant à l'adhérence de $\gS(\pi_0)=
{\rm Aut}_{\rm lac}(\pi_0)$ dans $\hat{\hat\gS}(\pi)$,
ou encore à $\hat\gS(\pi_0)$.  Celle-ci contient toujours $\pi$.
$$
{\rm Aut}_{\rm lac}(\pi,\pi_0^\natural) \isom \hat\gS(\pi_0).\leqno(9)
$$
Pour toute [pré?]discrétification $\pi_0^\natural$, le sous-groupe de
$\hat{\hat\gS}(\pi)$ des automorphismes à lacets qui fixent $\pi_0^\natural$
est noté $\hat\gS(\pi_0^\natural)$.  C'est donc l'image inverse
d'un sous-groupe de $\hat{\hat\gT}(\pi)$, noté \break
$\hat\gT(\pi_0^\natural)$.\footnote{NB. L'ensemble des classes de discrétification de $\pi$
se décrit en termes de groupes {\it extérieurs} définis par $\pi$
comme ${\rm Isomext}_{\rm lac}(\hat\pi_{g,\nu},\pi$) divisé par
$\hat\gT_{g,\nu}$; on a d'ailleurs une application de degré 2
${\rm Isomext}(\hat\pi_{g,\nu},\pi)/\hat\gT_{g,\nu}^+ \to
{\rm Isomext}(\hat\pi_{g,\nu},\pi)/\hat\gT_{g,\nu}$, ce qui permet pour 
toute classe de discrétification de définir ses deux {\it orientations} 
et de parler des classes de discrétification {\it orientées}.}
On a donc une inclusion de structures d'extensions:
\[\begin{tikzcd}
	1 & \pi & {\hat\gS (\pi^\natural_0)} & {\hat\gT (\pi^\natural_0)} & 1 \\
	1 & \pi & {\hat\gS (\pi)} & {\hat\gT (\pi)} & {1.}
	\arrow[from=1-1, to=1-2]
	\arrow[from=1-2, to=1-3]
	\arrow[from=1-3, to=1-4]
	\arrow[from=2-1, to=2-2]
	\arrow[from=2-2, to=2-3]
	\arrow[from=2-3, to=2-4]
	\arrow[from=2-4, to=2-5]
	\arrow[from=1-4, to=1-5]
	\arrow[hook', from=1-3, to=2-3]
	\arrow[hook', from=1-4, to=2-4]
\end{tikzcd}\leqno{(10)}\]
(On montre par voie arithmético-géométrique que l'inclusion
$\hat\gT\to\hat{\hat\gT}$ n'est jamais un isomorphisme).
On trouve ainsi une application 
$$
{\hbox{Discrét}}'(\pi)\to{\hbox{sous-groupes fermés de}}\ 
\hat{\hat\gT}(\pi),\leqno{(11)}
$$
$$
[\pi_0^\natural\ \mapsto \hat\gT(\pi_0^\natural)]
$$
évidemment compatible à l'action de $\hat{\hat\gS}(\pi)$
(transport de structure) -- opérant à droite via
$\hat{\hat\gT}(\pi)$ et ses automorphismes intérieurs sur lui-même.%${}^3$

On trouve ainsi une classe de conjugaison bien déterminée de
sous-groupes fermés de $\hat{\hat\gT}(\pi)$, qu'on appelle ses
sous-groupes de Teichmüller ``{\it géométriques}''.  Il se pourrait
d'ailleurs que $\hat\gT_{g,\nu}$ soit son propre normalisateur
dans $\hat{\hat\gT}_{g,\nu}$ ce qui équivaut à l'assertion
que (11) est bijective: la donnée d'une classe de discrétifications
de $\pi$ serait équivalente à celle d'un sous-groupe
de Teichmüller géométrique dans son groupe de
Teichmüller étendu $\hat{\hat\gT}$.\footnote{c'est 
complètement déconnant et 
ultra-faux; un moment d'égarement! Cela appara\^\i t clairement
par la suite\dots La question judicieuse (avec laquelle j'ai d\^u sur
le coup confondre) c'est si $\hat\gT$ est {\it invariant} dans
$\hat{\hat\gT}$, i.e. si le normalisateur est $\hat{\hat\gT}$ tout entier.}

Notons qu'on a des homomorphismes canoniques
$$
\hat{\hat\gT}\ {\buildrelHi\over\to}\ \hat{\mathbf{Z}^*}\leqno{(12)}
$$
$$
\hat{\hat\gT} \to \gS_I\leqno(13)
$$
(où $I=I(\pi)$ est l'ensemble des classes de conjugaison des
sous-groupes à lacets de $\pi$), induisant sur
$\hat\gT(\pi_0^\natural)$ des homomorphismes correspondants -- d'où
la définition des sous-groupes ${\hat{\hat\gT}}^!$,
${\hat{\hat\gT}}^+$, ${\hat{\hat\gT}}^{!+}$ et de même pour $\hat\gT$.
Notons que $Hi$ ne prend sur $\hat\gT$ que les valeurs $\pm 1$.%${}^4$

Le sous-groupe $\hat\gT^+$ des automorphismes extérieurs
du groupe à lacets profinis $\pi$, fixant une classe
de discrétifications $\pi_0^\natural$ et de multiplicateur
$+1$, joue un rôle très particulier.  A l'opposé de ce
qu'on peut conjecturer sur $\hat\gT$ (dont $\hat\gT^+$ est
un sous-groupe ouvert d'indice 2), on supposerait plutôt
que $\hat\gT^+$ est invariant dans $\hat{\hat\gT}$ [voir note en bas de page
précédente].  En tout état de cause, les sous-groupes ainsi définis 
dans $\hat{\hat\gT}$ via classes de discrétification de $\pi$
(peut-être n'y en a-t-il qu'un seul et unique!)${}^5$ s'appelleront
les sous-groupes de Teichmüller géométriques 
{\it stricts}.  Le choix d'un tel sous-groupe de $\hat{\hat\gT}$
est, en tout état de cause, un élément de structure nettement
plus faible que celui d'une classe de discrétification, et
même que celui d'un sous-groupe de Teichmüller géométrique
(pas strict).  Pour préciser les relations entre ces
deux notions, rappelons d'abord que $\hat\gT^+=\Sigma$
se déduit de $\hat\gT$ comme noyau de $Hi|\hat\gT:\hat\gT
\to\{\pm 1\}$.  D'autre part (pour un sous-groupe de
Teichmüller géométrique strict choisi $\hat\gT^+$), considérons
$$\cN_\Sigma=\cN={\rm Norm}_{{\hat{\hat\gT}}}\ (\Sigma)\leqno{(14)}$$
où $\Sigma=\hat\gT^+$ ([$\cN_\Sigma$ est] peut-être toujours égal
à $\hat{\hat\gT}$ tout entier), évidemment (si $\Sigma$ provient
d'un $\hat\gT$)
$$\Sigma = \hat\gT^+ \subset \hat\gT\subset \cN_\Sigma.\leqno{(15)}$$
Le groupe profini $\cN/\Sigma$ associé à $\Sigma$ se note
$\GG_\Sigma$ (si $\Sigma$ est unique, on note $\GG_\pi$),
et les $\hat\gT$ donnant naissance au même $\Sigma$ 
correspondent à une classe de conjugaison d'éléments
d'ordre 2 de $\GG_\Sigma$.  Il est clair en tous cas qu'on
a une application injective
$$
\{\hbox{ensemble des sous-groupes de Teichmüller géométriques\ }
\hat\gT\ {\rm dans}\ \hat{\hat\gT}\ {\rm tels\ que}\ \hat\gT^+=\Sigma\}
$$
\centerline{$\downarrow$}
$$
\{\hbox{ensemble des éléments d'ordre 2 dans\ }\GG_\Sigma\}
\leqno{(16)}
$$
et que son image est stable par conjugaison.  Montrons que deux
éléments de l'image sont conjugués dans $\GG_\Sigma$.
En effet, soient $\hat\gT$, $\hat\gT'\supset\Sigma$ tels que
$\Sigma=\hat\gT^+=\hat\gT'^+$.  Il existe $g\in\hat{\hat\gT}$ tel
que ${\hat\gT}'={\rm Int}(g)\hat\gT$, et on aura alors 
$\hat\gT'^+={\rm Int}(g)\hat\gT^+$, i.e. $g\in\cN$, OK. 

Les éléments d'ordre 2 ainsi obtenus dans $\GG_\Sigma$
s'appellent les involutions canoniques. On en donnera une
interprétation conjecturale remarquable plus bas.  L'ensemble
$\hat{\hat\gT}/\cN$ s'identifie à l'ensemble des sous-groupes de
Teichmüller géométriques stricts (une fois choisi
l'élément ``origine'' $\Sigma$). Plus intrinsèquement,
on aura:
$$
\{\hbox{Ensemble des sous-groupes de Teichmüller géométriques 
stricts de}\ \hat{\hat\gT}(\pi)\}
$$
$$\uparrow$$
$$
{\rm Isom}_{\rm lac}(\hat\pi_{g,\nu},\pi)/\cN_{g,\nu}\leqno{(17)}
$$
où on pose, comme de juste,
$$
\cN_{g,\nu}={\rm Norm}_{{\hat{\hat\gT}}_{g,\nu}}(\hat\gT^+_{g,\nu})\leqno{(18)}
$$
(peut-être égal à $\hat{\hat\gT}_{g,\nu}$ tout entier !).
\vskip .3cm
Considérons maintenant le {\it topos modulaire sur} $\mathbf{Q}$ des courbes
algébriques de type $(g,\nu)$, noté $M_{g,\nu,\mathbf{Q}}$ ou simplement
$M_{g,\nu}$.  Si nous choisissons un revêtement universel
$\widetilde{M_{g,\nu}}$ (d'où un revêtement universel de Spec$\,\mathbf{Q}$,
i.e. une clôture algébrique $\overline{\mathbf{Q}}$ de $\mathbf{Q}$),
on peut préciser le $\pi_1(M_{g,\nu,\mathbf{Q}})$ comme le groupe
des $M_{g,\nu,\mathbf{Q}}$-automorphismes de ce revêtement et
l'appeler le {\it groupe de Teichmüller arithmétique}
de type $(g,\nu)$ (relatif au choix de $\widetilde{M_{g,\nu,\mathbf{Q}}}$).
On aura donc une suite exacte
$$
1 \to \pi_1(M_{g,\nu,\overline{\mathbf{Q}}}) \to \pi_1(M_{g,\nu})
\to {\rm Gal}(\overline{\mathbf{Q}}/\mathbf{Q}) \to 1,\leqno(19)
$$
(où $\overline{\mathbf{Q}}$ et les points base sont explicités comme
il a été dit).  Notons que par la théorie transcendante
de Teichmüller on a un isomorphisme (défini modulo l'opération induite
par $\pi_1(M_{g,\nu})$ sur $\pi_1(M_{g,\nu,\overline{\mathbf{Q}}})\,$):
$$
\pi_1(M_{g,\nu,\overline{\mathbf{Q}}}) \isom \hat\gT^+_{g,\nu}\leqno{(20)}
$$
(où $\hat\gT^+_{g,\nu}$ est le compactifié 
profini de $\gT^+_{g,\nu}$).  

Considérons d'autre part le schéma $U_{g,\nu}$ sur
$M_{g,\nu}$, courbe de type $(g,\nu)$ ``universelle''.  On a donc,
en choisissant un revêtement universel $\widetilde{U_{g,\nu}}$
de celle-ci {\it au-dessus} du revêtement universel choisi
$\widetilde{M_{g,\nu}}$ de $M_{g,\nu}$, un diagramme commutatif:
\[\begin{tikzcd}
	1 & \pi & {\pi_1(U_{g, \nu})} & {\pi_1(M_{g, \nu})} & 1 \\
	1 & \pi & {\hat{\hat\gS}(\pi)} & {\hat{\hat\gT}(\pi)} & 1
	\arrow[from=1-1, to=1-2]
	\arrow[from=2-1, to=2-2]
	\arrow[from=1-2, to=1-3]
	\arrow[from=2-2, to=2-3]
	\arrow[from=1-3, to=1-4]
	\arrow[from=2-3, to=2-4]
	\arrow[from=1-4, to=1-5]
	\arrow[from=2-4, to=2-5]
	\arrow[from=1-4, to=2-4]
	\arrow[from=1-3, to=2-3]
	\arrow["\sim"', from=1-2, to=2-2]
\end{tikzcd}\leqno{(21 - 22)}\]
et de même pour $U_{g,\nu,\overline{\mathbf{Q}}}\to M_{g,\nu,
\overline{\mathbf{Q}}}$:
$$
1\to\pi\to\pi_1(U_{g,\nu,\overline{\mathbf{Q}}})\to
\pi_1(M_{g,\nu,\overline{\mathbf{Q}}})\to 1.\leqno{(21')}
$$
On a bien s\^ur un isomorphisme de groupes à lacets
$$
\pi \isom \hat\pi_{g,\nu}\leqno{(23)}
$$
dont on voudrait déterminer l'indétermination de
fa\c con précise.%${}^7$

Notons $M_{g,\nu,\mathbf{C}}$ le topos modulaire de Teichmüller
complexe, défini à l'aide de la surface $C^\infty\ $
$U_{g,\nu}$; il est muni du revêtement universel canonique
$\widetilde{M_{g,\nu,\mathbf{C}}}$ (\S 21; $\widetilde{M_{g,\nu,\mathbf{C}}} \isom 
E_g/A_{g,\nu}^o$) de groupe d'automorphismes
$\gT^+_{g,\nu}$ justement -  $U_{g,\nu,\mathbf{C}}$ étant lui aussi
muni d'un revêtement universel au-dessus du précédent,
($\widetilde{U_{g,\nu,\mathbf{C}}}=E_g/A_{g,\nu_+1}^o$).  Ceci donne
naissance aux suites exactes des groupes discrets
(dans le contexte topologique général)
\[\begin{tikzcd}
	1 & {\pi_{g, \nu}} & {{\gS^+_{g, \nu}}} & {\gT^+_{g, \nu}} & 1 \\
	1 & {\pi_{g, \nu}} & {\Aut_{\text{lac}}(\pi_{g, \nu})} & {\Autext_{\text{lac}}(\pi_{g, \nu})} & 1
	\arrow[from=1-1, to=1-2]
	\arrow[from=2-1, to=2-2]
	\arrow[from=1-2, to=1-3]
	\arrow[from=2-2, to=2-3]
	\arrow[from=1-3, to=1-4]
	\arrow[from=2-3, to=2-4]
	\arrow[from=1-4, to=1-5]
	\arrow[from=2-4, to=2-5]
	\arrow[from=1-4, to=2-4]
	\arrow[from=1-3, to=2-3]
	\arrow["\sim"', from=1-2, to=2-2]
\end{tikzcd}\leqno{(24)}\]
(la première suite s'envoyant dans la seconde par un isomorphisme
de structures d'extensions [avec les identifications]
$\gS^+_{g,\nu}=\pi_1(U_{g,\nu,\mathbf{C}})$ et
$\gT^+_{g,\nu}=\pi_1(M_{g,\nu,\mathbf{C}})$)
qui par passage aux complétés profinis
donne la suite exacte analogue pour les multiplicités algébriques sur $\mathbf{C}$.
\[\begin{tikzcd}
	1 & {\pi_{g, \nu}} & {{\hat{\gS}^+_{g, \nu}}} & {\hat{\gT}^+_{g, \nu}} & 1 \\
	& {\pi_{g, \nu}} & {\Aut_{\text{lac}}(\hat{\pi}_{g, \nu})} & {\Autext_{\text{lac}}(\hat{\pi}_{g, \nu})} & {1.}
	\arrow[from=1-1, to=1-2]
	\arrow[from=1-2, to=1-3]
	\arrow[from=2-2, to=2-3]
	\arrow[from=1-3, to=1-4]
	\arrow[from=2-3, to=2-4]
	\arrow[from=1-4, to=1-5]
	\arrow[from=2-4, to=2-5]
	\arrow[from=1-4, to=2-4]
	\arrow[from=1-3, to=2-3]
	\arrow[shift left=1, from=1-2, to=2-2]
	\arrow[shorten <=3pt, shorten >=2pt, no head, from=1-2, to=2-2]
\end{tikzcd}\leqno{(25)}\]
Cette situation provient d'ailleurs canoniquement d'une
situation analogue sur $\overline{\mathbf{Q}}_0=$ clôture algébrique
de $\mathbf{Q}$ dans $\mathbf{C}$, le choix de $\widetilde{U_{g,\nu,\mathbf{C}}}$
définissant un $\widetilde{U_{g,\nu,\overline{\mathbf{Q}}_0}}$,
d'où un $\widetilde{M_{g,\nu,\overline{\mathbf{Q}}_0}}$.  Pour définir
un isomorphisme entre cette suite (25) et la suite exacte
(21') (munie canoniquement de deux homomorphismes dans (22)) il faut 
\begin{enumerate}
 
    \item[1)] choisir un isomorphisme $\overline{\mathbf{Q}} \isom \overline{\mathbf{Q}_0}$ -- l'indétermination est dans ${\rm Gal}(\overline{\mathbf{Q}}/\mathbf{Q})$;
ceci permet d'identifier $\overline{\mathbf{Q}}$ et $\overline{\mathbf{Q}}_0$;
\vskip .2cm
    \item[2)] Choisir un isomorphisme (sur $\overline{\mathbf{Q}}=
\overline{\mathbf{Q}}_0$) de $(\widetilde{U_{g,\nu}})_0$ 
(construit par voie transcendant sur $\mathbf{C}$ et descendu
à $\overline{\mathbf{Q}}_0$) avec $\widetilde{U_{g,\nu,\overline{\mathbf{Q}}}}$ --
l'indétermination est dans $\pi_1(M_{g,\nu,\overline{\mathbf{Q}}})$.
En résumé, on a une classe d'isomorphismes de
(21) et (25), définie modulo automorphismes intérieurs
dans le groupe profini $\pi_1(U_{g,\nu})$.  Une fois
choisi l'isomorphisme $\overline{\mathbf{Q}} \isom \overline{\mathbf{Q}}_0$,
les isomorphismes correspondants transforment la 
{\it classe de discrétifications orientée} standard
de $\hat\pi_{g,\nu}$ en exactement une classe de 
discrétifications orientée de $\pi$.
\end{enumerate}

Changeant le choix de l'isomorphisme en son complexe conjugué,
on trouve la quasi-discrétification orientée opposée.%${}^8$

On voit d'autre part que l'image de $\pi_1(M_{g,\nu,\overline{\mathbf{Q}}})\subset 
\pi_1(M_{g,\nu})$ dans $\hat{\hat\gT}(\pi)$ n'est autre que le
$\hat\gT^+$ correspondant à une quelconque des classes
de quasi-rigidification discrètes choisies -- le sous-groupe ne dépend
pas du choix de cette classe -- ce qui ne fait qu'exprimer que l'image
de $\pi_1(M_{g,\nu,\overline{\mathbf{Q}}})$ dans $\hat{\hat\gT}(\pi)$, considérée
comme sous-groupe de l'image du groupe plus grand $\pi_1(M_{g,\nu})$,
y est invariante -- ce qui résulte du fait que $\pi_1(M_{g,\nu,\overline{\mathbf{Q}}})$ est invariant dans $\pi_1(M_{g,\nu,\mathbf{Q}})$.  Ainsi
$\pi$ est muni canoniquement (non d'une quasi-rigidification discrète
orientée, ce qui dépend du choix d'un isomorphisme $\overline{\mathbf{Q}}
 \isom \overline{\mathbf{Q}}_0$, i.e. d'un plongement $\overline{\mathbf{Q}}
\hookrightarrow \mathbf{C}$), mais du moins d'un groupe de Teichmüller
géométrique strict $\Sigma\subset \hat{\hat\gT}(\pi)$, à
savoir l'image de $\pi_1(M_{g,\nu,\overline{\mathbf{Q}}})$ (isomorphe à [??])
Ceci posé, on a un homomorphisme canonique d'extensions de groupes
\[\begin{tikzcd}
	1 & {\pi_1(M_{g, \nu,\overline{\mathbf{Q}}})} & {\pi_1(M_{g, \nu})} & {\Gal(\overline{\mathbf{Q}}/\mathbf{Q})} & 1 \\
	1 & \Sigma & {\cN_\Sigma} & {\GG_\Sigma} & {1,}
	\arrow[from=1-1, to=1-2]
	\arrow[from=1-2, to=1-3]
	\arrow[from=2-2, to=2-3]
	\arrow[from=1-3, to=1-4]
	\arrow[from=2-3, to=2-4]
	\arrow[from=1-4, to=1-5]
	\arrow[from=2-4, to=2-5]
	\arrow[from=1-4, to=2-4]
	\arrow[from=1-3, to=2-3]
	\arrow[shift left=1, from=1-2, to=2-2]
	\arrow[shorten <=3pt, shorten >=2pt, no head, from=1-2, to=2-2]
	\arrow[from=2-1, to=2-2]
\end{tikzcd}\leqno{(26)}\]
où $\cN_\Sigma={\rm Norm}_{\hat{\hat\gT}(\pi)}(\Sigma)$ et
$\GG_\Sigma$ est le groupe des automorphismes extérieurs
``arithmétiques'' de $\pi$ (muni de $\Sigma$).

Pour le choix d'un isomorphisme $\overline{\mathbf{Q}} \isom \overline{\mathbf{Q}}_0$,
i.e. d'un plongement $\overline{\mathbf{Q}}\hookrightarrow \mathbf{C}$, l'image
de la conjugaison complexe de ${\rm Gal}(\overline{\mathbf{Q}}_0/\mathbf{Q})$
n'est autre que l'élément d'ordre 2 de $\GG_\Sigma$
associé au sous-groupe de Teichmüller géométrique, associé
à la quasi-discrétification (orientée, mais peu importe)
canonique sur $\pi$ (définie par $\overline{\mathbf{Q}} \isom \overline{\mathbf{Q}}_0$).

La conjecture naturelle ici, c'est que
$${\rm Gal}(\overline{\mathbf{Q}}/\mathbf{Q})\to\GG_\Sigma\leqno(27)$$
du groupe de Galois vers les automorphismes extérieurs arithmétiques
de $(\pi,\Sigma)$ soit un isomorphisme.  On peut la compléter 
par la conjecture, encore plus hardie, que de plus $\Sigma$ est
invariant dans $\hat{\hat\gT}(\pi)$ ($\Leftrightarrow \Sigma_{g,\nu}$
invariant dans $\hat{\hat\gT}_{g,\nu}$), ce qui signifierait donc que
$\cN_\Sigma=\hat{\hat\gT}$, ou encore que\footnote{conjectural!}
$$
\pi_1(M_{g,\nu}){\buildrel\sim\over\to}
\hat{\hat\gT}(\pi)={\rm Autext}_{\rm lac}(\pi) \leqno{(28)}
$$
et ${\rm Gal}(\overline{\mathbf{Q}}/\mathbf{Q})$ s'identifierait au quotient 
``arithmétique'' du groupe des automorphismes extérieurs
à lacets de $\pi$.

Soit maintenant $U$ une courbe algébrique de type $g,\nu$
sur $\overline{\mathbf{Q}}$.  Elle est définie par un point de
$M_{g,\nu,\overline{\mathbf{Q}}}$ sur ${\rm Spec}\,  \overline{\mathbf{Q}}$,
et comme fibre de $U_{g,\nu}$ en ce point.  Choisissons
un revêtement universel de $U$, d'où un groupe
$\pi(U)$, canoniquement isomorphe au groupe $\pi$ ci-dessus, une
fois choisi un isomorphisme entre $\widetilde{U_{g,\nu}}$ et
le revêtement universel de $U_{g,\nu}$ déduit de
$\widetilde{U}$, ce qui donne une indétermination dans 
$\pi_1(U_{{g,\nu},\overline{\mathbf{Q}}}) \isom \hat\gS^+$.
Ainsi, une fois choisi un isomorphisme $\overline{\mathbf{Q}}
 \isom \overline{\mathbf{Q}}_0$, on trouve sur $\pi_1(U)$ une
quasi-discrétification orientée [voir ${}^8$] (évidente d'ailleurs
à définir directement, par voie transcendante), changée en son opposée
par changement de l'isomorphisme $\overline{\mathbf{Q}} \isom 
\overline{\mathbf{Q}}_0$ par conjugaison complexe.  D'autre part,
le sous-groupe $\hat\gT^+$ de $\hat{\hat\gT}(\pi)$,
correspondant à cette quasi-discrétification (orientée,
peu nous chaut) ne dépend pas du choix de l'isomorphisme
en question $\overline{\mathbf{Q}} \isom \overline{\mathbf{Q}}_0$, il
est canoniquement associé à $\pi$ en tant que groupe
fondamental d'un $U$ sur un corps algébriquement
clos (N.B. on pourrait prendre un corps algébriquement
clos de caractéristique zéro quelconque, pas la peine
que ce soit sur $\overline{\mathbf{Q}}$, cf. plus bas).

Si la conjecture sur $\GG$ est vérifiée, il en
résulterait que la donnée d'un groupe profini à lacets
$\pi$ de type $(g,\nu)$, muni d'un groupe de Teichmüller
géométrique strict $\Sigma\subset \hat{\hat\gT}(\pi)$
(ce qui peut-être n'est pas une structure supplémentaire
du tout -- $\Sigma$ serait uniquement déterminé
par $\pi$!) équivaudrait à la donnée
d'une extension algébriquement close $\overline{\mathbf{Q}}$
de $\mathbf{Q}$ (dont le groupe de Galois serait $\cN_{{\hat{\hat\gT}}}
(\Sigma)/\Sigma=\GG_\Sigma$) et le $\GG_0$-torseur ${\rm Isom}
(\overline{\mathbf{Q}},\overline{\mathbf{Q}}_0)$ s'identifierait à 
l'ensemble ($\subset {\rm Isom}(\hat\pi_{g,\nu},\pi)/\hat\gT^\circ_{g,\nu}$)
des quasi-rigidifications {\it orientées} donnant naissance à
$\pi$\dots) et d'un revêtement universel de $U_{g,\nu,\overline{\mathbf{Q}}}$
ou simplement d'un revêtement universel $\widetilde{U_{g,\nu}}$
de $U_{g,\nu}=U_{g,\nu,\mathbf{Q}}$.
\footnote{si on se donne seulement $\pi$ comme groupe extérieur,
avec $\Sigma$, ce qui revient à la donnée d'un revêtement
universel de $M_{g,\nu}$.}
La donnée d'un isomorphisme
$\overline{\mathbf{Q}} \isom \overline{\mathbf{Q}}_0$ revient donc à celle
d'une quasi-rigidification orientée sur $\pi$ compatible
avec $\Sigma$ (condition peut-être automatiquement
satisfaite\dots).  Ceci dit, la donnée d'une courbe
algébrique de type $g,\nu$ sur un corps algébriquement
clos $\overline{\mathbf{Q}}$, et d'un revêtement universel de celle-ci
reviendrait à la donnée d'une donnée précédente
(à savoir un revêtement universel d'un $M_{g,\nu}$,
{\it plus} un point de $M_{g,\nu,\overline{\mathbf{Q}}}$ sur $\overline{\mathbf{Q}}$
définissant un revêtement universel)
\footnote{NB. Si on renonce à choisir $\widetilde{U}$, ceci
revient à travailler avec les groupes à lacets {\it extérieurs}.}
, et cette dernière
donnée s'identifierait à un germe de scindage dans l'extension
$$1\to\pi_1(M_{g,\nu,\overline{\mathbf{Q}}})\to
\pi_1(M_{g,\nu})\to\pi_1(\mathbf{Q})\to 1.$$
L'interprétation profinie (en termes des groupes {\it extérieurs}
à lacets $\pi$) est alors un
$(\pi,\Sigma)$, et un germe de scindage de 
$$1\to\Sigma\to\cN_\Sigma\to
\GG_\Sigma\to 1.$$
Si on veut une courbe sur une sous-extension finie $K'$
de $\overline{\mathbf{Q}}/\mathbf{Q}$, correspondant à un sous-groupe
d'indice fini $\Gamma'\subset \GG_\Sigma$, il s'agira 
d'un scindage partiel $\Gamma'\to\cN_\Sigma$.

Mais s\^urement, si cette description des courbes
algébriques anabéliennes est pleinement fidèle, elle
n'est pas 2-fidèle, i.e. il faut des
conditions sur ce scindage, qu'il faudra examiner
par la suite. 

La description d'une courbe algébrique de type $g,\nu$
sur un corps fixé, de type fini sur $\mathbf{Q}$, $K$ quelconque,
muni d'une extension algébriquement close $\overline K$
(donc $\GG={\rm Gal}(\overline K/K)\to
\GG_0={\rm Gal}(\overline{\mathbf{Q}}/\mathbf{Q})$, où $\overline{\mathbf{Q}}$
est la clôture algébrique de $\mathbf{Q}$ dans $\overline K$),
en termes de $\GG\leftarrow\GG_0$, serait la suivante:
donnée d'un groupe extérieur à lacets $\pi$
de type $g,\nu$, d'un sous-groupe de Teichmüller
$\Sigma$ dans $\hat{\hat\gT}(\pi)$, d'un isomorphisme
$\overline{\mathbf{Q}} \isom $ corps défini par cette situation
(ayant comme groupe de Galois $\cN_\Sigma/\Sigma$, de
sorte qu'on trouve un isomorphisme $\GG_0 \isom 
\cN_\Sigma/\Sigma$), enfin d'un relèvement de
$\GG\to\GG_0$ (qui décrit une action
arithmétiquement extérieure de $\GG$ sur $(\pi,\Sigma)$)
en une action extérieur $\GG\to\cN_\Sigma$
de $\GG$ sur $(\pi,\Sigma)$ avec des conditions qu'il
faudra essayer de dégager (nécessaires et
suffisantes conjecturalement) sur ce relèvement.

Notons que tout plongement $\overline K\hookrightarrow
\mathbf{C}$ définit canoniquement sur le groupe extérieur $\pi=
\pi_1(U_{\overline K})$ une quasi-discrétification
orientée, remplacée par l'opposée quand on remplace
le plongement par le complexe conjugué.  Je dis que
ces quasi-discrétifications définissent toutes
$\Sigma$ comme groupe des automorphismes et que la 
quasi-discrétification définie par $\overline K
\hookrightarrow \mathbf{C}$ ne dépend que de sa restriction en
$\overline{\mathbf{Q}}\hookrightarrow \mathbf{C}$ -- de fa\c con plus
précise, c'est celle qu'on définit directement à l'aide
de $\overline{\mathbf{Q}}\hookrightarrow \mathbf{C}$, i.e. de
$\overline{\mathbf{Q}} \isom \overline{\mathbf{Q}}_0$.  Mais il y a lieu
d'examiner aussi la fa\c con d'obtenir des {\it discrétifications}.
Ainsi, si un $U$ est défini sur $\overline{\mathbf{Q}}_0$, [c'est]  un $\pi$
extérieur de type $(g,\nu)$, muni d'une quasi-discrétification
orientée $\pi_0^{\natural +}$, et d'un germe de scindage de
$$1\to\Sigma\to\cN_\Sigma\to\GG_\Sigma
\to 1,$$
où $\Sigma={\rm Autext}_{\rm lac}(\pi,\pi_0^{\natural+})$, qui
fait opérer extérieurement le noyau du groupe défini par $\GG_\Sigma$
sur $\pi$, les points rationnels sur $\overline{\mathbf{Q}}_0$
correspondant aux classes de $\pi$-conjugaison des germes
de relèvement de cette opération extérieure en une
vraie opération.  Supposons donc donné un tel point,
i.e. on a un sous-groupe ouvert $\GG\subset \GG_\Sigma$
qui opère bel et bien sur $\pi$, l'opération définie
mod automorphismes intérieurs (lui-même unique car $\pi^\Gamma
=\{1\}$!)  Dans cette situation, il faudrait définir
dans $\pi \isom \pi_1(U_P,P)$ une {\it discrétification
orientée} $\pi_0\subset \pi$, et pas seulement une
quasi-discrétification orientée!  (Bien s\^ur, elle
doit être dans la classe qu'on s'est donnée d'avance
de discrétifications orientées, qu'on avait justement
notée $\pi_0^{\natural+}$.  On y reviendra -- ainsi qu'à 
la situation analogue sur un corps de base $K$ de type
quelconque\dots)

Je réfère au début du \S\ suivant (\S 27) pour le changement
de terminologie, la ``quasi-discrétification'' devenant
une ``prédiscrétification''.  Mais il y a lieu
d'introduire aussi une notion plus fine, correspondant
au cas d'un $\pi_1$ (profini) d'une variété algébrique
$X$ définie sur un corps algébriquement clos 
$\overline K$, quand on plonge $\overline K$ dans $\mathbf{C}$: 
il y a une discrétification mod automorphismes intérieurs --
on l'appellera une {\it prédiscrétification stricte}.  Dans le
cas d'un $\pi$ à lacets, l'espace homogène sous
$\hat{\hat\gT}(\pi)$ de celles-ci s'identifie à
$${\rm Isom}_{\rm lac}(\hat\pi_{g,\nu},\pi)/(\hat\pi_{g,\nu}Dot
\gT_{g,\nu}) \isom {\rm Isomext}_{\rm lac}(\hat\pi_{g,\nu},\pi)/
\gT_{g,\nu},$$ 
et cette action également ne dépend que du
groupe profini à lacets extérieur défini par
$\pi_{g,\nu}$, et équivaut à celle d'une ``base extérieure''
de $\pi$ mod action  de $\gT_{g,\nu}$, i.e. d'un
isomorphisme de $\pi$ avec le $\hat\pi_{g,\nu}$ ``type'',
mod action de $\gT_{g,\nu}$.  L'application
\vskip .2cm
\centerline{Bases extérieures de $\pi$ $\to$ Prédiscrétifications
de $(\pi)$}
\vskip .2cm
\noindent fait donc du membre de gauche un torseur relatif à droite sur celui
de droite, de groupe $\gT_{g,\nu}$.

Une prédiscrétification de $(\pi,\Sigma)$ correspond à un choix
d'un isomorphisme de la clôture algébrique
$\overline{\mathbf{Q}}_{(\pi,\Sigma)}=\overline{\mathbf{Q}}$ associé à
$(\pi,\Sigma)$, avec $\overline{\mathbf{Q}}_0$.   Quand on se donne
$(\pi,\Sigma)$ comme correspondant à une courbe algébrique
sur $\overline{\mathbf{Q}}$, i.e. qu'on se donne un germe de relèvement
de $\GG_\Sigma$ dans $\cN_\Sigma\subset \hat{\hat\gT}(\pi)$,
i.e. un germe d'actions extérieures de $\GG_\Sigma$
sur $\pi$, toute prédiscrétification doit donner naissance à une
discrétification stricte et même à une discrétification
quand on remonte un germe d'action (``admissible'') de
$\GG_\Sigma$ sur $\pi$\dots















%%%%%%%%%%%%%%%%%%%%%%%%%%%%%%%%%%%%%%%%%%%%%%%%%%%%%%%%%%%%%%%
\chapter*{\S \space 27. --- CHANGEMENT DE TYPE $(g,\nu)$: a) BOUCHAGE DE TROUS}\thispagestyle{empty}
\addcontentsline{toc}{section}{27. Changement de type $(g,\nu)$: a) Bouchage de trous (et diagrammes remarquables p.174)}
\label{sec:27}
\section*{}

Je vais changer de terminologie, en appelant pré\-discréti\-fica\-tion
ce que j'avais appelé (un peu péjorativement!) quasi-dis\-crétifi\-cation.
En effet, on prévoit qu'une pré\-discrétifi\-cation -- compatible
avec $\Sigma$ -- qui revient moralement au plongement (modulo la conjugaison
complexe) dans $\mathbf{C}$ de la clôture algébrique $\overline{\mathbf{Q}}$
de $\mathbf{Q}$ canoniquement associé à $\bigl(\pi,\Sigma(\subset 
\hat{\hat\gT}(\pi))\bigr)$, donne naissance, dès que l'action
extérieure arithmétique de $\GG_\Sigma=
\cN_\Sigma/\Sigma$ sur $\pi$ est relevée en un germe 
d'action sur $\pi$, à une vraie discrétification de
$\pi$.  Itou pour les prédiscrétifications orientées.  Le
groupe $\cN_\Sigma/\Sigma$ mérite aussi un nom -- je vais
l'appeler le {\it groupe de Teichmüller arithmétique}
associé à $(\pi,\Sigma)$\footnote{moralement, $\Sigma=\hat\gT^+$, mais il n'y a pas de $\hat\gT^!$}
, et ses éléments seront
appelés les automorphismes arithmétiquement extérieurs
de $\pi$ -- déduits d'un automorphisme extérieur
(norma\-li\-sant $\Sigma$) en négligeant les automorphismes
extérieurs ``géométriques'' (i.e. justement ceux dans
$\Sigma$) -- comme les automorphismes extérieurs ordinaires
étaient décrits en négligeant les automorphismes intérieurs
de $\pi$.  On fera attention que le caractère ``multiplicateur''
$$
Hi:\hat{\hat\gT}(\pi)\to \hat{\mathbf{Z}}^*
$$
est trivial sur $\Sigma$, par construction de $\Sigma$, et
passe à $\GG_\Sigma$:
$$
Hi_{{}\atop\Sigma}:\GG_\Sigma \to \hat{\mathbf{Z}}^*.
$$
Bien s\^ur, ce caractère s'appelera encore ``multiplicateur'',
ou ``caractère cyclotomique''.  Si notre conjecture
fondamentale est vraie${}^1$, ce caractère identifie
$(\GG_\Sigma)_{\rm ab}$ à $\hat{\mathbf{Z}}^*$.
Par contre, si $I=I(\pi)$, l'homomorphisme canonique
$$\hat{\hat\gT}\to \gS_I$$
n'est pas trivial sur $\Sigma$, mais induit un homorphisme surjectif:
$$\Sigma\to \gS_I,$$
d'où un sous-groupe $\Sigma^!$ tel que
$$
\Sigma/\Sigma^!\buildrel\sim\over\to\gS_I.
\footnote{moralement, $\Sigma^! \isom  \hat\gT^{!+}$}
$$
On voit de suite
que tout $\gamma\in\hat{\hat\gT}$ qui normalise $\Sigma$
normalise $\Sigma^!=\Sigma\cap\hat{\hat\gT}^!$, d'où
$\Sigma^!$ est aussi invariant dans $\cN_\Sigma$.
Soit $\cN_\Sigma^!=\cN_\Sigma\cap\hat{\hat\gT}^!$,
on a un diagramme cartésien de sous-groupes de $\hat{\hat\gT}$:
\[\begin{tikzcd}
	{\cN^!} && \cN \\
	\\
	{\Sigma^!} && \Sigma
	\arrow["{\GG_\Sigma}", hook, from=3-1, to=1-1]
	\arrow["{\gS_I}", hook, from=1-1, to=1-3]
	\arrow["{\gS_I}", hook, from=3-1, to=3-3]
	\arrow["{\GG_\Sigma}"', hook, from=3-3, to=1-3]
\end{tikzcd}\]
donnant un homomorphisme injectif $\cN/\Sigma^!\buildrel\sim\over\to
\cN/\cN^!\times\cN/\Sigma$ (ici $\cN/\cN^!=\gS_I$,
$\cN/\Sigma=\GG_\Sigma$ et $\Sigma/\Sigma^!{\buildrel\sim \over\to}
\cN/\cN^! \isom \gS_I$), dont on voit de suite qu'il est
bijectif
$$\cN/\Sigma^!{\buildrel\sim \over\to}\cN/\cN^!\times\cN/\Sigma \isom 
\gS_I\times\GG_\Sigma,
$$
et on a par suite aussi
$$
\cN^!/\Sigma^!{\buildrel\sim \over \to} \cN/\Sigma=\GG_\Sigma,
$$
i.e. le groupe $\GG_\Sigma$ des automorphismes
arithmétiquement extérieurs de $(\pi,\Sigma)$ peut se décrire aussi via
les automorphismes extérieurs induisant l'identité sur $I$.

Dans le cas $(g,\nu)=(0,3)$, on a $\Sigma^!=\{1\}$, donc
$\Sigma{\buildrel\sim \over\to}\gS_3$.  $\cN$ est le normalisateur
de $\gS_3$ dans $\hat{\hat\gT}$, $\cN^!{\buildrel\sim\over\to}
\GG_\Sigma$ est son centralisateur, et $\cN$ s'identifie au produit
direct des deux.

Soit maintenant $I'\subset  I$, $\nu'={\rm card}(I')$, et
considérons le groupe extérieur $\pi'$ déduit
de $\pi$ par ``bouchage de trous'' en $I\setminus I'$, d'où un homomorphisme
$$\pi\to \pi'.$$
Considérons les bases de $\pi$ pour lesquelles, les $\nu'$ premiers $l_i$
soient dans des groupes à lacets $L_{i'}$ ($i'\in I'$) -- on les
appelle adaptées à $I'$; elles forment un torseur $\subset 
{\rm Isom}_{\rm lac}(\hat\pi_{g,\nu},\pi)$ sous le sous-groupe
$\hat{\hat\gS}_{g;(\nu,\nu')}$ de $\hat{\hat\gS}_{g,\nu}={\rm Aut}_{\rm lac}
(\hat\pi_{g,\nu})$, formé des $u\in\hat{\hat\gS}_{g,\nu}$ dont l'image dans 
$\gS_I$ invarie l'ensemble des $\nu'$ premiers éléments (ou encore
l'ensemble complémentaire des $\nu-\nu'$ derniers).

Pour une telle base, on trouve une base correspondante de
$\pi'$.  La donnée de $(\pi,I')$ équivaut à celle
d'un torseur sous $\hat{\hat\gS}_{g;(\nu,\nu')}$,
celle d'un $\pi'$ à la donnée d'un torseur sous
$\hat{\hat\gS}_{g,\nu'}$, et le passage de $\pi$ à $\pi'$
est décrit par le changement de groupe d'opérateurs\footnote{D'où aussi: toute base, toute
discrétification, discrétification orientée de $\pi$
en définit une de $\pi'$, itou pour les classes de $\pi, \pi'$ conjugaison 
i.e. pour les prédiscrétifications (éventuellement orientées)
strictes, et aussi pour les adhérences des classes i.e. pour
les prédiscrétifications et prédiscrétifications orientées.
Il n'y a que le cas des arithmétisations qui demande une analyse 
plus fine.}
$$\hat{\hat\gS}_{g;(\nu,\nu')}\to \hat{\hat\gS}_{g,\nu'}.
$$
Cet homomorphisme envoie $\hat\gS_{g;(\nu,\nu')}$ dans $\hat\gS_{g,\nu'}$, et 
même $\gS_{g;(\nu,\nu')}$ dans $\gS_{g,\nu'}$, aussi $\pi$ dans
$\pi'$.  Il s'ensuit que toute prédiscrétification de $\pi$
en définit une de $\pi'$, de même pour les prédiscrétifications
strictes, discrétifications, bases exté\-rieures
``adaptées à $I'$ ''.  Enfin, si on a une 
``pré\-arith\-méti\-sation''${}^2$
de $\pi$, i.e. un $\Sigma\subset \hat{\hat\gT}(\pi)$, en déduit-on
une préarithmétisation de $\pi'$ -- i.e. (par exemple) si deux
prédiscrétifications de $\pi$ sont compatibles i.e. ont
le même groupe $\Sigma\subset \hat{\hat\gT}(\pi)$ d'automorphismes
extérieurs, en est-il de même de leurs images?

Pour y voir plus clair, on va écrire sous forme de diagramme
les ensembles remarquables associés à $\pi$, et ceci de deux
fa\c cons, l'une sans utiliser de structure particulière sur
$I=I(\pi)$, l'autre en utilisant un ordre total, ce qui
permet, dans le torseur ${\rm Isom}_{\rm lac}(\hat\pi_{g,\nu},\pi)$
sous $\hat{\hat\gS}_{g,\nu}$, de définir le sous-torseur
sous $\hat{\hat\gS}^!_{g,\nu}$ qu'on peut noter 
${\rm Isom}^!_{\rm lac}(\hat\pi_{g,\nu},\pi)$.
\moveleft 1.4cm \vbox{
\[\begin{tikzcd}
	& {\text{Bases de}~(\pi) \isom \Isom_{\text{lac}}(\hat{\pi}_{g, \nu}, \pi)} \\
	& {[\text{Bases de}^!(\pi) \isom \Isom^!_{\text{lac}}(\hat{\pi}_{g, \nu}, \pi)]} \\
	{\begin{matrix} \text{discrétifications orientées}~(\pi) \\ \isom {\rm Isom}_{\rm lac}(\hat\pi_{g,\nu},\pi)/\gS^+_{g,\nu} \\ \isom {\rm Isom}^!_{\rm lac}(\hat\pi_{g,\nu},\pi)/\gS^{!+}_{g,\nu} \end{matrix}} && {\begin{matrix} \text{Bases extérieures de}~\pi \\ \isom {\rm Isom}_{\rm lac}(\hat\pi_{g,\nu},\pi)/\hat\pi_ {g,\nu} \\ \isom {\rm Isomext}_{\rm lac}(\hat\pi_{g,\nu},\pi) \end{matrix}} \\
	& {\begin{matrix} \text{discrétifications orientées strictes}~(\pi) \\ \isom {\rm Isom}_{\rm lac}(\hat\pi_{g,\nu},\pi)/\gS^+_{g,\nu} Dot\hat\pi_{g,\nu} \\ \isom {\rm Isomext}_{\rm lac}(\hat\pi_{g,\nu},\pi)/\gT^+_ {g,\nu} \\ \isom {\rm Isom}^!_{\rm lac}(\hat\pi_{g,\nu},\pi)/\gS^{!+}_ {g,\nu}Dot\hat\pi_{g,\nu} \\ \isom {\rm Isomext}_{\rm lac}(\hat\pi_{g,\nu},\pi)/\gT^{!+}_ {g,\nu} \end{matrix}} \\
	& {\begin{matrix} \text{prédiscrétifications orientées}~(\pi) \\ \isom {\rm Isom}_{\rm lac}(\hat\pi_{g,\nu},\pi)/\hat\gS^+_{g,\nu} \\ \isom {\rm Isomext}_{\rm lac}(\hat\pi_{g,\nu},\pi)/\hat\gT^+_{g,\nu} \\ \isom {\rm Isom}^!_{\rm lac}(\hat\pi_{g,\nu},\pi)/\hat\gS^{!+}_{g,\nu} \\ \isom {\rm Isomext}^!_{\rm lac}(\hat\pi_{g,\nu},\pi)/\hat\gT^{!+}_ {g,\nu} \end{matrix}} \\
	& {\begin{matrix} \text{préarithmétisations de}~\pi \\ \isom {\rm Isom}_{\rm lac}(\hat\pi_{g,\nu},\pi)/M_{g,\nu}{}^3 \\ \isom {\rm Isomext}_{\rm lac}(\hat\pi_{g,\nu},\pi)/\cN_{g,\nu} \\ \isom {\rm Isom}^!_{\rm lac}(\hat\pi_{g,\nu},\pi)/M^!_{g,\nu} \\ \isom {\rm Isomext}^!_{\rm lac}(\hat\pi_{g,\nu},\pi)/\cN^!_ {g,\nu} \end{matrix}}
	\arrow[from=1-2, to=2-2]
	\arrow["{\gS^+_{g, \nu}}", from=2-2, to=3-1]
	\arrow["{\hat{\pi}_{g, \nu}}"', from=2-2, to=3-3]
	\arrow["{\gT_{g, \nu}}"', from=3-3, to=4-2]
	\arrow[from=3-1, to=4-2]
	\arrow["{\GG_{g, \nu}}", from=5-2, to=6-2]
	\arrow["{\GG_\Sigma}"', from=5-2, to=6-2]
	\arrow[from=4-2, to=5-2]
\end{tikzcd}\]
}
N.B. On pose
$$\GG_{g,\nu}=\cN_{g,\nu}/\Sigma_{g,\nu}=\cN^!_{g,\nu}/\Sigma^!_{g,\nu}$$
(où $\Sigma_{g,\nu}=\hat\gT^+_{g,\nu}$, $\Sigma^!_{g,\nu}=
\hat\gT^{!+}_{g,\nu}$).  On peut aussi le décrire comme le
groupe $\GG_\Sigma$ associé à un groupe extérieur
profini à lacets $\pi$ {\it muni} d'une prédiscrétification
orientée $\alpha$ (sans plus).  Si on a deux tels couples
$(\pi,\alpha)$, $(\pi',\alpha')$, d'où $\GG_\Sigma$
et $\GG_{\Sigma'}$ ($\Sigma={\rm Autext}_{\rm lac}
(\pi,\alpha)$, $\Sigma'={\rm Autext}_{\rm lac}(\pi',\alpha')$), on
trouve en effet un isomorphisme {\it canonique}:
$$\GG_\Sigma{\buildrel\sim \over\to} \GG_{\Sigma'}$$
en prenant n'importe quel isomorphisme extérieur à lacets $u$
de $\pi$ avec $\pi'$ transformant $\alpha$ en $\alpha'$ et
l'isomorphisme associé $\GG_\Sigma\to\GG
_{\Sigma'}$ ne dépend évidemment pas du choix de $u$.

Ceci posé, les couples $(\pi,\Sigma)$ d'un groupe profini à
lacets de type $(g,\nu)$, muni d'une pré\-arithmé\-tisation,
avec comme morphismes les iso\-mor\-phismes arithmé\-tique\-ment
extérieurs (relatifs à $\Sigma$ et $\Sigma'$\dots), forment
un groupoïde connexe $\Pi_{g,\nu}$, ayant un objet
``origine'' défini à isomorphisme unique près comme
étant $(\hat\pi_{g,\nu},\Sigma_{g,\nu})$, où au choix n'importe
quel $(\pi,\Sigma_\alpha)$ provenant d'un $(\pi,\alpha)$
($\alpha$ une prédiscrétification orientée de $\pi$),
et dont le groupe des automorphismes est justement $\GG_{g,\nu}$.

On constate qu'à l'exception des deux espaces homogènes
(sous $\hat{\hat\gS}(\pi)$, mais pas 
nécessairement sous $\hat{\hat\gS}(\pi,I')$,
voire sous $\hat{\hat\gS}^!(\pi)$) Bases$(\pi)$ et Bases ext$(\pi)$,
les quatre autres sont en fait des espaces homogènes sous
$\hat{\hat\gS}^!$, et s'expriment comme quotients du
$\hat{\hat\gS}^!_{g,\nu}$-torseur ${\rm Isom}^!(\hat\pi_{g,\nu},\pi)$,
par les quatre sous-groupes $\gS^!_{g,\nu}$, $\gS^!_{g,\nu}Dot
\hat\pi_{g,\nu}$, $\hat\gS^!_{g,\nu}$, $M^!_{g,\nu}$
(ce dernier défini comme image inverse de $\cN^!_{g,\nu}$ dans
$\hat{\hat\gS}_{g,\nu}$, tout comme $M_{g,\nu}$ est
défini comme image inverse de $\cN_{g,\nu}$).  On trouve d'autre part,
si $I'=\{i_0,\ldots,i_{\nu'-1}\}$, un homomorphisme
$\hat{\hat\gS}^!_{g,\nu}\to\hat{\hat\gS}^!_{g,\nu'}$, envoyant
$\gS^!_\gn$ dans $\gS^!_{g,\nu'}$, en envoyant $\pi_\gn$ dans
$\pi_{g,\nu'}$, donc $\hat\pi_\gn$ dans $\hat\pi_{g,\nu'}$,
$\hat\gS^!_\gn$ dans $\hat\gS^!_{g,\nu'}$, et le sous-groupe $\gS^!_\gn
Dot\hat\pi_\gn$
de $\hat\gS^!_\gn$ dans le sous-groupe $\gS^!_{g,\nu'}Dot
\hat\pi_{g,\nu'}$ de $\hat\gS^!_{g,\nu'}$.  Enfin, comme
$\Sigma^!_\gn$ s'envoie dans $\Sigma^!_{g,\nu'}$,
$\cN^!_\gn$ s'envoie dans le normalisateur de $\Sigma^!_{g,\nu'}$
dans $\hat{\hat\gT}^!_{g,\nu'}$, je dis que ce n'est autre
que $\cN^!_{g,\nu'}$.  Changeant de notation, ceci revient au
\vskip .3cm
{
Lemme\footnote{pas prouvé}. --- \it $\cN^!_\Sigma={\rm Norm}_{{\hat{\hat\gT}^!}} (\Sigma^!)
\bigl(={\rm Norm}_{{\hat{\hat\gT}}}(\Sigma^!)\cap\hat{\hat\gT}^! \bigr)$.
}
\vskip .3cm
Évidemment on a $\cN^!_\Sigma\subset {\rm Norm}_{{\hat{\hat\gT}}}(\Sigma^!)
\cap\hat{\hat\gT}^!$; inversement soit $g\in\hat{\hat\gT}^!$ qui normalise
$\Sigma^!$, montrons qu'il normalise $\Sigma$, i.e. qu'il est $\in\cN$
(donc $\in \cN^! = \cN \cap \hat{\hat\gT}^!$)\dots.

C'est pas clair.  J'ai pourtant envie de prouver la
%\vfill\eject
\vskip .3cm
{
Conjecture. --- \it L'application canonique
$${\hbox{Prédiscrét}}^+(\pi)\to
{\hbox{Prédiscrét}}^+(\pi')$$ est {\it bijective}.  
Pour que deux prédiscrétifications
orientées $\alpha$, $\beta$ de $\pi$ aient même groupe
d'automorphismes extérieurs $\Sigma_\alpha=\Sigma_\beta$,
il faut qu'il en soit ainsi pour leurs images $\Sigma_{\alpha'}$ et
$\Sigma_{\beta'}$, i.e. que $\Sigma_{\alpha'}=\Sigma_{\beta'}$,
de sorte que la bijection précédente induit une bijection
$${\hbox{Préarithmétisation}}(\pi)\buildrel\sim\over\to
{\hbox{Préarithmétisation}}(\pi').$$
Enfin, les bijections précédentes sont compatibles avec
l'homomorphisme des groupes d'opérateurs $\hat{\hat\gT}
(\pi,I')\to\hat{\hat\gT}(\pi'){}^4$, a fortiori avec
$\hat{\hat\gT}^!(\pi)\to\hat{\hat\gT}^!(\pi')$, ce
qui implique qu'elle induit (pour une préarithmétisation
$\Sigma\subset  \hat{\hat\gT}(\pi)$ donnée de $\pi$, donnant $\Sigma'\subset 
\hat{\hat\gT}(\pi')$ dans $\pi'$ avec $\Sigma^!\to
\Sigma'^!$)${}^5$ un homomorphisme $\cN^!_\Sigma\to
\cN^!_{\Sigma'}$, et l'homomorphisme correspondant
$$\GG_\Sigma=\cN^!_\Sigma/\Sigma^!\to
\GG_{\Sigma'}=\cN^!_{\Sigma'}/\Sigma'^!$$
est un isomorphisme.
}
\vskip .3cm
Pour s'en convaincre il suffit de regarder le cas où
$I'=\{i\}$ (et si on note $\pi = \pi_\gn$, $I'$ réduit
au dernier élément de $I$).  On a alors un homomorphisme
$$\hat{\hat\gT}(\pi,i)\to\hat{\hat\gS}(\pi')$$
(les automorphismes {\it extérieurs} à lacets de $\pi$
fixant $i$ définissent des automorphismes bien déterminés
de $\pi'$ -- pas seulement extérieurs, i.e. $\hat{\hat\gS}(\pi,i)
\to\hat{\hat\gS}(\pi')$ est trivial sur $\pi$, et passe
donc au quotient en $\hat{\hat\gT}(\pi,i)\to
\hat{\hat\gS}(\pi,i)$).

J'admets, en analogie avec le cas discret, que cet homomorphisme
est un {\it iso\-mor\-phisme}, de sorte qu'on a une suite exacte
\footnote{rappelons qu'on suppose $\pi$ anabélien, i.e. $2g+\nu \geq 3$}
$$1\to \pi'\to\hat{\hat\gT}(\pi,i)
\to\hat{\hat\gS}(\pi,i)\to 1$$
ou encore
$$1\to \hat\pi_{g,\nu-1}\to\hat{\hat\gT}_{g;(\nu,\nu-1)}
\to\hat{\hat\gS}_{g,\nu-1}\to 1$$
qui contient la suite exacte
$$1\to \pi_{g,\nu-1}\to\gT_{g;(\nu,\nu-1)}
\to\gS_{g,\nu-1}\to 1$$
comme sous-suite exacte.  Il est commode de travailler plutôt avec
\[\begin{tikzcd}
	1 & {\pi_{g,\nu-1}} & {\gT^{!+}_{g,\nu}} & {\gS^{!+}_{g,\nu-1}} & 1 \\
	1 & {\hat\pi_{g,\nu-1}} & {\hat\gT^{!+}_{g,\nu}} & {\hat\gS^{!+}_{g,\nu-1}} & 1 \\
	1 & {\hat\pi_{g,\nu-1}} & {\hat{\hat\gT}^!_{g,\nu}} & {\hat{\hat\gS}^!_{g,\nu-1}} & {1.}
	\arrow[from=1-1, to=1-2]
	\arrow[from=2-1, to=2-2]
	\arrow[from=3-1, to=3-2]
	\arrow[from=1-2, to=1-3]
	\arrow[from=1-3, to=1-4]
	\arrow[from=1-4, to=1-5]
	\arrow[from=2-2, to=2-3]
	\arrow[from=3-2, to=3-3]
	\arrow[from=2-3, to=2-4]
	\arrow[from=3-3, to=3-4]
	\arrow[from=2-4, to=2-5]
	\arrow[from=3-4, to=3-5]
	\arrow[hook', from=1-4, to=2-4]
	\arrow[hook', from=2-4, to=3-4]
	\arrow[hook', from=1-3, to=2-3]
	\arrow[hook', from=2-3, to=3-3]
	\arrow[hook', from=1-2, to=2-2]
	\arrow[shift left=1, shorten <=2pt, shorten >=2pt, no head, from=2-2, to=3-2]
	\arrow[shorten <=2pt, shorten >=2pt, no head, from=2-2, to=3-2]
\end{tikzcd}\]
L'application Prédiscrét$^+(\pi)\to$
Prédiscrét$^+(\pi')$ s'identifie alors à $\hat{\hat\gT}^!_{g,\nu}
/\hat\gT^!_{g,\nu}$, on lit sur le diagramme
que c'est $ \isom  \hat{\hat\gS}^!_{g,\nu-1}/\hat\gS^{!+}_{g,\nu-1}$,
d'où la bijectivité sur les ensembles de pré\-discréti\-fica\-tions
orientées.

Dans le groupe $\hat{\hat\gT}^!_{g,\nu}=\gT$, on a un sous-groupe
invariant $\hat\pi_{g,\nu-1}=\pi'$, et un sous-groupe
$\Sigma$ ($=\hat\gT^{!+}_{g,\nu}$) entre $\pi'$ et $\gT$,
donnant un sous-groupe $\Sigma' \isom \hat\gS^{!+}_{g,\nu-1}$ dans 
$\gT^!=\gT/\pi'$ ($ \isom \hat{\hat\gS}^!_{g,\nu-1}$), et on sait bien 
qu'alors $\gT/\Sigma \isom \gT'/\Sigma'$, et que le normalisateur
$\cN$ de $\Sigma$ dans $\gT$ est l'image inverse du
normalisateur $\cN'$ de $\Sigma'$ dans $\gT'$, de sorte que
$\cN/\Sigma \isom \cN'/\Sigma'$, ce qui prouve ce qu'on voulait.
\vskip .3cm
{
Corollaire. --- \it On a des isomorphismes canoniques:
$$\GG_{g,\nu}{\buildrel\sim \over \to}\GG_{g,\nu-1}
{\buildrel\sim \over\to}  Dots$$
de sorte que si $g\ge 2$, on a canoniquement $\GG_{g,\nu}
 \isom \GG_{g,0}$; d'autre part $\GG_{1,\nu}
 \isom \GG_{1,1}$ ($g=1$, $\nu\ge 1$), $\GG_{0,\nu}
 \isom \GG_{0,3}$ ($\nu\ge 3$).\footnote{On voit aussi que $\Sigma$ invariant dans $\hat{\hat\gT}$
 équivaut à $\Sigma'$ invariant dans $\hat{\hat \gT'}$.}
 }
\vskip .3cm
NB. Le ``fait'' admis est loin d'être évident --
même que $\hat{\hat\gT}_{g;(\nu,\nu-1)}
\to\hat{\hat\gT}_{g,\nu-1}$ soit surjectif ou que
$\hat{\hat\gT}_{g;(\nu,\nu-1)}\to\hat{\hat\gS}
_{g,\nu-1}$ soit injectif, est loin d'être évident,
et est peut-être tout à fait faux!  Le fait admis
revient à un énoncé d'existence d'un foncteur en sens
inverse ``forage de trous'' qui en tout état de cause
reste incompris.

















%%%%%%%%%%%%%%%%%%%%%%%%%%%%%%%%%%%%%%%%%%%%%%%%%%%%%%%%%%%%%%%
\chapter*{\S \space 28. --- CHANGEMENT DE TYPE $(g,\nu)$ (SUITE): PASSAGE À UN REVÊTEMENT FINI}\thispagestyle{empty}
\addcontentsline{toc}{section}{28. Changement de type $(g,\nu)$ (suite): b) passage à un revêtement fini (la conjecture hâtive grince\dots)}
\label{sec:28}
\section*{}

Soit $\pi$ un groupe profini à lacets de type $(g,\nu)$, anabélien 
comme toujours, $\pi'$ un sous-groupe d'indice fini.  On sait (ou
on vérifie, par passage au cas discret) que muni des traces
sous $\pi$ des sous-groupes à lacets de $\pi$, $\pi'$ est
un groupe à lacets.  Ses sous-groupes à lacets sont aussi
les sous-groupes $L'$ tels que
\begin{enumerate}
    \item[a)] $L'={\rm Centr}_{\pi}(L')\cap \pi'$
    \item[b)] $L={\rm Centr}_{\pi}(L')$ est un sous-groupe à lacets dans $\pi$.
\end{enumerate}
On pose $d(L')=[L:L']$, et on a ainsi une fonction
$d:I'=I(\pi')\to \mathbf{N}^*$, et une application $I'\buildrel
\phi \over{\to} I$, telles que $\forall i\in I$, on ait:
$$
\sum_{{i'\in I'}\atop{i' \, {\hbox {au dessus de}}\, i}} d(i')= n
\ \ \ \ \ ({\hbox  {où} }\ n=[\pi:\pi']).  \leqno{(1)}
$$
On aura la formule de Hurwitz
$$
2g'-2 = n(2g-2)+\sum_{i'\in I'}\bigl (d(i')-1\bigr),\leqno{(2)}
$$
où la somme à droite est égale à $(n\,{\rm card}(I)-{\rm card}(I'))$, i.e.
$$
2g'-2 + {\rm card}(I')=n\bigl(2g-2+{\rm card}(I)\bigr).\leqno{(3)}
$$
(NB.  $2g-2+{\rm card}(I)$ est l'opposé de la caractéristique
d'Euler-Poincaré à supports compacts de la courbe
dont $\pi$ est le groupe fondamental\dots).\footnote{NB. Dire qu'on est dans le cas anabélien
signifie que $-{\rm EP}_!$ [i.e. l'opposé de la caractéristique
d'Euler-Poincaré à support compact] est $\ge 1$, et cette relation est
donc conservée par passage à un revêtement. L'entier $-{\rm EP}_!$
mesure par sa positivité stricte le degré d'anabélianité
en quelque sorte..}
\vskip .2cm
De fa\c con évidente, toute discrétification de $\pi$ en
définit une de $\pi'$ -- et il en est de même pour les
discrétifications orientées, modulo un peu de cohomologie,
ce qui revient à dire, dans le cas discret par exemple,
que $T \isom  H^2_!(U,\mathbf{Z})
\to H^2_!(U',\mathbf{Z}) \isom  T'$ peut s'écrire
sous la forme $n\theta^{-1}$, où $\theta$ est un isomorphisme bien
déterminé (l'isomorphisme trace) $T' \isommap T$.

Il n'est pas clair pour moi à première vue si l'application
$$
\hbox {Discrét}(\pi)\to \hbox {Discrét}(\pi')
$$
est surjective, ou injective.  L'injectivité pour tout $\pi'$
d'indice fini dans $\pi$ signifierait que deux discrétifications
$\pi_0$, $\pi'_0$ de $\pi$ qui sont commensurables, i.e. telles que
$\pi_0\cap\pi'_0$ soit d'indice fini dans $\pi_0$ et dans
$\pi'_0$, sont égales.

Supposons que $\pi'$ soit un sous-groupe invariant dans $\pi$,
et posons $G=\pi/\pi'$.
\vskip .3cm
{
Conjecture. --- \it L'application Discrét$(\pi)\to$
Discrét$(\pi')$ est injective, et son image est formée des
discrétifications $\pi'_0\subset \pi'$ telles que
$G\to{\rm Autext}_{\rm lac}(\pi') \isom {\rm Autext}_{\rm lac}
(\hat\pi'_0) \isom \hat{\hat\gT}(\hat\pi'_0)$ se factorise
par $\gT(\pi'_0)={\rm Autext}_{\rm lac}(\pi'_0)$ (et en fait,
même par $\gT(\pi'_0)^+$).  En d'autres termes, $\forall
g\in\pi$, $\exists h\in\pi'$ tel que ${\rm Int}(hg)(\pi'_0)
\subset  \pi'_0$, i.e. $\pi'Dot\pi_0=\pi$, où $\pi_0$
est le normalisateur de $\pi'_0$ dans $\pi$.
}
\vskip .3cm
La condition pour qu'un $\pi'_0$ soit dans l'image est évidemment
nécessaire.  Montrons qu'elle est suffisante, et que le $\pi_0$
donnant naissance à $\pi'_0$ par $\pi_0\cap\pi'$ est unique.
On a une action extérieure de $G$ sur $\pi'_0$, qui
définit l'action extérieure sur $\hat\pi'_0 \isom \pi'$,
or $\pi$ se récupère canoniquement à partir de cette dernière
(car ${\rm Centre}(\hat\pi')=1$), et on voit que c'est le complété
profini de l'extension $\pi_0$ de $G$ par $\pi'_0$, définie
par l'action extérieure de $G$ sur $\pi'_0$. On a donc bien une
discrétification généralisée
$\pi_0$ de $\pi$, au sens de la seule
structure de groupe profini, et elle induit $\pi'_0$ -- mais il faut
voir qu'elle est compatible avec la structure à lacets de $\pi$.
Mais celle-ci s'explicite, à partir de la structure à
lacets de $\pi'$, en prenant les sous-groupes à lacets $L'$
de $\pi'$, et leurs centralisateurs dans $\pi$.  Faisant
itou pour $\pi_0\supset\pi_0'$, on devrait trouver une structure
à lacets sur $\pi_0$, donnant lieu aux mêmes $d_i$.
J'ai l'impression que \c ca ne doit pas être très vache à prouver --
ce serait comme si on savait déjà que toute extension ``sans torsion''
d'un groupe fini par un groupe à lacets est de fa\c con canonique
un groupe à lacets\dots
\vskip .3cm
{
Corollaire. --- \it En tout cas, l'application
$$\hbox {Discrét}(\pi)\to\hbox {Discrét}(\pi')$$
est \emph{injective}. 
}
\vskip .3cm
N.B. Elle ne doit pas toujours être bijective,
car il doit y avoir une action extérieure d'un $G$ sur un $\hat\pi'_0$,
i.e. un homomorphisme $G\to{\hat{\hat\gT}}(\hat\pi'_0)$,
qui ne se factorise pas par $\gT(\pi'_0)$, et qui pourtant
est aussi bonne du point de vue groupe profini que celles provenant d'un $\pi$
et d'un sous-groupe invariant $\pi'$.  En fait, partons d'une
telle situation {\it discrète} $\pi_0\supset \pi'_0$,
d'où $G = \pi_0/\pi'_0$ et $G\to{\rm Autext}
_{\rm lac}(\pi'_0)=\gT(\pi'_0)\subset {\hat{\hat\gT}}(\hat\pi'_0)$,
et {\it conjuguons} $G$ par un élément $\gamma$
de ${\hat{\hat\gT}}(\pi'_0)$, de fa\c con à trouver
$G\to\hat{\hat\gT}(\hat\pi'_0)$ qui ne se factorise
pas par $\gT(\pi'_0)$.  On trouve une extension
$\pi^\natural$ de $G$ par $\hat\pi'_0$,
{\it isomorphe} à $\hat\pi_0$ (en tant qu'extension de $G$
par $\hat\pi'_0$), donc la structure à lacets de $\hat\pi'_0$
en définit une sur $\pi^\natural$ (de fa\c con à être
induite par cette dernière), mais  pourtant la discrétification
choisie $\pi_0$ de $\hat\pi_0$ n'est pas induite par une de $\pi$.
Pour faire la construction, il suffit de trouver un élément 
$\gamma\in {\hat{\hat\gT}}(\pi'_0)$ tel que ${\rm Int}(\gamma)\cdot G
\not\subset \gT$ i.e. tel que $G$ ne stabilise pas $\gamma^{-1}(\pi'_0)\subset 
\hat\pi'_0$.
Un tel $\gamma$ existe toujours\dots

Bien s\^ur, il n'y a pas de raison, pour deux discrétifications
$\pi_0$, $\pi_1$ de $\pi$, que si $\pi_0$, $\pi_1$ sont $\pi$-conjugués
(i.e. définissent la même prédiscrétification stricte
de $\pi$), il en soit de même pour $\pi'_0$ et $\pi'_1$
pour l'action intérieure de $\pi'$ (il faudrait que $\pi_0$,
$\pi_1$ soient conjugués par $\pi'$, et pas seulement par $\pi$).
Par contre, je présume que si $\pi_0$, $\pi_1$ sont ``adhérents''
l'un à l'autre dans l'espace des discrétifications
de $\pi$, i.e. s'ils définissent une même prédiscrétification,
il en est de même pour $\pi'_0$, $\pi'_1$, et que l'application
$$
\hbox {Prédiscrét}(\pi)\to\hbox{Prédiscrét}(\pi')\leqno{(4)}
$$
qu'on obtient ainsi, est encore bijective, et qu'elle passe à
son tour au quotient, pour définir une application bijective
$$
\hbox {Préarith}(\pi)\to\hbox {Préarith}(\pi'),\leqno{(5)}
$$
et que si $\Sigma$, $\Sigma'$ se correspondent par cette
dernière, on a un isomorphisme canonique
$$
\GG_\Sigma\to\GG_{\Sigma'},\leqno{(6)}
$$
compatible avec les actions de ces groupes sur l'ensemble
$P_\Sigma$ des prédiscrétifications de $\pi$ compatibles
à $\Sigma$, et l'ensemble $P_{\Sigma'}$ des prédiscrétifications
de $\pi'$ compatibles à $\Sigma'$ (qui sont respectivement des torseurs
à gauche sous $\GG_\Sigma$, $\GG_{\Sigma'}$).  En même temps,
on trouvera donc que $\Sigma$ est unique pour $\pi$, i.e.
invariant dans $\hat{\hat\gT}(\pi)$, si et seulement si
$\Sigma'$ est unique dans $\pi'$, i.e. invariant dans 
$\hat{\hat\gT}(\pi')$.

En tout cas, la conjecture fondamentale qui interprète
les $\GG_{g,\nu}$ comme $\GG_0={\rm Gal}$ $(\overline{\mathbf{Q}}_0/\mathbf{Q})$
aurait au moins comme conséquence que l'application
$\hbox {Discrét}^+(\pi)\to\hbox {Discré}^+(\pi')$
passe au quotient de deux fa\c cons, pour donner un
diagramme commutatif:
\[\begin{tikzcd}
	{\text{Discrét}^+(\pi)} & {\text{Discrét}^+(\pi')} \\
	{P =~\text{Prédiscrét}^+(\pi)} & {P' =~\text{Prédiscrét}^+(\pi')} \\
	{A =~\text{Arithm}(\pi)} & {A' =~\text{Arithm}(\pi')}
	\arrow[from=3-1, to=3-2]
	\arrow[from=2-1, to=2-2]
	\arrow[from=1-1, to=1-2]
	\arrow[from=1-2, to=2-2]
	\arrow[from=2-2, to=3-2]
	\arrow[from=2-1, to=3-1]
	\arrow[from=1-1, to=2-1]
\end{tikzcd}\]
et que le carré inférieur est cartésien (sans préjuger de
l'injectivité ou de la bijectivité de l'application
$A\to A'$).  Donc il faudrait que si $\pi_0$,
$\pi_1$ sont deux discrétifications orientées de $\pi$,
qui définissent la même prédiscrétification
(orientée), alors il en soit de même pour leurs traces 
sur $\pi'$, $\pi'_0$ et $\pi'_1$, relativement à $\pi'$.
Mais déjà si $\pi_0$ et $\pi_1$ sont $\pi$-conjugués
(i.e. définissent la même discrétification orientée,
{\it stricte}), ce n'est pas tellement clair -- mais si, car on
aura $\pi=\pi'Dot\pi_0$ donc si $\pi_1= {\rm int}(u)\ \pi_0$,
écrivant $u=u'u_0$ avec $u'\in\pi'$, $u_0\in\pi_0$, on aura
$\pi_1={\rm int}(u'){\rm int}(u_0)\pi_0={\rm int}(u')\pi_0$,
donc $\pi'_1={\rm int}(u')\pi'_0$ donc $\pi'_0$ et $\pi'_1$
sont conjugués.  T\^achons de procéder de même dans le cas
général d'un $u\in\hat\gS(\pi_0)^+$ tel que
$\pi_1=u(\pi_0)$, en écrivant si possible $u=u'u_0$,
avec $u'\in\hat\gS(\pi_0,\pi'_0)^+$, et
$u_0\in\gS(\pi_0)^+$, où on définit $\gS(\pi_0,\pi'_0)^+$
comme le groupe des automorphismes discrets de $\pi_0$
qui {\it stabilisent} $\pi'_0$, et $\hat\gS(\pi_0,\pi'_0)^+$
est son compactifié profini.  On aurait alors $\pi_1=
u'(u_0(\pi_0))=\\cdotu'(\pi_0)$, où $\\cdotu'$ désigne
l'image de $u'$ dans $\hat\gT(\pi'_0)$.  En tous cas,
il n'y a aucun problème si $\pi'$ est un sous-groupe
{\it caractéristique} de $\pi$, car alors on a un homomorphisme
discret $\gS(\pi_0)\to\gS(\pi'_0)$,
définissant l'homomorphisme 
$\hat\gS(\hat\pi_0)\to\hat\gS(\hat\pi'_0)$.  Comme les
sous-groupes ouverts caractéristiques de $\pi$ sont cofinaux,
on est ramené (pour les questions de factorisabilité
de $D(\pi)\to D(\pi')$ en $P(\pi)\to P(\pi')$
et $A(\pi)\to A(\pi')$ et de bijectivité des
applications $P(\pi)\to P(\pi')$ et
$A(\pi)\to A(\pi')$) au cas où
$\pi'$ est un sous-groupe {\it caractéristique}.
On a alors factorisabilité de $D(\pi)\to D(\pi')$
en $P(\pi)\to P(\pi')$.  Montrons que
cette application est injective\dots Cela signifie (a)
que l'image inverse dans $\hat{\hat\gS}(\hat\pi_0)$ de
$\hat\gS(\hat\pi'_0)^+$ est $\hat\gS(\hat\pi_0)^+$, la surjectivité
signifie (b) que tout élément de $\hat{\hat\gS}(\hat\pi'_0)$
est congru mod $\hat\gS(\pi'_0)^+$ à un élément
de l'image de $\hat{\hat\gS}(\pi_0)$.  La factorisabilité
en $A(\pi)\to A(\pi')$ signifie que (c) par l'application
précédente $\hat{\hat\gS}(\pi_0)\hookrightarrow
\hat{\hat\gS}(\pi'_0)$, (NB. il est immédiat que c'est injectif)
envoie $M(\pi_0)={\rm Norm}_{\hat{\hat\gS}(\pi_0)}(\hat\gS
^+(\pi_0)$) dans $M(\pi'_0)$, l'injectivité de $A(\pi)
\to A(\pi')$ que (d) l'on a même que $M(\pi)$
est exactement l'image inverse de $M(\pi')$, la surjectivité
de $A(\pi)\to A(\pi')$ que (e) tout élément
de $\hat{\hat\gS}(\pi'_0)$ est congru mod $M(\pi'_0)$ à
un élément de $\hat{\hat\gS}(\pi_0)$ (c'est plus faible
que (b)) -- et ces conditions réunies impliquent
que l'homomorphisme canonique 
$$M(\pi_0)/\hat\gS(\pi_0)=\GG_{\pi_0}\to\GG_{\pi'_0}
=M(\pi'_0)/\hat\gS(\pi'_0)$$
est un isomorphisme -- il suffit même pour ceci d'avoir (c)
(pour pouvoir définir cette application) et (a) (pour son
injectivité), et (b) et le renforcement (d) de (c)
(pour sa surjectivité).  Donc on voit que tout est 
suspendu aux propriétés des homomorphismes d'inclusions de groupes:
\[\begin{tikzcd}
	{\hat\gS^+(\pi_0)} & {M(\pi_0)} & {\hat{\hat\gS}(\pi_0)} \\
	{\hat\gS^+(\pi'_0)} & {M(\pi'_0)} & {\hat{\hat\gS}(\pi'_0)}
	\arrow[hook', from=1-3, to=2-3]
	\arrow[hook, from=1-2, to=1-3]
	\arrow[hook, from=2-2, to=2-3]
	\arrow["{?}", hook', from=1-2, to=2-2]
	\arrow[hook', from=1-1, to=2-1]
	\arrow[hook, from=1-1, to=1-2]
	\arrow[hook, from=2-1, to=2-2]
\end{tikzcd}\leqno{(7)}\]
à charge de prouver (a) et (b) (qui ne concernent que le
carré composé) et (c).

Peut-être les propriétés (a), (b) et (c) sont-elles tout
à fait fausses -- pourtant (a) (qui correspond à l'injectivité
de $P(\pi)\to P(\pi')$) semble assez plausible.  La
propriété (b) (qui exprimerait la surjectivité de
$P(\pi)\to P(\pi')$) est beaucoup plus problématique,
voir fausse.  L'ennui, c'est que le groupe $\hat{\hat\gS}(\pi_0)$
peut être ``beaucoup plus petit'' que $\hat{\hat\gS}(\pi'_0)$,
on s'en rend compte en passant au quotient par le sous-groupe
$\hat\pi'_0=\pi'$ (contenu dans le plus petit des groupes
de (7), et invariant dans le plus grand), on trouve sur la
deuxième ligne les groupes $\hat\gT^+(\pi'_0)$,
$\cN(\pi'_0)$ et $\hat{\hat\gT}(\pi'_0)$, sur la première
des extensions des groupes correspondants pour $\pi_0$
($\hat\gT^+(\pi_0)$, $\cN(\pi_0)$ et $\hat{\hat\gT}(\pi_0)$)
par le groupe fini $G=\pi_0/\pi'_0 \isom  \pi/\pi'$, notées
par des $\tilde{}$, de sorte que l'on a:
\[\begin{tikzcd}
	G & {\tilde{\hat\gT}^+(\pi_0)} & {\tilde\cN(\pi_0)} & {\tilde{\hat{\hat\gT}}(\pi_0)} \\
	& {\hat\gT^+(\pi'_0)} & {\cN(\pi'_0) } & {\hat{\hat\gT}(\pi'_0)}
	\arrow[from=1-1, to=2-2]
	\arrow[hook, from=1-1, to=1-2]
	\arrow[hook, from=1-2, to=1-3]
	\arrow[hook, from=2-2, to=2-3]
	\arrow[hook', from=1-2, to=2-2]
	\arrow["{?}", hook', from=1-3, to=2-3]
	\arrow[hook, from=1-3, to=1-4]
	\arrow[hook, from=2-3, to=2-4]
	\arrow[hook', from=1-4, to=2-4]
\end{tikzcd}\leqno{(8)}\]
qui montre que les groupes de la première ligne
{\it normalisent} le sous-groupe $G\subset \hat\gT^+(\pi'_0)$
(NB en fait on a $G\subset \gT^+(\pi'_0)$, et $\pi_0$
se reconstitue à partir de $\pi'_0$ et de ce sous-groupe
de $\pi'_0$) -- on doit pouvoir montrer sans trop de mal
que dans (8), $\tilde{\hat\gT}^+(\pi_0)$ et 
$\tilde{\hat{\hat\gT}}(\pi_0)$ sont justement les normalisateurs
de $G$ dans $\hat\gT^+(\pi'_0)$ et dans $\hat{\hat\gT}(\pi'_0)$
(ce qui impliquerait bien (a), d'ailleurs) -- mais je ne vois
aucune raison plausible que (b) soit vrai -- ce qui
signifierait, essentiellement, qu'il n'y a pas plus de
``transcendance arithmétique'' définie par le
(``très gros''!) $\hat{\hat\gT}(\pi'_0)$ (mod $\hat\gT(\pi'_0)^+$),
que celle définie par le (``bien plus petit'') groupe
$\tilde{\hat{\hat\gT}}(\pi_0)={\rm Norm}_{\hat{\hat\gT}(\pi_0)}(G)$\dots
D'ailleurs ces conditions (a), (b) ne sont pas impératives
pour que la conjecture fondamentale reliant les $\pi_{g,\nu}$
à $\GG_0$ soit cohérente -- par contre il faudrait
absolument avoir (c) pour pouvoir au moins définir
$A(\pi)\to A(\pi')$ et (si $\Sigma$, $\Sigma'$
se cor\-res\-pon\-dent) $\GG_\Sigma\to\GG_{\Sigma'}$,
dans le cas $\pi'$ caractéristique dans $\pi$ (et dans tous les
cas si on admet de plus (a), qui semble assez plausible,
et donne les {\it injectivités} qu'il faut).  Mais on se
demande bien pourquoi un élément de $\hat{\hat\gT}(\pi'_0)$
simplement parce qu'il est dans le normalisateur $R$ de $G$
dans $\hat{\hat\gT}(\pi'_0)$, donc aussi $R$, et qu'il normalise
$R\cap\hat\gT^+(\pi'_0)$, devrait normaliser aussi
$\hat\gT^+(\pi'_0)$ lui-même!  Il est vrai qu'on a fait
des hypothèses draconiennes au départ (partant d'un sous-groupe
{\it caractéristique} $\pi'_0$ de $\pi_0$) qui doivent bien se
refléter par des propriétés particulières de
$G\subset \gT^+(\pi'_0)$.  Peut-être finalement l'astuce de
se ramener à des sous-groupes caractéristiques pour examiner
la situation n'est-elle pas si astucieuse.  Pour y voir plus
clair, on pourrait déjà essayer de comprendre le cas
où (sans suppose $\pi'$ caractéristique), on suppose $\pi'$
d'indice $2$ dans $\pi$, donc invariant et $G \isom \mathbf{Z}/2\mathbf{Z}$,
ou bien on prend le noyau de l'homomorphisme canonique
$\pi\to(\pi_{ab})_2$ ($ \isom (\mathbf{Z}/2\mathbf{Z})^{2g}$ si $\nu=0$,
$ \isom (\mathbf{Z}/2\mathbf{Z})^{2g-\nu-1}$ si $\nu\ge 1$), en prenant par exemple
$\pi=\hat\pi_{0,3}$ -- situation de la courbe de Fermat
$x^2+y^2+z^2=0$ (revêtement octaédral de $\P^1_{\overline{\mathbf{Q}}}
\setminus\{0,1,\infty\}$\dots)
 
Mais admettons provisoirement les hypothèses d'injectivité
(relativement anodines), {\it plus} la condition (c), pas
anodine du tout -- d'où si $\Sigma$ et $\Sigma'$ se
correspondent, un homomorphisme injectif
$$
\GG_\Sigma\hookrightarrow\GG_{\Sigma'},\leqno{(9)}
$$
et voyons ce qu'on pourrait en tirer, en admettant même,
provisoirement (pour voir) que (9) est un isomorphisme,
c'est-à-dire toute la force de la conjecture principale
$\GG_0  \isom  \GG_{g,\nu}$.

Soit $\pi$ un groupe profini à lacets de type $(g,\nu)$,
$\pi'$ de type $(g',\nu')$.  Appelons ``cor\-res\-pondance''
entre $\pi$ et $\pi'$ un couple formé d'un $\gG$ profini
à lacets et d'homomorphismes à lacets 
$\gG\buildrel p\over\to\pi$, $\gG\buildrel q\over\to
\pi'$ qui sont donc chacune composée d'un morphisme ``bouchage de trous'',
et d'un isomorphisme avec un sous-groupe d'indice fini.
Soient $D_p$, $D_q\subset D=I(\pi'')$ les sous-ensembles
de $D$ qui correspondent aux ``trous bouchés par $p$''
resp. par $q$, on suppose s'il le faut que $D_p\cap
D_q=\emptyset$ (sinon on pourrait factoriser par un même
quotient $\tilde\pi''$ de $\pi''$).  En d'autres termes, si
$\overline{D}_p$, $\overline{D}_q$ sont les complémentaires
de $D_p$, $D_q$ dans $D$ (de sorte qu'on a 
$\overline{D}_p\setminus I=I(\pi)$, $\overline{D}_q\setminus I'=I(\pi')$),
on a $\overline{D}_pUp\overline{D}_q=D$.

On va supposer maintenant $\pi$ muni d'un $\Sigma\in A(\pi)$,
$\pi'$ muni d'un $\Sigma'\in A(\pi')$, on appellera
``correspondance arithmétique'' entre $\pi$ et $\pi'$
un quadruplet $(\gG,p,q,\Sigma_\gG)$ où $(\gG,p,q)$ sont comme
dessus, et $\Sigma_\gG\in A(\gG)$, tel que l'on ait
$\Sigma_\gG=p^*(\Sigma)=q^*(\Sigma')$.  Si $P_\Sigma$, $P_{\Sigma'}$,
$P_{\Sigma_G}$ sont respectivement les torseurs sous
$\GG_\Sigma$, $\GG_{\Sigma'}$, $\GG_{\Sigma''}$ qu'on sait,
on a deux diagrammes d'isomorphismes de torseurs:
\[\begin{tikzcd}
	& {P_{\Sigma_G}} &&& {\GG_{\Sigma_G}} \\
	{P_\Sigma} && {P_{\Sigma'}} & {\GG_\Sigma} && {\GG_{\Sigma''}}
	\arrow["{p^*_P}", from=2-1, to=1-2]
	\arrow["{q^*_P}"', from=2-3, to=1-2]
	\arrow["{p^*_{\GG}}", from=2-4, to=1-5]
	\arrow["{q^*_{\GG}}"', from=2-6, to=1-5]
\end{tikzcd}\]
d'où par composition un isomorphisme 
$$
(P_\Sigma,\GG_\Sigma)\buildrel\sim\over\to(P_{\Sigma'},
\GG_{\Sigma'}).
$$
On appelle ``isomorphisme arithmétiquement extérieur''
de $(\pi,\Sigma)$ avec $(\pi',\Sigma')$, tout isomorphisme
de torseurs qu'on peut obtenir de cette fa\c con.  Deux
correspondances sont dites équivalentes du point de vue
arithmétiquement extérieur, si elles définissent
le même isomorphisme arithmétiquement extérieur.
La composition est définie par composition des actions
sur $(P_\Sigma,\GG_\Sigma)$ -- on voit que \c ca provient d'une
correspondance.  On trouve donc un groupoïde, dont on
vérifie sans peine qu'il est connexe.
\footnote{On l'appellera le groupoïde des courbes 
arithmétiques extérieures.}
Je dis que les automorphismes de $(\pi,\Sigma)$ ``sont'' les automorphismes
de $(P_\Sigma,\GG_\Sigma)$ définis par des éléments
de $\GG_\Sigma$.  C'est facile\dots

Dans ce groupoïde, il y a un système transitif d'isomorphismes
entre les $(\hat\pi_0,\Sigma_{\pi_0})$, où $\pi_0$ est un
groupe discret à lacets) plus généralement
entre les $(\pi,\Sigma_\alpha)$,
où $\alpha\in P(\pi)$ est un prédiscrétification
de $\pi$ -- définissons donc un isomorphisme arithmétiquement
extérieur de $\pi_{g,\nu}$, $\pi$ \dots  Le groupe de ces 
automorphismes est noté $\GG_0$.  Le groupoïde des
courbes virtuelles arithmétiques extérieures
s'identifie donc à la catégorie des torseurs sous $\GG_0$.
Une prédiscrétification (on pourrait aussi l'appeler
une rigidification arithmétiquement extérieure)
n'est pas autre chose qu'un isomorphisme arithmétiquement
extérieur entre cet élément de référence, et $\pi$.
















%%%%%%%%%%%%%%%%%%%%%%%%%%%%%%%%%%%%%%%%%%%%%%%%%%%%%%%%%%%%%%%
\chapter*{\S \space 29. --- CRITIQUE DE L'APPROCHE PRÉCÉDENTE}\thispagestyle{empty}
\addcontentsline{toc}{section}{29. Critique de l'approche précédente (on rajuste les notions et les conjectures)}
\label{sec:29}
\section*{}


L'approche des paragraphes précédents semble finalement très
brutale.  J'ai même des doutes si la conjecture sur
les propriétés du foncteur ``bouchage de trous'' est vraie
telle quelle.  Il est vrai que pour un groupe profini à lacets
$\pi$, si on fixe $i\in I(\pi)$, on a un homomorphisme canonique
$$\hat{\hat\gT}(\pi,i)=\hat{\hat\gT}(\pi)_i\to
\hat{\hat\gS}(\pi'),\leqno{(1)}$$
(où $\hat{\hat\gT}(\pi)_i$ est le stabilisateur de $i$ dans
$\hat{\hat\gT}(\pi)$), mais il est problématique si c'est
un isomorphisme -- et même si c'est injectif, ou si c'est
surjectif.  Mais, choisissant une discrétification $\pi_0
\subset \pi$, d'où itou $\pi'_0$ pour $\pi'$, l'homomorphisme
précédent induit bel et bien un isomorphisme
$$\gT(\pi_0)_i \isom \gS(\pi'_0)\leqno{(2)}$$
d'où
$$\hat\gT(\pi_0)\buildrel\sim\over\to\hat\gS(\pi'_0)\leqno{(3)}$$
et par suite, la suite exacte $1\to\hat\pi'_0\to
\hat\gS(\pi'_0)\to\hat\gT(\pi'_0)\to 1$
donne :
$$1\to\hat\pi'_0\to\hat\gT(\pi_0)\to
\hat\gT(\pi'_0)\to 1\leqno{(4)}$$
et par suite on trouve un homomorphisme injectif
$$\hat\pi'_0\to\hat\gT(\pi_0)_i\hookrightarrow\hat{\hat\gT}
(\pi_0)_i  \isom  \hat{\hat\gT}(\pi)$$
(où $\hat\pi'_0=\pi'$)
donc un homomorphisme 
$$i_{\pi_0}:\pi'\to\hat{\hat\gT}(\pi)_i\leqno{(5)}$$
dont le composé avec (1) est l'injection canonique
$\pi'\hookrightarrow \hat{\hat\gS}(\pi')$.  Rempla\c cant
la discrétification $\pi_0$ par une
autre, $\pi_1$, on trouve a priori un autre homomorphisme
$i_{\pi_1}:\pi'\to\hat{\hat\gT}(\pi)_i$.  On peut supposer que
$\pi_1=u(\pi_0)$, $u\in\hat{\hat\gT}(\pi_0)_i$
et par transport de structure on trouve
$$i_{\pi_1}=i_{u(\pi_0)}={\rm Int}(u)\circ i_{\pi_0}\circ
u_{\pi'}^{-1}$$
où $u_{\pi'}$ est l'automorphisme de $\pi'$ défini par
$u$ via (1).  Dire que $i_{\pi_0}$ est indépendant du
choix de la discrétification $\pi_0$, revient aussi à dire
que son image dans $\hat{\hat\gT}(\pi)_i$ est un sous-groupe
invariant -- et alors l'action de $\hat{\hat\gT}(\pi)_i$
sur le sous-groupe invariant $i_{\pi_0}(\pi)=i(\pi)$
via automorphismes intérieurs, n'est autre que celle
définie par (1).  Dans ce cas, on trouve par passage au
quotient un homomorphisme
$$\hat{\hat\gT}(\pi)_i/\pi'\to\hat{\hat\gT}(\pi')\leqno{(6)}$$
dont l'injectivité resp. surjectivité équivaudrait à celle
de (1).  Mais il n'est pas évident du tout qu'il en soit toujours ainsi.

S'il n'en était pas ainsi, il s'imposerait de regarder le sous-groupe
de $\hat{\hat\gT}(\pi)_i$ formé des $\gamma\in\hat{\hat\gT}(\pi)_i$
qui normalisent le sous-groupe $i_{\pi_0}(\pi')$, et tels que
l'action induite sur $i_{\pi_0}(\pi')$ correspond à celle
donnée par l'action (1) de $\hat{\hat\gT}(\pi)$ sur $\pi'$.
Ce sous-groupe $H_{\pi_0}$ (qui contient $\hat\gT(\pi_0)_i$)
dépend donc a priori de la discrétification choisie.
Rempla\c cant $\pi_0$ par $u(\pi_0)$ (où $u\in\hat{\hat\gT}
(\pi)_i$) le remplace par ${\rm Int}(u)\cdot H$ --
donc $H$ tout au moins ne change pas, si on fait varier
$\pi_0$ dans une classe de prédiscrétifications.

On peut faire des choix plus symétriques, en considérant
{\it pour tout} $i\in I$ le quotient
correspondant $\pi'_i$ de $\pi$, d'où, pour une 
discrétification donnée $\pi_0$, des homomorphismes
$$i_{\pi_0,i}:\pi'_i\to\hat{\hat\gT}(\pi)_i\subset 
\hat{\hat\gT}(\pi),$$
et on définit $H_{\pi_0}\subset \hat{\hat\gT}(\pi)$ comme
le sous-groupe des $\gamma\in\hat{\hat\gT}(\pi)$ qui
``permutent les $i_{\pi_0,i}$ entre eux'' dans un sens évident.
On a donc
$$H_{\pi_0}\cap\hat{\hat\gT}(\pi)_i=H_{\pi_0,i}\leqno{(7)}$$
et
$$H\supset\hat\gT(\pi_0).\leqno{(8)}$$
Le point que j'ai en vue, c'est que le sous-groupe de $\hat{\hat\gT}
(\pi_0)$ image de $\pi_1(M_\gn)$, doit non seulement normaliser
$\hat\gT(\pi_0)^+$, mais de plus être contenu dans un $H$.
C'est là une condition que j'ai rencontrée
par la bande, à la faveur de
l'hypothèse (peut-être bien h\^ative) que (1) est un
isomorphisme, qui impliquait (facilement) que l'on avait $H=
\hat{\hat\gT}(\pi_0)$ tout entier.  Il est possible que
$\hat{\hat\gT}(\pi_0)$ soit un groupe à tel point démesuré
et pathologique, qu'il ne pourra jamais être
question de dire des choses 
raisonnables (et vraies) sur le groupe tout entier, (tel que la bijectivité
de (1) par exemple) et qu'on soit obligé de travailler avec des sous-groupes
plus petits, qui restent proches du discret (avec quand-même des aspects
supplémentaires ``arithmétiques'', d\^u au $\GG_0=
{\rm Gal}(\overline{\mathbf{Q}}_0/\mathbf{Q})$!).  En fait, il y a (pour $I=
I(\pi)\ne\emptyset$, i.e. $\nu\ne 0$) dans le groupe $\hat\gT(\pi_0)$
une structure simpliciale d'extensions successives, qui va
être respectée par l'action extérieure du groupe de Galois,
et dont il faudrait tenir compte.  Elle fait partie de la ``structure
à l'$\infty$'' dans le $\pi_1$ des multiplicités modulaires
$M_\gn$, qui même pour $\nu=0$ est sans doute non triviale,
et il est possible qu'il faille en tenir compte, pour arriver
à mettre le doigt sur $\GG_\mathbf{Q}$.
\vskip .2cm
Sans essayer de donner d'emblée une description a priori
de $\GG_\mathbf{Q}$ ``dans les $\hat{\hat\gT}_\gn$'', on va procéder
de  fa\c con plus inductive, en partant de la présence de $\GG_\mathbf{Q}$
(pour des raisons arithmético-géométriques), et en essayant
de dégager des propriétés de cette présence peut-être assez fortes
pour finir par donner une caractérisation purement algébrique.
Rappelons que, via le choix de $U_\gn$ (différentiable),
on avait pu construire, par voie transcendante, un
$\widetilde{M_{\gn,\mathbf{C}}}$ et un $\widetilde{U_{\gn,\mathbf{C}}}=
\widetilde{M_{g,\nu+,\mathbf{C}}}$, d'où un $\widetilde{M_{\gn,\mathbf{Q}}}$, et
un $\widetilde{U_{\gn,\mathbf{Q}}}$, d'où un $\pi_1(U_{\gn,\mathbf{Q}})$,
avec une filtration en trois crans, dont les facteurs
sont respectivement canoniquement isomorphes à
$$\hat\pi_\gn,\ \hat\gT(\pi_\gn)^+,\ \GG_\mathbf{Q}={\rm Gal}(\overline{\mathbf{Q}_0}/\mathbf{Q}).\leqno{(9)}$$
Ce groupe s'envoie (on présume injectivement) dans $\hat{\hat\gS}(\pi_\gn)$,
induisant un isomorphisme entre son sous-groupe
$\pi_1(U_{\gn,\overline{\mathbf{Q}}_0})$ et $\hat\gS(\pi_\gn)$;
désignons son image par $M_\gn$.  Donc c'est un sous-groupe fermé
$$\hat\gS_\gn\subset M_\gn\subset \hat{\hat\gS}_\gn\leqno{(10)}$$
et $\hat\gS^+_\gn$ est normal dans $M_\gn$ (mais pas $\hat\gS_\gn$ !),
La donnée d'un tel $M_\gn$ équivaut à celle d'un $\cN_\gn$
$$\hat\gT_\gn\subset \cN_\gn\subset \hat{\hat\gT}_\gn,\leqno{(11)}$$
avec $\hat\gT^+_\gn$ normal dans $\cN_\gn$.  On pose\footnote{Dans $\GG_\gn$, on a un élément canonique 
d'ordre 2 $\tau_\gn$, correspondant aux éléments de $\hat\gT_\gn
\setminus \hat\gT_\gn^+ = \hat\gT_\gn^-$.}:
$$\GG_\gn=\cN_\gn/\hat\gT^+_\gn \isom M_\gn/\hat\gS^+_\gn.
 \leqno{(12)}$$
On a un homomorphisme surjectif
$$\GG_\mathbf{Q}\to\GG_\gn\leqno{(13)}$$
dont on présume qu'il est bijectif -- i.e. que les
$\GG_\gn$ sont canoniquement isomorphes entre eux.

Soit maintenant $\pi_0$ un groupe discret à lacets de type
$\gn$, alors on définit des sous-groupes $M_{\pi_0}$,
$\cN_{\pi_0}$ :
$$\hat\gS(\pi_0)\subset M_{\pi_0}\subset \hat{\hat\gS}(\pi_0)\leqno{(14)}$$
$$\hat\gT(\pi_0)\subset \cN_{\pi_0}=M(\pi_0)/\hat\pi_0\subset 
\hat{\hat\gT}(\pi_0)\leqno{(15)}$$
(et on pose $\GG_{\pi_0}=M_{\pi_0}/\hat\gS(\pi_0)$),
en utilisant un isomorphisme $\pi_0\buildrel \sim \over
\to\pi_\gn$ et en procédant par transport de structure
-- le résultat n'en dépend pas.  Plus généralement,
partons d'un couple $(\pi,\alpha)$ d'un groupe $\pi$ profini,
muni d'une {\it prédiscrétification} $\alpha$, qu'on peut donc
interpréter comme une classe d'isomorphismes
$\hat\pi_\gn\to\pi$, définie modulo composition
à droite par un $u\in\hat\gS_\gn$.  Alors par
transport de structure on en déduit des groupes $M_\alpha$,
$\cN_\alpha$
$$\hat\gS_\alpha\subset M_\alpha\subset \hat{\hat\gS}(\pi)\leqno{(16)}$$
$$\hat\gT_\alpha\subset \cN_\alpha=M_\alpha
/\pi\subset \hat{\hat\gT}(\pi)\leqno{(17)}$$
avec $\hat\gS(\pi)^+$ normal dans $M_\alpha$, i.e. $\hat\gT^+(\pi)$
normal dans $\cN_\alpha$.  On pose
$$\GG_\alpha=M_\alpha/\hat\gS^+_\alpha \isom \cN_\alpha/\hat\gT^+_\alpha.
\leqno{(18)}$$
On a un élément canonique
$$\tau_\alpha\in\GG_\alpha,\ \ \tau_\alpha^2=1,\ \ \tau_\alpha\ne 1.
\leqno{(19)}$$

Soit maintenant $\pi$ un groupe profini à lacets de type $(\gn)$.
On appelle {\it arithmétisation} de $\pi$ la donnée d'un
élément de
$$A(\pi)={\rm Isom}_{\rm lac}(\hat\pi_\gn,\pi)/M_\gn
 \isom {\rm Isomext}_{\rm lac}(\hat\pi_\gn,\pi)/\cN_\gn.
\leqno{(20)}$$
Soit $P(\pi)$ l'ensemble des prédiscrétifications orientées de $\pi$:
$$P(\pi)={\rm Isom}_{\rm lac}(\hat\pi_\gn,\pi)/\hat\gS
^+_\gn \isom {\rm Isomext}_{\rm lac}(\hat\pi_\gn,\pi)/\hat\gT^+_\gn;
\leqno{(21)}$$
alors $\GG_\gn$ opère librement à droite sur $P(\pi)$, et
$$A(\pi) \isom  P(\pi)/\GG_\gn\leqno{(22)}$$
-- $P(\pi)$ est un torseur relatif sur $A(\pi)$, de groupe $\GG_\gn$.
Pour $a\in A(\pi)$, soit $M_a$ le sous-groupe de $\hat{\hat\gS}
(\pi)$ des automorphismes de $(\pi,a)$, il opère sur le
$\GG_\gn$-torseur $P_a$ des discrétifications orientées
de $\pi$ sur $a$.  Pour $\alpha\in P_a$, soit $\hat\gS_\alpha
\subset \hat{\hat\gS}(\pi)$; alors $\hat\gS^+_\alpha$ ne dépend
pas de $\alpha$ (on le notera $\hat\gS_a$);
\footnote{mais attention, 
$\hat\gS_\alpha$ dépend de $\alpha$, i.e. $\tau_\alpha\in\GG_a$ 
dépend de $\alpha$.}
c'est aussi le noyau de l'opération de $M_a$
sur $P_a$.  On a
$$\hat\gS^+_a\subset M_a\subset \hat{\hat\gS}(\pi)\leqno{(23)}$$
d'où encore
$$\hat\gT^+_a\subset \cN_a\subset \hat{\hat\gT}(\pi)\leqno{(24)}$$
en posant 
$$\cN_a=M_a/\pi,\ \hat\gT^+_a=\hat\gS^+_a/\pi.\leqno{(25)}$$
On pose
$$\GG_a=M_a/\hat\gS^+_a \isom  \cN_a/\hat\gT^+_a.\leqno{(26)}$$
Alors $\GG_a$ s'identifie au commutant de $\GG_\gn$ opérant
sur $P_a$, i.e. à $\GG_\gn$ ``tordu'' par le torseur $P_a$.
Le choix d'un $\alpha\in P_a$ revient donc au choix d'un tel
isomorphisme
$$\GG_a \isom  \GG_\gn\leqno{(27)}$$
qui est changé par automorphisme intérieur si on change $\alpha$\dots

On a la catégorie des groupes à lacets extérieurs arithmétisés
de type $\gn$ -- c'est un groupoïde connexe, avec une origine
fixée $(\hat\pi_\gn,a_\gn)$ ($a_\gn$ arithmétisation ``canonique''),
dont le groupe des automorphismes est $\cN_\gn$, extension de
$\GG_\gn$ par $\hat\gT_\gn^+ \isom  \pi_1(M_{\gn,\mathbf{Q}})$ (si
$\GG_\mathbf{Q}\to\GG_\gn$ est injectif!) (calculé par rapport
au rev\^ etement universel canonique de $M_{\gn,\mathbf{Q}})$\dots
\vskip .2cm
Se donner un objet de cette catégorie, c'est essentiellement
la même chose (à isomorphisme unique près) que de se donner
un revêtement universel $\widetilde{M_{\gn,\mathbf{Q}}}$ de
$M_{\gn,\mathbf{Q}}$ -- ou un torseur sous $\cN_\gn$; on considère
alors la famille de courbes de type $\gn$ sur $\widetilde{
M_{\gn,\mathbf{Q}}}$
$$U_\gn\times_{M_\gn}\widetilde{M_{\gn,\mathbf{Q}}}=U_{\gn}(\widetilde{
M_{\gn,\mathbf{Q}}})\leqno{(28)}$$
comme étant ``la'' courbe algébrique dont le $\pi_1$
extérieur (qui est donc le $\pi_1$ de ce schéma\dots)
soit le groupe extérieur à lacets donné.

Si on prenait les groupes à lacets arithmétisés,
pas extérieurs, on aurait encore un groupoïde
connexe avec ``origine'' $(\hat\pi_\gn,a_\gn)$, avec un groupe
d'automorphismes qui est $M_\gn \isom \pi_1(U_{\gn,\mathbf{Q}})$.
La donnée d'un objet de cette catégorie revient à la donnée
d'un revêtement universel (non seulement de $M_{\gn,\mathbf{Q}}$,
mais) de $U_{\gn,\mathbf{Q}}$, soit $\widetilde{U_{\gn,\mathbf{Q}}}$, qui
sert de revêtement universel
de référence pour la courbe
relative (28), permettant alors de préciser son groupe
fondamental comme un vrai groupe à lacets (pas seulement
un groupe extérieur).

On peut enfin regarder aussi la catégorie des groupes profinis
à lacets arithmétisés, où on prend comme morphismes
les {\it isomorphismes arithmétiquement extérieurs}.
On trouve encore un groupoïde connexe avec origine marquée
$(\hat\pi_\gn,a_\gn)$, dont le groupe des automorphismes est 
maintenant $\GG_\gn$.  Maintenant les groupes $\hat\pi_0$
($\pi_0$ groupe à lacets discret de type $\gn$) sont 
canoniquement isomorphe entre eux.  La catégorie est
(conjecturalement, admettant que $\GG_\mathbf{Q}\to \GG_\gn$
soit un isomorphisme) équivalente à celle des revêtement 
universels de ${\rm Spec}\,\mathbf{Q}$, i.e. à celle des clôtures
algébriques de $\mathbf{Q}$, l'élément origine correspondant à $\overline{\mathbf{Q}}_0$.

Quand on se donne un $\pi$ avec arithmétisation $a$, alors il
lui correspond donc canoniquement une clôture algébrique
$\overline{\mathbf{Q}}$ de $\mathbf{Q}$.  Le $\GG_\mathbf{Q}$-torseur des isomorphismes
${\rm Isom}(\overline{\mathbf{Q}}_0,\overline{\mathbf{Q}})$, i.e. des plongements
$\overline{\mathbf{Q}}\hookrightarrow \mathbf{C}$, est en correspondance 1-1
avec l'une ($P_a$) des prédiscrétifications orientées
de $\pi$ donnant naissance à $a$.

Notons que tout $\alpha$ définit un élément $\tau_\alpha\in
\GG_a$, $\tau_\alpha^2=1$ ($\tau_a\ne 1)$, d'où une application
canonique
$$P_a\to {}_2\GG_a.\leqno{(29)}$$
On voit que cette application est compatible avec l'action du
groupe $\pm 1$ sur $P_a$,
$$\tau_\alpha=\tau_{-\alpha}.$$
S'il est vrai que $\GG_\mathbf{Q}\buildrel\sim\over\to \GG_\gn$,
alors (29) induit par passage au quotient une application
bijective

\noindent (30)\ \ \ $P_a/\pm 1\buildrel\sim\over\to$ ensemble
des éléments d'ordre $2$ de $\GG_a$

(où $P_a/\pm 1=P_a^\natural=$ ensemble des prédiscrétifications
-- pas orientées -- sur $a$).

\noindent Cela provient
du fait connu que dans $\GG_\mathbf{Q}$, les seuls éléments
d'ordre $2$ sont les conjugués de $\tau$, et que le
centralisateur de $\tau$ dans $\GG_\mathbf{Q}$ est réduit à
$\{1,\tau\}$.  On peut dire que la donnée d'une
discrétification (pas orientée) $\alpha^\natural$
sous l'arithmétisation $a$, revient à la donnée d'une
valuation archimédienne sur la cl\^ oture
algébrique $\overline{\mathbf{Q}}$ de $\mathbf{Q}$
définie par $(\pi,a)$ (i.e. d'un isomorphisme 
$\overline{\mathbf{Q}}_1 \isom \overline{\mathbf{Q}}_0$ {\it modulo conjugaison
complexe}).













%%%%%%%%%%%%%%%%%%%%%%%%%%%%%%%%%%%%%%%%%%%%%%%%%%%%%%%%%%%%%%%
\chapter*{\S \space 30. --- PROPRIÉTÉS DES $\cN_{g,\nu}$, $\GG_{g,\nu}$}\thispagestyle{empty}
\addcontentsline{toc}{section}{30. Propriétés des $\cN_{g,\nu}$, $\GG_{g,\nu}$}
\label{sec:30}
\section*{}

\begin{enumerate}
    \item[a)] {\bf Propriétés liées aux sous-groupes finis de Teichmüller}
\end{enumerate}

On aimerait dégager des propriétés, inspirées par le contexte
géométrico-arithmétique, mais qui puissent se formuler de
fa\c con purement algébrique -- et qui soient assez fortes peut-être
pour finir par caractériser les $\cN_{g,\nu}$.  

Revenons à un $(\pi,a)$, groupe à lacets arithmétisé,
heuristiquement, il correspond à la donnée d'une clôture
algébrique $\overline{\mathbf{Q}}$ de $\mathbf{Q}$, et (si $\pi$ est  donné
extérieurement) d'une famille algébrique, paramétrée
par un revêtement universel $\widetilde{M_{g,\nu,\overline{\mathbf{Q}}}}$ de
$M_{g,\nu,\overline{\mathbf{Q}}}$, de courbes algébriques de
type $(g,\nu)$.  Dans cette optique, réduire cette famille
à {\it une} courbe algébrique -- i.e. se donner une courbe
algébrique de type $(g,\nu)$ sur $\overline{\mathbf{Q}}$ -- doit revenir
à se donner un noyau de relèvement de l'homomorphisme surjectif
$$\cN_a\to\GG_a\leqno(1)$$
(de noyau $\hat\gT^+_a$).  La donnée d'un tel relèvement
sur un sous-groupe ouvert $\Gamma'$ revient à la donnée
d'une courbe algébrique de type $g,\nu$, définie
sur l'extension finie $K'\subset \overline{\mathbf{Q}}$ définie par $\Gamma'$.
Il s'agit ici non pas de relèvements continus quelconques, 
mais de relèvements ayant des propriétés
particulières (dont certaines fort profondes, du genre ``Weil''\dots
mais {\it peut-être} conséquences de propriétés
beaucoup plus simples).  Les propriétés à dégager
devraient en tout cas être stables par passage à un sous-groupe
ouvert plus petit.  Parmi ces propriétés, il y aurait que ces
germes de relèvements pourraient à leur tour se remonter 
à $\hat{\hat\gS}(\pi)$ lui-même -- de fa\c con également
``admissible'' en un sens à préciser -- et même qu'il y
aurait ``beaucoup'' de classes de $\pi$-conjugaison de tels germes
de relèvements, pour un germe d'action extérieure sur $\pi$
déjà donné -- ce qui correspond au fait naïf que la
courbe algébrique $U$ sur $\overline{\mathbf{Q}}$ définie par cette
action extérieure a ``beaucoup de points'' rationnels sur 
$\overline{\mathbf{Q}}$.  En fait, il suffirait, à la limite, de parler des
propriétés de ces relèvements plus complets en un germe d'une
vraie action de $\GG_a$ sur $\pi$ (pas seulement extérieure) --
dont les actions extérieures vont se déduire par composition avec
$$M_a\to\cN_a \isom M_a/\pi.$$
On suppose donc donné un sous-groupe fermé
$$\Gamma\subset M_a\subset \hat{\hat\gS}(\pi)$$
tel que
$$\Gamma\to\GG_a=M_a/\pi\leqno(a)$$
est injectif, et a comme image un sous-groupe ouvert de $\GG_a$ --
étant entendu que deux sous-groupes conjugués sous $\pi$ --
voire même parfois, sous $M_a$ -- ne sont pas considérés
comme essentiellement distincts.  On considère, en même temps
que $\Gamma$, ses sous-groupes ouverts $\Gamma'$.  On veut surement,
pour de tels sous-groupes ouverts
$$\pi^{\Gamma'}=\{1\}.\leqno(b)$$
Si on désigne par $\tilde\Gamma$ l'image de $\Gamma$ dans $\cN_a$,
de sorte qu'on a une extension
$$1 \to \pi \to E \to \tilde \Gamma
\to 1\leqno(1')$$
(où $\tilde\Gamma$ est isomorphe à $\Gamma$); on peut considérer
$\Gamma$ comme une section de cette extension, $\Gamma'$ comme 
une section partielle.  L'hypothèse $\pi^{\Gamma'}=\{1\}$
assure la {\it rigidité} de la catégorie des points de $U$
à valeurs dans $\overline{\mathbf{Q}}$, paradigmée par celle des germes de scindage
de l'extension (1').

On peut aussi regarder les ensembles $(\hat\gS_a^+)^{\Gamma'}=
{\rm Centr}_{M_a}(\Gamma')\cap\hat\gS^+_a$ et
$(\hat\gT^+_a)^{\Gamma'}={\rm Centr}_{\cN_a}(\tilde\Gamma')\cap
\hat\gT^+_a$; on a donc une suite exacte
$$1\to \pi^{\Gamma'}\to\bigl(\hat\gS^+_a\bigr)
^{\Gamma'}\to\bigl(\hat\gT^+_a\bigr)^{\Gamma'}\leqno(2)$$
qui, compte tenu de l'hypothèse $\pi^{\Gamma'}=\{1\}$, donne
$$\bigl(\hat\gS^+_a\bigr)^{\Gamma'}\hookrightarrow\bigl(\hat\gT^+_a\bigr)
^{\Gamma'}.\leqno(3)$$
Le deuxième membre de (3), par notre dictionnaire hypothétique,
devrait être cano\-niquement isomorphe au groupe des automorphismes
de la courbe $U$ sur $K'$, donc être un groupe fini, et même
la limite inductive (pour $\Gamma'$ décroissant) -- qui est le groupe
des automorphismes de $U$ sur $\overline{\mathbf{Q}}$ -- est finie.  Désignant
par un exposant $\natural$ le germe de groupe correspondant, on veut donc que
$$Z=\bigl(\hat\gT^+_a\bigr)^{\Gamma^\natural} \bigl(={\rm Centr}_{\cN_a}
(\tilde \Gamma^\natural)\cap\hat\gT^+_a\bigr)\ {\rm 
soit\  un\  groupe\  fini} \leqno(c)$$
-- ce qui fait pendant à (b), et exprime
que les groupes d'automorphismes des points de $M_{g,\nu,\overline{\mathbf{Q}}}$
sur $\overline{\mathbf{Q}}$ sont finis.

En fait, on voudrait que $Z$ soit conjugué dans $M_a$ à un sous-groupe
de $\gT^+$ (si on suppose qu'on dispose d'une discrétification $\pi_0$
de $\pi$, permettant de définir $\gT$ -- sinon, on peut exprimer
cette propriété en disant qu'il existe une discrétification
$\pi_0$ de $\pi$, compatible avec $a$, telle que $Z\subset \gT(\pi_0)\ldots$)

Considérons maintenant un sous-groupe fini quelconque
$G\subset \hat\gT^+_a$.  [N.B.  si on prend
seulement $G\subset M_a$, de sorte que l'image de $G$
dans $\GG_a=M_a/\hat\gT^+_a$ est un sous-groupe fini, alors s'il est
vrai que $\GG_a \isom \Gal(\overline{\mathbf{Q}}, \mathbf{Q})$, cette image doit être d'ordre $1$ ou $2$,
et dans le deuxième cas, doit définir un $\tau\in\GG_a$ correspondant
à une {\it prédiscrétification} (pas orientée) bien déterminée
de $\pi$.  Ce cas devrait être encore réutilisée dans la suite$\ldots$]
Supposons alors que [ce n'est sans doute {\it pas} automatique
-- si on ne suppose {\it pas} $\cN_a={\rm Norm}_{\hat{\hat\gT}}(\hat\gT_a)$],
que $G$ est contenu dans $\gT^+$, pour une discrétification convenable
dans la classe $a$, et que l'action extérieure ainsi obtenue de $G$
sur un $\pi_0$ discret de type $(g,\nu)$ soit toujours 
{\it réalisable}.  Elle
est donc réalisable aussi pour une action de $G$ sur une structure complexe,
donc algébrique, ce qui signifie qu'on peut trouver un
germe de relèvement {\it admissible}
$$\GG_a\to \cN_a$$
qui centralise $G$.  Considérons d'autre part
$${\rm Centr}_{\cN_a}(G)\to\GG_a\leqno(4)$$
dont le noyau est ${\rm Centr}_{\hat\gT^+_a}(G)=\hat\gT_a^{+G}$.
Si on se pla\c cait dans le contexte discret (avec une discrétification
$\pi_0\subset \pi$ invariante par $G$) on aurait, par les conjectures
standard topologiques, que $\hat\gT_a^{+G}$ est le groupe des
automorphismes, dans la catégorie isotopique, de l'action de $G$
sur une surface $U$ de type $(g,\nu)$, décrite par l'opération
extérieure donnée de $G$ sur $\pi_0$.  Ce groupe n'a aucune
raison d'être fini -- s'il l'était, l'homomorphisme (4) serait
à noyau fini, donc serait un isomorphisme des noyaux des groupes,
et il y aurait un germe de relèvement unique $\GG_a^\natural\to
M_a$ qui centraliserait $G$ -- ce qui signifierait qu'il y a une seule
fa\c con de réaliser l'opération topologique de $G$ sur $U$
par une opération analytique complexe, donc algébrique -- or ce n'est
surement pas le cas, l'ensemble des points fixes de $G$ opérant 
sur l'espace de Teichmüller $\widetilde{ M_{g,\nu}}$ n'est pas réduit
à un point -- c'est (dans le contexte algébrique) une multiplicité
schématique qui peut être de dimension quelconque -- elle est de 
dimension $>0$ en tout cas, si le quotient $U/G$ n'est pas de genre $0$.

A retenir en tout cas, comme propriétés plausibles:
\vskip .2cm
\noindent (d) Pour tout sous-groupe fini $G$ de $\hat\gT^+_a$ (ou
du moins si $G\subset \gT^+$, quand on dispose d'une discrétification
$\pi_0\subset \pi$ dans $a$), regardant ${\rm Centr}_{\hat{\hat\gT}}(G)=
Z(G)$, $\cN_a\cap Z(G)\to \GG_a$ a une image 
ouverte (et il y a m\^ eme des
germes de relèvements admissibles $\GG_a\to Z(G)$).

Ceci implique une propriété non triviale pour les $g\in M_a$ en tant
qu'éléments de $\hat{\hat\gT}(\pi)$, ou mieux de $\cN'={\rm Norm}
_{\hat{\hat\gT}}(\hat\gT)$ [à savoir: $\exists n\in\mathbf{N}^*$
tel que $g^n$ soit congru mod $\hat\gT^+_a$ à un élément
de $Z(G)$\dots].  Considérons en effet l'image $Z'_G$ de $Z(G)\cap\cN'$
dans $\Gamma'=\cN'/\hat\gT$.  On a évidemment $\GG_a\subset \Gamma'$,
et l'image de $Z(G)\cap\cN_a$ dans $\GG_a$ n'est autre que $Z'_G\cap
\GG_a$.  Dire que celle-ci est ouverte, i.e. d'indice
fini, implique donc que $\forall
g\in\GG_a$, $\exists n\in\mathbf{N}^*$ tel que $g^n\in Z'_G$.

Notons que dans $\gT(\pi_{g,\nu})$ il n'y a qu'un nombre fini de classes
de conjugaison de sous-groupes finis, donc si on regarde
leurs centralisateurs $Z_G$ dans ${\hat{\hat\gT}}(\pi_{g,\nu})$
et leurs images dans
$$\Gamma'_{g,\nu}={\rm Norm}_{\hat{\hat\gT}_{g,\nu}}(\hat\gT^+_{g,\nu})
/\hat\gT^+_{g,\nu},$$
on ne trouve qu'un nombre fini de sous-groupes de $\Gamma'_{g,\nu}$,
soit $Z'_{g,\nu}$ leur intersection.  On voit donc que le sous-groupe
$\GG_{g,\nu}$ de $\Gamma'_{g,\nu}$ est tel que $\GG_{g,\nu}\cap
Z'_{g,\nu}$ soit ouvert dans $\GG_{g,\nu}$.
\vskip .2cm
Passons maintenant à des sous-groupes finis de $\hat{\hat\gS}(\pi)$
lui-même -- ou du moins de $M_a$ -- ou, ce qui revient presque
au même, de $\hat\gS^+_a$.  On va, comme tantôt, se borner à des
$G\subset \gS(\pi_0)$, pour une discrétification convenable $\in a$,
car autrement il n'y aurait rien à dire.  On sait qu'alors $G$
est cyclique -- et correspond à une action de $G$ sur un
$U$ topologique, {\it avec point fixe}.  On peut la réaliser de
fa\c con complexe, ce qu'on exprime en disant qu'il existe un relèvement
admissible $\GG_a^\natural \to M_a$, qui centralise $G$.
On sait d'ailleurs, si $G\ne 1$, que
$$\pi^G=\{1\}\leqno(5)$$
(ou plutôt, on le sait dans le cas discret -- on l'admet dans
le contexte profini) -- ce qu'on peut encore interpréter comme
une propriété de rigidité -- ainsi
$${\rm Centr}_{\hat{\hat\gS}}(G)=\hat{\hat\gS}^G\to\hat{\hat\gT}^G$$
est {\it injectif}, donc si on a $\Gamma\subset \cN_a$ qui est dans l'image,
alors il existe un relèvement $\Gamma\to\hat{\hat\gS}$ unique
qui centralise $G$, i.e. tel que $\Gamma\to\hat{\hat\gS}^G$.
Mais en fait ce qu'on saura, pour un $\Gamma\subset M_a$ donné
(décrivant une courbe algébrique via son action extérieure sur un $\pi_1$)
c'est que $\Gamma\subset M_a^G$ (i.e. l'action extérieure
commute à une action extérieure donnée de $G$, i.e. $G$
opère sur la courbe algébrique) -- et on voudrait néanmoins
en conclure que lorsqu'on a relevé $G\to\hat\gT^+$
ou $G\to\hat\gS^+$ (i.e. quand on s'est donné un point
fixe de l'action de $G$ sur $U$) alors l'action de $\Gamma^\natural$ se
remonte automatiquement en $\Gamma^\natural\to M_a$ de fa\c con
à commuter\dots\dots(i.e. $\Gamma^\natural\to M_a^G$).  Est-ce
là une propriété du choix de $M_a$ ou de celui du relèvement
$\Gamma^\natural\to\cN_a$, ou de $G$??

Dans le cadre discret, sauf erreur, si on a une action fidèle
discrète de $G$ fini sur $\pi_0$, alors $\gS(\pi_0)^G\to\gT(\pi_0)
^G$ (qui est injectif, par $\pi_0^G=1$) est aussi surjectif.  Non,
il y {\it a} erreur -- \c ca signifierait tel quel que si $G$ opère
fidèlement sur une surface topologique $U$ de type $g,\nu$,
avec un point fixe $P$, alors les automorphismes qui commutent à $G$
{\it fixent} $P$ (au lieu de permuter seulement les points fixes
entre deux\dots).  Donc il ne faut {\it pas} s'attendre à ce que tout
tout élément  dans $\cN_a^G$, ni même dans $\hat\gT_a^G$, se remonte à
$M_a^G$ -- mais plutôt ceci: pour un $G\subset \gT_a(\pi_0)$
sous-groupe fini donné, les classes de $\pi$-conjugaison
de ``remontages'' à $\hat\gS$, qui s'interprètent comme des
points fixes d'une action de $G$ sur quelque $U$, forment un
ensemble fini, sur lequel $\hat{\hat\gT}(\pi)^G$ opère
de fa\c con naturelle, et le stabilisateur d'un point $P$ i.e.
d'une classe de conjugaison de relèvements se remonte de fa\c con
unique en $\hat{\hat\gS}^G$.  Il faudrait réexaminer ceci
de fa\c con plus soigneuse par la suite.  Mais il me semble
qu'on ne trouve pas ici de nouvelles propriétés des
$\Gamma\subset M_a$ ni de $M_a$ lui-même, i.e. de $\GG_a$
comme sous-groupe profini de $\Gamma'={\rm Norm}_{\hat{\hat\gT}}
(\hat\gT)/\hat\gT$.














%%%%%%%%%%%%%%%%%%%%%%%%%%%%%%%%%%%%%%%%%%%%%%%%%%%%%%%%%%%%%%%
\chapter*{\S \space 31. --- DIGRESSION SUR LES RELÈVEMENTS D'UNE ACTION EXTÉRIEURE D'UN GROUPE FINI $G$ SUR UN GROUPE PROFINI À LACETS $\pi$}\thispagestyle{empty}
\addcontentsline{toc}{section}{31. Digression sur les relèvements d'une action extérieure d'un groupe fini sur un groupe profini à lacets}
\label{sec:31}
\section*{}

On suppose l'action extérieure {\it fidèle}, i.e. $G\subset \hat{\hat\gT}
(\pi)=\hat{\hat\gT}$, et que $G=G^+$.  On suppose de plus qu'il existe
une discrétification invariante $\pi_0$, i.e. telle que $G\subset 
\gT(\pi_0)$.  Si on suppose $G$ cyclique, alors sauf erreur il est
prouvé que la situation est {\it réalisable} topologiquement (donc aussi
de fa\c con analytique complexe\dots)

Dans le cas discret, la signification des classes de $\pi_0$ conjugaison de 
relèvement\hfill\break 
$G\to\gS(\pi_0)$ est bien comprise, de même
que pour les relèvements partiels.  On trouve une réalisation
canonique [si $U^!\ne\emptyset$] du groupoïde fondamental
associé au groupe extérieur $\pi_0$, par un groupoïde fini
$[U^!]$ sur lequel $G$ opère au sens strict, dont les
points correspondent aux classes de $\pi_0$-conjugaison
de sections partielles $\ne 1$ maximales.  Le groupe $\gT(\pi_0)^G$
opère de fa\c con également canonique sur ce groupoïde.  Si
$P\inU^!$ correspond à un relèvement $G\to\gS(\pi_0)$,
alors un élément $\\cdotu\in\gT(\pi_0)^G$ est dans l'image
de $\gS(\pi_0)^G$, i.e. peut se remonter en $u\in\gS(\pi_0)$ commutant à $G$,
si et seulement si $\\cdotu(P)=P$, i.e. (si $\\cdotu$ est remonté de fa\c con
quelconque en $v$), si et seulement si $v(G)$ est $\pi_0$-conjugué à $G$, 
i.e.  si et seulement s'il existe $g\in\pi_0$ tel que $v(G)=
{\rm int}(g)(G)$, i.e. $u\buildrel {\rm def}\over ={\rm int}(g^{-1})v$ 
fixe $G$  (auquel cas bien s\^ur il centralise, par l'hypothèse 
$\\cdotu\in\gT^G$).  Donc notre brillante assertion est une tautologie, et 
donc le relèvement est unique.
Comme $U^!$ est fini, le stabilisateur $\gT(\pi_0)^G_P$ de $P$ dans
$\gT(\pi_0)^G$ est d'indice fini, et c'est donc ce sous-groupe
d'indice fini qui se remonte gaillardement et canoniquement.

Que peut-on dire dans le contexte profini?  Bien s\^ur, on a une
application canonique
\vskip .2cm
classes de $\pi_0$-conjugaison\ \ \ \ \ \ \ \ \ \ \ \ \ \ \ \ 
\ \ \ \ \ \ \  classes de $\pi_0$-conjugaison

de relèvements de \ \ \ \ \ \ \ \ \ \ \ \ \ \ \ \ \ \ \ \ \ \ \ \ \ \ \ \ \ 
\ \ \ des relèvements de

$G$ en $G\to\gS(\pi_0)$\ \ \ \ \ \ \ \ \ \ \ \ \ \ \  \ \ \ \
$\longrightarrow$
\ \ \ \ \ \ \ \ \ \ $G$ en $G\to \hat{\hat\gS}(\pi_0)$,

i.e. de scindage de l'extension\ \ \ \ \ \ \ \ \ \ \ \ \ \ \ \ \ \ i.e. de 
scindage de l'extension

$1\to\pi_0\to E\to G \to 1$\ \ \ \ 
\ \ \ \ \ \ \ \ \ \ \ \ \ \ \ \ \ \ \ \ \ \ \ \ $1\to\hat\pi_0=\pi
\to \hat{E} \to G \to 1$.
\vskip .2cm
Il faudrait examiner d'abord
\begin{enumerate}
    \item[a)]  la question de la bijectivité de cette application,
    \item[b)]  si $\pi^G=1$ (rigidité), pour $G\ne \{1\}$. (pour un relèvement donné, dans $E$ pour simplifier).
\end{enumerate}

J'ai envie de conjecturer sans vergogne qu'il en est ainsi -- ce
qui impliquerait par exemple que pour tout tel relèvement de $G$, il y a un 
sous-groupe {\it ouvert} $(\gT^G)_P$ du groupe 
(peut-être vraiment immense a priori!) $\hat{\hat\gT}^G$, qui se remonte 
canoniquement de fa\c con à commuter à $G$\dots.

Il faudrait manifestement faire, en même temps qu'une théorie
des opérations extérieures des groupes finis sur des groupes discrets
à lacets (qui est pour le moment extrêmement conjecturale)
une théorie analogue dans le cas profini -- j'aurai sans doute à y
revenir par la suite.















%%%%%%%%%%%%%%%%%%%%%%%%%%%%%%%%%%%%%%%%%%%%%%%%%%%%%%%%%%%%%%%
\chapter*{\S \space 32. --- RETOUR SUR LES ASPECTS ARITHMÉTIQUES DU BOUCHAGE DE TROUS : RELATIONS ENTRE $\GG_{g,\nu}$ et $\GG_{g,\nu-1}$}\thispagestyle{empty}
\addcontentsline{toc}{section}{{\bf 32.} Retour sur les aspects arithmétiques du bouchage \\ de trous: relations entre $\GG_{g,\nu}$ et $\GG_{g,\nu-1}$}
\label{sec:1}
\section*{}


Bien s\^ur, on a que
$$\cN_{g,\nu}\to\gS_\nu$$
est surjectif (puisque $\gT_{g,\nu}\to\gS_\nu$ l'est), donc on aura
$$1\to\cN_{g,\nu}^!\to\cN_{g,\nu}\to\gS_\nu\to
1\leqno(1)$$
et le diagramme cartésien de sous-groupes de $\hat{\hat\gT}_{g,\nu}$:
\[\begin{tikzcd}
	{\hat\gT^+_{g,\nu}} && {\cN_{g,\nu}} \\
	\\
	{\hat\gT^{!+}_{g,\nu}} && {\cN^!_{g,\nu}}
	\arrow[hook, from=3-1, to=1-1]
	\arrow[hook, from=1-1, to=1-3]
	\arrow[hook, from=3-1, to=3-3]
	\arrow[hook, from=3-3, to=1-3]
\end{tikzcd}\leqno{(2)}\]
[$\cN_{g,\nu}= \cN_{g,\nu}^! \cdot \hat\gT_\gn^+$] donnera un isomorphisme
$$\cN_{g,\nu}^!/\hat\gT^{!+}_{g,\nu}\buildrel\sim\over\to
\cN_{g,\nu}/\hat\gT^+_{g,\nu}=\GG_{g,\nu}.\leqno(3)$$
Dans le cas d'un $(\pi,a)$, ($\pi$ groupe extérieur à lacets,
$a$ une arithmétisation), on aura de même:
$$
\begin{cases}
\cN^!_a/\hat\gT^+_a\buildrel\sim\over\to\cN_a/\hat\gT^+_a
=\GG_a \\ 
\cN_a=\cN^!_a \cdot \hat\gT^+_a.
\end{cases}\leqno(4)
$$
Considérons maintenant pour tout $i\in I$ le stabilisateur $\cN_i$
de $i$ dans $\cN$, et le groupe à lacets (pas extérieur!) $\pi_i$, de type
$(g,\nu-1)$, déduit de $\pi$ par ``bouchage du trou $i$''
\footnote{On écrit ici $\cN$ etc. au lieu de $\cN_a$;
on suppose $(g,\nu-1)$ aussi anabélien, i.e. $2g+\nu\ge4.$}.
On a alors, pour une discrétification choisie $\alpha$ compatible
avec l'arithmétisation, un homomorphisme injectif
$$\pi_i'\buildrel {\phi_i}\over\to\hat\gT^{+!}\subset \cN^!\subset 
\hat{\hat\gT}_i,\leqno(5)$$
tel que le composé
$$\pi_i' \to\hat{\hat\gT}_i \to \hat{\hat\gS}(\pi_i') 
(\hookleftarrow \pi_i')$$
soit l'inclusion canonique.  Ceci posé, on veut
\footnote{Plutôt, {\it il est vrai que!}}
que l'image de (5) (qui n'est peut-être pas invariante dans $\hat\gT_i$) soit
{\it invariante dans} $\cN_i$,
\footnote{(***)}{Cela signifie aussi que $\phi_i$ ne dépend pas du 
choix de la discrétification de classe $\alpha$\dots}
ou ce qui revient au même, dans $\cN^!$ (car $\cN_i=\cN^!Dot\hat\gT_i^+$,
or $\hat\gT_i^+$ invarie cette image).  Revenant à la situation 
universelle, cela signifie:

\noindent (5') $\cN_{g,\nu}^!\subset \hat{\hat\gT}_{g,\nu}$ est contenu dans
le normalisateur des $\pi_i'\rinto\hat\gT^{!+}_{g,\nu}$ ($1\le i\le \nu-1$),

\noindent qui sont donc invariants dans $\cN^!_{g,\nu}$
\footnote{Il suffit me semble-t-il de le prouver pour
{\it un} $i$ pour le déduire pour les autres.}
De plus,
$\hat{\hat\gT}^!_{g,\nu}\buildrel{\psi_i}\over\to\hat{\hat\gS}^!
_{g,\nu-1}$ induit un {\it isomorphisme}
$$ \cN_{g,\nu}^!\buildrel\sim\over\to M_{g,\nu-1}^!.\leqno(6)$$
Cette assertion se décompose en deux: tout d'abord que $\psi_i$ applique
bien $\cN^!_{g,\nu}$ dans $M^!_{g,\nu}$ -- et ceci résulte de la
définition arithmético-géométrique de ces groupes -- cela
implique d'autre part, puisque $\psi_i$ induit aussi un isomorphisme
$$\hat\gT^+_{g,\nu}\buildrel\sim\over\to\hat\gS^+_{g,\nu-1},$$
qu'il induit un homomorphisme
$$\diagram
\GG_{g,\nu} & \buildrel{\lambda_i}\over\rArr & \GG_{g,\nu-1} \\
\wr\,|      &                                &     \wr\,|    \\
\cN_{g,\nu}/\hat\gT^+_{g,\nu} &   &  M_{g,\nu-1}/\hat\gS   \\
\enddiagram\leqno(7)$$
et l'injectivité (resp. surjectivité) de (6) équivaut à celle
de (7).  D'ailleurs (7) s'insère, par construction, dans un diagramme
commutatif:
\[\begin{tikzcd}
	& {\Gamma_{\mathbf{Q}}} \\
	{\GG_{g, \nu}} && {\GG_{g, \nu - 1}}
	\arrow["{\theta_{g, \nu}}"', from=1-2, to=2-1]
	\arrow[from=2-1, to=2-3]
	\arrow["{\theta_{g, \nu - 1}}", from=1-2, to=2-3]
\end{tikzcd}\leqno{(8)}\]
avec $\theta_{g,\nu}$ et $\theta_{g,\nu-1}$ surjectifs, donc il est bel
et bien évident que (6) est {\it surjectif}.  La bijectivité signifie
que $\theta_{g,\nu}$ et $\theta_{g,\nu-1}$ ont même noyau (alors qu'a
priori il se pourrait que $\theta_{g,\nu}$ ait un noyau plus petit, i.e.
corresponde à une représentation de $\GG_{\mathbf{Q}}$ ``moins infidèle''
que $\theta_{g,\nu-1}$).  D'ailleurs, il est évident (même, a priori,
sans utiliser la définition ailleurs explicitée de $\cN_{g,\nu}$) que
l'homomorphisme $\lambda_i$ de (7) ne dépend pas du choix de
$0\le i\le \nu-1$ -- par exemple puisque l'on passe de l'un à l'autre
en appliquant des opérations de $\hat\gT^+$ (puisque $\hat\gT^+$
opère transitivement sur $I$) et que $\hat\gT^+$ opère trivialement
dans $\GG_{g,\nu}$\dots

On peut dire que (6) décrit $M^!_{g,\nu-1}$ (donc $\cN^!_{g,\nu-1}$)
en termes de $\cN_{g,\nu}$ -- mais l'inverse est moins clair,
faute de savoir si $\hat{\hat\gT}_{g,\nu}\buildrel
{\psi_i}\over\to\hat{\hat\gS}_{g,\nu-1}$ est
injectif; si on le savait, on pourrait décrire $\cN^!_{g,\nu}$
comme l'image inverse de $M^!_{g,\nu-1}$\dots  Il ne serait pas
impossible d'ailleurs que (6) soit faux, i.e. que les
$\theta_{g,\nu}$ soient infidèles, mais de moins en moins quand
on fait augmenter $\nu$ -- en passant à la limite projective
$\theta_{g,\infty}$, seulement, aurait-on (peut-être!)
une représentation fidèle de $\GG_{\mathbf{Q}}$?  Mais jusqu'à
indication contraire, je préfère travailler hypothétiquement
avec (6), ce qui s'exprime par les suites exactes fondamentales
$$1\to\hat\pi_{g,\nu-1}\buildrel{\phi_i}\over\to
\cN^!_{g,\nu}\buildrel{\psi'_i}\over\to\cN^!_{g,\nu-1}
\to 1
\footnote{On pourrait l'inclure dans une suite exacte un
peu plus grande, avec $(\cN_{g,\nu})_i$ et $\cN_{g,\nu-1}$\dots}
\leqno(9)$$
qui étend la suite exacte
$$1\to\hat\pi_{g,\nu-1}\buildrel{\phi_i}\over\to
\hat\gT^!_{g,\nu}\buildrel{\psi'_i}\over\to
\hat\gT^!_{g,\nu-1}\to 1\leqno(10)$$
(et tient lieu de la suite exacte peut-être défaillante
$$ ??\quad 1\to\hat\pi_{g,\nu-1}\to\hat{\hat\gT}^!_{g,\nu}
\to\hat{\hat\gT}^!_{g,\nu-1}\to 1\quad ??).$$
Quand $(\pi,a)$ est un groupe extérieur à lacets muni d'une
arithmétisation -- ce qu'on pourrait appeler une ``courbe
algébrique virtuelle'' -- alors ce qui précède
permet de définir, sur chacun des $\pi'_i$ de type $(g,\nu-1)$
associés aux $i\in I(\pi)$, une arithmétisation canoniquement
associée à $a$, soit $a_i$, et on trouve alors une suite exacte
$$1\to \pi'_i\to(\cN_a)_i\buildrel
{\psi_i}\over\to\cN_{a^+_i}\to 1\leqno(11)$$
telle que l'on ait
$$(\hat\gT^+_a)_i=\psi^{-1}_i(\hat\gT^+_{a'_i}),\leqno(12)$$
induisant
\[\begin{tikzcd}
	1 & {\pi'_i} & {\cN^!_a} & {\cN^!_a} & 1 \\
	&& {\hat\gT^{+!}_a = {\psi'_i}^{-1}(\hat\gT^{+!}_{a_i})} & {\hat\gT^{+!}_{a'_i}}
	\arrow[from=1-1, to=1-2]
	\arrow[from=1-2, to=1-3]
	\arrow["{\psi^!_i}", from=1-3, to=1-4]
	\arrow[hook, from=2-3, to=1-3]
	\arrow[from=1-4, to=1-5]
	\arrow[hook, from=2-4, to=1-4]
\end{tikzcd}\leqno{(11')}\]
et induisant par passage au quotient
$$\GG_a\buildrel\sim\over\to\GG_{a_i}.\leqno(13)$$
De plus, l'application canonique
$${\hbox{Discrét}}^+(\pi)\to{\hbox{Discrét}}^+(\pi'_i)$$
définit par passage aux quotients
$$\P_a\buildrel\sim\over\to\P_{a_i}\leqno(14)$$
compatible avec les actions de $\GG_a$, $\GG_{a'_i}$ et (13).

Je ne fais ici aucune assertion sur une soi-disant
bijectivité entre ensemble des arithmétisations de $\pi$,
et ensemble des arithmétisations de $\pi'_i$ -- ce qui
reviendrait à la bijectivité de
$$\hat{\hat\gT}/\cN_a\to\hat{\hat\gS}'/M_{a'} \isom 
\hat{\hat\gT}'/\cN_{a'},$$
qui n'aurait guère de raison d'être que si on admettait
$\hat{\hat\gT}(\pi)\buildrel\sim\over\to \hat{\hat\gS}
(\pi'_i)$, qui me semble bien problématique.

{\it Cependant, on trouve, par le foncteur ``bouchage de
trous'', une équivalence entre la catégorie des groupes
profinis à lacets } {\bf extérieurs arithmétisés}
{\it de type $(g,\nu)$, et des groupes profinis à lacets
(pas extérieurs!) } {\bf arithmétisés}, {\it de
type $(g,\nu-1)$}.
\vskip .2cm
On peut se proposer d'essayer de préciser le type de
propriétés qui vont caractériser $\cN_{g,\nu}$ dans
$\cN'_{g,\nu}={\rm Norm}_{\hat{\hat\gT}_{g,\nu}}(\hat\gT
^+_{g,\nu})$, ou encore $\GG_\gn$ dans $\GG'_\gn=\cN'_\gn/
\hat\gT^+_\gn$.  On a des homomorphismes naturels
$$
\begin{cases}
\GG'_\gn\to{\rm Autext}(\hat\gT^+_\gn) \\
	 \GG'_\gn\to{\rm Autext}(\hat\gT^{+!}_{\gn}) 
\end{cases}
\leqno(15)
$$
et je présume que $\GG_\gn$ pourra se décrire comme image 
inverse d'un sous-groupe fermé convenable de l'un ou de
l'autre des seconds membres, i.e. qu'on peut le décrire
en termes des propriétés d'opérations extérieures
sur $\hat\gT^+_\gn$ ou sur $\hat\gT_\gn$.  Les conditions
(e) en tout cas, sont bien de ce type (propriété de normaliser
des sous-groupes invariants $\pi'_i$ de $\gT^{+!}_\gn$\dots).  Bien
s\^ur, on pourrait poser des conditions sur des automorphismes
extérieurs (de ${\hat{\gT'}}^{+!}_\gn$, disons), qui soient
stables par passage successifs à des ${\hat{\gT'}}^{+!}_{g,
\nu-1}$, ${\hat{\gT'}}^{+!}_{g,\nu'-1}$ etc\dots.et qui soient
engendrées par les conditions (5') et (6).  Mais il n'est pas dit
du tout que cela suffira à décrire les $\GG_\gn\subset 
\GG'_\gn$, ne serait-ce que parce que la condition devient
vide pour le cas limite $\nu=0$ (si $g\ge 2$; ou pour les cas
$g=1$, $\nu=1$, ou $g=0$, $\nu=3$).  Il est possible qu'il
faille faire intervenir des propriétés des $\pi_1$
des $M_{g,\nu,\overline{\mathbf{Q}}}$, liées à la compactification.
Ce n'est guère que dans le cas de $(g,\nu)=(0,3)$ qu'il
ne faut pas s'attendre du tout à ce genre de condition.
\vskip .2cm
Appelons ``courbe algébrique potentielle'' (sous-entendu, 
sur une clôture algébrique non précisée de $\mathbf{Q}$) la
donnée d'un groupe extérieur profini $\pi$ à lacets de
type $(g,\nu)$, muni d'une arithmétisation $a$, et d'un
germe de relèvement ``admissible''
$$\GG_a^\natural\to\cN_a.\leqno(16)$$
Elles forment une catégorie (pour les isomorphismes, pour le
moment); si on se donne un torseur $P$ sous $\GG_\gn$ (ce qui, moralement,
revient à se donner une clôture algébrique $\overline{\mathbf{Q}}$
de $\mathbf{Q}$\dots) les courbes potentielles de type $(g,\nu)$
{\it relatives} à ce torseur (moralement, correspondant à
des courbes algébriques sur $\overline{\mathbf{Q}}$\dots)  sont
celles munies en plus d'un isomorphisme (``intérieur'')
$$\P_a\buildrel\sim\over\to P\leqno(17)$$
(définissant un isomorphisme
$$\GG_a\to \Gamma_P={\rm Aut}_{\GG_\gn}(P).)\leqno(18)$$
Cette fois-ci, on s'attend à trouver des ensembles d'isomorphismes
{\it finis}, au lieu de torseurs sous des groupes profinis
considérables
\footnote{Le cas où $P$ est le torseur trivial est celui des
$\pi$ [extérieurs ?] munis d'une {\it prédiscrétification}
orientée $\alpha$; d'où une suite exacte:
$$1\to\hat\gT_\alpha\to\hat\cN_\alpha\to
\GG_\alpha\to 1$$}.

On considère les {\it points} de $\pi$, comme les classes de
$\pi$-conjugaison de relèvements ``admissibles'' de (16) en
$$\GG_a^\natural\to M_a\subset \hat{\hat\gT}(\pi)\leqno(19)$$
i.e. un germe de vraie action de $\GG_a$ sur $\pi$
avec $\pi^{\GG_a^\natural}=1$ ceci pour l'admissibilité.
\vskip .2cm
\noindent {\it Question liminaire:} La connaissance de $\pi$,
de $\Sigma_a=\hat\gT^+_a\subset \hat{\hat\gT}(\pi)$, et d'un sous-groupe
$\Gamma\subset \hat{\hat\gT}(\pi)$ (normalisant $\Sigma_a$,
tel que $\Gamma'\cap\Sigma_a=\{1\}$) permet-elle de retrouver
l'arithmétisation de $\pi$ -- et, pour commencer, de retrouver
$\cN_a$ (dans lequel $\Gamma \cdot \Sigma$ est d'indice fini)?  Il 
suffirait, pour pouvoir répondre par l'affirmative, de savoir
que tout isomorphisme extérieur $\pi\to\pi_\gn$,
qui envoie $\Sigma$ sur $\hat\gT_\gn=\Sigma_\gn$, et tout
sous-groupe d'indice fini $\Gamma'Dot\Sigma$ de $\cN_a$
dans $\cN_\gn$, est compatible avec les arithmétisations.
Or ceci revient exactement à:
\vskip .2cm
(g) (facultatif, quand-même!) Pour tout sous-groupe ouvert
$\cN'$ de $\cN_\gn$, les éléments de $\hat{\hat\gT}_\gn$ 
normalisant $\hat\gT_\gn$ et qui transportent $\cN'$ dans $\cN_\gn$, 
sont dans $\cN_\gn$ (a fortiori $\cN_\gn$ serait son propre
normalisateur dans ${\rm Norm}_{\hat{\hat\gT}_\gn}(\hat\gT_\gn)$).
\vskip .2cm
Avec les ``points de $U$'' (définis comme classes de
conjugaison de relèvements (19)) ``admissibles'' i.e. satisfaisant
(20), on fait un groupoïde, ayant comme groupe fondamental
extérieur $\pi$, et sur lequel $\Gamma\subset \cN_a$
opère strictement (NB. pour se reposer, on se donne maintenant
$\Gamma$ lui-même, pas seulement un germe -- moralement, cela
signifie qu'on a une courbe définie sur une sous-extension
finie $K$ de $\overline{\mathbf{Q}}/{\mathbf{Q}}$\dots
Les points fixes de $\Gamma$ correspondent aux points rationnels
sur $K$, les points fixes sous un sous-groupe fermé $\Gamma'$
aux points rationnels sur la sous-extension $K'$ de
$\overline{\mathbf{Q}}/{\mathbf{Q}}$ associée à $\Gamma'$\dots)

Soit maintenant $I'\subset  I=I(\pi)$ une partie de $I$,
stable par $\Gamma$.  On trouve par ``bouchage de trous en $I'$''
un groupe extérieur à lacets $\pi'$, et un homomorphisme
extérieur
$$\pi\to\pi'.\leqno(20)$$

D'ailleurs $\pi'$ hérite d'une arithmétisation $a'$ par $\pi$ --
et on aura
$$\begin{cases}\GG_a\buildrel\sim\over\to \GG_{a'} \\
      P \isom \P_a\buildrel\sim\over\to\P_{a'} \end{cases}\leqno(21)$$
donc $\pi'$ est défini sur la même $P$ que $\GG_a$ (i.e.
en faisant des trous dans $U$ pour trouver $U'$, on n'a pas
dérangé le corps de base algébrique absolu 
$\overline{\mathbf{Q}}$\dots).  D'ailleurs on aura un homomorphisme canonique
$$\hat{\hat\gT}(\pi)\to\hat{\hat\gT}(\pi')$$
induisant
\[\begin{tikzcd}
	{\cN_a} && {\cN_{a'}} \\
	\\
	{\hat{\gT}_a} && {\hat{\gT}_{a'}}
	\arrow[from=3-1, to=3-3]
	\arrow[from=1-1, to=1-3]
	\arrow[hook, from=3-1, to=1-1]
	\arrow[hook, from=3-3, to=1-3]
\end{tikzcd}\leqno{(22)}\]
(induisant justement $\GG_a\to\GG_{a'}$ par passage
aux quotients), et on trouve, en composant
$$\Gamma\hookrightarrow \cN_a\to\cN_{a'}$$
un homomorphisme également injectif
$$\Gamma\hookrightarrow \cN_{a'}\leqno(23)$$
qui est une quasi-section de $\cN_{a'}$ sur $\GG_{a'}$.  Donc
sous réserve d'admissibilité, on trouve sur $\pi'$ une
structure de courbe algébrique potentielle, relative
au même $P$.  
\vskip .2cm
Je suis vraiment gêné aux entournures, faute d'avoir une 
définition en forme d'``admissible'' -- je vais y revenir très 
vite -- mais pour le moment, j'ai envie de noter que, si 
$I'\ne\emptyset$, le groupe extérieur $\pi'$ peut se décrire par 
un vrai groupoïde, ayant $I'$ comme ensemble d'objets, et
sur lequel $\Gamma$ opère en sens strict (ceci est de l'algèbre
pure, indépendamment des histoires d'arithmétisation\dots)
Le fait que $\Gamma$ opère trivialement sur les $i'\in I'$
implique que pour tout $i'\in I'$, on a un homomorphisme canonique
$$\Gamma\to \pi'(i')\leqno(24)$$
qui relève son action extérieur, d'où une classe
de $\pi'$-conjugaison de relèvements de l'action extérieure
de $\Gamma$ sur $\pi'$ en une vraie action de $\Gamma$ sur
$\pi'$ -- on espère qu'elle satisfait $\pi'^\Gamma=1$ -- et on a donc
une application canonique
$$I'\to{\rm Pts}(\pi',\alpha',\Gamma).\leqno(25)$$
{\it Je dis que cette application est injective}.  Ceci est
``évident'' quand on interprète ``action extérieure
admissible'' par ``réalisable par une vraie courbe algébrique'',
et ``relèvements admissibles'' par ``réalisables par des vrais
points rationnels sur $K$, corps des invariants de $\Gamma$''
\footnote {On est ramené au cas ${\rm Card}(I')=2$\dots}.
Mais on voudrait
bien s\^ur des raisons internes à la donnée des $(\cN_\gn)$, et
des propriétés de ces données!  Ce point étant admis (en
mettant entre parenthèses les deux définitions essentielles
d'admissibilité, sur lesquelles on va revenir plus bas) on
trouve un foncteur ``bouchage de trous'' (il faut faire aussi les 
restrictions anabéliennes habituelles):
\vskip .2cm
Courbes algébriques potentielles\spc Courbes algébriques potentielles

sur $P$ relatives à un sous-groupe ouvert $\Gamma$
$\ \ \to \ \ \quad \ $ sur $P$ relatives à $\Gamma$

de $\GG_P$, et munies d'un ensemble\spc munies d'une partie de l'ensemble

$I'$ de points à l'infini \spc\quad\quad\quad
des ``points invariants sous $\Gamma$''
\vskip .2cm
et sauf erreur, il est devenu évident que ce foncteur est une
{\it équivalence de catégories} [pour les isomorphismes] 
et comme conséquence,
il y a le foncteur en sens inverse: forer des trous en des
``points'' invariants sous $\Gamma$ (ou en des points quelconques,
quitte à passer à un $\Gamma'$ plus petit).

Mais je me rends compte que ce n'est pas du tout évident --
je vais essayer d'élucider la situation axiomatiquement.
On a fixé un genre $g$, on suppose donné, pour $\nu$
variable (tel que $(g,\nu)$ anabélien) des sous-groupes
fermés $\cN_\gn\subset \hat{\hat\gT}_\gn$, normalisant
$\hat\gT_\gn$, et satisfaisant la condition essentielle
que les $\psi_i:\hat{\hat\gT}_\gn\to\hat{\hat\gS}_{g,\nu-1}$
induisent
$$M_\gn\buildrel\sim\over\to M_{g,\nu-1},$$
ce qui permet de définir la notion d'arithmétisation
d'un $\pi$ profini de type $(g,\nu)$ anabélien, et la théorie du
bouchage d'un nombre quelconque de trous, et du forage d'un
seul trou, dans la catégorie des groupes de type $(g,\nu)$
arithmétisés.

On suppose d'autre part donné une sous-catégorie pleine 
\footnote {On la supposera stable par passage à des sous-groupes
ouverts.} 
de la catégorie des groupes profinis (qui pourrait se réduire aux
sous-groupes ouverts de $\GG_g$, valeur commune des $\GG_\gn$, ou des
sous-groupes ouverts du groupe $\GG_a$, $a$ une arithmétisation),
et une notion d'opérations extérieures ``admissibles'' de tels
$\Gamma$ sur des $\pi$ arithmétisés.  On suppose

\noindent (1) Si $\Gamma$ opère admissiblement sur $\pi$,
tout sous-groupe ouvert aussi (et inversement);

\noindent (2)\footnote{Il suffit de la poser pour ${\rm Card}(I')=1$, i.e.
bouchage d'{\it un} trou -- dumoins si on sait que la condition
d'admissibilité ne dépend que de l'action des noyaux des groupes.}
Si de plus $\Gamma$ invarie une partie $I'$ de 
$I=I(\pi)$, alors en ``bouchant $I'$'', l'opération
de $\Gamma$ sur $\pi'$ est admissible.
\vskip .2cm
Quand on a une action {\it effective} (pas extérieure) de $\Gamma$ sur un
$(\pi,a)$ de type $(g,\nu)$, alors le foncteur ``forage de
trous'' donne une action extérieure sur un $(\pi^\circ,
a^\circ)$ de type $(g,\nu+1)$ et on dit que l'action de 
départ est admissible, si l'action {\it extérieure}
déduite l'est.  On trouve ainsi une {\it équivalence}
entre la catégorie des systèmes $(\pi,a,\Gamma,\psi,i)$
d'un $(\pi,a)$ de type $(g,\nu+1)$, avec une opération
{\it extérieure} admissible $\psi$ d'un $\Gamma$ dessus,
et un $i\in I(\pi)$ stable par $\Gamma$, avec la catégorie
des $(\pi',a',\Gamma,\psi')$ des $(\pi',a')$ de type $(g,\nu)$,
avec une {\it vraie} action admissible $\psi'$ de $\Gamma$ dessus.
La condition (2) assure que si $\Gamma$ opère effectivement,
de fa\c con admissible, alors l'action extérieure déduite
est admissible (mais l'inverse ne sera pas vrai -- il y aura
des relèvements ``pathologiques'' d'une action extérieure
admissible donnée\dots).  On suppose de plus

\noindent (3) Si $\Gamma$ opère effectivement de fa\c con admissible
sur $\pi$, alors $\pi^\Gamma=\{1\}$.
\vskip .2cm
Cela implique que la catégorie des $\Gamma$-points de $\pi$,
ou si on veut des sections de $B_{\pi,\Gamma}( \isom  B_u)$
sur $B_\Gamma$, est rigide.  Mais considérons
la catégorie limite inductive de la catégorie des
sections de $B_{\pi,\Gamma} \isom  B_u$ sur des $B_{\Gamma'}$
($\Gamma'$ sous-groupe ouvert); elle est discrète et
correspond à l'ensemble
$${\rm Pt}_{ad}(B_{\pi,\Gamma^\natural}) \isom  \varinjlim
{\rm Pt}_{ad}(B_{\pi,\Gamma'})$$
où l'on prend la limite inductive sur les ouverts $\Gamma'$ de $\Gamma$.
\vskip .2cm
Si ${\rm Pt}(B_{\pi,\Gamma})\ne\emptyset$, alors (du seul fait
que $\pi^{\Gamma'}=\{1\}$ pour tout sous-groupe ouvert de
$\Gamma$) le groupe extérieur $\pi$ peut être décrit canoniquement
par un groupoïde profini $\underline{{\rm Pt}}(B_{\pi,\Gamma})$
ayant ${\rm Pt}(B_{\pi,\Gamma})$ comme ensemble d'objets
\footnote{Le groupoïde à opérateurs stricts
 $\underline{{\rm Pt}}(B_{\pi,\Gamma})$ dépend fonctoriellement 
 de $(\pi,\Gamma)$, pour des morphismes extérieurs [??].},
liés par le groupe extérieur $\pi$, sur lequel $\Gamma$ opère
en sens strict, le stabilisateur $\Gamma_P$ de tout point étant
ouvert, et $\Gamma_P\to{\rm Aut}(P)$ ($ \isom \pi$ modulo
automorphismes intérieurs)
étant continue.  On a ici que si à tout
$P\in{\rm Pt}(B_{\pi,\Gamma})$ on associe la classe d'isomorphie
de sections de $B_{\pi,\Gamma}$ sur $B_\Gamma$,
on trouve une bijection -- ce qui caractérise le
$\Gamma$-groupoïde en question à {\it isomorphisme} unique
près.  On supposera:
\vskip .2cm
\noindent (4) Si $\Gamma$ opère admissiblement sur $(\pi,a)$,
alors il existe un sous-groupe ouvert $\Gamma'$ de $\Gamma$
dont l'opération extérieure se relève en une opération
effective admissible -- i.e. ${\rm Pt}_{ad}(B_{\Gamma,\pi})\ne
\emptyset$.
\vskip .2cm
Notons que dans la situation envisagée, du bouchage d'un trou
$i\in I(\pi)$ d'un $\pi$, en plus de l'homomorphisme
$$\hat{\hat\gT}(\pi)_i\buildrel{\psi_i}\over\longrightarrow
\hat{\hat\gS}(\pi'_i)$$
associé, donnant par composition
$$\hat{\hat\gT}(\pi)_i\buildrel{\psi'_i}\over\longrightarrow
\hat{\hat\gT}(\pi'_i)$$
d'où une action extérieure de $\hat{\hat\gT}(\pi)$ sur $\pi'_i$,
de fa\c con que les $\pi\buildrel p_i\over \to\pi_i'$
commutent aux actions extérieures de $\hat{\hat\gT}
(\pi)$, on a un homomorphisme canonique d'extensions,
provenant de cette action extérieure et de l'homomorphisme
$p_i:\pi\to\pi'_i$;
\[\begin{tikzcd}
	1 & \pi & {\hat{\hat\gS}(\pi)_i} & {\hat{\hat\gT} (\pi)_i} & 1 \\
	1 & {\pi'_i} & {\hat{\hat\gS}(\pi'_i)} & {\hat{\hat\gT}(\pi'_i)} & 1
	\arrow[from=2-4, to=2-5]
	\arrow[from=1-4, to=1-5]
	\arrow[from=2-3, to=2-4]
	\arrow[from=1-3, to=1-4]
	\arrow[from=2-2, to=2-3]
	\arrow[from=1-2, to=1-3]
	\arrow[from=2-1, to=2-2]
	\arrow[from=1-1, to=1-2]
	\arrow["{p_i}", from=1-2, to=2-2]
	\arrow["{\phi_i}"', from=1-3, to=2-3]
	\arrow["{\psi'_i}", from=1-4, to=2-4]
	\arrow["{[\psi_i]}"', from=1-4, to=2-3]
\end{tikzcd}\]
qui n'est {\it pas} le composé 
$$\hat{\hat\gS}(\pi)_i\to\hat{\hat\gT}(\pi)_i
\buildrel{\psi_i}\over\to \hat{\hat\gS}(\pi_i')$$
(il peut se définir chaque fois qu'on a un groupe $\Gamma$,
i.e. $\hat{\hat\gT}(\pi)$, qui commute extérieurement
sur deux groupes extérieurs $\pi$, $\pi'$, et qu'on a un
homomorphisme $p_i\ \pi\to\pi'$ qui commute à l'action
de $\Gamma$, et tel que le centre de $\pi$ et le centralisateur
de son image dans $\pi'$ soient triviaux\dots)  C'est aussi ici
l'homomorphisme de ``transport de structure'', qui pour tout
automorphisme à lacets $u$ de $\pi$, associe l'automorphisme
correspondant de $\pi'_i$.  On a oublié de préciser:
\vskip .2cm
\noindent (g) L'homomorphisme canonique $(\hat{\hat\gS}_\gn)_i
\to\hat{\hat\gS}_{g,\nu-1}$ associé à $i\in [0,\nu-1]$
envoie $(M_\gn)_i$ dans $M_\gn$, donc il s'insère dans
un homomorphisme de suites exactes:
\[\begin{tikzcd}
	1 & \pi & {M_a(\pi)_i} & {\cN_a(\pi)_i} & 1 \\
	1 & {\pi'_i} & {M_{a'}(\pi'_i)} & {\cN_{a'}(\pi'_i)} & 1
	\arrow[from=2-4, to=2-5]
	\arrow[from=1-4, to=1-5]
	\arrow[from=2-3, to=2-4]
	\arrow[from=1-3, to=1-4]
	\arrow[from=2-2, to=2-3]
	\arrow[from=1-2, to=1-3]
	\arrow[from=2-1, to=2-2]
	\arrow[from=1-1, to=1-2]
	\arrow[from=1-2, to=2-2]
	\arrow["{\phi_i}"', from=1-3, to=2-3]
	\arrow[from=1-4, to=2-4]
\end{tikzcd}\leqno{(26)}\]
Ceci posé, si on a une opération extérieure admissible
$\Gamma\to \cN_a(\pi)$ de $\Gamma$ sur $\pi$, fixant $i$,
donc aussi par exemple $\cN_a(\pi)\to\cN_{a'}(\pi')$ sur
$\pi'_i$, on peut considérer que tout relèvement $\Gamma
\to M_a(\pi)$ pour $\pi$ définit par composition
un relèvement $\Gamma\to M_{a'}(\pi'_i)$ pour
$\pi'_i$.  On conclut aisément de (2) que si le premier est
admissible, le deuxième l'est.  On trouve donc
$${\rm Pt}_{ad}(B_{\pi,\Gamma})\to{\rm Pt}_{ad}(B_{\pi'_i,
\Gamma})\leqno(27)$$
et en passant à la limite, une application de $\Gamma$-ensembles:
$${\rm Pt}_{ad}(B_{\pi,\Gamma^\natural})\to{\rm Pt}_{ad}
(B_{\pi'_i,\Gamma^\natural})\leqno(28)$$
-- qui correspond d'ailleurs à un homomorphisme de
$\Gamma$-groupoïdes
$$\underline{{\rm Pt}}_{ad}(B_{\pi,\Gamma^\natural})\to
\underline{{\rm Pt}}_{ad}(B_{\pi'_i,\Gamma^\natural}).\leqno(29)$$
Ceci posé, soit $P\in{\rm Pt}_{ad}(B_{\pi'_i,\Gamma})$ le
``point canonique'' de ${\rm Pt}_{ad}(B_{\pi'_i,\Gamma})=
\bigl({\rm Pt}_{ad}(B_{\pi'_i,\Gamma^\natural})\bigr)^\Gamma$.
On demande:
\vskip .2cm
\noindent (5) L'application (27) induit une {\it bijection}
$${\rm Pt}_{ad}(B_{\pi,\Gamma})\to{\rm Pt}_{ad}
(B_{\pi'_i,\Gamma})\setminus\{P\}$$
ou encore (passant à la limite) une bijection de
$\Gamma$-ensembles
$${\rm Pt}_{ad}(B_{\pi,\Gamma^\natural})
\buildrel\sim \over\to{\rm Pt}_{ad}
(B_{\pi'_i,\Gamma^\natural})\setminus\{P\}.$$
\vskip .2cm
Ceci signifie trois choses:

\noindent (a) L'homomorphisme canonique $\Gamma\to
M_{a'}(\pi'_i)$, composé de $\Gamma\to\cN_a(\pi)$
et de $\cN_a(\pi)\buildrel\sim\over\to M_{a'}(\pi'_i)$,
n'est pas $\pi'_i$-conjugué à un homomorphisme composé
$\Gamma\buildrel\lambda\over\to M_a(\pi)\buildrel
{\phi_i}\over\to M_{a'}(\pi'_i)$, où
$\Gamma\buildrel\lambda\over\to M_a(\pi)$ est un relèvement
admissible.
\vskip .2cm
\noindent (b) Si $\lambda,\mu:\Gamma\to M_a(\pi)$ sont
deux relèvements admissibles de $\Gamma\to\cN_a(\pi)$
tels que $\phi_i\circ\lambda$ et $\phi_i\circ\mu:\Gamma\to
M_a(\pi)$ soient $\pi'_i$-conjugués, alors $\lambda$ et
$\mu$ sont déjà $\pi$-conjugués.
\vskip .2cm
\noindent (c) Tout relèvement admissible de $\Gamma\to
\cN_{a'}(\pi'_i)$ en $\Gamma\to M_{a'}(\pi'_i)$, provient
par composition d'un relèvement admissible $\Gamma\to
M_a(\pi)$.
\vskip .2cm
Gr\^ace à ceci: l'équivalence de cat\/egories entre systèmes
$(\pi,a,i,\Gamma,\theta:\Gamma\to\cN_a)$ et les
systèmes $(\pi',a',\Gamma,\theta':\Gamma\to\cN_{a'},P)$,
où $P\in{\rm Pt}(B_{\pi',\Gamma^\natural})^\Gamma$, se
précise de fa\c con parfaite au niveau des points:  les
points $\Gamma$-rationnels de $B_{\pi}$ s'identifient à
l'un des pts $\Gamma$-rationnels de $B_{\pi'}$, distincts de $P$.

Ceci nous permet alors (ce qui n'était pas faisable dans le
contexte discret!) d'étendre l'équivalence de
catégories en une équivalence:
\vskip .2cm
\noindent [Catégorie des systèmes
$(\pi,\alpha,I',\Gamma,\theta:\Gamma\to\cN_\alpha)$ d'un
$\pi$ arithmétisé par $\alpha$ de type $(g,\nu)$ avec 
$2g+(\nu-{\rm card}(I')) \ge 3$, de $I'\subset  I(\pi)$, d'une 
action {\it admissible} extérieure de $\Gamma$ sur $\pi$ telle que 
$\Gamma$ {\it fixe chaque point de $I'$}]
$$|\,\wr\wr\leqno(30)$$
[Catégorie des systèmes $(\pi',\alpha',I',\Gamma,\theta':
\Gamma\to\cN_{\alpha'})$ d'un $\pi'$ arithmétisé par 
$\alpha'$ de type $(g,\nu')$ [$\nu'=\nu-{\rm card}(I')$] avec 
opération extérieure admissible de $\Gamma$ dessus, et une
partie $I'\subset  {\rm Pt}_{ad}(B_{\pi',\Gamma})$ (donc $I'\subset 
{\rm Pt}_{ad}(B_{\pi',\Gamma})^\Gamma$), i.e. $I'$ formé de
points invariants par $\Gamma$, i.e. ``$\Gamma$-rationnels''].
\vskip .2cm
On a alors un foncteur quasi-inverse: forage de trous en $I'$!  Il
est à noter que dans cette approche, on a d\^u se borner au cas
d'un ensemble de points $I'\subset  I(\pi)$, non seulement stable
par $\Gamma$, mais inclus dans $I^\Gamma$, ou encore à une partie
$I'\subset  {\rm Pt}(B_{\pi',\Gamma^\natural})$ non seulement
stable par $\Gamma$, mais même dans ${\rm Pt}(B_{\pi',\Gamma^
\natural})^\Gamma$.  Pour montrer que sans cette restriction,
le foncteur naturel correspondant est néanmoins une
équivalence, on est ramené à ceci:
\vskip .2cm
\noindent (6) Soit $(\pi,a,I'\subset  I(\pi))$ avec $(\pi,a)$
groupe à lacets extérieur profini arithmétisé de
type $(g,\nu)$, d'où $(\pi',a')$ -- soit $\Gamma$, avec
$\Gamma'$ sous-groupe ouvert opérant sur $(\pi,a)$ de
fa\c con admissible ($\Gamma\to\cN_a(\pi)$), et laissant
stable $I'$, donc il opère de fa\c con admissible sur
$(\pi',a')$.  Supposons donné un {\it relèvement} de cette
action de $\Gamma'$ en une action admissible de $\Gamma$
sur $\cN_{a'}$, qui invarie $I'\subset  {\rm Pt}(B_{\pi',\Gamma^
\natural}) \isom {\rm Pt}(B_{\pi',{\Gamma'}^\natural})$
[cf. le diagramme],
\[\begin{tikzcd}
	{(\cN_a)_{I'}} && {\Gamma'} \\
	\\
	{\cN_{a'}} && \Gamma
	\arrow[hook', from=1-3, to=3-3]
	\arrow[from=1-3, to=1-1]
	\arrow[from=3-3, to=1-1]
	\arrow[from=3-3, to=3-1]
	\arrow[from=1-1, to=3-1]
\end{tikzcd}\]
Alors il existe une action admissible unique de $\Gamma$
sur $(\pi,a)$, qui prolonge celle de $\Gamma'$ et qui donne
naissance à celle donnée sur $\pi'$.
\vskip .2cm
L'unicité est-elle de toutes fa\c cons claire?  Considérons
l'extension $E$ de $\Gamma$ par $L={\rm Ker}\bigl(
(\cN_a)_{I'}\to\cN_a\bigr)$, image inverse de
l'extension $(\cN_a)_{I'}$ (de $\cN_{a'}$ pas $L$) via
$\Gamma\to\cN_{a'}$, on a un scindage partiel de cette
extension au dessus du sous-groupe $\Gamma'$ de $\Gamma$,
et l'assertion est que ce scindage se {\it prolonge, de
fa\c con unique} à $\Gamma$.  On peut supposer $\Gamma'$ invariant
dans $\Gamma$, et on identifie $\Gamma'$ à un sous-groupe
de $E$.  Toutes les sections de $E$ sur
$\Gamma$ s'identifient à des sous-groupes, sections 
$\tilde \Gamma$ de $E$.  Pour un tel $\tilde\Gamma$,
on a bien s\^ur $\tilde\Gamma\subset {\rm Norm}_{E}(\Gamma')$,
d'ailleurs on a évidemment:
$${\rm Norm}_{E}(\Gamma')\cap L=L^{\Gamma'}\leqno(31)$$
et je présume qu'on doit avoir $L^{\Gamma'}=1$ (qui généralise
la condition (3) plus haut\dots)  {\it Or cette condition implique
l'unicité} -- savoir $\tilde\Gamma={\rm Norm}_{E}(\Gamma')$
et l'existence signifie que

${\rm Norm}_{E}(\Gamma')
\to\Gamma$ est un épimorphisme (donc un isomorphisme\dots).

Mais il faudra essayer de préciser, dans certains contextes
(au moins celui des actions arithmétiquement fidèles,
i.e. $\Gamma\to\GG_a$ injectif -- qui correspond
normalement au cas des courbes algébriques définies sur
des extensions {\it algébriques} de ${\mathbf{Q}}$) -- la notion
d'action ``admissible''.  Pour ceci, on doit revenir sur la relation
entre courbes (potentielles) et revêtements finis, ce qui
donne aussi une fa\c con de faire varier $g$.
\vskip .2cm
Mais j'ai envie d'abord de reprendre sous un autre aspect
(peut-être plus général) le formalisme précédent, qui 
peut-être aussi s'applique au cas des groupes {\it discrets} à 
lacets (je pense au formalisme des $U^!$, lié aux actions
extérieures de groupes finis sur de tels groupes à lacets).
Soit ${\cal X}$ un ensemble $\ne\emptyset$ (moralement, un ensemble
de ``points'' d'une courbe algébriques, ou d'une surface
topologique\dots);  on suppose donné, pour toute partie finie
$I$ de ${\cal X}$, de complémentaire $U_I={\cal X}\setminus I$,
un groupoïde $\Pi_{U_I}$, ayant $U_I$ comme ensemble
d'objets.  De plus, pour $J\supset I$ i.e. $U_J\subset U_I$,
on suppose donné un homomorphisme de groupoïdes
$$\Pi_{U_J}\to\Pi_{U_I}$$
qui sur les objets soit l'inclusion $U_J\hookrightarrow \mathbf{C}U_I$.
On supposera la transitivité (stricte) i.e. [l'existence d']un foncteur
covariant de la catégorie des parties $U$ de ${\cal X}$
complémentaires de parties finies 
\footnote{Si ${\cal X}$ est fini et $I={\cal X}$, on suppose que 
cependant $\Pi_{U_I}(=\Pi_\emptyset)$ n'est pas le groupoïde vide,
mais un groupoïde (qui s'envoie dans les précédents\dots)
mais on ne pourra pas exiger la transitivité stricte Il faudrait
peut-être faire des hypothèses anabéliennes sur $\Pi_{U_I}$.}
(avec les inclusions), vers la catégorie des groupoïdes,
qui sur l'ensemble d'objets coïncide avec le foncteur
évident\dots On suppose de plus que pour tout $i\in I$, on se donne
un groupoïde $\Pi_i$ (ou $\Pi_{D^*_i}$), et pour $i\in I
\in{\gP}_f(??)$ un homomorphisme de groupoïdes
$$\Pi_{D^*_i}\to\Pi_{U_I}$$
compatible avec les morphismes de transition $\pi_{U_I}\to
\pi_{U_J}$, $I\supset J$.  On suppose que pour $I$ fixé, on
trouve sur $\Pi_{U_I}$ une structure de groupoïde
à lacets par $$\Pi_{D^*_i}\to\Pi_{U_I}$$
sans d'ailleurs exclure le cas où $I=\emptyset$,
et où ($\Pi_{\cal X}$ étant connexe, disons) le genre 
{\it est} $0$.  On suppose de plus que si $J\supset I$ i.e.
$U_J\subset U_I$, et pour $i\in J\setminus I$, l'homomorphisme
composé
$$\Pi_{D^*_i}\to\Pi_{U_J}\to\Pi_{U_I}$$
soit ``constant'', de telle fa\c con que $\Pi_{U_I}$ se
déduise de $\Pi_{U_J}$ par ``bouchage des trous'' en
$J\setminus I$.

Jusqu'à présent, tout ceci pouvait se visualiser par
exemple en partant d'une surface topologique orientable
$X$, et d'une partie ${\cal X}$ de $X$ rencontrant toute
composante connexe, en prenant pour $\Pi_I$ la restriction
à ${\cal X}\setminus I$ du groupoïde fondamental de 
$X\setminus I$ (si $I\ne{\cal X}$; si ${\cal X}=I$, ce qui
exige ${\cal X}$ fini, on prendra le groupoïde fondamental 
de $X\setminus I=X\setminus{\cal X}$, qu'on aura du mal à
envoyer dans les autres avec transitivité {\it stricte}, qu'à
cela ne tienne!) Mais, on est surtout interessé au cas 
${\cal X}$ infini.  Où on prend $X$ courbe
algébrique sur un corps algébriquement clos, ${\cal X}$ partie
de $X$ rencontrant toute composante connexe, et on définit
les $\Pi_{U_I}$ comme précédemment.  

Soit maintenant $\Gamma$ une groupe qui opère sur la situation 
({\it strictement}, s'entend), donc sur l'ensemble ${\cal X}$,
les groupoïdes $\Pi_{U_I}$, les $\Pi_{D_i^\ast}$ (NB qu'il 
permute entre eux), en commutant aux homomorphismes 
$$\Pi_{U_I}\to\Pi_{U_J}{\rm\ et\ les\ }\Pi_{D^*_i}
\to\Pi_{U_I}.$$
(On aura des difficultés, si ${\cal X}$ est fini, pour
$\Pi_{U_I}$ quand $I={\cal X}$, pour la commutation {\it stricte} des
$\Pi_{U_{\cal X}}\to\Pi_{U_J}$\dots mais passons\dots)	
Nous supposons définie une notion de sous-groupe
{\it admissible} de $\Gamma$ (ils sont en tout cas $\ne \{1\}$) telle
que si $\Gamma'$ est admissible alors tout sous-groupe
$\Gamma''\supset\Gamma'$ aussi.  

On suppose
\vskip .2cm
\noindent ($1^\circ$) $\forall P\in{\cal X}$, $\Gamma_P\ne\{1\}$,
et $\Gamma_P$ d'indice fini dans $\Gamma$ (i.e. l'orbite de
$P$ finie).
\vskip .2cm
Nous voulons des conditions qui assurent que ${\cal X}$ et les
$\Pi_{U_I}$ etc. peuvent se re\-cons\-trui\-re à isomorphisme
canonique près à l'aide des données purement
groupoïdiques ou topossiques (à opérateurs
$\Gamma$) correspondantes.  On peut le dire en langage topossique
savant, mais on va l'exprimer en termes de théorie des
groupes à lacets.  Soit $I$ une partie de ${\cal X}$ stable
par $\Gamma$, on voudrait poser des conditions qui permettent
d'exprimer (pour tout tel $I$) la situation des 
groupoïdes $\Pi_V$ associés aux $V\supset U_I$ 
(i.e. les $U_J$ avec $J\subset  I$), les $\Pi_{D_i^\ast}$ 
($i\in I$) et l'opération de $\Gamma$ dessus, en 
termes des seules opérations de $\Gamma$ sur le groupe 
extérieur à lacets $\pi_1(\Pi_{U_I})$ associé à $I$
ou plutôt (car cela est immédiat, par l'opération de forage
de trous), en sens inverse, montrer comment, sous réserve
d'anabélianité, disons de $\Pi_{\cal X}$ (pour fixer les idées),
on peut plus ou moins reconstituer $\Pi_{U_I}$ et
l'action de $\Gamma$ dessus\dots, à partir de $\Pi_{\cal X}$ (ou 
plutôt, du topos $B_{\Pi_{\cal X}}$), et de l'action de $\Gamma$ 
dessus.  On veut au moins une description intrinsèque des éléments
de ${\cal X}$ et de l'action de $\Gamma$ dessus, via cette action 
``molle'' -- qui, pour ${\cal X}$ connexe,  revient encore à une 
action extérieure de $\Gamma$ sur un groupe à lacets ({\it sans} 
lacets!) $\pi$\dots

Une première condition, sous forme faible, est que (supposant
${\cal X}$ connexe, et appelant $\pi(I)$ le groupe extérieur
$\pi_1(\Pi_{U_I})$, pour tout partie finie $I$ de ${\cal X}$) que
pour $\pi(I)$ anabélien (ce qui sera le cas pour $I$ ``assez
grand'', du moins si ${\cal X}$ est infini, donc qu'on puisse
prendre $I$ de cardinal arbitrairement grand\dots) l'application
évidente
$$
\begin{cases}&\text{classes de}~$\pi(I)$-\text{conjugaison} \cr
	 U_I\to{\rm Pt}(B_{\pi(I),\Gamma^\natural})
	 \buildrel\sim\over \to&\text{de germes de scindage de}\cr
           &\text{l'extension}~$E(I)$
           \end{cases}
           \leqno(2^\circ)$$
(compatible avec les actions de $\Gamma$) soit {\it injective}.
Mais pour aller plus loin, il faudrait pouvoir donner une
caractérisation de l'image.  Notons (nous pla\c cant
dans le contexte profini désormais) que si on se donne
sur les $\pi(I)$ des arithmétisations, compatibles avec les
homomorphismes extérieurs de bouchage de trous $\pi(I)\to
\pi(J)$, et avec l'action de $\Gamma$ sur les $\pi(I)$, on a
donc un homomorphisme canonique de $\Gamma$ dans le groupe
commun $\GG$ des automorphismes arithmétiquement
extérieurs de ces arithmétisations (et même
$\Gamma\to M_{\pi(\theta)=\pi}$, ce qui est
déjà une donnée nettement plus forte).  On peut
donc supposer donnée une notion de {\it relèvement admissible}
d'une telle action arithmétiquement extérieure, de telle
fa\c con que l'application ($2^\circ$) soit une bijection
$$U_I= X \textbackslash I\buildrel\sim\over\to
{\rm Pt}_{ad}(B_{\pi(I),\Gamma^\natural}).$$
Ceci signifie d'ailleurs, pratiquement, que la notion
d'admissibilité satisfait aux conditions ($1^\circ$) à
($6^\circ$) vues ci-dessus.
\vskip .2cm
Quant à savoir ce qu'il y a lieu d'appeler opération
extérieure ``admissible'' de $\Gamma$ sur un groupe
extérieur profini à lacets, cela reste pour le moment 
conjectural.  On pourrait à titre expérimental
conjecturer que la notion qui suit marcherait.  Appelons 
``admissible'' toute telle action
$$\Gamma\to\hat{\hat\gT}(\pi)$$
qui respecte une arithmétisation $a$, i.e. qui se factorise
par le $\cN_a$ correspondant, telle que l'image de
$\Gamma\to\cN_a/\hat\gT=\GG_a$ soit ouverte,
et que pour tout homomorphisme de bouchage de trous en $i\in
I(\pi)$, $\pi\to\pi'_i$, $\cN(\pi)_i\to
M(\pi'_i)$, l'homomorphisme correspondant de $\Gamma_i$
ni d'aucun sous-groupe ouvert $\Gamma$ de $\Gamma_i$, ne
normalise un $L_j'$, $j\in I(\pi'_i)=I\setminus\{i\}$;
peut-être même faudrait-il imposer que $\pi^{({\Gamma'}^+)}
=\{1\}$, où ${\Gamma'}^+$ est le noyau de $\Gamma'
\buildrelHi\over\to\hat{\mathbf{Z}}^*$ -- en tout cas on s'attend
à ce que l'on ait alors $\pi^{\Gamma'}=\{1\}$, et si ce
n'était le cas, il faudrait l'introduire dans la définition --
pour chaque opération de bouchage de trous relatif à
$I'\subset  I(\pi)$, et un choix d'un $i\in I'$, permettant
de définir $\Gamma_{(I',i)}\to M(\pi(I'))$)
\footnote{Plus généralement, pour $i\in I'\subset  I=I(\pi)$ on
impose que la [??] action de $\Gamma_I'$ sur $\pi(I')$ correspondant
à $i$ ne normalise aucun sous-groupe $L_j$ ($j\in I\setminus I'$).
On est ramené pour ceci au cas où $I'=I\setminus\{j\}$, donc
où $\pi(I')$ n'a qu'une seule classe de lacets\dots (si $g\ge 1$)}.


Notons qu'une vraie action de $\Gamma$ sur $\pi$ (pas seulement
extérieure) qui relève une action donnée
``admissible'', est elle-même admissible (en ce sens qu'elle
définit une action extérieure {\it admissible} sur un
$\pi'$ de type $(g,\nu+1)$), si et seulement si cette
action ou plutôt son germe, ne normalise aucun $L_i$
($i\in I(\pi)$, et qu'il en soit de même pour l'action
induite sur chaque $\pi(I')$ ($I'\subset  I=I(\pi)$) [il suffit
de prendre les $\pi(I')$ pour $I'$ de la forme $I\setminus
\{j\}$, $j\in I$, du moins si $g\ne 0$] -- {\it et}
qu'enfin l'action de $\Gamma^\natural$ sur $\pi'(I)$ (déduite
de $\pi'$ en bouchant les trous en $I$, de sorte que
$\pi'(I)$ a une seule classe de lacets; $\pi'(I)$ se
déduit aussi de l'action effective de $\Gamma$ sur $\pi(I)$ --
avec 0 classe de lacets -- en faisant ``un trou''
correspondant -- relatifs à un point $i\in I$), ne normalise
pas de sous-groupe lacets\dots  Mais je présume que la première
condition (action de $\Gamma^\natural$ ne normalisant aucun des $L_i$)
implique les autres.  Mais il faut dans la définition
d'admissibilité aussi tenir compte de la condition
($4^\circ$), qui implique l'existence de suffisamment de
relèvements\dots

J'en arrive (péniblement!) à une
\vskip .3cm
{Conjecture provisoire profinie\footnote{Canulé, cf. plus bas\dots}. --- \it
Pour $\Gamma$ groupe profini donné, une action extérieure
sur un $(\pi,a)$ arithmétisé anabélien de type $(g,\nu)$
est dite {\it admissible}, si
\vskip .2cm
a) L'homomorphisme $\Gamma\to\GG_a$ a une image ouverte
\vskip .2cm
b) Les actions {\it effectives} déduites par bouchages de trous
($i'\in I'\subset  I=I(\pi)$)

ne normalisent pas de sous-groupe à lacets.
\vskip .2cm
c) Il existe une infinité de classes de $\pi$-conjugaison de
germes de scindages de l'extension de $\Gamma$ par $\pi$, qui ne
normalisent aucun sous-groupe à lacets de $\pi$\footnote{Cette condition c) exclut sans doute des cas comme
$\Gamma =\cN_{g,\nu}$ avec $(g,\nu)\ne (0,3)$, car l'extension
$M_{g,\nu}$ de $\cN_{g,\nu}$ par $\hat\pi_{g,\nu}$ n'admet sans
doute pas de germe de scindage - ce qui est équivalent (?) au fait
que $U_{g,\nu}$ sur $\M_{g,\nu}$ n'admet pas de multisection étale\dots}.
\vskip .3cm
Une action effective de $\Gamma$ sur $\pi$ est dite effective,
si l'action extérieure est effective, et si elle ne
normalise aucun sous-groupe à lacets.

Ceci posé:
\vskip .2cm
a) Si $\Gamma$ opère effectivement de fa\c con admissible, on a
$$\pi^\Gamma=\{1\}$$
(peut-être même $\pi^{\Gamma^+}=\{1\}$).
\vskip .2cm
b) Si $i\in I$ est stable par $\Gamma$ opérant extérieurement
de fa\c con admissible, alors l'action effective sur $\pi'=
\pi(i)$ ne normalise aucun $L_j'$ ($j\in I\setminus\{i\}$).
\vskip .2cm
c) Si $\Gamma$ opère effectivement sur $\pi$ de fa\c con
admissible, en laissant fixe $i\in I$, alors l'action 
correspondante effective (par passage au quotient) sur $\pi'=\pi(i)$
ne normalise aucun $L_j$, $j\in I\setminus\{i\}$).
\vskip .2cm
d) Si $\Gamma$ opère effectivement sur $\pi$ de fa\c con admissible,
alors l'application
$${\rm Pt}_{ad}(B_{\pi,\Gamma^\natural})\to
{\rm Pt}_{ad}(B_{\pi',\Gamma^\natural})$$
définie par c) est {\it injective}, et le complémentaire
de son image est égal à $\{P\}$, où $P$ est défini par le
``trou'' $i$.
}
\vskip .2cm
Mais cette dernière partie de l'assertion, allant au-delà
de la seule {\it injectivité} -- et caractérisant
l'image, me semble maintenant tout à fait douteuse --
en effet, il suffit de regarder un schéma $S$ de paramètres,
de type fini sur ${\mathbf{Q}}$ (un $K(\pi,1)$ de préférence,
par exemple une courbe algébrique) et de prendre pour
$\Gamma$ (non le groupe fondamental de son point
générique, i.e. d'un corps, mais) le groupe fondamental de
$S$ lui-même.  La donnée d'une action extérieure admissible
de $\Gamma$ sur un $\pi$ correspond moralement à celle
d'une famille de courbes $U$ de type $g,\nu$ paramétré
par $S$.
\footnote {Les scindages d'extensions correspondent aux
section de $U$ sur $S$.}
La condition c) est vérifiée par exemple si
$U$ provient d'une courbe algébrique sur le corps de base
lui-même, donc pas de problème  -- et il est manifeste que la 
surjectivité déconne.  La difficulté provient du fait que
deux actions {\it distinctes} de $U$ sur $S$ peuvent ne pas
être {\it disjointes}!

Il faut pouvoir parler de sections ``strictement distinctes''
i.e. distinctes en tout point de $S$ -- ce qui, en traduction
profinie, revient à comparer deux scindages de l'extension
de $\Gamma$ par $\pi$, avec les germes de scindages sur
$\GG'\subset \GG$ de $\Gamma$ (en tant qu'extension de 
$\GG_{\mathbf{Q}}$ par une partie géométrique) qui correspondent
aux points de $S$ -- i.e. les (germes de scindage)
{\it admissibles} justement.  On a l'impression de tourner dans
un cercle vicieux ou presque -- il semblerait qu'il ne
faudrait pas trop vouloir avaler à la fois -- traiter d'emblée
{\it tous} les groupes profinis $\Gamma$ à la fois --
mais plutôt se borner d'abord à ceux qui moralement,
correspondent à des $\pi_1$ de schémas de type fini sur ${\mathbf{Q}}$,
des $K(\pi,1)$ disons, ou même des variétés élémentaires
à la Artin -- et dans les définitions
resp. conjectures se tirer par les lacets des souliers, en
récurrant sur la dimension.  Au premier cran donc
(dimension $0$) on se bornerait à des actions de groupes
$\Gamma$ qui soient (modulo tout au moins passage à un
sous-groupe ouvert) ``arithmétiquement fidèles'', par
exemple $\Gamma\to\GG_a$ injectif.  Dans ce cas-là,
la conjecture telle qu'elle est énoncée tantôt
semble raisonnable, ou du moins pas nécessairement 
déconnante.  Le test décisif, il est vrai, serait la
possibilité d'obtenir les ``courbes'' ainsi définies
comme des revêtements de ${\P}^1\setminus\{0,1,\infty\}$\dots















%%%%%%%%%%%%%%%%%%%%%%%%%%%%%%%%%%%%%%%%%%%%%%%%%%%%%%%%%%%%%%%
\chapter*{\S \space 33. --- DIGRESSION TOPOLOGIQUE}\thispagestyle{empty}
\addcontentsline{toc}{section}{{\bf 33.} Digression topologique : Anti-involutions des surfaces orientées algébroïdes}
\label{sec:33}
\section*{}

{\bf Anti-involutions des surfaces orientées algébroïdes}\footnote{Finalement je ne regarde que
les surfaces compactes -- pourtant le
cas non compact serait aussi très intéressant à regarder -- 
pour essayer de retrouver par voie algébrique, sur l'automorphisme
extérieur d'ordre 2 du $\pi_1$, la disposition des ``points à 
l'infini'' de la courbe sur les composantes connexes de $X^\sigma
=\hat U$.}

On appelle surface algébroïde une surface $ U$ de la
forme $X\setminus S$, $X$ surface compacte orientable, $S$ fini.
Alors $X$ est determinée à homéomorphisme unique près
comme ``compactifié pur'' de $ U$; on la note $\hat  U$.
Les homéomorphismes de $ U$ avec lui-même s'identifient
aux homéomorphismes de $X$ qui appliquent $S$ dans lui-même.
Les orientations de $X$ sont en correspondance 1-1 avec celles
de $ U$.  Le anti-involutions de $ U$ (supposées orientées)
sont en correspondance 1-1 avec celles de $X$ qui conservent $S$.

On trouve alors une équivalence entre la catégorie des surfaces
orientées algébroïdes $ U$ munies d'une anti-involution
$\sigma$, et des surfaces à bord compactes $Y$, munies
d'une partie finie $T$; à $( U,\sigma)$ correspond
$(\hat  U/\sigma,(\hat  U\setminus U)/\sigma)$ -- et à
$(Y,T)$ correspond $\tilde Y\setminus\tilde T$, où $\tilde Y$
est le ``double orienté'' de $Y$ (qui est une surface
orientée compacte), et $\tilde T$ est l'image inverse de $T$
dans $\tilde Y$.

Nous nous intéressons maintenant au cas où $ U$ est 
connexe de type $(g,\nu)$, en examinant d'abord le cas $\nu=0$,
i.e. $ U=\hat  U$, $S=\emptyset$ (auquel le cas général
se ramènera).  La condition $\nu=0$ correspond au cas où
$T=\emptyset$ -- donc les $ U=X$ envisagés s'identifient
aux doublements orientés de certaines variétés à
bord compactes $Y$.  Lesquelles?  Evidemment il faut que $Y$
soit {\it connexe}, et {\it non orientable}
\footnote{Ce n'est pas vrai que $Y$ soit nécessairement
non orientable -- seulement dans le cas où $X^\sigma=\emptyset$.}.
Donc $Y$ est caractérisé, à homéomorphisme près,
par son genre $\gamma$ et le nombre $\mu$ des composantes
connexes du bord -- il peut se déduire du plan projectif
réel $Y_0$ en y découpant $\gamma$ rondelles disjointes,
et y recollant des rubans de M\"obius, puis en découpant encore
$\mu$ rondelles ouvertes -- ce qui donne
$$Hi(Y)= Hi(Y_0) -\gammaHi_!({\hbox{rondelles ouvertes}})+
\gammaHi_!({\hbox{rubans de M\"obius ouverts}})$$
$$- \mu Hi_!({\hbox{rondelles ouvertes}})$$
[où $Hi(Y_0)=1$ et $Hi_!($rondelle ouverte$)=1$].
Or le ruban de M\"obius ouvert (déduit de $Y_0$ en enlevant
une rondelle fermée) a un $Hi_!$ égal à $Hi(Y_0)
-Hi_!($rondelle fermée$)$, soit $1-1=0$, d'où
$$Hi(Y)=1-(\gamma+\mu),\leqno(1)$$
pour une surface $Y$ compacte connexe avec un bord à 
$\mu$ composantes non orientable de genre $\gamma$.

D'autre part, on a
$$Hi(X)=Hi(X\setminus X^\sigma);$$
or $X^\sigma$ est une réunion de cercles donc son $Hi$ est nul.  Or,
comme $X\setminus X^\sigma$ est un revêtement étale
d'ordre $2$ de $Y$, on a
$$Hi(X\setminus X^\sigma)=2Hi({\rm Int}(Y))$$
enfin
$$Hi(Y)=Hi(Y\setminus\partial Y)=Hi({\rm Int}(Y));$$
pour la même raison que tantôt ($Hi(\partial Y)=0$),
d'où enfin
$$Hi(X)=2Hi(Y)=2\bigl(1-(\gamma+\mu)\bigr),\leqno(2)$$
ou encore, puisque $Hi(X)=2-2g$
$$g=\gamma+\mu.\leqno(3)$$
Donc on trouve
\vskip .3cm
{
Proposition\footnote{Non, on ne trouve qu'une sous-catégorie pleine
de celle de tous les $(X,\sigma)$.  Il y a d'autre part
aussi les doublements orientés $\coprod_{\partial Y} Y$ des
surfaces {\it orientables} compactes à bord non vide --
si $Y$ est de type $(\gamma,\mu)$, $X$ est de genre $g$ avec
$g=2\gamma+\mu-1$ (ou $\gamma\ge 0$, $\mu\ge 1)$.}. --- 
\it La catégorie des surfaces
orientées connexe compactes de genre $g$ munies d'une
anti-involution $\sigma$ est équivalente à celle des
variétés compactes à bord {\it non orientables}, de
type $(\gamma,\nu)$ ($\gamma$ le genre, $\mu$ le nombre
de trous), avec $\gamma+\nu=g$.
}
\vskip .3cm
{
Corollaire\footnote{\c Ca ne marche qu'en se limitant aux $(X,\sigma)$ tels
que $X/\sigma$ non orientable -- cf. ci-contre (i.e. (*))
pour le cas orientable.}. ---
\it Il y a exactement $g+1$ classes
d'isomorphisme de systèmes $(X,\sigma)$, classifiés
par $\sigma\le \mu\le g$, où $\mu={\rm card}(\pi_0(X^\sigma))$.
}
\vskip .3cm
Si $X=X_g$ est une surface orientée compacte de genre $g$,
cela signifie aussi que dans le groupe $A_g={\rm Aut}(X_g)$,
il y a exactement $g+1$ classes de conjugaison d'éléments
$\sigma$ satisfaisant
$$\sigma^2=1,\ \ {\rm sg}(\sigma)=+1\leqno(4)$$
(i.e. qui soient des anti-involutions).  Elles fournissent
$g+1$ classes de conjugaison d'éléments de $A_g/A^\circ_g=
\gT_g$, satisfaisant les mêmes relations.  Nous montrerons
(on l'espère!) que ces classes sont distinctes et que tout
$\sigma\in\gT_g$ satisfaisant les relations (4) est dans l'une de
ces classes.

On admettra le
\vskip .3cm
{
Lemme. --- \it Toute anti-involution du 
disque unité ou de $\mathbf{C}$ est conjugué de $z\mapstochar\to
\bar z$, et par suite l'ensemble de ses points fixes est homomorphe
à $\mathbf{R}$, et en particulier est {\it connexe}.
}
\vskip .3cm
{
Théorème. --- \it Soit $ U$ une surface orientée
séparée connexe, paracompacte, {\it non isomorphe à}
$S^2$, $\sigma$ une anti-involution de $ U$, $\tilde  U$
un revêtement universel de $ U$, d'où $\pi={\rm Aut}
(\tilde  U/ U)$ et une extension $E$ de ${\mathbf{Z}}/2\mathbf{Z}$
par $\pi$, formée des automorphismes topologiques de
$\tilde  U$ compatibles avec la relation d'équivalence
définie par $\tilde  U\setminus U$, et induisant sur $ U$
l'automorphisme ${\rm id}_ U$ ou $\sigma$.  Alors:
\vskip .2cm
a) (Pour mémoire) $ U^\sigma$ est une sous-variété
fermée de $ U$ de dimension $1$.
\vskip .2cm
b) Pour tout $x\in U^\sigma$, on trouve une classe de
$\pi$-conjugaison de scindages de l'extension $E$,
à la fa\c con habituelle, d'où une application
$$U^\sigma\to{\rm Sc}(E,{\mathbf{Z}}/2),\leqno(5)$$
où ${\rm Sc}(E,{\mathbf{Z}}/2)$ désigne l'ensemble des
classes de $\pi$-conjugaison de scindages de $E$
sur ${\mathbf{Z}}/2\mathbf{Z}$, ou encore des éléments d'ordre $2$ de
$E$ qui ne sont pas dans $\pi$.
Cette application se factorise en une {\it bijection}
$$\pi_0( U^\sigma) \isom {\rm Sc}(E,{\mathbf{Z}}/2).\leqno(6)$$
\vskip .2cm
c) Soit $Z_i$ une composante connexe de $ U^\sigma$, 
correspondant à un scindage $\sigma_i\inE^-$
d'ordre $2$.  Alors
$$\pi^{\sigma_i}=\{1\}\ \ \ {\hbox{si et seulement si}}\ \ Z_i \isom 
{\R}.\leqno(7)$$
\indent d) Si $Z_i\not \isom {\mathbf{R}}$, i.e. $Z_i \isom  S^1$, alors,
pour une orientation choisie de $Z_i$, désignant par $g_i$
l'élément de $\pi=\pi_1( U)$ qu'il définit (défini à
conjugaison près), et par $\phi_i:{\mathbf{Z}}\to\pi$
l'homomorphisme $\phi_i(n)=g_i^n$ de ${\mathbf{Z}}$ dans $\pi$, on a:
\vskip .2cm
$1^\circ$) $\phi_i$ est injectif;
\vskip .2cm
$2^\circ$) quitte à remplacer $\phi_i$ par un conjugué, on a
$$\pi^{\sigma_i}={\rm Im}\,\phi_i.\leqno(8)$$
}
\noindent Démonstration.  On trouve l'application (5) en
prenant, pour tout $x\in U^\sigma$, l'opération canonique
de ${\mathbf{Z}}/2\mathbf{Z}$ sur le revêtement universel $\tilde  U(x)$ ponctuée   
en $x$, et en prenant les opérations
correspondantes sur $\tilde  U$, déduites par les isomorphismes
$\tilde  U \isom \tilde  U(x)$.  On voit que les opérations
$\sigma'$ obtenues sur $\tilde  U$ (pour $x$ fixé) sont
celles pour lesquelles ${\tilde  U}^{\sigma'}$ a un point
au-dessus de $x$ -- donc l'ensemble des $x$ qui donnent naissance
à la classe d'un $\sigma'$, sont les éléments de l'image
de ${\tilde  U}^{\sigma'}$ par la projection $\tilde  U
\to U$.  Comme par le lemme ${\tilde  U}^{\sigma'}$
est non vide et connexe, il s'ensuit que son image dans $ U^\sigma$
l'est aussi -- on va voir que son image est exactement une
composante connexe de $ U^\sigma$ -- ce qui à la fois prouvera
la factorisabilité de (5) par $\pi_0( U^\sigma)$ (assez
triviale de toutes fa\c cons) et le fait que l'application 
déduite de (6) est {\it bijective}.

Soit $\tilde Z_i$ un revêtement universel de $Z_i$, et prenons le
revêtement universel correspondant $\tilde  U_i$ de $ U$,
de sorte qu'on a
\[\begin{tikzcd}
	{\widetilde{Z}_i} && {U_i} \\
	\\
	{Z_i} && U
	\arrow[from=1-3, to=3-3]
	\arrow[from=1-3, to=1-1]
	\arrow[from=3-3, to=3-1]
	\arrow[from=1-1, to=3-1]
\end{tikzcd}\leqno{(9)}\]
et par fonctorialité $\sigma$ opère aussi sur ce diagramme
(en opérant trivialement sur $Z_i$, $\tilde Z_i$), soit
$\sigma_i$ son opération sur $\tilde  U_i$.  On voit donc que
$\tilde Z_i$ s'envoie dans ${\tilde  U}_i^{\sigma_i}$,
qui s'envoie donc {\it sur} $Z_i$ -- ce qui achève déjà
de prouver b).

Considérons l'image de $\tilde Z_i$ dans $\tilde  U_i|
Z_i$; c'est une partie {\it ouverte} et fermée
(comme image d'un homomorphisme de revêtements étales
sur la même composante $Z_i$), donc c'est a fortiori une
partie ouverte et fermée de ${\tilde  U}_i^{\sigma_i}$,
et comme cet espace est connexe, il lui est égal.
De plus $\tilde Z_i\to{\tilde  U}^{\sigma_i}$ fait
de $Z_i$ un revêtement étale de son image ${\tilde  U}_i^{\sigma_i}$
dans ${\tilde  U}^{\sigma_i}|Z_i$, et comme ${\tilde  U}^{\sigma_i}$ 
est simplement connexe, on trouve finalement
$$\tilde Z_i\buildrel\sim\over\to{\tilde  U}_i^{\sigma_i}.
\leqno(10)$$
\indent Lorsque $Z_i$ est simplement connexe, i.e. $\tilde Z_i
 \isom  Z_i$, cela signifie aussi que ${\tilde  U}_i^{\sigma_i}$
est un homéomorphisme (donc $\tilde Z_i=Z_i\to
{\tilde  U}_i^{\sigma_i}$ est l'homéomorphisme inverse).  Comme,
pour $x\in Z_i$ et $\tilde x\in({\tilde  U}_i^{\sigma_i})_x$,
les $\tilde x'$ de ${\tilde  U}_i^{\sigma_i}$ au-dessus de
$x$ sont les éléments de la forme $\tilde x\gamma$, avec
$\gamma\in\pi_i^{\sigma_i}$  ($\pi_i={\rm Aut}({\tilde  U}_i/ U)$),
il s'ensuit que l'on a bien
$$\pi_i^{\sigma_i}=1\leqno(11)$$
ce qui est essentiellement la formule (7).  Dans le cas $Z_i$
non simplement connexe, posant
$$T_i={\rm Aut}(\tilde Z_i/Z_i) \isom \pi_1(Z_i;\tilde
Z_i)\leqno(12)$$
on trouve un homomorphisme canonique
$$\phi_i:T_i\to\pi_i\leqno(13)$$
et l'injectivité dans (10) équivaut au fait que (13) est 
injectif.  Bien entendu, le choix d'un générateur $g_i^\circ$
de $T_i$ équivaut au choix d'une orientation de $Z_i$, et
$g_i=\phi_i(g_i^\circ)\in\pi$ est alors l'élément correspondant
de $\pi_1( U)$ dont il est question dans l'énoncé
(moins précis, car on n'y parle que de classe de conjugaison).
Il est clair que $\sigma_i$ (opérant trivialement sur $T_i$,
et sur $\pi_i$ par transport de structure) commute à
l'homomorphisme injectif $\phi_i$, donc $\phi_i$ induit
$T_i\to\pi_i^{\sigma_i}$.  Je dis que c'est
en fait un {\it isomorphisme}
$$\phi_i:T_i\buildrel\sim\over\to\pi_i^{\sigma_i}\leqno(14)$$
-- il reste à prouver la surjetivité.  Mais si $\gamma\in
\pi_i^{\sigma_i}$, alors $\tilde x\gamma\in{\tilde  U}_i^{\sigma_i}$,
donc $\tilde x\gamma\in\phi_i(\tilde Z_i)$ ce qui signifie que
$\tilde x\gamma=\tilde x\phi_i(g)$ pour $g\in T_i$, donc
$\gamma\in{\rm Im}\,y_i$, cqfd.
\vskip .3cm
{
Corollaire 1. --- \it Soit $\sigma_i\in\gT_g$ ($1\le i\le
g$), correspondant à une anti-involution $\sigma_i$ de
$X_g$ telle que ${\rm Card}\,\pi_0(X_g^{\sigma_i})=i$ et
$X_g/\sigma$ non orientable.  Alors 
\vskip.2cm
a) Il y a exactement $i$ classes de $\pi_g$-conjugaison
d'automorphismes effectifs d'ordre $2$ de $\pi_g$ dans la
classe $\sigma_i$ (ce qui prouve que si $i\ne j$, $\sigma_i$
et $\sigma_j$ ne sont pas conjugués dans $\gT_g$\dots);
\vskip .2cm
b) Si $u_i\in\gS_g={\rm Aut}_{\rm lac}(\pi_g)$ est d'ordre $2$
dans la classe $\sigma_i$, alors
$$\pi_g \isom {\mathbf{Z}};\leqno(14)$$
\indent c) Si $u_i,u'_i\in\gS_g$ sont d'ordre $2$, de classe
$\sigma_i$, alors $\exists h\in\gS_g^+$ tel que
$$u'_i=hu_ih^{-1}={\rm int}(h)u_i.\leqno(15)$$
(NB. Nécessairement, l'image de $h$ dans $\gT_g^+$ sera dans
$(\gT^+_g)^{\sigma_i}={\rm Centr}_{\gT_g}(\sigma_i)^+$.)
}
\vskip .3cm
\noindent Démonstration.  a) et b) sont des cas particuliers
du théorème, appliqué à une anti-involution $\sigma_i$
de $X_g$, avec $\pi_0(X_g^{\sigma_i})$ de cardinal $i$.
Soit $Y_{g,i}=X_g^{\sigma_i}$ surface compacte à bord
non orientable connexe de type $(g-i,i)$; il est bien connu et
immédiat que le groupe des automorphismes d'une telle variété
est transitif sur $\pi_0(\partial Y)$, donc en remontant à
$(X_g,\sigma_i)$, que le groupe des automorphismes de 
$(X_g,\sigma_i)$ est transitif sur $\pi_0(X_g^{\sigma_i})$,
qu'on peut interpréter aussi comme l'ensemble des classes
de $\pi_g$-conjugaison de $u_i$, comme dans b), c).  Cela
implique donc qu'il existe $\cdot h\in\gT_g^+$, commutant à
$\sigma_i$, tel que -- désignant par $h$ un relèvement
dans $\gS_g^+$ -- on ait $u'_i$ $\pi$-conjugué à
${\rm int}(h)u_i$, ce qui signifie aussi qu'on peut
(quitte à modifier $h$) le choisir de fa\c con qu'on ait (15).
\vskip .3cm
{
Corollaire 2. --- \it Sous les conditions du corollaire
précédent, posant $\pi_g^{u_i}=T$ ($ \isom {\mathbf{Z}}$), on a un
diagramme de suites exactes
\[\begin{tikzcd}
	1 & T & {\gS_g^{+u_i}} & {\gT_g^{+\sigma_i}} \\
	1 & T & {\gS_g^{u_i}} & {\gT_g^{\sigma_i}}
	\arrow[from=1-1, to=1-2]
	\arrow[from=2-1, to=2-2]
	\arrow[from=1-2, to=1-3]
	\arrow[from=1-3, to=1-4]
	\arrow[from=2-2, to=2-3]
	\arrow[from=2-3, to=2-4]
	\arrow[hook', from=1-4, to=2-4]
	\arrow[hook', from=1-3, to=2-3]
	\arrow[shift left=1, shorten <=2pt, shorten >=2pt, no head, from=1-2, to=2-2]
	\arrow[shorten <=2pt, shorten >=2pt, no head, from=1-2, to=2-2]
\end{tikzcd}\leqno{(16)}\]
et l'indice de $\gS_g^{+u_i}$ dans $\gT_g^{+\sigma_i}$ (et de
$\gS_g^{u_i}$ dans $\gT_g^{\sigma_i}$) est $i$.
}
\vskip .3cm
Comme les deux dernières flèches verticales sont des inclusions
de sous-groupes d'indice $2$, il suffit de traiter l'assertion
concernant $\gS^+$, $\gT^+$.  Or $\gT_g^{+\sigma_i}$ opère
trivialement sur l'ensemble des classes de $\pi$-conjugaison
de $u'_i$, d'après c); d'autre part le stabilisateur dans
$\gT_g^{+\sigma_i}$, pour cette action, de $u_i$, est formé
des $\dot\gamma\in\gT_g^{+\sigma_i}$ tels que ${\rm int}(\gamma)u_i$
soit $\pi$-conjugué à $u_i$, i.e. tels qu'il existe
$\alpha\in\pi$, avec ${\rm int}(\gamma)u_i={\rm int}(\alpha)u_i$,
i.e. $\alpha^{-1}\gamma\in\gS_g^{+u_i}$, ce qui signifie que
$\dot\gamma$ est dans l'image de $\gS_g^{+u_i}$, cqfd.

J'ai envie de construire une igure géométrique où on
puisse mettre en évidence simultanément
des anti-involutions topologiques qui donnent naissance aux
$\sigma_i\in\gT_g$ (à conjugaison près, et aux divers
$u_i\in \gS_g$ associés à un $\sigma_i$ (ce qui sera alors
facile, par la recette générale).  Nous savons que
$\sigma_i\in\gT_g$ s'obtient en regardant un $X_g$ comme
doublement orienté d'un $Y=Y_{g-i,i}$, donc en partant du
plan projectif réel $Y_{0}$, en y faisant $g$ trous, dont
on rebouche $g-i$ par des rubans de M\"obius, en laissant
les $i$ autres trous tels quels.  Soient $D_j$ $(1\le j\le g)$
les disques disjoints fermés correspondant aux ``trous'' donc
$$V_{0,j}=Y_0 \setminus \cup_j D_j^\circ\leqno(17)$$
est une variété à bord (non orientable), contenue à
la fois dans $Y_0$ et dans $Y=Y_{g-i,i}$, et coïncidant
même avec $Y$ en les points de
$$\partial Y = \bigcup_{g-i+1\le j\le g} \partial D_j.$$
Soit $X_0$ la sphère orientée qui revêt $Y_0$, et
$U_0=X_0\setminus V_0$ sa restriction sur $Up D_j$,
c'est donc le complémentaire d'une réunion de $2g$
disques
$$U_0=X_0\setminus\bigcup_{1\le j\le g} \Delta_j\leqno(18)$$
où $\Delta_j\to D_j$ est un revêtement trivial
à $2$ feuillets de $D_j$ ($\Delta_j=\Delta'_j\amalg 
\Delta''_j \isom  D_j\times\epsilon_j$, où $\epsilon_j$
est un ensemble à $2$ éléments qui s'identifie à l'ensemble
des deux orientations de $D_j$\dots.).  Soit d'autre part
$M_j$ ($1\le j\le g-i$)
le ruban de M\"obius dont le bord a été recollé à
$V_0$ par un homomorphisme
$$\partial M_j \isom  \partial D_j.\leqno(19)$$
Soit
$$M=\coprod_{1\le j\le g-i} M_j\leqno(20)$$
de sorte que
$$Y_{g-i,i}=V_0\amalg_{\partial M} M.\leqno(21)$$

Soit donc $X=X_g$ le doublement orienté de $Y_{g-i,i}$
défini par (21), soit $\tilde M=\coprod_{1\le j\le g} \tilde M_j$
l'image inverse de $M$ dedans, qui est un revêtement étale
de degré $2$ de $M$:
$$X_{g,i}=X_g\setminus {\rm Int}(\tilde M),\leqno(22)$$
et on aura
$$X_g=X_{g,i}\amalg_{\partial\tilde M} \tilde M\leqno(23)$$
pour un homéomorphisme bien déterminé de $\partial\tilde M$
avec une partie ouverte et fermée de $\partial X_{g,i}$.
D'ailleurs, on aura une application continue canonique:
$$U_0\to X_{g,i}\leqno(24)$$
qui définit un homéomorphisme de $X_{g,i}$ avec la
surface obtenue en contractant les $\partial \Delta_j\subset  U_0$
($\partial \Delta_j=(\partial D_j)\times \epsilon_j$), 
pour $g-i+1\le j\le g$ à l'aide des projections
$\partial \Delta_j  \isom  (\partial D_j)\times\epsilon_j
\to D_j$.

D'autre part, chaque $\tilde M_j$ ($1\le j\le g-i$), 
revêtement des orientations du ruban du M\"obius, est isomorphe
au cylindre $S^1\times I$, et son anti-involution canonique
est sans point fixe, et s'identifie à $(z,t)\to
(-z,-t)$ (où $S^1$ est identifié aux nombres complexes
de module $1$, et $I$ à $[-1,+1]$).  D'ailleurs, 
$X_0$ orienté avec son anti-involution de revêtement de $Y_0$
s'identifie à la sphère ordinaire (dans un espace
vectoriel euclidien de dimension $3$), avec l'anti-involution
$x\mapstochar\to -x$.  En résumé:
\vskip .3cm
{
Proposition. --- \it On peut obtenir (pour $1\le i\le g$
fixé) les couples $(X_g,\sigma_i)$, à homéomorphisme
près, en prenant la sphère (euclidienne orientée)
$X_0$, avec son antipodisme standard $\tau$, en prenant
un ensemble de $2g$ disques $D_j$ ($j\in J$) mutuellement
disjoints, stable par $\tau$, et une partie $I'\subset 
J/\tau=I$ ($I$ de cardinal $g$) avec ${\rm card}(I')=i$,
et en procédant ainsi: pour tout $j\in I$, soient
$\Delta_j=D'_jUp D''_j$ la réunion des deux disques
correspondants, et $S_j=\partial\Delta_j/\tau$, de sorte que
$S_j$ est un cercle; soit
$$T_j=S_j\times I^{\epsilon_j}$$
(où $I=[-1,+1]$, $\epsilon_j=\{j\in I\mid j\ {\rm sur}\ i\}
 \isom \ $l'ensemble des orientations de $S_i$, $I^{\epsilon_j}$
tordu par $\epsilon_j$, de sorte qu'on a un homéomorphisme
canonique:
$$\partial T_j \isom \partial\Delta_j,\leqno(25)$$
de sorte que $X_g=U_0\amalg_{\partial \Delta}T$ (où 
$\Delta=\coprod_1^g \Delta_j$, $T=\coprod_1^g T_j$) est
{\it orientée}.  [NB. $T_j$ est canoniquement orienté
et l'isomorphisme (25) {\it respecte} comme il se doit l'orientation,
i.e. $\partial T \isom  \partial V_0$ la renverse.]
Ceci posé, l'antipodisme $\tau$ induit sur
chaque $\tilde M_i \isom  S_j\times\epsilon_j$
l'anti-involution canonique provenant de cette expression
des $\partial T_i$, qui échange les deux composantes connexes.
Pour tout $j\in I$, on prolonge $\dot\tau|\partial T_j$
à $T_j$ en deux anti-involutions $\dot\tau_j$,
$\dot\tau'_j$ de $T_j$ de telle fa\c con que
\vskip .2cm
a) $\dot\tau_j$ soit sans points fixes,
\vskip .2cm
b) $\dot\tau'_j$ ait un ensemble de points fixes homéomorphe
à $S_j$ par la projection $T_j^{\dot\tau_j}\to
S_j$ (cf. plus bas pour des choix particuliers explicites --
on verra qu'on peut même supposer $T_j^{\dot\tau_j}=S_j\times
\{0\}$).

Soit, pour toute partie $I'\subset  I$, $\tau_{I'}$ l'anti-involution
de $X_g=V_0\amalg_{\partial T}T$ (où $V_0=X_0\setminus
Up_{j\in J} D_j'$) qui coïncide avec $\tau$ sur $V_0$,
avec $\tau'_j$ sur $T_j$ pour $j\in I'$, avec $\tau_j$ pour $j\in
I\setminus I'$.  Alors on a
$$X_g^{(\tau_{I'})}=\bigcup_{j\in I'}(S_j\times\{0\})\leqno(26)$$
donc si ${\rm card}(I')=i$ ($0\le i\le g$), alors $\tau_{I'}$
est un $\sigma_i$.
}
\vskip .3cm
Choix de $\dot\tau_j$, $\dot\tau'_j$.  On choisit un isomorphisme
$S_j \isom  U\buildrel{\rm def}\over =\{z\in{\mathbf{C}}\mid
|z|=1\}$, d'où une orientation de $S_j$, et une bijection
$\epsilon_j \isom \{\pm 1\}$, d'où $I^{\epsilon_j} \isom 
I=[-1,+1]$ et on prend
$$
\begin{cases}
\dot\tau(z,t)=(-z\,{\rm exp}({\rm i}\pi t),-t)& \\
	 \dot\tau'(z,t)=(z,-t).& 
\end{cases}\leqno(27)
$$
[NB. Pour la définition des $\dot\tau_j'$, on n'a pas besoin
du choix d'un isomorphisme $S_j \isom  U$.]  

On a bien $\dot\tau^2=\dot\tau'^2= id,\ \dot\tau\vert\partial\tau
=\dot\tau'\vert\partial\tau=\tau\vert\partial\tau.$

D'ailleurs on note que:
$$(\dot\tau'\dot\tau)^2=\dot\tau'\dot\tau\dot\tau'\dot\tau=
\bigl((z,t)\mapstochar\to(z\,{\rm exp}(2{\rm i}\pi t),t)
\bigr).$$
Ce n'est pas l'application identique -- l'ensemble de ses
points fixes est égal à l'ensemble des $(z,t)$ tels que
$t\in\{-1,0,+1\}$ -- i.e. 
$$T_j^{(\dot\tau'_j\dot\tau_j)^2}=S_j\times [\partial I^{\epsilon_j}
Up\{0\}].\leqno(28)$$
Soit
$$\dot\rho_j=\dot\tau'_j\dot\tau_j,\ \ [(z,t)\mapstochar\to
(-z\,{\rm exp}({\rm i}\pi t),t)]\leqno(29)$$
-- c'est un automorphisme de $T_j$ qui est {\it l'identité}
sur $\partial T_j$.  Pour toute partie $I'$ de $I=J/\tau$, soit
$$\rho_{I'}={\hbox{l'automorphisme de}}\ X_g\ {\hbox{qui est
l'identité sur}}\ V_0=X_g\setminus\bigcup_{j\in I\setminus I'} T_j,
\leqno(30)$$
$${\hbox{et qui est}}\ \rho_j\ {\rm sur}\ T_j\ {\rm pour}\ 
j\in I'  
\footnote{Posant $\rho_j=\rho_{\{j\}}$, on aura simplement
$\rho_J=\prod_{j\in J}\rho_j$.  Sauf erreur, $\rho_j$ engendre
le groupe $S\Gamma^{!+}(T_j) (??) \isom  \mathbf{Z}$, donc les $\rho_j$
engendre un groupe $ \isom  \mathbf{Z}^I$. NB. On a $\rho_j=\tau_{\{j\}}\tau_
\emptyset$.}.\spc $$
On aura donc
$$\tau_{I'}\tau_{I''}=\rho_{I'_0}\rho^{-1}_{I''_0}\leqno(32)$$
où $I'_0=I'\setminus I'\cap I''$, $I''_0=I''\setminus I'\cap I''$;
d'autre part on aura evidemment
$$[\rho_J,\rho_K]=1 \ {\rm pour}\  J,K\subset  I .\leqno(33)$$
\vskip .3cm
{\bf Remarque.} Au lieu d'un 
isomorphisme $S_j \isom  \U$ supposons donné
plutôt sur $S_j$ une structure de torseur sous ${\U}^{\epsilon_j}$
(${\U}$ tordu par $\epsilon_j$, gr\^ace à l'automorphisme
d'ordre $2$,\hfill\break
$z\mapstochar\to z^{-1}=\bar z$ de $\U$).
On peut donc définir
$${\rm exp}:{\mathbf{R}}^{\epsilon_j}\to{\U}^{\epsilon_j}\leqno{(36\ [sic])}
$$
où $I^\epsilon_j\subset {\R}^{\epsilon_j}$, de fa\c con évidente,
d'où des anti-involutions $\dot\tau_j,\dot\tau'_j:T_j
\buildrel\sim\over\to T_j$ par les formules (27).

Si par exemple on choisit des disques $D'_j$ tels que leurs
bords soient des {\it cercles} eu\-cli\-diens, alors il y a sur
chaque $\partial D'_j$ une structure de torseur sous un
groupe ${\U}^{\epsilon_j}$, invariante par antipodisme, et
qui passe donc au quotient.  Dès lors tout automorphisme
de $(X_0,(D'_j))$ qui respecte cette structure supplémentaire
de torseur -- et notamment tout automorphisme qui respecte
la structure {\it métrique}, opère sur $X_g$ en commutant au
système des $\tau_j$, $\tau'_j$ au sens évident.

Également, si on retient sur $X$ la structure conforme seulement,
et si on choisit dans chaque $D_j$ un ``centre'' $a_j$, de
fa\c con compatible avec l'involution, alors le choix des
$a_j\in {\rm Int}(D_j')$ définit une structure de torseur
sur $M_j$ et ces structures sont
invariantes par transformations conformes qui respectent
l'ensemble de points $a_j$. [NB. On ne suppose plus
nécessairement que ($\partial D_j$ soit un cercle, seulement 
que $\partial D_j$ pas trop sauvage.  Si $\partial D_j$
est un cercle cette définition coïncide avec la
précédente si et seulement si $a_j$ est le centre du cercle.]

Pour définir $\tau_j$ sur $T_j$, il suffit de nettement moins de
données que d'une structure de torseur topologique sur $T_j$.
Ecrivant, pour $(z,t)\in S_j\times I^{\epsilon_j}$
$$\tau_j(z,t)=\bigl(u_t(z),-t\bigr)\leqno(37)$$
où $u_t:S_j\buildrel\sim\over\to S_j$ est un 
homéomorphisme dépendant contin\^ument de $t$, écrivant
que $\tau_j|\partial T_j=\tau|\partial T_j$ on trouve la condition
\vskip .2cm
a) $u_t={\rm id}$ si $t\in\partial I^{\epsilon_j} \isom \epsilon_j$
(\c ca s'écrit, si $S_j$ est orienté, $u_1=u_{-1}=0$);
\vskip .2cm
écrivant que $\tau_j^2={\rm id}$, on trouve la condition
\vskip .2cm
b) $u_{-t}=u_t^{-1}$\ \ ($t\in I^{\epsilon_j}$),
\vskip .2cm
et écrivant que ${T_j}^{\tau_j}=\emptyset$ on trouve la condition
\vskip .2cm
c) $u_0$ sans points fixes.
\vskip .2cm
Si une orientation est choisie, les $j\mapstochar\to
u_j$ satisfaisant a), b), c) correspondent aux applications
$[0,1]\to {\rm Aut}(S_j)$ par $t\mapstochar\to u_t$,
telles que $u_1={\rm id}$, $u_0$ sans point fixe {\it et d'ordre $2$}
(le cas envisagé plus haut est celui-ci où $t\mapstochar
\to u_{1-t}$ provient d'une représentation continue
$\mathbf{R} \mapstochar\to {\rm Aut}(S_j)$).
\vskip .3cm
{\bf Remarques.} On peut se proposer de déterminer
la structure de toutes les anti-involutions $\tau$, sur un
cylindre orienté $T \isom  S\times I^{\epsilon_j}$
($\epsilon={\rm Or}(S)$) qui n'ont pas de point fixe
sur le bord -- ce qui implique déjà que $\tau$
permute les deux composantes connexes du bord.  Plus
généralement, les anti-involutions $\tau$ d'une surface
{\it à bord} orientée $X$, n'ayant pas de point fixe
sur $\partial X$, correspondent aux variétés à bord 
munies d'un partie à la fois ouverte et fermée
$(\partial Y)'$ de $\partial Y$ -- en associant à
une telle $(Y,(\partial Y)')$ son {\it ``doublement''
orienté} relativement à $(\partial Y)'$ -- à
$X,\sigma$ correspondant $(X/\sigma,\ {\rm Im}X^\sigma
\to X/\sigma)$.  Si $X$ est connexe compact,
$Y$ est compacte {\it non orientable}
\footnote{Pas vrai! Si $Y$ est orientable de type $(\gamma,j)$,
avec $0\le i\le j$, on aura $g+\nu/2 =2\gamma+j-1, \nu=2(j-i)$,
i.e. $i=j-\nu/2$ comme dans le cas ci-contre [celui du texte qui suit].}
; supposons que son
type soit $(\gamma,j)$, et soit $i={\rm card}\bigl(\pi_0
((\partial Y)')\bigr)={\rm card}(\pi_0(X^\sigma))$.
Donc $0\le i\le j$, et $Hi(Y)=1-(\gamma+j)$,
et on voit de suite que 
$$ Hi(X) =Hi_!\bigl(X\setminus (X|\partial Y)\bigr)=Hi_!
\bigl( X|{\rm Int}(Y)\bigr)$$
(car le $Hi$ d'un cercle est nul)
$$= 2Hi_!({\rm Int}(Y))=2Hi(Y)=2(1-(\gamma+j));$$
donc si $X$ est de type $(g,\nu)$, on aura $Hi(X)=2-2g, 
2-2g-\nu=2\bigl(1-(\gamma+j)\bigr)$, i.e.
$$g+\nu/2 = \gamma+j\leqno(38)$$
(cela exige que $\nu$ soit pair, ce qui était évident
a priori, car $\sigma$ doit permuter les éléments de
$\pi_0(\partial X)$ entre eux, sans y avoir de points fixes\dots).
Mais j'ai oublié de noter que $\nu$ est déterminé
en fonction de $(\gamma,j,i)$ où $i={\rm card}((\partial Y)')$ par
$$\nu=2(j-i),\leqno(39)$$
i.e. $i=j-\nu/2$.  Donc il faut ici (pour $g=0,\nu=2$)
chercher $(\gamma,j,i)$ avec $\gamma\in{\mathbf{N}}$, $0\le i\le j\in{\mathbf{N}}$,
tels que l'on ait $\nu=2(j-i)$, i.e. $j-i=1$ i.e. $i=j-1$,
et $\gamma+j=1$, ce qui donne la seule possibilité (comme
$i\ge 0$, donc $j\ge 1$)
\vskip .2cm
a) $\gamma=0$, $j=1$, $i=0$.
\vskip .2cm
Le cas a) correspond au cas où $T^\sigma=\emptyset$; il se déduit
du plan projectif réel en y faisant un trou à bord
(d'où ruban de M\"obius), et en prenant le doublement orienté.

Le cas où $T^\sigma\ne\emptyset$ donnera nécessairement
un quotient $Y$ orienté, on doit avoir que pour son type $(\gamma,[??])$,
le seul cas:
\vskip .2cm
b) $\gamma=0$, $j=2$, $i=1$ (quotient {\it orienté})
\vskip .2cm
\noindent déduit de la sphère à deux trous, i.e. du cylindre,
en prenant le doublement orienté par rapport à
{\it un} des trous.

C'est bien les deux cas donnés respectivement par 
$\dot\tau$ et $\dot\tau'$ dans les formules (27).
\vskip .2cm
Je voudrais maintenant construire une situation sous les
conditions de la proposition mettant en évidence un maximum de 
symétries -- il est vrai que l'on pourrait travailler avec
tous les automorphismes de $X_0$ commutant à l'antipodisme
$\tau$, invariant l'ensemble des $D_j$, et respectant
(disons) des structures de torseur topologique sur l'ensemble
des $\partial D_j$ -- ou ce qui revient naturellement au même
(à indétermination de multiplication par $2$ près)
\footnote{Non, {\it sans} indétermination, en relevant à 
la sphère de fa\c con à repsecter l'orientation.}
les automorphismes
de $Y_0$ qui invarient $D= \cup_{i\in I} D_i$, et respectent
des structures de torseur sur les composantes connexes
$\partial D_j$ de $\partial D$.  On prévoit que (travaillant
modulo isotopie) on aura un groupe qui sera voisin d'un groupe 
de tresses, et sans doute calculable sans grand mal --
et en le mettant ensemble avec le groupe engendré par
les opérateurs précédents, on trouvera peut-être un démarrage 
pour engendrer par exemple $\gT_g$ par générateurs
(anti-involutifs) et relations.
\vskip .2cm
Considérons le sous-groupe $G$ de $A_g={\rm Aut}(X_g)$
engendré par les $\tau_{I'}$, $I'\subset  I$.  Soit
${H}$ le sous-groupe engendré par les $\rho_j$,
($j\in I$).   On a
$$\tau_{I'}\rho_J\tau_{I'}=\tau_{I'}(\tau_{\{j\}}\tau_{\emptyset})
\tau_{I'}=(\tau_{I'}\tau_{\{j\}})(\tau_{\emptyset}\tau_{I'})
\in{H}$$
par la formule (32) donc ${H}$ est un sous-groupe invariant.
Les formules (32) montrent que $G/{H}$ est un groupe
{\it commutatif}, et même qu'il est isomorphe à
$\pm 1$ par le caractère d'orientation tous les $\tau_I'$ sont 
égaux mod ${H}$).

Considérons comme élément de référence de ${H}$
l'élément
$$\tau=\tau_{\emptyset}\leqno(40)$$
(anti-involution sans points fixes de $X_g$).  On trouve alors
par (32) que pour $I'\subset  I$,
$$\tau_{I'}=\rho_{I'}\tau=\bigl(\prod_{j\in I'}\rho_j\bigr) \cdot\tau
\footnote{On veut du reste, comme $\tau^2=1$, 
$\tau_{I'}^2=1$, que $\tau_{I'}=\tau_{I'}^{-1}=\tau\rho_{I'}^{-1}$
donc $\tau\rho_{I'}\tau=\rho_{I'}^{-1}$, ce qui est (42).},
\leqno(41)$$
d'ailleurs on aura, pour $j\in I$,
$$\tau\rho_i\tau=\tau_{\emptyset}(\tau_{\{i\}}\tau_{\emptyset})\tau_{
\{0\}}=\tau_{\emptyset}\tau_{\{j\}}=\rho_j^{-1}$$
$$\tau\rho_i\tau^{-1}=\rho_i^{-1}.\leqno(42)$$
Ainsi $G$ apparaît comme le produit semi-direct de
${H} \isom {\mathbf{Z}}^I$, et de $\{\pm 1\} \isom \{1,\tau\}$
y opérant par la symétrie $\rho\mapstochar\to
\rho^{-1}$.  Donc pour {\it tout} $\rho\in{H}$,
on a $(\tau\rho)^2=1$, i.e. pour tout $\sigma\in{H}^{-1}$,
$\sigma$ {\it est une anti-involution}.  Comme $\sigma$
coïncide avec $\tau$ sur $V_0=X_g\setminus \cup_{i\in I}
{\rm Int}(T_i)$, on voit que l'ensemble des points fixes de
$\sigma$ est contenu dans $\cup_{i\in I}{\rm Int}(T_i)$,
et pour calculer l'indice de $\sigma$, i.e. ${\rm card}\bigl(
\pi_0(X_g^\sigma)\bigr)$, il suffit de prendre la somme des
indices dans les $T_j$.  Or dans $T_j$, on a par (29)
$\rho_j(z,t)=(-z\ {\rm exp}({\rm i}\pi t),t)$, et par récurrence,
$$\rho_j^{n_j}(z,t)=\bigl((-1)^{n_j}z\ {\rm exp}({\rm i}\pi n_jt),t
\bigr)\leqno(43)$$
donc
$$\tau\rho_j^{n_j}(z,t)=\bigl((-1)^{n_j+1}z\ {\rm exp}({\rm i}\pi 
(n_j+1)t),-t\bigr)\leqno(44)$$
et $(z,t)\in T_j$ est point fixe de $\tau\rho_j^{n_j}$ si et seulement
si $t=0$, et $n_j$ est {\it impair}, donc
$$
T_j^{\tau\rho_j^{n_j}}=
\begin{cases}
\emptyset& \text{si}~$n_j$~\text{pair} \\
	   S_j\times\{0\}&\text{si}~$n_j$~\text{impair}
\end{cases}\leqno(45)
$$
donc
$${\hbox{Indice de}}\ u=\tau\prod_j\rho_j^{n_j} =\leqno(46)$$
\centerline {= cardinal de l'ensemble $I'_n$ des $j\in I$ tels que
$n_j$ soit impair.}
\vskip .2cm
Il en résulte pour des raisons générales que $u$ est
conjugué dans $A_g={\rm Aut}(X_g)$ (par un élément de
$A_g^+$) à $\tau_{I'}=\rho_{I'}\tau=\tau\rho_{I'}^{-1}$, ou
aussi $\rho_{I'}^{-1}\tau=\tau\rho_{I'}$.

Mais si $\tau\rho'$, $\tau\rho''\in G$, ($\rho',\rho''\in
{H}$), et si $\rho\in {H}$, la relation
$$\rho(\tau\rho')\rho^{-1}=\tau\rho''$$
équivaut à
$$\tau\rho\tau\rho'\rho^{-1}=\rho''$$
(où $\tau\rho\tau=\rho^{-1}$), i.e. $\rho^2=\rho'\rho''^{-1}$;
donc $\tau\rho'$ et $\tau\rho''$ sont conjugués par un élément
de ${H}$ si et seulement si $\rho'\rho''^{-1}\in{H}^2$ --
ce qui précise l'observation précédente\dots
\vskip .2cm
La fa\c con la plus riche en symétries simples pour disposer
les $2g$ trous antipodiques $D_j$ me semble la suivante.  On
considère la sphère euclidienne, avec l'action du groupe
diédral $\D_{2g}$ -- par exemple quand c'est la sphère
de Riemann qui est considérée comme riemannienne, par le
choix de antipodisme comme étant
$$\tau z=-{{1}\over{\bar z}};\leqno(47)$$
on prend l'action type du groupe diédral (avec comme pôles
les points $0$, $\infty$, et comme équateur le cercle unité
$U=\{z\in{\mathbf{C}}\mid |z|=1\}$), en écrivant $\D_n$ comme
$\subset  O(2,{\mathbf{R}})$, comme produit semi-direct de $\{\pm 1\}$
par $\mu_n({\mathbf{C}})=\mu_n$, le couple $(\xi,\alpha)$
($\xi\in\mu_n$, $\alpha\in\{\pm 1\}$) opérant par
$(\xi,\alpha)(z)=\xi z^\alpha$.  On peut d'ailleurs
l'élargir en un groupe $\tilde\D_n \isom  \D_n\times{\mathbf{Z}}/2\mathbf{Z}$, où
le deuxième facteur ${\mathbf{Z}}/2\mathbf{Z}$ est engendré par l'antipodisme
(47) (qui commute à $\D_n$ -- de même d'ailleurs que
l'anti-involution $z\mapstochar\to 1/{\bar z}$, qui a comme
ensemble de points fixes $U$ et n'est autre que la symétrie
par rapport à l'équateur, leur composé $z\mapstochar
\to -z$ étant la symétrie par rapport à
l'axe des pôles\dots) donc on regarde les transformations
$$u_{\xi,\alpha,\beta}=u_{\xi,\alpha}\tau^\beta,\ 
\xi\in\mu_n,\  \alpha\in\{\pm 1\}, \ \beta\in{\mathbf{Z}}/2\mathbf{Z},$$
donc
$$
u_{\xi,\alpha,\beta}\, z=
\begin{cases}
u_{\xi,\alpha}(z)=\xi z^\alpha &si
					       $\beta=0$
              u_{\xi,\alpha}({{-1}\over{\bar z}})=-\xi{\bar
                 	      z}^{-\alpha}&si $\beta=1$
\end{cases}                 	      
$$
mais on se rappellera que $\tau$ renverse l'orientation,
donc est à manier avec réserve pour ce qui concerne le
``transport de structure'' dans la situation présente.
Ici $n=2g$, $\D_{2g}$ est d'ordre $4g$, et $\tilde \D_n$
d'ordre $8g$.  On prend sur l'équateur une trajectoire
de $\mu_n=\mu_{2g}$, par exemple justement l'ensemble $\mu_{2g}$,
de racines $2g$-ièmes de l'unité lui-même (en tant que
sous-ensemble de la sphère) comme l'ensemble des ``centres''
des disques $D'_i$.  On choisit un disque $D'_0$
autour du point $P_0$ (assez petit pour ce qui va suivre)
\footnote{Il faut simplement que $D'_0$ ne rencontre
pas $\xi D'_0$, où $\xi ={\rm exp}(2{\rm i}\pi/2g)$.},
et on prend les transformés de $D'_0$ par les $\xi\in \mu_n$.
Ces choix étant faits, la surface $X_g$ est déterminée
sans ambiguité, et le groupe $\D_{2g}$ y opère par transport
de structure, en permutant entre eux les $g$ cylindres $T_i$,
correspondant aux éléments de $J/\tau=J/\{\pm 1\}$, i.e.
aux paires d'éléments antipodiques de $S$, i.e.
aux ``diagonales'' du polygone à $2\mu$ côtés qu'ils
déterminent sur l'équateur.  Le groupe $\D_{2g}$ normalise
le groupe $G$; de fa\c con précise on aura, pour $u\in
\D_{2g}$,
$$u\tau_{I'}u^{-1}=\tau_{u(I')}\leqno(49)$$
pour $I'\subset  I=J/\tau$, en tenant compte de l'opération
de $\D_{2g}$ sur $I$.  On aura donc en particulier, désignant
par $\tau_g=\tau_\emptyset$ l'extension de l'antipodisme de la sphère
$X_0$ (ou plutôt de $\tau |V_0)$ en un
antipodisme sans points fixes de $X_g$, noté $\tau$
précédemment
$$u\tau_g u^{-1}=\tau_g\leqno(50)$$
i.e. $\tau_g$ commute à l'action de $\D_n$, et
$$u\rho_j u^{-1}=\rho_{u(j)}.\leqno(51)$$
On peut donc dire que $\D_{2g}$ opère sur $G$, d'où
un produit semi-direct $\D_{2g}\cdotG$, qui opère donc
sur $X_g$.  

Quant à la question de prolonger de même l'action sur $X_0$
de $\widetilde\D_{2g}$ tout entier en une action sur $X_g$,
\c ca a été fait sans crier gare, à $\tau_{X_0}$
correspondant naturellement $\tau_{X_g}=\tau_g$, qui a en
effet le bon go\^ut de commuter à l'action de $\D_{2g}$.
[Il pourrait sembler plus naturel, il est vrai, dans un esprit
de ``transport de structure (envers et contre tout?)'', de faire
correspondre à $\tau_{X_0}$ l'opération $\tau_I$ donnée
par (27), qui en chaque $T_j$ serait égal à $\tau_j$
(et sur $V_0$, bien s\^ur, coïncide avec $\tau_{X_0}$),
$\tau_j(z,t)=(z,-t)$, l'ensemble des points fixes de $\tau_I$
étant formé des $g$ cercles médians $S_j\times\{0\}$ des
$g$ tubes $T_j$.  On aura par (41) $\tau_I=\rho_I\tau_g$, donc 
$\tau_I$ commute également à $\D_n$, puisque $\tau$ et
$\rho_I=\prod_{i\in I} \rho_i$ y commutent.  Mais il ne semble
pas important pour le moment quelle convention nous
adoptons.]  On peut donc dire aussi que le groupe $\widetilde\D_{2g}$
opère sur ${H}$ -- cette opération prolongeant
celle de $\tau_g$, identifié maintenant à un
élément de $\widetilde\D_{2g}$, i.e. à l'antipodisme 
dans $\widetilde\D_{2g}$ -- et le produit semi-direct
$$\widetilde\D_{2g} \cdot {H}\supset G=\langle 1,\tau_g\rangle \cdot
{H}\leqno(52)$$
opère sur $X_g$.
\vskip .2cm
Il faudrait maintenant, dans cette voie:
\vskip .2cm
1) Expliciter l'action extérieure de ce produit semi-direct
sur le groupe fondamental $\pi_g$, avec une attention toute
particulière à l'action du groupe $({\mathbf{Z}}/2{\mathbf{Z}})^3\subset 
\widetilde\D_{2g}$ qui stabilise un des cylindres $T_j$ -- qui dans 
l'espace euclidien de dimension $3$ s'interprète comme le groupe
des changements de signe relatif au système d'axes
orthonormés correspondant.
\vskip .2cm
2) Etendre l'action sur $\pi_g$ du groupe de Teichmüller
(plus ou moins) ``spécial'' de données chacun des $T_j$, en
l'action d'un groupe analogue d'un ensemble plus grand obtenu en lui
rajoutant une ``lanière'' $L$ (d'où un tore à un trou,
et son groupe de Teichmüller spécial, qui s'introduisent
de fa\c con naturelle)
\footnote{C'est essentiellement un ${\rm SL}(2,{\mathbf{Z}})$ --
plutôt une extension centrale remarquable de ${\rm SL}(2,{\mathbf{Z}})$
par ${\mathbf{Z}}$, qui rappelle celle de ${\rm SL}(2,{\R})$ par
${\mathbf{Z}}$\dots(revêtement universel de ${\rm SL}(2,{\R})$).}
voire l'ensemble encore plus grand obtenu en mettant également
la lanière antipodique $L'$ (cet ensemble se présente
comme un cylindre $(L \cup L')$ ou ``buse'', où on
aurait mis un tube $(T_j)$ en travers, et a la structure
topologique d'un tore à deux trous).
Il est possible qu'il faille considérer de près ce dernier,
pour étudier les relations entre les éléments de
$\gT_g$ provenant des ensembles précédents\dots. Un travail
amusant sera de se débrouiller pour écrire les générateurs
du groupe $\pi_g$, et surtout la fameuse relation, dans une
disposition géométrique relative des ``anses'', qui est une
disposition ``panachée'' -- et non plus sagement à la queue-leu-leu!
\vskip .2cm
En attendant d'entrer ainsi dans le vif de la structure du
groupe de Teichmüller, je vais déjà essayer de décrire
des générateurs et relations pour le sous-groupe intéressant
qu'on vient d'écrire, $\widetilde\D_{2g} \cdot {H}$.
Je vais prendre les sempiternels générateurs $\epsilon_0$,
$\epsilon_1$ de $\D_{2g}$
\footnote{NB. $\epsilon_0$ bouge le sommet du repère,
$\epsilon_1$ l'arête et pas le sommet, donc $\epsilon_1(j_0)=j_0$.},
$$\epsilon_0^2=\epsilon_1^2=1,\ \ (\epsilon_0\epsilon_1)^{2g}=1$$
et y joindre $\tau=\tau_g$, et $\rho_0\tau=\tau'$ ($\rho_0=
\rho_{j_0}$, $j_0$ point marqué de $I=J/\tau$) satisfaisant
$$\tau^2={\tau'}^2=1,$$
$$(\tau\epsilon_0)^2=(\tau\epsilon_1)^2=1$$
exprimant la commutation de $\tau$ à $\epsilon_0$, $\epsilon_1$;
$$(\epsilon_1\tau')^2=1,$$
i.e. $\epsilon_1$ commute à $\tau'$,
$$\bigl((\epsilon_0\epsilon_1)^g\tau'\bigr)^2=1$$
(l'involution $(\epsilon_0\epsilon_1)^g$ commute à $\tau'$);
sauf erreur, \c ca fait un ensemble de générateurs et
relations pour $\widetilde\D_{2g} \cdot {H}$ -- en résumé
$\epsilon_0$, $\epsilon_1$, $\tau$, $\tau'$:
$$
\begin{cases}
\epsilon_0^2=\epsilon_1^2=\tau^2={\tau'}^2=1& \\
    (\tau\epsilon_0)^2=(\tau\epsilon_1)^2=(\tau'\epsilon_1)^2=1& \\
    (\epsilon_0\epsilon_1)^{2g}=1& \\
    \bigl(\tau'(\epsilon_0\epsilon_1)^g\bigr)^2=1.& 
\end{cases}\leqno(52)
$$
C'est peut-être pas très astucieux comme choix de
générateurs, en ce sens que $\epsilon_0$, $\epsilon_1$ ne sont
pas du tout sur le même pied que $\tau$, $\tau'$ -- ce ne sont pas
des {\it anti}-involutions.  Il vaudrait mieux prendre
$\epsilon'_0=\tau\epsilon_0$, $\epsilon'_1=\tau\epsilon_1$, de
fa\c con à obtenir:
$$
\begin{cases}
{\epsilon'}^2_0={\epsilon'}^2_1=\tau^2={\tau'}^2=1& \\
    (\tau\epsilon'_0)^2=(\tau\epsilon'_1)^2=\bigl(\tau'(\tau
    \epsilon'_1)\bigr)^2=1& \\
    (\epsilon'_0\epsilon'_1)^{2g}=1& \\
    \bigl(\tau'(\epsilon'_0\epsilon'_1)^g\bigr)^2=1.& 
\end{cases}
\leqno{(52\  bis)}
$$
Ce sont essentiellement les ``mêmes'' relations sauf la dernière
de la deuxième ligne.
\vskip .2cm
Peut-être ce petit jeu avec le tout petit groupe
$\widetilde\D_n$ opérant sur ${H}$ est un peu une amusette --
le groupe ${H} \isom {\mathbf{Z}}^g$ dans $\gT_g$ suggère
beaucoup la situation d'un tore maximal dans un groupe
semi-simple; on a vraiment envie d'en avoir
une caractérisation intrinsèque dans $\gT_g$ -- ou
dans $\gS_g$, ce qui est possible puisqu'en tant que groupe
de transformations topologiques effectives, il laisse fixe
les points de $V_0\subset  X_g$ -- par exemple les deux
pôles, qu'on peut prendre comme points base pour construire
revêtement universel et groupe fondamental.  Ce sont peut-être
les sous-groupes abéliens-libres maximaux.  On aimerait
étudier le normalisateur -- est-ce exactement ce qui provient
des automorphismes de $X_g\setminus \cup T^\circ_i \isom  V_0$?
Les ${\rm SL}(2,{\mathbf{Z}})$ associés aux ``lanières'' à
travers le tube $T_j$, jouent-ils un rôle analogue à
celui des groupes ${\rm SL}(2,K)$ où ${\rm GP}(1,K)$
``de rang $1$'' dans la théorie des groupes algébriques
réductifs? Si [?] ne ``mord'' pas à la situation, la soumettre 
peut-être à [?], qui est à l'aise tant avec les groupes 
discrets et leurs générateurs et relations, qu'avec la 
théorie des semi-simples -- et la topologie\dots














%%%%%%%%%%%%%%%%%%%%%%%%%%%%%%%%%%%%%%%%%%%%%%%%%%%%%%%%%%%%%%%
\chapter*{\S \space 33bis. --- ÉTUDE DES REVÊTEMENTS FINIS - RELATION ENTRE LES $\cN_{g,\nu}$, $ \Gamma_{g,\nu}$ POUR $g$ VARIABLE}\thispagestyle{empty}
\addcontentsline{toc}{section}{33bis. Étude des revêtements finis - relation entre les $\cN_{g,\nu}$, $ \Gamma_{g,\nu}$ pour $g$ variable}
\label{sec:33}
\section*{}

Soient $\pi$ un groupe extérieur à lacets profini arithmétisé
de type $(g,\nu)$, $\pi '$ un sous-groupe connexe de $\pi$, d'où une
application injective
$${\hbox{Discrét}}(\pi) \hookrightarrow {\hbox{Discrét}}(\pi');$$
on voudrait prouver qu'elle est com\-patible avec la re\-la\-tion
d'équi\-valence de l'arith\-métisa\-tion, de sorte  
que toute arithmétisa\-tion
de $\pi$ en donne une de $\pi'$, et de même 
pour les prédiscrétifications.  
On voudrait étab\-lir en même temps que les appli\-ca\-tions 
$P^+(\pi) \to P^+(\pi')$ sur les prédiscrétifications
orientées et $A(\pi) \to A(\pi')$ sur les arithmétisations 
sont injectives. Pour prouver ces points, on est ramené au cas 
où $\pi'$ invariant caractéristique (cf. \S 28, diagramme (7)).
Choisissant une discrétification $\pi_0$ de $\pi$, donc $\pi_0'$
de $\pi'$, on trouve donc $\hat{\hat\gS} (\pi_0) \to
\hat{\hat\gS}(\pi_0')$ et on a: 

L'homomorphisme  $\hat{\hat \gS} (\pi_0) \to
\hat{\hat \gS}(\pi_0')$  envoie $M(\pi_0)$ dans $M(\pi_0')$.

On aura donc un diagramme (variante de celui du \S 28):
\[\begin{tikzcd}
	{\hat {\gS^+}(\pi_0)} & {M(\pi_0)} & {\hat{\hat \gS}(\pi_0)} \\
	{\hat {\gS^+}(\pi_0')} & {M(\pi_0')} & {\hat{\hat \gS}(\pi_0')}
	\arrow[hook', from=1-3, to=2-3]
	\arrow[hook', from=1-2, to=2-2]
	\arrow[hook', from=1-1, to=2-1]
	\arrow[hook, from=1-1, to=1-2]
	\arrow[hook, from=2-1, to=2-2]
	\arrow[hook, from=2-2, to=2-3]
	\arrow[hook, from=1-2, to=1-3]
\end{tikzcd}\leqno{(18)}\]
qui implique qu'une discrétification $\pi_1$ sur $\pi$ qui donne
même prédiscrétification orientée que $\pi_0$ (resp. même 
arithmétisation) définit une discrétification $\pi_1'$ de $\pi'$
qui donne même prédiscrétification orientée (resp. même
arithmétisation) que $\pi_0'$.

Donc on a bien
$P^+(\pi) \to P^+(\pi')$, $ A(\pi) \to A(\pi')$
et il faut encore exprimer l'injectivité, qui revient aux relations
$$ \hat\gS^+(\pi_0) = \hat\gS^+(\pi_0') \cap \hat{\hat\gS}(\pi_0),
\quad M(\pi_0) = M(\pi_0') \cap \hat{\hat \gS}(\pi_0)\leqno(2) $$
i.e. un automorphisme à lacets de $\pi$ qui, sur $\pi'$, appartient à 
$\hat \gS (\pi_0')$, resp. à $M(\pi_0')$, est déjà dans
$\hat \gS(\pi_0)$, resp. dans $M(\pi_0')$. 

L'assertion repérée étant admise, l'assertion non
repérée  signifie aussi que l'homomorphisme canonique:
$$ \GG_a = M_a(\pi)/\hat\gS^+_a(\pi) \to
 \GG_a' = M_a'(\pi')/\hat\gS^+_{a'}(\pi') \leqno {(3)}$$
est {\it injectif}. Par construction (via $\GG_{\mathbf{Q}}$),
c'est aussi sur\-jec\-tif, donc on trouve\-rait:

\noindent l'ho\-mo\-mor\-phisme canonique (3) est bijectif et on trouverait 
aussi une bijection
$$ P_a(\pi) \to P_a'(\pi') \leqno{(4)}$$
entre arithmétisation de $\pi$ de type $a$ et arithmétisation de 
$\pi'$ de type $a'$, {\it compatible} avec l'isomorphisme (3).

Enfin, ces résultats dans les cas $\pi'$ invariant caractéristique
s'étendraient aussitôt
au cas d'un $\pi' \subset  \pi$ ouvert quelconque.

Procédant par exemple comme dans le \S 28 à coups de ``correspondances''
arithmétiques extérieures entre ``courbes potentielles'', on trouve
un groupoïde connexe (ponctué par 
les $\hat \pi_{g,\nu}$, par exemple, qui y sont canoniquement
isomorphes), dont le $\pi_1$ extérieur est la valeur commune des 
$\GG_{g*} = \GG_{g,\nu}$. Pour trouver un isomorphisme
canonique entre $\GG_{g,\nu}$ et $\GG_{g',\nu'}$,
on choisit une correspondance entre $\pi_{g,\nu}$ et $\pi_{g',\nu'}$,
par exemple on regarde  l'un et l'autre ( quitte à passer de $g,\nu$  
à $g, \tilde\nu$ avec $\tilde \nu \geq \nu$, et de même pour 
$g',\nu'$ à $g',\tilde\nu'$ avec $\tilde\nu' \geq \nu'$)
comme des sous-groupes [d'indices] finis du même $\pi_{0,3}$, d'où
$\GG_{0,3} {\buildrel\sim\over\to} \GG_{g,\tilde \nu}$,
$\GG_{0,3} {\buildrel\sim\over\to} \GG_{g',\tilde \nu'}$.

On aimerait maintenant voir ce qui, dans le yoga précédent,
est indépendant de toute conjecture.  Par exemple, le fait que
les homomorphismes surjectifs canoniques\hfill\break $\GG_{g,\nu}\to
\GG_{g,\nu'}$ ($g$  le même, $\nu'$ constant) n'est pas établi.
Cependant, si on savait que dans la situation du $(\pi,\pi')$
avec $\pi'$ invariant caractéristique, on a $\hat{\hat\gT}(\pi_0')\cap
\hat\gS^+(\pi_0)=\hat\gS^+(\pi_0)$ (qui est un énoncé de nature
relativement anodine sur des groupes à lacets profinis), on concluerait
à la {\it bijectivité} de $\GG_{\pi_0}\to\GG_{\pi_0'}$
dans cette situation, et on en concluerait que le noyau de
$\GG_\mathbf{Q}\to\GG_{0,3}$ s'envoie {\it trivialement} dans les
$\GG_{g,\nu}$ pour tout $\nu$ -- i.e. qu'il s'envoie dans les
$\GG_{g,\nu}$ par {\it l'intermédiaire} de $\GG_{0,3}$ -- et
que de plus $\GG_{0,3}\to\GG_{g,\nu}$ est un {\it isomorphisme}
pour tout $(g,\nu)$ avec $\nu$ {\it assez} grand -- il suffit que la
surface de type $(g,\nu)$ (ou de type $(g,\nu')$ avec un $\nu'\le \nu$)
puisse s'obtenir comme revêtement étale de $X_{0,3}$, i.e. que
$\pi_{g,\nu'}$ se réalise (avec sa structure à lacets) comme sous-groupe
d'indice fini de $\pi_{0,3}$.  Alors, considérant 
un $\pi''=\pi_{g'',\nu''}\subset 
\pi_{g,\nu'}$ d'indice fini et contenu dans $\pi_{0,3}$, on aura un 
diagramme commutatif:
\[\begin{tikzcd}
	{\GG_{0, 3}} && {\GG_{g'', \nu''}} \\
	\\
	{\Gamma_{\mathbf{Q}}} && {\GG_{g, \nu'}}
	\arrow[from=3-1, to=1-1]
	\arrow["\sim", from=1-1, to=1-3]
	\arrow["\sim"', from=3-3, to=1-3]
	\arrow[from=3-1, to=3-3]
\end{tikzcd}\]
d'où notre assertion $\GG_{0,3}\buildrel\sim\over\to\GG_{g,\nu'}$
et a fortiori l'homomorphisme surjectif $\GG_{0,3}\to
\GG_{g,\nu}$ qui le factorise ($\nu\ge\nu'$) est bijectif.  Notons
par exemple que les $\GG_{0,\nu}\to\GG_{0,3}$ ($\nu\ge 3$)
sont alors bijectifs.  Mais on notera qu'en tout état de cause,
on ne trouve de résultat pour un $g,\nu$ que si $\nu\ge 3$ --
donc ceci ne dit rien sur la fidélité éventuelle, par exemple de
$\GG_{0,3}\to\GG_{g,0}=\GG_g$ ($g\ge 2$), ou
$\GG_{0,3}\to\GG_{1,1}=\GG_1$.

Soit une courbe algébrique $U$ de type $g,\nu$
\footnote{On ne fait plus d'hypothèse conjecturale (telle que
l'injectivité dans (3)).},
définie sur un sous-corps $K$ fini sur $\mathbf{Q}$ de $\overline{\mathbf{Q}}_0$,
donc elle définit une action extérieure du sous-groupe ouvert $\Gamma=
\Gamma_K\subset \Gamma_\mathbf{Q}$ sur $\pi_1(U_{\overline{\mathbf{Q}}_0})$.  Quitte
à agrandir $K$, et à agrandir $\nu$ ou $\nu'$ (i.e. faire des trous)
pour trouver $U'$ {\it on peut supposer que $U'$ est un revêtement étale
de} $(U_{0,3})_K$.  On pose $\pi'=\pi_1(U'_{\overline{\mathbf{Q}}_0})$,
$\pi=\pi_1(U_{\overline{\mathbf{Q}}})$.  Soit alors $\theta$ le noyau de
$$\Gamma\ (\subset  \GG_\mathbf{Q})\longrightarrow \GG_{0,3},$$
il opère donc par automorphismes intérieurs sur $\hat\pi_{0,3}$,
donc l'image de $\theta'=\Gamma'\cap\theta$ dans le groupe des
automorphismes extérieurs de $\pi'\subset \hat\pi_{0,3}$ est
un sous-groupe fini -- a fortiori son image dans $\GG_{\pi'}$, et
dans $\cN(\pi)$, et dans $\GG_\pi$.  Il en est de même (choisissant
un point base rationnel sur $K$ dans $U'$) de l'opération
effective sur $\pi_1(U,P)$ car il suffit d'appliquer le
résultat précédent à $U\setminus\{P\}$.  On en conclut
aussi que les noyaux des homomorphismes $\GG_{0,\nu}\to
\GG_{0,3}$ ($\nu\ge 3$) sont {\it finis}.  On voit facilement
que pour tout V.A. [Valuation Arithmétique]
$A$ définie sur $K$, l'opération de $\theta'$
sur $\prod_{\ell}T_\ell(A)$ se fait à travers un groupe quotient fini.

Il devient très difficile de s'imaginer comment il pourrait se faire
que $\theta'$ n'opère pas en fait trivialement!  J'ai même l'impression que 
je peux montrer, gr\^ace au résultat du russe que m'a signalé Deligne,
que $\GG_\mathbf{Q}\to\GG_{0,3}$ est un isomorphisme! Cela signifierait
que les actions extérieures de $\GG_\mathbf{Q}$ sur des $\pi_1 \isom 
\overline{\pi_{g,\nu}}$ peuvent s'interpréter (au moins pour $\nu$
assez grand, $g$ étant fixé) comme des scindages de l'extension
$\cN_{g,\nu}$ de $\GG_{g,\nu}$ par $\hat\gT_{g,\nu}$ -- ou de
l'extension $M_{g,\nu}$ de $\GG_{g,\nu}$ par $\hat\gS_{g,\nu}$,
quand il s'agit d'actions effectives.

Bien entendu, la question essentielle qui se pose alors (admettant
$\GG_\mathbf{Q} \isom \GG_{0,3}$) est de caractériser $\GG_{0,3}$
algébriquement, ainsi que les $\cN_{g,\nu}$ -- et de donner une
description algébrique également des scindages ``admissibles'' --
i.e. correspondant bel et bien à des courbes algébriques
sur des corps de nombres algébriques, que ce sont exactement
les actions relevant l'action extérieure donnée,
qui ne normalisent aucun sous-groupe à lacets, ou encore telle
que $\hat\pi_{0,3}^{\Gamma''^+}$ (mais il est concevable
qu'il faille y ajouter de conditions plus subtiles, faisant intervenir
les Frobenius\dots).  On peut se proposer de trouver une description
des actions {\it extérieures} ``admissibles'', sans avoir à
passer par des actions effectives admissibles -- et à s'embarasser d'en
donner une définition plus ou moins plausible.  La difficulté
bien s\^ur provient du fait que si $\Gamma''$ opère {\it extérieurement}
sur $\hat\pi_{0,3}$, il n'opère pas extérieurement pour autant 
sur un sous-groupe ouvert donné $\pi'$ de $\hat\pi_{0,3}$.
Mais on va supposer justement qu'il {\it existe} une telle action,
de telle fa\c con que $\pi'\to\hat\pi_{0,3}$ soit
compatible avec les actions extérieures, et que l'action de
$\Gamma'$ sur $\pi$ se déduise (modulo isomorphisme) de [??].

Pour ce deuxième point, on a une réponse immédiate
(supposant connus déjà les $\cN_{g,\nu}$).  Soit un $(\pi,a)$
de type $g,\nu$, avec relèvement partiel de $\GG_a$ en
$\Gamma'_a\hookrightarrow \cN_a$.  Elle sera admissible si et seulement
si on peut trouver un $(\pi',a')$ de type $(g,\nu')$ et un plongement
de $\pi'$ comme sous-groupe d'indice fini de $\hat\pi_{0,3}$
compatible avec $a'$, et un sous-groupe ouvert $\Gamma''$ de $\GG_{0,3}$,
et une action effective admissible de $\Gamma''$ sur $\hat\pi_{0,3}$
qui relève l'action extérieure donnée, et qui invarie $\pi'$,
de fa\c con qu'à isomorphisme près, $(\pi,a)$, et le germe d'action
de $\Gamma'$ dessus se déduise de $(\pi',a')$ par l'action de
$\Gamma''$ dessus, par ``bouchage'' d'un paquet de trous (et oubli
d'une action effective au profit de l'action extérieure).
Il faut dans cette approche de simplement savoir préciser
algébriquement ce qu'on entend par action ``admissible'' de
$\Gamma''\subset \GG_{0,3}$ sur $\hat\pi_{0,3}$ -- sous-entendant
que \c ca doit correspondre aux points de $U_{0,3}$ rationnels
sur le corps de nombres défini par $\Gamma''$.  On peut conjecturer
celle-ci, par ``bouchage de trous''.

Il semble qu'on puisse sur le même principe donner une description
des $\cN_{g,\nu}$ (donc de $\GG_{g,\nu}$) en termes de $\GG_{0,3}$.
Utilisant les homomorphismes de transition $\hat{\hat\gT}_{g,\nu}
\to\hat{\hat\gT}_{g,\nu'}$, il suffit de le faire, quand
$g$ est fixé, pour des $\nu$ grands -- assez grands pour qu'il
existe une courbe algébrique de type $g,\nu$ sur $\mathbf{Q}$
qui soit un revêtement étale de $U_{0,3}$ sur $\mathbf{Q}$.
On considère donc un plongement correspondant de $\pi_{g,\nu}$
dans $\pi_{0,3}$, et on décrète que si on peut faire opérer 
extérieurement $\GG_{0,3}$ dans $\pi_{g,\nu}$, de fa\c con que
$\pi_{g,\nu}\to\pi_{0,3}$ commute à ces actions extérieures,
{\it alors} le sous-groupe de $\hat{\hat\gT}_{g,\nu}$ engendré
par $\hat\gT_{g,\nu}$ et $\GG_{0,3}$ est $\cN_{g,\nu}$.















%%%%%%%%%%%%%%%%%%%%%%%%%%%%%%%%%%%%%%%%%%%%%%%%%%%%%%%%%%%%%%%
\chapter*{\S \space 34. --- DESCRIPTION HEURISTIQUE PROFINIE DE LA CATÉGORIE DES COURBES ALGÉBRIQUES \\ DÉFINIES SUR DES SOUS-EXTENSIONS FINIES $K$ DE $\overline{\mathbf{Q}}_0/{\mathbf{Q}}$ (I.E. DE ${\mathbf{C}}/{\mathbf{Q}}$))}\thispagestyle{empty}
\addcontentsline{toc}{section}{34. Description heuristique profinie de la catégorie des courbes algébriques définies sur des sous-extensions finies $K$ de $\overline{\mathbf{Q}}_0/{\mathbf{Q}}$ (i.e. de ${\mathbf{C}}/{\mathbf{Q}}$)}
\label{sec:34}
\section*{}


On se borne aux courbes géométriques connexes (par commodité)
anabéliennes (par nécessité provisoire), cf. plus bas sur
la fa\c con de se débarrasser de cette restriction.  La donnée
d'un revêtement de $U_{\overline{\mathbf{Q}}_0}$ définit une
structure d'extension
$$1\to\pi\to\Sigma\to\Gamma^+\to 1
\leqno(1)$$
ou encore on a un homomorphisme
$$E\to\GG\leqno(2)$$
(image $\Gamma$ ouverte, de noyau appelé $\pi$) où
$\Gamma\subset \GG_{{\mathbf{Q}}} \isom \GG_{0,3}$ est le sous-groupe ouvert
correspondant à $K$.  Se donner une telle extension
(moyennant ${\rm Centre}(\pi)=1$) revient à se donner une action
extérieure de $\Gamma'$ sur $\pi$.  Une première question:
faut-il mettre la structure à lacets de $\pi$ dans les données
de l'objet (1) (ou (2)) censé décrire $U/K$ ?
\vskip .3cm
{
Conjecture. --- \it Ce n'est pas la peine -- la structure
à lacets de $\pi$ est la seule structure à lacets 
invariante par l'action extérieure de $\Gamma$ (ou de
$\Gamma^\natural$).  Les sous-groupes à lacets $L_i$ sont les
sous-groupes maximaux dans $\pi$, tels que ${\rm Norm}_E
(L_i)\to \Gamma$ ait comme image un sous-groupe ouvert
de $\Gamma$
\footnote{et tout sous-groupe $\ne \{1\}$ de $\pi$ qui a cette
propriété est contenu dans un unique $L_i$\dots}.
Je présume aussi que tout homomorphisme entre extensions
$E$ de $\Gamma^+$ par un $\pi$, $E'$ de $\Gamma$
par un $\pi'$ ($E$, $E'$ provenant de courbes
algébriques $U$, $U'$) et tel que l'image de $E$
dans $E'$ soit ouverte respecte nécessairement
la structure à lacets, et en fait provient d'un homomorphisme
(unique, on le sait) de courbes algébriques.
}
\vskip .3cm
En tout cas, si on admet la description des sous-groupes à
lacets, il sera clair que l'image par $u$ d'un $L_i$ sera
ou bien (1), ou bien contenu dans un unique $L_j$\dots

Ceci signifierait que le foncteur des courbes algébriques
sur $K$ ([avec comme] morphismes les morphismes dominants) vers les
groupes profinis extérieurs sur lesquels $\Gamma$ opère,
serait pleinement fidèle.  Le foncteur ``extension
du corps de base'' de $K$ à $K'$ cor\-res\-pond au foncteur
restriction d'un groupe extérieur (ou d'une structure
d'extension) de $\Gamma$ à $\Gamma'$ (sous-groupe ouvert).
Les revêtements étales finis de $U$ correspondent
aux $E$-ensembles ($\tilde U$ étant choisi)\dots

Pour décrire l'image essentielle de ce foncteur, on n'est
pas réduit aux conjectures.  On part de l'extension canonique
$E_{0,3}$ de $\GG_{0,3}=\GG$ par $\hat\pi_{0,3}$,
on prend un sous-groupe ouvert $E'$ de $E_{0,3}$,
d'image $\Gamma\subset \GG_{0,3}$, noyau $\pi'\subset \hat\pi_{0,3}$,
et dans la structure à lacets canonique de $\pi'$ de
genre $g$ (induite par celle de $\pi_{0,3}$), on prend un
$I\subset  I(\pi')$ tel que $2g+{\rm card}(I)\ge 3$, stable par
$\Gamma$, et on ``bouche les trous'' en $I(\pi')\setminus I$.
De même, pour décrire quand une action effective, relevant une
action extérieure admissible, est elle-même admissible --
i.e. peut-être obtenue à partir d'une courbe algébrique
$U$, {\it ponctuée} par un point rationnel sur $K$.
Ceci ne signifie pas pour autant, que si $U$ est donnée
par une action extérieure de $\Gamma$ sur un $\pi$, que
les classes de conjugaison de relèvements admissibles
de cette action proviennent bien toutes de points de
$U$ rationnels sur $K$.  Mais on voit de suite que ceci
sera le cas, dès que l'on admet la pleine fidélité
pour les {\it isomorphismes}.

(NB. Même pour les automorphismes
de $U_{0,3}=\P^1_{{\mathbf{Q}}}\setminus\{0,1,\infty\}$, cette
pleine fidélité n'est pas du tout claire.  Il faudrait prouver
que le commutant de $\GG_{0,3}$ dans $\hat{\hat\gT}_{0,3}$ est
réduit à $\gS_3$.  La situation est moins sans espoir,
quand on se donne, avec la structure de groupe profini extérieur
à opérateur $\Gamma$, une arithmétisation de $\pi$ (ce
qui suppose qu'on a explicité une structure à lacets)
invariante par $\Gamma$ -- donc dans $\hat{\hat\gT}(\pi)$ on
dispose d'un $\cN(\pi)$, et $\Gamma\hookrightarrow \cN(\pi)$.
Dans ce point de vue, pour un homomorphisme de $\Gamma$-groupes
extérieurs (arithmétisés) $\pi'\to\pi$, il est
sous-entendu qu'il est compatible avec les arithmétisations,
i.e. si $\pi''$ est l'image de $\pi'$ dans $\pi$, on veut 
que l'arithmétisation de $\pi''$ déduite de celle de
$\pi'$ par ``passage au quotient'', soit celle induite par $\pi$.
En particulier, pour les automorphismes extérieurs de $\pi$,
il est sous-entendu que non seulement ils commutent à
$\Gamma$, mais encore qu'ils sont dans $\cN(\pi)$ (ce qui
implique qu'ils sont dans $\hat\gT(\pi)$, car le centralisateur
dans $\pi$ de tout sous-groupe ouvert de $\GG_\pi$ est
$\{1\}$!)  Dans le cas de $U_{0,3}$, on a $\cN_{0,3} \isom 
\gS_3\times\GG_{0,3}$, et on sait que le centre de $\GG_{0,3}
 \isom \GG$ est triviale -- donc on trouve bien que le groupe des
automorphismes de cette structure est réduit à $\gS_3$!

Ce point de vue est néanmoins probablement superflu -- car
on présume que pour l'action extérieure donnée de
$\Gamma$, il y a une unique arithmétisation invariante
par $\Gamma$, et que les homomorphismes ``admissibles''
de $\Gamma$-groupes extérieurs ``admissibles'', tels qu'ils
ont été définis précédemment, respectent
automatiquement cette arithmétisation.  S'il n'en était
rien, il faudrait bien entendu introduire les arithmétisations
dans la structure.

Il se pose la question de trouver une description plus simpliste
des actions extérieures d'un $\Gamma$ sur un $\pi$ qui sont
``admissibles''.  Ici, on va partir d'un $\pi$ dont on fixe déjà
une structure à lacets et une arithmétisation $a$ (invariantes
par $\Gamma$), de sorte que $\Gamma\subset \cN(\pi)$,
$\Gamma\to\GG_\pi$ injective à image ouverte,
i.e. $\Gamma$ correspond à un scindage partiel (ou germe
de scindage) de
$$1\to \hat\gT(\pi)\to \cN(\pi)\to
\GG_\pi\to 1.$$
Soit $(g,\nu)$ le type de $\pi$; si $g\ge 1$ le critère
la plus simple, c'est que les ``$\Gamma^\natural$-points''
de $\pi'$ déduits de $\pi$ en bouchant tous les trous,
soient distincts.  (NB. Si $g=1$, en bouchant tous les trous
on tombe dans un cas abélien -- mais \c ca ne fait rien).
Si $g=0$, il n'y a pas de condition pour $\nu=3$, et
si $\nu>3$, mettant à part une partie $I'\subset  I(\pi)$
avec ${\rm card}(I')=3$, la condition c'est qu'en bouchant
les trous en $I\setminus I'$, les
$\Gamma^\natural$-points de $\pi$ déduits des points de
$I\setminus I'$ soient distincts.

On notera que dans cette optique conjecturale, dans les cas
limites $(g,0)$ ($g\ge 2$), $(1,1)$, $(0,3)$, on n'impose aucune
condition a priori sur les relèvements.
C'est peut-être très brutal -- et il se pourrait qu'on
trouve des relèvements qui ne correspondent pas à une
courbe algébrique -- i.e. qui en fait ne sont {\it pas}
admissibles, même s'ils le paraissent.

Pour y comprendre quelque chose, il me semble qu'il faut
revenir à une interprétation des germes de scindages dits
``admissibles'' d'une extension telle que
$$1\to\hat\gT_\gn\to\cN_\gn\to
\GG_\gn\to 1$$
où $\GG_\gn \isom \GG_{\mathbf{Q}} \isom \GG_{0,3}$, comme correspondants
aux points algébriques d'une {\it variété} (plutôt ici,
une multiplicité) modulaire $M_{g,\nu,{\mathbf{Q}}}$, dont
$\cN_{g,\nu}$ est le groupe fondamental arithmétique,
et $\hat\gT_\gn$ le groupe fondamental géométrique.
J'aimerais examiner cette situation de plus près, par la suite.
Pour le moment, il semble prudent de ne pas faire de conjectures
h\^atives pour une description {\it directe}
(i.e. {\it pas} via $U_{0,3}$) des opérations extérieures
``admissibles''.  On travaillera donc pour le moment avec cette
notion sous la forme constructive (via $U_{0,3}$).

Si on a un $\pi$ extérieur de type $(g,\nu)$ avec opération
extérieure admissible de $\Gamma\subset \GG$, on prévoit qu'il y
aura une prédiscrétification stricte canonique sur $\pi$
(pas invariante pas $\Gamma$, bien s\^ur -- mais telle que
l'arithmétisation correspondante le soit).  On va même,
pour une action effective admissible de $\Gamma^\natural$ sur $\pi$
(relevant l'action extérieure donnée), définir une
discrétification orientée correspondante canonique $\pi_0
\subset \pi$ (remplacée par une conjuguée, quand on conjugue
le relèvement $\Gamma^\natural\to E$ par
un $g\in\pi$) -- toutes ces discrétifications orientées
(correspondant aux différents ``points'' de $B_{\pi,\Gamma^\natural}$)
définissent une même prédiscrétification orientée --
et même une même prédiscrétification orientée {\it
stricte}.

L'action effective de $\Gamma^\natural$ sur $\pi$ peut s'obtenir
(à isomorphisme près) comme suit: on prend un sous-groupe
ouvert $E'\subset E_{0,3}$, d'où extension
$1\to\pi'\to E'\to\Gamma'
\to 1$ avec $\pi'\subset \hat\pi_{0,3}$, donc
$\pi'=\hat\pi'_0$ ($\pi'_0=\pi'\cap\pi_{0,3}$).  On a une
opération de $\Gamma'$ sur $I(\pi')=I(\pi'_0)$, on la
prend triviale (quitte à passer à un sous-groupe
ouvert de $\Gamma'$), on choisit $i\in I'\subset  I(\pi')$,
on bouche les trous en $I'$, d'où $\pi$, avec
discrétification orientée $\pi_0$, et une action 
extérieure de $\Gamma'$ dessus, qui est relevée en une
action effective gr\^ace à $i$.  On trouvera bien ainsi un
``réseau'' -- une discrétification orientée dans $\pi$ -- je dis qu'il
ne dépend pas des choix qui ont été faits -- en particulier,
que les automorphismes de $(\pi,a,\Gamma^\natural)$ transforment
$\pi_0$ en lui-même.  De plus, si on a deux points $x$,
$y$ de $B_{\pi,\Gamma^\natural}$, d'où $\pi(x)$,
$\pi(y)$, parmi les classes de chemins de $x$ à $y$
(qui font un bitorseur sous $\pi(y)$, $\pi(x)$) il y a
en a pour lesquelles $\pi(x)\to\pi(y)$ envoie
$\pi_0(x)$ dans $\pi_0(y)$ -- quand on se limite à
ceux-ci, on trouve un sous-groupoïde du groupoïde
fondamental $\Pi(\pi,\Gamma^\natural)$, qui est cette fois-ci
un groupoïde {\it connexe} à lacets.  On fera attention
que $\Gamma$ n'opère {\it pas} sur ce groupoïde,
bien qu'il opère sur l'ensemble de ses objets.
















%%%%%%%%%%%%%%%%%%%%%%%%%%%%%%%%%%%%%%%%%%%%%%%%%%%%%%%%%%%%%%%
\chapter*{\S \space 35. --- L'INJECTIVITÉ DE $\GG_\mathbf{Q} \to
{\rm Autext_{lac}}(\hat\pi_{0,3})$}\thispagestyle{empty}
\addcontentsline{toc}{section}{{\bf 35.} L'injectivité de $\GG_\mathbf{Q}\to
{\rm Autext_{lac}}(\hat\pi_{0,3})$}
\label{sec:35}
\section*{}

\vskip .3cm
{
Théorème 1\footnote{Démontré modulo le lemme 2 plus bas.}. --- \it
Soit $\GG={\rm Gal}(\overline{\mathbf{Q}}_0/{\mathbf{Q}})$, où ${\overline{\mathbf{Q}}_0}$
est la clôture algébrique de ${\mathbf{Q}}$ dans ${\mathbf{C}}$, considérons
l'homomorphisme canonique
$$\GG\buildrel\Theta\over\to\hat{\hat\gT}_{0,3}=
{\rm Autext}_{\rm lac}(\hat\pi_{0,3}).\leqno(1)$$
Cet homomorphisme est {\it injectif}.
}
\vskip .3cm
\noindent Démonstration.
\vskip .3cm
{
Lemme 1. --- \it Soit $x\in{\mathbf{Q}}$, $x \neq 0,1$ i.e.
$x \in U_{0,3}({\mathbf{Q}})$; choisissons un chemin sur ${\P}^1({\mathbf{C}})$
de $P=-\bar j$ (point base pour définir $\pi_{0,3}=
\pi_1(U_{0,3}({\mathbf{C}}),P)$,  (NB. $U_{0,3}={\P}^1({\mathbf{Q}})\setminus
\{0,1,\infty\}$), d'où un isomorphisme $\pi_1(U_{0,3}
(\overline{\mathbf{Q}}_0),x) \isom \hat\pi_{0,3}$, et par transport de
structure une action effective de $\GG$ sur
$\hat\pi_{0,3}$ (pas seulement extérieure).  Alors
$K={\rm Ker}\,\Theta$ opère trivialement sur $\hat\pi_{0,3}$.
}
\vskip .3cm
\noindent Démonstration.  Il suffit de voir que l'opération
{\it extérieure} de $K$ sur $\pi_1(\overline V)$ où
$V=U_{0,3}\setminus\{x\}$, est triviale.  D'après le
résultat de Belyi, il existe un morphisme ({\it défini
sur} ${\mathbf{Q}}$, ceci est essentiel)
$${\P}^1{\mathbf{Q}}\to{\P}^1{\mathbf{Q}},$$
étale au-dessus de $U_{0,3}={\P}^1{\mathbf{Q}}\setminus\{0,1,\infty\}$,
et tel que $x\mapstochar\to 0$.  Soit $U'$ le
revêtement étale de $U_{0,3}=U$ induit, $\pi'=
\pi_1(\overline{U}')$ son groupe fondamental géométrique,
sur lequel $\GG$ donc $K\subset \GG$ opère extérieurement.
Alors $\pi(\overline V)$ est un quotient de $\pi'$, avec
respect des opérations extérieures de $\GG$, et il suffit
de prouver que $K$ opère trivialement sur $\pi'$.  Mais
$\pi'$ s'identifie à un sous-groupe ouvert de $\hat\pi_{0,3}=
\pi$ (avec respect des opérations extérieures de $K$), sur
lequel l'opération extérieure de $K$ est triviale.  Il
s'ensuit que l'opération extérieure de $K$ sur $\pi'$ se
fait à travers un groupe quotient {\it fini}.
\[\begin{tikzcd}
	1 & {\pi'} & {E'_K} & K & 1 \\
	1 & \pi & {E_K} & K & 1
	\arrow[from=1-1, to=1-2]
	\arrow[from=1-2, to=1-3]
	\arrow[from=1-3, to=1-4]
	\arrow[from=1-4, to=1-5]
	\arrow[from=2-1, to=2-2]
	\arrow[from=2-2, to=2-3]
	\arrow[from=2-3, to=2-4]
	\arrow[from=2-4, to=2-5]
	\arrow[from=1-2, to=2-2]
	\arrow[from=1-3, to=2-3]
	\arrow[shift left=1, shorten <=2pt, shorten >=2pt, no head, from=1-4, to=2-4]
	\arrow[shorten <=2pt, shorten >=2pt, no head, from=1-4, to=2-4]
\end{tikzcd}\]
[En effet, l'action extérieure de $K$ se fait à travers
l'action effective du groupe extérieur $E_K$,
laquelle par construction de $K$ se fait par un homomorphisme
$E_K\to\Pi/Z$ ($Z={\rm Centre}
(\pi)$), et l'action de $E'_K$ induite se fait par le
composé $E'_K\to E_K\to
\pi/Z$, dont l'image est dans $\cN/Z$, où
$\cN$ est le normalisateur de $\pi'$ dans $\pi$.  Donc
l'image de $K$ dans ${\rm Autext}(\pi')$ est contenue dans
celle de $\cN/Z\to{\rm Autext}(\pi')$,
or $\cN/\pi'$ est fini.]

Repassant à $\pi_1(\overline V)$, on voit donc que l'image
de $K$ dans ${\rm Autext}(\pi_1(\overline V))$ est finie, donc
l'image de $K$ dans ${\rm Aut}(\hat\pi_{0,3})$ est finie.  Elle est
d'ailleurs formée d'automorphismes intérieurs, donc le
lemme 1 sera conséquence du
\vskip .3cm
{
Lemme 2. --- \it Tout automorphisme intérieur d'ordre
fini de $\hat\pi_{0,3}$ (groupe profini libre à deux
générateurs) est trivial.  De fa\c con plus précise:
$\hat\pi_{0,3}$ a un centre réduit à $\{1\}$, et tout
élément de $\hat\pi_{0,3}$ d'ordre fini est réduit à $1$.
}
\vskip .3cm
Ceci est un énoncé d'algèbre profini pure, que je reporte
pour plus tard, pour en terminer avec la partie ``géométrique''
de la démonstration du théorème.
\vskip .3cm
{
Lemme 3. --- \it Pour tout ouvert non vide $V\subset 
{\P}^1{\mathbf{Q}}$, considérant l'action extérieure de $\GG$ sur
$\pi_1(\overline V)$, la restriction de celle-ci à $K$
est triviale.
}
\vskip .3cm
Quitte à passer à un ouvert plus petit, on peut supposer par Belyi
que $V$ est un revêtement étale de $U_{0,3}$ --
et le raisonnement précédent montre alors que l'action
extérieure de $K$ se fait via un groupe quotient fini --
mais cela n'est pas suffisant pour notre propos, et n'implique
pas par lui-même que cette action soit triviale.  D'ailleurs,
au point où j'en suis, on aurait pu remplacer $U$ par
n'importe quelle courbe algébrique quasi projective lisse
géométriquement connexe -- pour trouver que $K$ opère
sur le groupe extérieur $\pi_1(\overline V)$ via un groupe
fini.  Mais ici l'hypothèse $V\subset {\P}^1{\mathbf{Q}}$
implique qu'il existe $y\in V({\mathbf{Q}})$, soit $x \in U({\mathbf{Q}})$
son image dans $U=U_{0,3}$.  Prenant $y$, $x$ comme
points base pour les groupes fondamentaux, on trouve maintenant
un homomorphisme effectif de groupes fondamentaux
$$\pi'=\pi_1(\overline V,y)\hookrightarrow\pi( \isom \hat\pi_{0,3})
=\pi_1(\overline{U},x),$$
compatible avec une action {\it effective} de $\GG$ sur ces groupes.
Par le lemme 1 l'action induite de $K$ sur $\pi=\pi_1(\overline
{U},x)$ est triviale, donc aussi son action sur le sous-groupe
$\pi'$.  A fortiori l'action extérieure est triviale, cqfd.
\vskip .2cm
On peut maintenant prouver le
\vskip .3cm
{
Lemme 4. --- \it $K=\{1\}$ (i.e. le théorème!)
}
\vskip .3cm
En effet, sous les conditions du lemme 2, l'action de 
$\GG$ sur l'ensemble $I=S(\overline{\mathbf{Q}}_0)$, où
$S={\P}^1_{\mathbf{Q}}\setminus V$, est déduite de l'action extérieure
de $\GG$ sur $\pi_1(\overline V)$ si ${\rm card}(I)\ge 2$, comme
l'action sur les classes de conjugaison de ``sous-groupes
lacets''.  Comme $K$ opère trivialement sur le groupe
extérieur, il opère trivialement sur $I$.  Ceci étant
vrai pour tout $V$, on voit que l'action de $K$ sur ${\P}^1
(\overline{\mathbf{Q}}_0)=\overline{\mathbf{Q}}_0 \cup \{\infty\}$ est
triviale, i.e. son action sur $\overline{\mathbf{Q}}_0$ l'est,
donc $K=\{1\}$, cqfd.
\vskip .2cm
\hfill Ouf!
\vskip .3cm
Il reste à reporter la démonstration du lemme 2, que je vais
reformuler sous une forme plus générale:
\vskip .3cm
{
Théorème 2. --- \it Soit $\pi$ un groupe profini libre
sur un ensemble fini $I$ de cardinal $\ge 2$.  Alors:
\vskip .2cm
a) Le centre de $\pi$ est égal à $\{1\}$.
\vskip .2cm
b) Il n'y a pas dans $\pi$ d'élément d'ordre fini autre que
$1$.
}
\vskip .3cm
Finalement je cale sur a) -- tout comme je ne vois pas pourquoi
le centre de $\GG$ (et de tout sous-groupe ouvert) doit être
réduit à $\{1\}$.  Consulter Deligne à ce sujet!

Pour b), voici un expédient.  Soit $K$ un corps algébriquement
clos de caractéristique $0$, et considérons le corps
des fonctions $L=K(X)$ de $U={\P}^1_K\setminus
\{n+1\ {\rm points}\}$ ($n={\rm card}(I)$).  Alors $\pi \isom 
\pi_1(U)$, et c'est un quotient de $E_L=
{\rm Gal}(\bar L/L)$.  Comme $\pi$ est libre, $E_L
\to \pi$ se relève en $\pi\to E_L$,
et il suffit de voir que $E_L$ n'a pas d'élément
d'ordre fini $\ne 1$.  Mais d'après Artin, il n'y a pas
d'automorphisme d'ordre fini $\ne 1$ d'un corps algébriquement
clos $\bar L$, sauf dans le cas où $\bar L$ est extension
quadratique d'un corps ordonné maximal $R$, qui soit le
corps des invariants de $\tau$ (donc $\tau$ d'ordre $2$
exactement).

Mais en l'occurence on aurait $R\supset L$, et $L\supset K$,
or dans $K$ l'élément $-1$ est un carré, donc $R$ ne
peut être ordonné -- absurde!
\vskip .3cm
{
Corollaire 1 (du théorème 1). --- \it Soit $X$ une
courbe algébrique (lisse, géométriquement connexe,
quasi-projective), sur $k$ extension finie de ${\mathbf{Q}}$.  (Je
présume que l'action extérieure de $E^{\bar k}_k$
sur $\pi_1(X_{\bar k})$ est fidèle si $U$ anabélien --
à défaut de pouvoir le prouver, j'énonce:) Alors il
existe une partie ouverte non vide $V$ de $X$ telle que
l'action extérieure de $E^{\bar k}_k$ sur
$\pi_1(V_{\bar k})$ soit fidèle.
}
\vskip .3cm
\noindent Démonstration.  D'après Belyi, on sait qu'on peut
trouver $V\subset  X$ qui soit un revêtement étale de
$U=(U_{0,3})_k$, je dis que ce $V$ là convient.  Soit donc
$K$ le noyau de l'opération extérieure de $\Gamma=E
^{\bar k}_k$ sur $\pi'=\pi_1(V_{\bar k})$.  Soit $y$ un point 
fermé de $V$, rationnel sur l'extension finie $k'$ de $k$
correspondant à un sous-groupe ouvert $\Gamma'=E
^{\bar k}_{k'}$ de $\Gamma$, soit $K'=\Gamma'\cap K$.  Soit
$x$ l'image de $y$ dans $U$: on a
$$\pi'=\pi_1(V_{\bar k},y)\hookrightarrow\pi=\pi_1(U_{\bar
k},x),$$
et par construction l'action extérieure de $K'$ sur $\pi'$
est triviale, donc $K'$ opère sur $\pi'$ par automorphismes
intérieurs.  Or on a le 
\vskip .3cm
{
Lemma 5\footnote{prouvé seulement modulo vérification de
(2), (4) ci-dessous -- pour le Corollaire 1; (2) est d'ailleurs
suffisant\dots}. --- \it Soit $\pi$ un groupe profini à lacets, $\pi'$ un sous-groupe
ouvert; alors tout automorphisme $u$ de $\pi$ qui laisse stable
$\pi'$ et induit sur $\pi'$ un automorphisme intérieur (resp.
l'identité) est intérieur (resp. l'identité).
}
\vskip .3cm
Admettons pour l'instant ce lemme -- il en résulte que les
éléments de $K'$, qui opèrent sur $\pi$ en induisant sur
$\pi'$ des automorphismes intérieurs, induisent sur $\pi$
lui-même des automorphismes intérieurs, i.e. que l'action
extérieure de $K'$ sur $\pi$ est triviale.  D'après le
théorème 1, on sait d'autre part que l'action extérieure
de $\Gamma'$ sur $\pi$ est fidèle, donc $K'=\{1\}$.  Il en
résulte que $K$ est {\it fini}.  Mais par Artin on sait que les
seuls sous-groupes finis $\ne \{1\}$ de $\Gamma$ sont ceux
engendrés par une ``conjugaison complexe'' $\tau$,
correspondant à un sous-corps ordonné maximal entre $k$
et $\bar k$.  Mais pour un tel $\tau$ on a $Hi(\tau)=-1$,
donc l'action extérieure de $\tau$ ne peut être triviale
-- on gagne.  En fait, la démonstration a montré ceci:
\vskip .3cm
{
Corollaire. --- \it Soient $k$ un corps de caractéristique
$0$, $U$ une courbe algébrique (lisse etc.) sur $k$, telle
que $E^{\bar k}_k = \Gamma$ opère fidèlement sur
$\pi_1(U_{\bar k})$ (opération extérieure).  Alors pour
tout revêtement étale géométriquement connexe $V$ de
$U$, l'opération extérieure de $\Gamma$ sur $\pi_1(V_
{\bar k})$ est également fidèle.
}
\vskip .3cm
Reste à prouver le lemme 5.  Il suffit de le prouver dans le cas
``respé'' -- l'autre s'en déduit aussitôt.  Si $\pi$ a au
moins une classe de lacets (cas d'une courbe algébrique
affine i.e. non propre), alors $\pi$ est libre -- et le
lemme 5 est valable justement pour de tels groupes.  Si
$(l_i)_{1\le i\le \nu}$ est un système de générateurs,
soit $L_i$ le sous-groupe fermé engendré par $l_i$,
nous admettrons que $\forall n\in{\mathbf{N}}^*$
$$L_i={\rm Centr}_{\pi}(L_i^n)\ 
\footnote{pas prouvé!}.
\leqno(2)$$
Ceci posé, si l'automorphisme $u$ de $\pi$ induit
l'identité sur $\pi'$, $\exists n\in{\mathbf{N}}^*$ tel que
$\forall 1\le i\le \nu$, $u(l_i^n)=l_i^n$, donc
$u(l_i)$ centralise $l_i^n=u(l_i)^n$,
donc par (2) on a $u(l_i)\in L_i$, i.e. $u(L_i)\subset  L_i$,
mais on a ${\rm Aut}(L_i) \isom \hat{\mathbf{Z}}^*$, et un automorphisme
d'un $\hat{\mathbf{Z}}$-module libre de rang $1$ est connu quand on le
conna\^ \i t sur  ${}_uL_i\buildrel\sim\over\leftarrow L_i$.
Donc $\forall i$ on a $u(l_i)=l_i$, donc $u={\rm id}$, cqfd.

Dans le cas où $\pi$ n'est pas libre, prenons un bon
$(x_i,y_i)_{1\le i\le g}$ -- de sorte que $\pi$ soit défini
par la relation génératrice
$$\prod_{1}^g [x_i,y_i]=1.\leqno(3)$$
Soient $\Lambda_i$, $\Lambda'_i$ les sous-groupes fermés
engendrés par $x_i$, $y_i$.  (Ils sont d'ailleurs tous
conjugués sous ${\rm Aut}(\pi)$\dots).  J'admets que ces groupes
sont $ \isom \hat{\mathbf{Z}}$ (i.e. que les homomorphismes surjectifs
$\hat{\mathbf{Z}}\to\Lambda_i$, $\hat{\mathbf{Z}}\to\Lambda'_i$
envoyant $1$ dans les $x_i$, $y_i$ sont injectifs -- c'est
d'ailleurs trivial en passant à $\pi_{\rm ab}$) {\it et}
qu'on a, en analogie avec (2), pour tout $n\in{\mathbf{N}}^*$
$${\rm Centr}_{\pi}(\Lambda^n_i)=\Lambda_i,\ \ \ 
{\rm Centr}_{\pi}({\Lambda'}^n_i)=\Lambda'_i,$$
et on termine comme ci-dessus.
\vskip .2cm
\noindent {\bf Remarque:} Le résultat le plus fort de
fidélité dans la direction du présent paragraphe, concernant
des actions de $\GG$ et de ses sous-groupes ouverts, serait le
suivant: si $V$ est une courbe algébrique anabélienne sur
l'extension finie $k$ de ${\mathbf{Q}}$, avec la clôture algébrique
$\bar k$, alors non seulement l'action extérieure de
$E^{\bar k}_k=\Gamma$ sur $\pi_1(V_{\bar k})$ devrait
être fidèle, i.e.
$$\Gamma\to\hat{\hat\gT}(\pi)
\footnote{Cet homomorphisme se factorise automatiquement
par le sous-groupe $\cN(\pi)$ qui normalise $\hat\gT(\pi)$.}
$$
injective, mais même l'homomorphisme composé
$$\Gamma\to\cN(\pi)\to
\cN(\pi)/\hat\gT^+(\pi) \isom \GG_\pi$$
devrait être injectif (auquel cas ce sera même un isomorphisme,
puisqu'il est surjectif par construction même de $\cN(\pi)$,
$\GG_\pi$\dots)  Des raisonnements heuristiques faits précédemment 
(moins convaincants sans doute que ceux du présent paragraphe!) semblent
indiquer que ce serait le cas au moins lorsque $V$ est
un revêtement étale de $U=(U_{0,3})_k$, auquel cas
en effet l'homomorphisme $\Gamma\to\GG_\pi$
s'insère dans un diagramme commutatif
\[\begin{tikzcd}
	& \Gamma \\
	{\GG_{\pi_0}} & {\GG_\pi} & {\GG_{\pi'}}
	\arrow[from=1-2, to=2-2]
	\arrow[from=2-2, to=2-3]
	\arrow[from=1-2, to=2-3]
	\arrow[from=1-2, to=2-1]
\end{tikzcd}\]
où $\pi_0=\pi_1(U_{\bar k})$, de sorte que $\pi$ est
un sous-groupe ouvert de $\pi_0$, et $\pi'$ est un sous-groupe
ouvert convenable, {\it caractéristique} dans $\pi_0$.
{\it Si} on pouvait montrer que $\GG_{\pi_0}\to
\GG_{\pi'}$ est injectif (ce qui est plus ou moins une histoire
d'algèbre profinie), il en serait de même du composé
$\Gamma\hookrightarrow\GG_{\pi_0}\to\GG_{\pi'}$,
donc aussi de $\Gamma\to\GG_{\pi}$.  Donc modulo
cette hypothèse sur les groupes profinis, et utilisant
Belyi, on trouve que pour toute courbe algébrique $V$ sur $k$, 
il existe $V'\subset  V$ ouvert non vide, tel que
$$\Gamma\to\GG_{\pi'}$$
(où $\pi'=\pi_1(V'_{\bar k})$) soit injectif.  Quant à
savoir si $\Gamma\to\GG_{\pi}$ est lui-même déjà
injectif -- ou ce qui revient au même, si $\Gamma\to
\GG_g$ pour $g\ge 2$ et $\Gamma\to\GG_{1,1}$ sont
injectifs, je n'ai pas de raison heuristique plausible pour
m'en convaincre à présent -- peut-être est-ce tout à
fait faux?  L'argument plus ou moins convaincant rappelé
précédemment (à supposer qu'on arrive à le justifier)
montrerait seulement que si $\Gamma\to\GG_\pi$
est injectif pour {\it un} $\pi$ (ce qui ne dépend que 
de son type $(g,\nu)$) alors il l'est pour les $\pi'$
ouverts dans $\pi$.  On le sait à
présent (modulo peu de chose, tout au moins) pour les types
$(0,\nu)$ $(\nu\ge 3)$ exactement -- ni plus ni moins -- et
on pourrait peut-être le déduire pour les types $(g,\nu)$
qui s'en déduisent par ``revêtement fini''.  Mais il est
clair déjà qu'on n'obtient pas les types $(g,0)$ comme
cela $(g\ge 2)$, ni même aucun $(g,\nu)$ avec $g\ge 1$,
$\nu\in\{0,1,2\}$.  C'est dire qu'on est loin du compte\dots
Le premier cas bien intéressant serait donc le cas $(1,1)$
(tore à un trou!), où $\hat\gT^+_{1,1} \isom  
{\rm SL}(2,{\mathbf{Z}})^{\hat{}}$ opérant extérieurement sur
$\hat\pi_{1,1}$ (groupe libre à deux générateurs, encore
-- comme par hasard) -- l'action ``arithmétique'' de $\GG_{\mathbf{Q}}$ 
sur $\hat\pi_{1,1}$ (i.e. l'action extérieure {\it mod} 
$\hat\gT^+_{1,1} \isom {\rm SL}(2,{\mathbf{Z}})^{\hat{}}$\ ) est-elle fidèle?














%%%%%%%%%%%%%%%%%%%%%%%%%%%%%%%%%%%%%%%%%%%%%%%%%%%%%%%%%%%%%%%
\chapter*{\S \space 36. --- L'ISOMORPHISME $\GG_\mathbf{Q}\buildrel\sim\over\to
\GG_{1,1}$ ET L'INJECTIVITÉ DE $\GG_\mathbf{Q}\to {\rm Autext}(\hat\gT_{1,1}^+)
 \isom  {\rm Autext}(SL(2,\mathbf{Z})\,\hat{}\,)$}\thispagestyle{empty}
\addcontentsline{toc}{section}{36. L'isomorphisme $\GG_\mathbf{Q}\buildrel\sim\over\to
\GG_{1,1}$ \\ et l'injectivité de $\GG_\mathbf{Q}\to {\rm Autext}(\hat\gT_{1,1}^+)
 \isom  {\rm Autext}(SL(2,\mathbf{Z})\,\hat{}\,)$}
\label{sec:36}
\section*{}

Soit\footnote{Les réflexions du présent paragraphe, un peu cahin caha,
seront reprises de fa\c con moins pesante au paragraphe suivant.} $\pi_{0,3}'$ le groupe quotient de $\GG_{0,3}$ défini par
les relations
$$l_1^2=l_\infty^3=1;\leqno(1)$$
c'est donc le groupe ``carto\-graphi\-que orien\-té pour structures
triangulées'', les $\pi_{0,3}'$-ensembles finis cor\-res\-pondant aux
cartes orien\-tées finies (pouvant avoir des boucles aplaties) dont
les faces sont des triangles ou des mono-angles.  L'opération extérieure
de $\GG_\mathbf{Q}$ (mais non celle de $\gS_3$ !) sur $\hat\pi_{0,3}$ passe au
quotient en une action sur $\hat\pi_{0,3}'$.  On peut améliorer le
théorème 1 du paragraphe précédent par la
\vskip .3cm
{
Proposition. --- \it L'action extérieure de $\GG_\mathbf{Q}$ sur $\hat\pi_{0,3}'$ est fidèle.
}
\vskip .3cm
Pour le montrer, on va plonger $\pi_{0,3}$ comme sous-groupe d'indice
fini dans $\pi_{0,3}'$, d'où un plongement analogue 
$$\hat\pi_{0,3}\hookrightarrow\hat\pi_{0,3}'$$
qui sera compatible avec l'action extérieure de $\GG_\mathbf{Q}$.  

Considérons pour cela le schéma quotient $Y=\P^1_\mathbf{Q}/\gS_3=X/\gS_3$
avec les trois points $a_0, a_1, a_\infty$ de $Y$, rationnels sur $\mathbf{Q}$,
$a_0$ correspondant à la trajectoire $\{0,1,\infty\}$,
$a_1$ à la trajectoire $\{-1,{1\over 2},2\}$ et
$a_\infty$ à la trajectoire $\{j,\bar j\}, (j={\rm exp}
{{2i\pi}\over 3})$ de $\gS_3$ (ce qui épuise l'ensemble des
trajectoires ``singulières'' géométriques).  On sait que $Y$
est une droite projective (sur $\overline{\mathbf{Q}}$ a priori), qu'on épingle
par $a_0, a_1, a_\infty$ comme points $0,1,\infty$, donc $Y \isom 
\P_\mathbf{Q}^1$; l'homomorphisme $X\to Y$ s'identifie donc à 
un morphisme bien déterminé
$$f:\ X=\P^1_\mathbf{Q}\to \P^1_\mathbf{Q}=Y\leqno(3)$$
qui fait de $X$ un revêtement galoisien de $Y$, de groupe $\gS_3$, 
étale au dessus de $\P^1_\mathbf{Q}\setminus\{0,1,\infty\}$, avec comme
indices de ramifications en ces points 2,2,3.  Du point de vue de
la géométrie des cartes (se pla\c cant sur le corps de base $\mathbf{C}$),
on considère la carte déterminée sur $X$ par le triangle 
sphérique $(0,1,\infty)$ (sur l'axe réel comme équateur) -- qui
est donc la carte pondérée universelle -- comme image inverse
de la carte universelle (ayant un seul sommet 0, une seule arête
repliée $0\to 1$, une seule face, de centre $\infty$).  Tout
revêtement étale topologique de $U_{0,3}(\mathbf{C})\subset  X$ donne
ainsi un revêtement étale topologique de  $U_{0,3}(\mathbf{C})\subset  Y$,
ayant au dessus de 1 la ramification 2, au dessus de l'infini la
ramification 3 exactement, ce qui correspond au foncteur ``oubli''
de la pondération, où une carte triangulaire pondérée est
considérée comme une carte tout court.  Du point de vue des
groupes fondamentaux, on trouve ainsi une équivalence de catégories:

\centerline{Revêtements étales de $X\setminus\{0,1,\infty\}=U_{0,3}
 \isom \pi_{0,3}$-ensembles}

\noindent (4)\spc\spc $\downarrow \wr$

\centerline{Revêtements de $Y\setminus\{0,1,\infty\}=U_{0,3}$
/(l'objet $X\setminus\{0,1,\infty\}$ de la dite catégorie)}

\centerline{$ \isom \pi_{0,3}'$-ensembles/$E$}
\vskip .2cm
\noindent où:

\noindent - les revêtements de $Y\setminus\{0,1,\infty\}$ considérés
sont ceux dont la ramification est subordonnée à la signature
$2\{1\}+3\{\infty\}$,

\noindent - $E$ est le $\pi_{0,3}'$-ensemble correspondant à l'objet
$X\setminus\{0,1,\infty\}$ de la catégorie des revêtements
étales de $Y\setminus\{0,1,\infty\}$
à ramification subordonnée à $2\{1\}+3\{\infty\}$.
\vskip .2cm
Ici, le point base de $Y\setminus\{0,1,\infty\}$ choisi pour décrire
les revêtements (à ramification subordonnée à\dots) par un groupe
fondamental à ramification, étant encore $P=-\bar j$, on aura:
$$E=f_{\mathbf{C}}^{-1}(P)\leqno(5)$$
et on peut expliciter ainsi la catégorie ($\pi_{0,3}'$-ens.)/$E$, où
$E$ est un espace homogène, isomorphe au quotient de $\pi_{0,3}'$
par le sous-groupe $\pi_{0,3}'^0$, noyau de l'homomorphisme surjectif
idoine,

\spc\spc $\pi_{0,3}'\to\gS_3$

\spc\spc $l_0\mapstochar\to$\ éléments d'ordre 2

\noindent (6)\spc \ \ \ \ \ \ \ \ \quad\quad\quad
$l_1\mapstochar\to$\ éléments d'ordre 2

\spc\spc $l_\infty\mapstochar\to$\ éléments d'ordre 3

\noindent [en marge:] A calculer!.
\vskip .2cm
On choisit un $Q_0\in E$ comme ``origine'' de $E$ -- pour pouvoir
identifier $E$ comme $\pi_{0,3}'$-ensemble à $\pi_{0,3}'/\pi_{0,3}'^0
 \isom \gS_3$ -- alors ${\rm Ens}(\pi_{0,3}')/E$ s'identifie, par le
foncteur $F\to F_{Q_0}$, à Ens$(\pi_{0,3}'^0)$.

On trouve donc en résumé une équivalence de catégories
(dépendant du choix de $Q_0$)
$$\pi_{0,3}-{\rm ensembles}\buildrel\sim\over\to
\pi_{0,3}'^0-{\rm ensembles}$$
qui est elle-même décrite par un bitorseur sous
$(\pi_{0,3}'^0,\pi_{0,3})$ et, à isomorphisme (non unique!) près
par un isomorphisme
$$\pi_{0,3}\buildrel\sim\over\to\pi_{0,3}'^0,$$
les isomorphismes ainsi obtenus (pour des origines variables du
bitorseur) formant exactement une classe de conjugaison par $\pi_{0,3}'^0$,
i.e. un {\it isomorphisme extérieur} $\pi_{0,3}\buildrel\sim\over
\to\pi_{0,3}'^0$.  Quand de plus le choix de $Q_0$ varie,
on trouve un composé $\ \pi_{0,3}\to\pi_{0,3}'^0\to
\pi_{0,3}'$ exactement une classe de $\pi_{0,3}'$-conjugaison, i.e. un
homomomorphisme extérieur.
\vskip .2cm
J'ai beaucoup turbiné pour pas grand chose -- à défaut d'avoir 
écrit les fonctorialités [?] très générales [??] pour les ``groupes
fondamentaux avec ramification''.  Par exemple le bifoncteur $I$  
mystérieux de tantôt est formé des classes de chemins de
$P\in X(\mathbf{C})\setminus\{0,1,\infty\}$ (point base pour calculer
$\pi_{0,3}$) vers $Q_0$ (jouant le rôle d'un nouveau point base,
ayant le mérite de s'envoyer sur celui -- $P$ -- qui sert à calculer
$\pi_{0,3}'$, groupe fondamental à ramification sur 
$Y(\mathbf{C})\setminus\{0,1,\infty\}$\dots). Il serait peut-être plus commode
de définir directement un foncteur en sens inverse,
\vskip .1cm
\noindent (7) \ Revêtements de $Y(\mathbf{C})\setminus\{0,1,\infty\}$\ \ \ 
\ \ \ \ $\longrightarrow$\ \ \ \ \ Revêtements
étales de $X(\mathbf{C})\setminus\{0,1,\infty\}$

subordonnés à la signature $2\{1\}+3\{\infty\}$
\vskip .1cm
\noindent par image inverse, suivi d'une {\it normalisation}.

Ici on utilise le fait que l'objet $X(\mathbf{C})\setminus\{0,1,\infty\}$ de
la catégorie des revêtements à ramification maximum imposée,
réalise justement un maximum sur les points $1,\infty$ où la
condition intervient.  On trouve que le foncteur correspondant
$$\pi_{0,3}'-{\rm ensembles}\to\pi_{0,3}-{\rm ensembles}
\leqno{(7\  {\rm bis})}$$
a les propriétés d'exactitude d'un foncteur associé à un morphisme
de topos galoisiens
$$B_{\pi_{0,3}}\to B{\pi_{0,3}'}$$
(commute aux limites inductives, exact à gauche) et est donc défini
par un $(\pi_{0,3},\pi_{0,3}')$-ensemble qui soit un torseur (à droite)
pour $\pi_{0,3}'$
\footnote{Ce bitorseur est aussi l'ensemble des $Y_\mathbf{C}\setminus\{0\}$
homomorphismes du ``revêtement universel'' à
ramification imposée de cet espace
sur le revêtement universel ordinaire de $X(\mathbf{C})\setminus\{0,1,\infty\}$.}.
Le choix d'une origine pour ce torseur définit alors aussi un 
homomorphisme correspondant $\pi_{0,3} \to \pi_{0,3}'$.
\vskip .2cm
Revenons à la situation arithmétique sur $\mathbf{Q}$, où on dispose 
non seulement des groupes fondamentaux géométriques profinis
$\hat\pi_{0,3},\hat\pi_{0,3}'$, mais aussi d'extensions
\[\begin{tikzcd}
	1 & {\hat{\pi}_{0, 3}} & {E_{0, 3}} & \GG & 1 \\
	1 & {\hat{\pi}'_{0, 3}} & {E'_{0, 3}} & \GG & 1
	\arrow[from=1-1, to=1-2]
	\arrow[from=1-2, to=1-3]
	\arrow[from=1-3, to=1-4]
	\arrow[from=1-4, to=1-5]
	\arrow[from=2-1, to=2-2]
	\arrow[from=2-2, to=2-3]
	\arrow[from=2-3, to=2-4]
	\arrow[from=2-4, to=2-5]
\end{tikzcd}\leqno{(8)}\]
qui s'interprètent comme des groupes fondamentaux de $U_{0,3}$,
resp. de $U_{0,3}$ avec ramification subordonnée à $2\{1\}+
3\{\infty\}$.  Les raisonnements précédents s'étendent à ce
cadre et fournissent donc un homomorphisme d'extension
\[\begin{tikzcd}
	1 & {\hat{\pi}_{0, 3}} & {E_{0, 3}} & \GG & 1 \\
	1 & {\hat{\pi}'_{0, 3}} & {E'_{0, 3}} & \GG & 1
	\arrow[from=1-1, to=1-2]
	\arrow[from=1-2, to=1-3]
	\arrow[from=1-3, to=1-4]
	\arrow[from=1-4, to=1-5]
	\arrow[from=2-1, to=2-2]
	\arrow[from=2-2, to=2-3]
	\arrow[from=2-3, to=2-4]
	\arrow[from=2-4, to=2-5]
	\arrow[hook', from=1-3, to=2-3]
	\arrow[hook', from=1-2, to=2-2]
	\arrow[shift left=1, shorten <=2pt, shorten >=2pt, no head, from=1-4, to=2-4]
	\arrow[shorten <=2pt, shorten >=2pt, no head, from=1-4, to=2-4]
\end{tikzcd}\leqno{(9)}\]
défini modulo composition par un automorphisme intérieur de $\pi_{0,3}'$
[$\hat\pi_{0,3}\to\hat\pi_{0,3}$ s'insère dans une suite
exacte
$$1\to\hat\pi_{0,3}\to\hat\pi_{0,3}'\to\gS_3
\leqno{(9\ {\rm bis})}$$ 
et itou pour les groupes discrets,
$1\to\pi_{0,3}\to\pi_{0,3}'\to\gS_3$].
\vskip .2cm
Ceci dit, soit $K$ le noyau de l'opération extérieure de $\GG$ sur 
$\pi_{0,3}'$.  On voit alors que l'opération extérieure de $K$
sur le sous-groupe ouvert $\hat\pi_{0,3}$ se fait par un groupe fini
(en fait par l'intermédiaire de $\hat\pi_{0,3}'/\hat\pi_{0,3} \isom 
\gS_3$\dots) -- ce qui implique, par le théorème 1 du paragraphe
précédent, que le groupe $K$ lui-même est fini.  Donc par Artin
on a $K=1$ ou $K=(1,\tau)$, $\tau$ la conjugaison complexe.  Mais
comme $Hi (\tau)=-1$, il est évident que l'opération extérieure
de $\tau$ sur $\pi_{0,3}'$ (où même sur $\pi'_{{0,3}\,{\rm ab}}$ ($ \isom 
\mathbf{Z}/2\mathbf{Z}\times\mathbf{Z}/3\mathbf{Z}$) n'est pas triviale.  Cela prouve la proposition.
\vskip .2cm
J'ai envie maintenant d'interpréter $\pi_{0,3}'$ comme 
$\gT_{1,1}^+/$(centre) (le centre est isomorphe à $\{\pm 1\}$), et itou
pour $\hat\pi_{0,3}' \isom \hat\gT_{1,1}/\{\pm 1\}$, isomorphisme qui soit
compatible avec l'action extérieure de $\GG$.  On a (pour mémoire)
$$\gT_{1,1}\buildrel\sim\over\to GL(2,\mathbf{Z}),\leqno(10)$$
l'isomorphisme étant obtenu ainsi
$$\gT_{1,1}\buildrel\sim\over\to\gT_{1,0}\buildrel\sim\over\to
{\rm Autext}(\pi_{1,0}) \isom  {\rm Autext}(\mathbf{Z}^2) \isom  GL(2,\mathbf{Z})\leqno(11)$$
[où le premier isomorphisme est celui de bouchage de trou].

On a donc 
$$\gT_{1,1}^+ \isom  SL(2,\mathbf{Z})\leqno(12)$$
et 
$${\rm Centre}(\gT_{1,1}) \isom  \{\pm 1\}\leqno(13)$$
(s'identifie au groupe des homothéties de $SL(2,\mathbf{Z})$).  Il est connu
qu'on a dans $SL(2,\mathbf{Z})/\{\pm 1\}$ deux générateurs $\lambda_2,\lambda_3$
satisfaisant respectivement 
$$\lambda_3^3 =1, \lambda_2^2=1\leqno(14)$$
et tels que ces relations soient une {\it présentation} de
$SL(2,\mathbf{Z})/\{\pm 1\}$.  Sauf erreur ces éléments proviennent d'éléments
$u_4, u_6$ de $SL(2,\mathbf{Z})$ lui-même, satisfaisant
$$u_6^6=1, u_4^4 =1,\leqno(15)$$
et qui correspondent aux seuls éléments d'ordres 6 et 4 (à conjugaison
et passage à l'inverse près).  En fait, tout élément d'ordre
fini de $SL(2,\mathbf{Z})$ est conjugué à une puissance de $u_4$ ou $u_6$.  Les
sous-groupes engendrés respectivement par $u_4$ et $u_6$, cycliques 
d'ordre 4 et 6, sont les groupes des automorphismes des deux courbes
elliptiques exceptionnelles (d\^\i tes ``anharmoniques'')
(en caractéristique 0).

Je n'ai pas de formule sous la main pour ``placer'' $u_4$ et $u_6$,
de fa\c con qu'ils engendrent le groupe $SL(2,\mathbf{Z})$ modulo son centre
(ce qui implique sans doute qu'ils engendrent $SL(2,\mathbf{Z})$) -- mais on fera
attention à la relation supplémentaire, s'ajoutant à (15)
$$u_4^2=u_6^3\leqno(16)$$
(c'est justement l'élément $-1$ du centre de
$SL(2,\mathbf{Z})$).  Je n'ai peut-être
pas besoin de la formule explicite pour $u_4$ et $u_6$.  On définit
alors
$$\pi_{0,3}'\buildrel\sim\over\to SL(2,\mathbf{Z})/\{\pm 1\}\leqno(17)$$
par 
$$l_1'\mapstochar\to\lambda_2,
\ l_\infty'\mapstochar\to\lambda_3\leqno(18)$$
(donc $l_0'\mapstochar\to(\lambda_3\lambda_2)^{-1}
=\lambda_2\lambda_3^{-1}
=\lambda_2\lambda_3^2$), où $l_0', l_1', l_\infty'$ sont les
générateurs de $\pi_{0,3}'$ correspondants à ceux $l_0,l_1,l_\infty$
de $\pi_{0,3}$, avec les relations de définition
$$l_\infty' l_1' l_0'=1,\ l_1'^2=1,\ l_\infty'^3=1,\leqno(19)$$
de sorte que $\pi_{0,3}'$ est bien le groupe de générateurs $l_1', 
l_\infty'$ satisfaisant à $l_1'^2=1,l_\infty'^3=1$.
\vskip .2cm
Je veux maintenant me convaincre que l'homomorphisme correspondant
à (17),
$$\hat\pi_{0,3}'\to SL(2,\mathbf{Z})\,\hat {}/\{\pm 1\} \isom 
\hat\gT_{1,1}^+/\{\pm 1\}\leqno(20)$$
est compatible avec l'opération extérieure de $\GG=\GG_\mathbf{Q}$.  Pour ceci,
j'ai envie de définir directement le composé avec $\hat\pi_{0,3}
\to\hat\pi_{0,3}'$, i.e. $\hat\pi_{0,3}\to
\hat\gT_{1,1}^+/\{\pm 1\},$ et même de le relever en un homomorphisme
$$\hat\pi_{0,3}\to  \hat\gT_{1,1}^+.\leqno(21)$$
Au niveau des groupes discrets, on définit bien 
$$\pi_{0,3}\to SL(2,\mathbf{Z})\leqno(22)$$
$${\rm par}\quad l_1\mapstochar\to u_4,
\ l_\infty\mapstochar\to u_6, \ 
l_0\mapstochar\to (u_6 u_4)^{-1}=u_4^{-1}u_6^{-1},$$
et $SL(2,\mathbf{Z}) \isom \gT_{1,1}^+$ s'identifie au quotient de $\pi_{0,3}$
par le sous-groupe engendré par les éléments 
$$l_1^4, l_\infty^6, (l_1^2)(l_\infty^3)^{-1}.$$
D'ailleurs $\pi_{0,3}$ peut s'interpréter comme un groupe de Teichmüller
$\gT_{0,4}^{!+}$
\footnote{En effet $\gT_\gn^!$ est extension de
$\gT_{g,\nu-1}^!$ par $\pi_{g,\nu-1}$ (si $(g,\nu-1)$ anabélien)
et itou pour les $\gT^+$; $\gT_{0,3}^{!+}=\{1\}$!},
et (22) comme un homomorphisme
$$\gT_{0,4}^{!+}\to\gT_{1,1}^+\leqno(23)$$
dont je soup\c conne qu'il se prolonge en un homomorphisme
$$\gT_{0,4}^!\to\gT_{1,1} \isom  SL(2,\mathbf{Z})\leqno(24)$$
[le premier membre étant] une extension de $(1,\tau)$ par $\pi_{0,3}$,
qui n'est autre que le groupe cartographique triangulé pondéré
non orienté.

J'interprète le premier membre de (23) comme le $\pi_1$ ``géométrique
transcendant'' du topos modulaire $M_{0,4}^!$, classifiant les droites
projectives avec quatre points disctincts numérotés de 1 à 4;
et le deuxième membre de (23) est le $\pi_1$ transcendant du topos
mo\-du\-laire $M_{1,1}$, classifiant les courbes elliptiques (avec une origine
fixée).  On devrait donc pouvoir définir, au niveau des multiplicités
modulaires sur $\mathbf{Q}$, 
$$(M_{0,4}^!)_\mathbf{Q}\to (M_{1,1})_\mathbf{Q},\leqno(25)$$
qui donnerait naissance à (23). 

Mais on voit de suite qu'on a un isomorphisme canonique
$$M_{0,4}^! \isom  U_{0,3}\leqno(26)$$
(je laisse tomber les indices $\mathbf{Q}$), et la donnée de (25) revient donc
aussi à la donnée d'une famille de courbes elliptiques sur $U_{0,3}$.
Mais si $\lambda$ est la ``variable $\in U_{0,3}$'', en prenant ``le''
revêtement quadratique $E_\lambda$ de $\P^1$ ramifié en $0,1,\infty,
\lambda$, on trouve une courbe elliptique, avec quatre points marqués 
-- au dessus de $0,1,\infty, \lambda$ -- dont on peut prendre le point
au dessus de 0 comme origine -- alors les trois autres points
sont les trois points d'ordre 2, indexés 
respectivemnt par $1,\infty,\lambda$.

De fa\c con plus précise: sur un schéma de base quelconque $S$, 
sauf que 2 y soit inversible (caractéristique résiduelle différente
de 2), on a un foncteur qui va de la catégorie des systèmes
$(E,\alpha,\beta)$ d'une courbe elliptique (homogène) relative sur $S$,
et $\alpha,\beta$ deux sections de $E$ formant une base de ${}_2E$, en tant
que schéma localement constant en $\F_2$ -- modules (ou système local
de $\F_2$-modules) de $S$ (donnant naissance à $\gamma=\alpha+\beta$, comme
troisième larron - de sorte que $(\beta,\gamma)$ etc. sont en fait aussi
des bases) 
\footnote{Il revient au même de dire que l'on a trois sections
$\alpha,\beta,\gamma$ de ${}_2E$ sur $S$, qui sont distinctes en
tout $s\in S$ et partant $\ne 1$.}
-- vers la catégorie des fibrés en droites projectives $P$ sur $S$,
avec quatre sections $u_0,u_1,u_\infty,\lambda$ marquées disctinctes
en chaque point, en associant à $E$ le quotient $E/\pm1$, muni
des sections $u_0, u_1, u_\infty, \lambda$
qui sont images respectivement de $0,\alpha,\beta,
\gamma=\alpha+\beta$; et ce foncteur est {\it presque} une équivalence 
de catégories.  Ce qui lui manque pour l'être, c'est que pour une courbe
elliptique relative $E$, la symétrie $x\to -x$ de $E$ -- qui est
un automorphisme non trivial -- opère trivialement sur l'objet 
correspondant $E/\{\pm1\}$ 
\footnote{Les pages qui suivent sont inutilement compliquées,
avec l'introduction de $\M_{1,1}', \gT_{1,1}'$ etc. il suffit d'y 
aller brutalement avec la courbe
elliptique $y^2=\sqrt{x(x-1)(x-\lambda)}$ sur $U_{0,3}$,
pour avoir $U_{0,3}\to M_{1,1}$; cf. plus bas\dots}.
Donc il faut prendre le champ sur $(Sch)$, déduit de celui des courbes
elliptiques en commen\c cant par prendre ${\rm Isom}(E,F)'= {\rm Isom}
(E,F)/\pm1$, puis on prendra le champ associé à un préchamp
(les objets sur $S$ sont les ``courbes elliptiques relatives à symétrie
près sur $S$'').  Le champ est représentable par une multiplicité 
modulaire $M_{1,1}'$, ayant comme groupe fondamental géométrique
$SL(2,\mathbf{Z})\,\hat{}/\{\pm1\}$ justement, déduit par exemple des variétés
modulaires à rigidification de Jacobi déchelon $n$, $M_{1,1}[n]$ --
avec un groupe $SL(2,\mathbf{Z}/n\mathbf{Z})$ opérant dessus, de sorte que
$$M_{1,1} \isom  \bigl(M_{1,1}[n], SL(2,\mathbf{Z}/n\mathbf{Z})\bigr)\leqno(27)$$
et en outre que le sous-groupe central $\{\pm1\}$ de $SL(2,\mathbf{Z}/n\mathbf{Z})$
opère trivialement sur $M_{1,1}[n]$; donc on peut faire opérer
le groupe quotient $SL(2,\mathbf{Z}/n\mathbf{Z})'=SL(2,\mathbf{Z}/n\mathbf{Z})/\{\pm1\}$, et poser
$$M_{1,1}'=\bigl(M_{1,1}[n], SL(2,\mathbf{Z}/n\mathbf{Z})'\bigr)\leqno(28)$$
(ce qui manifestement ne dépend pas du choix de $n$, $n\ge 2$). On 
trouve ainsi un ho\-mo\-mor\-phisme
$$M_{0,4}^! \isom  U_{0,3}\longrightarrow M_{1,1}',\leqno(29)$$
qui fait de $M_{0,4}^!$ un revêtement galoisien de groupe $\gS_3$
de $M_{1,1}'$, et de fa\c con plus précise
$$M_{0,4}^!\buildrel\sim\over \to M_{1,1}[2]'\to M_{1,1}'
\leqno(30)$$
[la deuxième flèche définissant un] revêtement galoisien de
groupe $SL(2,\F_2) \isom \gS_3$), le groupe fondamental géométrique
de $M_{1,1}[2]$ étant d'ailleurs isomorphe au noyau de l'ho\-mo\-mor\-phisme
$SL(2,\mathbf{Z})'\,\hat{}\to SL(2,\mathbf{Z}/2\mathbf{Z})$ qui factorise l'homomorphisme
canonique $SL(2,\mathbf{Z})\,\hat{}\to SL(2,\mathbf{Z}/2\mathbf{Z})$.

Passant aux groupes fondamentaux pour $M_{0,4}^!=U_{0,3}$ et $M_{1,1}'$,
on trouve un homomorphisme de suites exactes
\[\begin{tikzcd}
	1 & {\hat{\pi}_{0, 3} = \gT^!_{0, 4}} & {E_{0, 3} = \cN_{0, 4}} & \GG & 1 \\
	1 & {\hat{\gT}^+_{1, 1}} & {E'_{1, 1}} & \GG & 1
	\arrow[from=1-1, to=1-2]
	\arrow[from=1-2, to=1-3]
	\arrow[from=1-3, to=1-4]
	\arrow[from=1-4, to=1-5]
	\arrow[from=2-1, to=2-2]
	\arrow[from=2-2, to=2-3]
	\arrow[from=2-3, to=2-4]
	\arrow[from=2-4, to=2-5]
	\arrow[from=1-2, to=2-2]
	\arrow[from=1-3, to=2-3]
	\arrow["\sim", from=1-4, to=2-4]
\end{tikzcd}\leqno{(31)}\]
([avec] $\hat\gT_{1,1}'^+=\hat\gT_{1,1}^+/\{\pm1\} \isom  
SL(2,\mathbf{Z})\,\hat{}/\{\pm1\})$, qui identifie $E_{0,3}$ à un sous-groupe
ouvert d'indice 6 dans $E_{1,1}'=E_{1,1}/\{\pm1\}$, et itou pour
$\hat\pi_{0,3}$ dans $\hat\gT_{1,1}'^+$.  On trouve ainsi un isomorphisme
$$\hat\pi_{0,3}\buildrel\sim\over\to {\rm Ker}\, 
(\gT_{1,1}'^+\to\gS_3) \isom  \bigl( {\rm Ker}
(\gT_{1,1}^+\to\gS_3)\bigr)/\{\pm1\} \leqno(32)$$
compatible avec les actions extérieures de $\GG$.  On peut dire aussi
qu'on a une structure d'extension
$$1\to\hat\pi_{0,3}\to\gT_{1,1}'^+\to\gS_3
\to 1\leqno(33)$$
(i.e. $1\to\hat\pi_{0,3}\to SL(2,\mathbf{Z})\,\hat{}/\{\pm1\}
\to SL(2,\mathbf{Z}/2\mathbf{Z})\to 1$) compatible avec les actions
de $\GG$ ($\GG$ opérant trivialement sur $\gS_3$).
\vskip .2cm
Je finis par m'apercevoir que ce qu'on obtient ici est loin de (22)
-- cet homomorphisme (22) n'a rien d'injectif, par contre il est
surjectif, et ses valeurs sont, non dans $SL(2,\mathbf{Z})/\{\pm 1\}$, mais dans
$SL(2,\mathbf{Z})$ lui-même!  Il faudra donc que je revienne encore sur une
fa\c con de donner un sens arithmético-géométrique remarquable à (22),
et son extension aux groupes profinis correspondants.  Pour le moment,
je m'en tiens à exploiter un peu (33).  Tout d'abord je note que les
arguments faits dans le cadre schématique marchent aussi dans le cas
transcendant, analytique complexe, et fournissent alors
$$\pi_{0,3}\buildrel\sim\over\to {\rm Ker}(\gT_{1,1}'^+
\to \gS_3) = {\rm Ker}\big(SL(2,\mathbf{Z})/\{\pm1\}
\to SL(2,\mathbf{Z}/2\mathbf{Z})\big) \leqno(34)$$
compatible avec (32), ou encore une suite exacte,
$$1\to\pi_{0,3}\to \gT_{1,1}'^+=SL(2,\mathbf{Z})/\pm1\to
\gS_3=SL(2,\mathbf{Z}/2\mathbf{Z})\to1. \leqno(35)$$
On aimerait interpréter (35) et (33), comme correspondant aux suites
exactes analogues liées au diagramme (9), reliant $\pi_{0,3}$ et
$\pi_{0,3}'$ -- en identifiant $\pi_{0,3}'$ à $\gT_{1,1}'^+$,
$\hat\pi_{0,3}'$ à $\hat\gT_{1,1}'^+$.  J'y reviendrai tantôt.
\vskip .2cm
\noindent {\bf Remarque:} On peut se demander si l'homorphisme
$$\hat \pi_{0,3}\to SL(2,\mathbf{Z})\,\hat{}/\{\pm1\}
 \isom \hat\gT_{1,1}^+/\{\pm1\}$$
se remonte
\footnote {(*)}{Oui, on peut remonter, et c'est plus ou moins trivial\dots
cf. plus bas\dots}
(de fa\c con plus ou moins naturelle) en un homomorphisme
$$\hat \pi_{0,3}\to SL(2,\mathbf{Z})\,\hat{} \isom  \hat\gT_{1,1}^+,$$
et itou pour les groupes discrets
$$\pi_{0,3}\to SL(2,\mathbf{Z}) \isom  \gT_{1,1}^+.$$
Bien s\^ur, comme $\pi_{0,3}$ et $\hat\pi_{0,3}$ sont libres (avec deux
générateurs) on peut toujours remonter -- d'exactement {\it quatre}
fa\c cons d'ailleurs (qui nécessairement, dans le cadre profini, 
respectent les réseaux discrets).  Mais peut-on le faire en respectant
les opérations de $\GG$?  Il suffirait pour cela qu'on puisse
remonter $U_{0,3}\to M_{1,1}'$ en $U_{0,3}\to
 M_{1,1}$ (NB. $M_{1,1}$ est une $\mathbf{Z}/2\mathbf{Z}$-gerbe au dessus de $M_{1,1}'$),
et je suspecte, d'après le yoga ``anabélien'' que j'essaye de 
développer, que l'inverse doit être vrai 
\footnote{Pas tout fait, cf. page suivante pour une formulation
plus raisonnable\dots} -- que tout relèvement de $\hat\pi_{0,3}
\to \hat\gT_{1,1}'^+$ commutant aux opérations de $\GG$
est défini par un tel relèvement $U_{0,3}\to M_{1,1}$.
 Or l'existence d'un tel relèvement $U_{0,3}\to M_{1,1}$
signifierait exactement l'existence d'une famille de courbes elliptiques
sur $U_{0,3}$, avec rigidification de Jacobi d'échelon 2, qui
corresponde à l'invariant tautologique $\lambda$.  Je doute qu'il en
existe une, comme je doute que le relèvement en termes de groupes
profinis à opération puisse se faire.

À vrai dire, comme $\hat\gT_{1,1}^+=SL(2,\mathbf{Z})\,\hat{}$  n'a plus
de centre trivial, il n'est plus raisonnable de vouloir
``tout exprimer'' par les opérations extérieures de $\GG$ sur
$\hat\gT_{1,1}^+$, il faut plutôt revenir à l'extension
$E_{1,1}=E_{M_{1,1}}$ de $\GG$ par $\hat\gT_{1,1}^+$, et la 
question est si l'homomorphisme 
$$E_{0,3}\longrightarrow E_{1,1}'=E_{1,1}/\{\pm1\}$$
se remonte en
$$E_{0,3}\buildrel ?\over\longrightarrow E_{1,1},$$
ce qui est plus fort que de trouver un relèvement
$\hat\pi_{0,3}\to \hat\gT_{1,1}^+$, compatible avec les
opé\-ra\-tions extérieures de $\GG$. Pour apprécier cette différence,
je note que $\GG$ opère sur l'ensemble $E$ des quatre relèvements
$\hat\pi_{0,3}\to\hat\gT_{1,1}^+$ (interprétés comme homomorphismes
extérieurs), et la question posée sous la forme faible est s'il
existe un élément de $E$ invariant par $\GG$.  S'il n'existait pas,
il y aurait en tous cas un sous-groupe ouvert de $\GG$, d'indice 2 ou 4,
qui laisserait invariant un élément -- donc, quitte à passer à
l'extension finie correspondante de $\mathbf{Q}$, on trouve un relèvement, et
quitte à passer à une extension un peu plus grande, les {\it quatre}
relèvements possibles commutent à l'action extérieure de $\GG$.  Par
contre, rien ne prouve que l'on puisse trouver un ``germe de relèvement''
de $E_{0,3}\to E_{1,1}'$ en $E_{0,3}^\natural\to
E_{1,1}^\natural$ (germes pris par rapport aux sous-groupes ouverts
de $E_{0,3}$ contenant $\hat\pi_{0,3}$, ou même tous les sous-groupes
ouverts).  L'obstruction à remonter sur $E_{0,3}$ lui-même, i.e.
à {\it scinder} une certaine extension de $E_{0,3}$ par $\{\pm1\}$,
est
$$\alpha\in H^2(E_{0,3},\mathbf{Z}/2\mathbf{Z}),\leqno(36)$$
et rien ne dit que cette classe de cohomologie puisse s'effacer, en passant
à un sous-groupe ouvert de $E_{0,3}$.

Mais en termes géométriques, les courbes elliptiques relatives
cherchées sur $U_{0,3}$ forment les sections d'un champ sur 
$(U_{0,3})_{\hbox {ét}}$ -- qui est une $\mathbf{Z}/2\mathbf{Z}$-gerbe -- l'obstruction
se trouve donc dans un groupe de Brauer, 
$$\beta\in H^2(U_{0,3},\mathbf{Z}/2\mathbf{Z}),\leqno(37)$$
ou comme $U=U_{0,3}$ est une courbe algébrique affine sur
un corps (il suffirait qu'elle ne soit pas de type $0,0$), il 
s'en suit que l'homomorphisme canonique (à coefficients de torsion
quelconques)
$$H^\ast(E_U,-)\longrightarrow H^\ast(U,-)$$
est un isomorphisme.  D'ailleurs, il doit être plus ou moins trivial
que $\beta$ est l'image de $\alpha$ -- ce qui confirme l'intuition
que si le relèvement est possible au niveau des groupes fondamentaux
profinis (arithmético-géométriques), il l'est aussi au niveau des
multiplicités modulaires elles-mêmes, donc qu'on a une existence d'une
courbe elliptique relative sur $U_{0,3}$ qui\dots

Il est vrai qu'il est bien connu qu'une classe
de cohomologie (37) à coefficients
de torsion s'efface, en passant à une extension finie du corps
de base ($\mathbf{Q}$ en l'occurence).  En fait, on a une suite spectrale
$$H^\ast(U_{0,3},\mathbf{Z}/2\mathbf{Z})\Leftarrow E^{p,q}_2=H^p\big(\mathbf{Q},H^q(\overline
{U_{0,3}},\mathbf{Z}/2\mathbf{Z})\bigr), \leqno(38)$$
et ici $H^q(\overline{U_{0,3}})=0$ pour $q\ge 2$, donc on trouve
une suite exacte
$$\ldots E^{0,1}_2=H^q(\mathbf{Q},H^1(\overline U))\to
E^{2,0}_2=H^q(\mathbf{Q},\mathbf{Z}/2\mathbf{Z})\to H^2(U_{0,3})\to$$
$$\to E^{1,1}_2=H^p(\mathbf{Q},H^1(\overline U) \isom \F_2^2)\to
E^{3,0}_2=H^3(\mathbf{Q},\mathbf{Z}/2\mathbf{Z})\to\ldots$$
\vskip .2cm
Je me rends compte enfin que l'existence d'un relèvement -- i.e. de
la courbe elliptique hypothétique sur $U_{0,3}$ -- est tout à fait
triviale, il suffit de le définir par l'équation
$$y=\sqrt{x(x-1)(x-\lambda)},$$
$${\rm i.e. }\quad y^2-x(x-1)(x-\lambda)=F(x,y;\lambda)=0,$$
ce qui définit une courbe plane affine, ou passer en coordonnées
projectives
$$F(x,y,z;\lambda)=y^2z-x^3+(1+\lambda)x^2z-\lambda xz^2=0.$$
Je ne vais pas approfondir la question ici, dans quelle mesure ce choix
est ``naturel''.

Il donne en tous cas un homomorphisme tout ce qu'il y a de précis
$$M_{0,4}^! \isom U_{0,3}\longrightarrow M_{1,1},\leqno(39)$$
d'où un homomorphisme d'extensions
\[\begin{tikzcd}
	1 & {\hat{\pi}_{0, 3} = \gT^!_{0, 4}} & {E_{0, 3} = \cN^!_{0, 4}} & \GG & 1 \\
	1 & {\hat{\gT}^+_{1, 1} = \Sl(2, \mathbf{Z})} & {E_{1, 1} = \cN_{1,1}} & \GG & 1
	\arrow[from=1-1, to=1-2]
	\arrow[from=1-2, to=1-3]
	\arrow[from=1-3, to=1-4]
	\arrow[from=1-4, to=1-5]
	\arrow[from=2-1, to=2-2]
	\arrow[from=2-2, to=2-3]
	\arrow[from=2-3, to=2-4]
	\arrow[from=2-4, to=2-5]
	\arrow[from=1-2, to=2-2]
	\arrow[from=1-3, to=2-3]
	\arrow[shift left=1, shorten <=2pt, shorten >=2pt, no head, from=1-4, to=2-4]
	\arrow[shorten <=2pt, shorten >=2pt, no head, from=1-4, to=2-4]
\end{tikzcd}\leqno{(39)}\]
défini modulo automorphisme intérieur par un élément de
$\hat\gT_{1,1}^+=SL(2,\mathbf{Z})\,\hat{}$ , et au niveau des groupes discrets,
$$\pi_{0,3}\longrightarrow \hat\gT_{1,1}^+$$
et même,
$$\pi_{0,3}\hookrightarrow \Gamma_2={\rm Ker}\bigl(
\hat\gT_{1,1}^+=SL(2,\mathbf{Z})\,\hat{}\to \gS_3=SL(2,\mathbf{Z}/2\mathbf{Z})\bigr)
\leqno(42)$$
(et itou pour les groupes profinis).  Ici $\pi_{0,3}$ est tel que le
sous-groupe de congruence du deuxième membre [?], soit $\Gamma_2$ soit
produit direct [??] sous-groupe $\pi_{0,3}$ et de son centre $\{\pm1\}$.
Mais en fait, il y a exactement quatre sous-groupes dans $\Gamma_2$ qui
réalisent cette décomposition.  Pour ne pas faire de jaloux, on
pourrait regarder l'intersection des quatre, qui est un sous-groupe
de $\Gamma_2$ d'indice un diviseur de 24=16 [?]
\footnote{NB. \c Ca doit être le noyau de $SL(2,\mathbf{Z})\to
SL(2,\mathbf{Z}/4\mathbf{Z}$).};
c'est l'intersection des noyaux de quatre homomorphismes
$\Gamma_2\to {\rm Centre}(\Gamma_2)=\{\pm1\}$. En tant que
sous-groupe de $\pi_{0,3}$, il est d'indice un diviseur de 8 [c'est
vrai et c'est même trivial\dots] -- \c ca ne m'étonnerait pas que ce
soit justement le groupe des commutateur de $\pi_{0,3}$, qui est d'indice 4
-- il y a là toute une situation à élucider\dots

Bien entendu, il faudrait aussi expliciter l'homorphisme (42)
(défini modulo automorphismes intérieurs) par ses valeurs sur
les générateurs $l_0,l_1,l_\infty$ -- j'y reviendrai peut-être
tantôt 
\footnote{Il n'est pas clair si $\pi_{0,3}$ est invariant en tant
que sous-groupe de $\gT_{1,1}^+=SL(2,\mathbf{Z})$, i.e. stable par l'action
extérieure de $\gS_3$ sur $SL(2,\mathbf{Z})$ -- je présume que oui,
puisque $\gS_3$ agit aussi extérieurement sur $\pi_{0,3}$ a priori\dots
$\gT_{1,1}^+/\pi_{0,3}$ serait une extension centrale intéressante
de $\gS_3$ par $\{\pm1\}$\dots}.
\vskip .3cm
{
Théorème. --- \it L'action extérieure naturelle de $\GG_\mathbf{Q}$ sur $\hat\gT_{1,1}^+$, et
même sur $\gT_{1,1}'^+=\gT_{1,1}^+/\{\pm1\}$, est fidèle.
}
\vskip .3cm
\noindent On utilise le fait qu'on a un homomorphisme extérieur injectif
$$\hat\pi_{0,3}\hookrightarrow \hat\gT_{1,1}'^+,$$
compatible avec l'action de $\GG$, en procédant comme pour la 
proposition du début (où $\hat\pi_{0,3}'$ jouait le rôle
de $\hat\gT_{1,1}'^+$).
\vskip .3cm
{
Corollaire. --- \it Les homomorphismes canoniques (surjectifs)
$$\GG_\mathbf{Q}\to \GG_{1,\nu} \ \ (\nu\ge 1),\leqno(43)$$
sont injectifs, donc des isomorphismes.
}
\vskip .2cm
Il suffit de le prouver pour $\GG_\mathbf{Q}\to\GG_{1,1}$ (puisque
$\GG_{1,1}$ est un quotient de $\GG_{1,\nu}$, $\nu\ge 1$\dots), or on
a un homorphisme canonique $\GG_{1,1}\to {\rm Autext}
\,\hat\gT_{1,1}^+$ et on peut considérer le composé,
$$\GG_\mathbf{Q}\to\GG_{1,1}\to {\rm Autext}\,\hat\gT_{1,1}^+
\to {\rm Autext}\,\gT_{1,1}'^+;$$
par le théorème précédent ce composé est injectif,
donc aussi $\GG_\mathbf{Q}\to \GG_{1,1}$, cqfd.
\vskip .2cm
\noindent {\bf Remarque.}  On a vraiment l'impression, avec la proposition du
début, et le résultat précédent baptisé ``théorème'', 
d'avoir démontré deux fois la même chose, et avec la même
démonstration encore!  Donc il est temps, après ce détour,
de s'assurer qu'il en est bien ainsi, i.e. que l'homorphisme
$\hat\pi_{0,3}\to\hat\gT_{1,1}'^+$ qu'on vient d'utiliser
s'identifie bel et bien à l'homomorphisme $\hat\pi_{0,3}\to
\hat\pi_{0,3}'$, moyennant un isomorphisme convenable (qui reste 
à décrire) commutant aux actions extérieures de $\GG$, qu'on se
proposait de construire (cf. (10) à (20)) -- et on l'a perdu en route,
en essayant de trouver $\pi_{0,3}'\buildrel\sim\over\to
\gT_{1,1}'^+$ par factorisation dans l'homomorphisme composé
$\pi_{0,3}\to \pi_{0,3}'\to\gT_{1,1}'^+$ [la première
flèche étant donnée par la surjection canonique] -- et on
s'est en un sens fourvoyé, car l'homomorphisme ``naturel''
$\pi_{0,3}\to \gT_{1,1}'^+$ sur lequel on est tombé n'était
{\it pas} du tout celui qu'on avait en vue: j'étais à côté de
mes pompes.  Donc il me faut revenir à la charge!














%%%%%%%%%%%%%%%%%%%%%%%%%%%%%%%%%%%%%%%%%%%%%%%%%%%%%%%%%%%%%%%
\chapter*{\S \space 37. --- THÉORIE DES MODULES DES COURBES ELLIPTIQUES VIA LEGENDRE (RIGIDIFICATION D'ÉCHELON 2)}\thispagestyle{empty}
\addcontentsline{toc}{section}{37. Théorie des modules des courbes elliptiques via Legendre (rigidification d'échelon 2)}
\label{sec:37}
\section*{}

Je me rends compte que j'ai pas mal compliqué des choses pourtant bien 
simples, au paragraphe précédent. D'abord un remords de topologie des
surfaces:
\vskip .3cm
{
Proposition\footnote{{\bf Variante}: $G$ opère sur $X$ avec sous-groupe
{\it invariant} $G$ satisfaisant les conditions ci-contre [i.e. ci-dessus]
on a alors, posant $Y=X/G$ avec 
opération de $G/G=H$ dessus: $\underline{\rm Rev}(X,G) \isom 
\underline{\rm Rev}\bigl((Y,\b d), H \bigr)$, qui donne
$\pi_1(X,G,x)\buildrel \sim \over \to \pi_1\bigl(
(Y,\underline d,H), y \bigr)$. Application: $\gS_{0,3} \isom 
\gT_{1,1}'( \isom  GL(2,{\mathbf{Z}})^+/\pm1)$ (où $\gS_{0,3}$ est une extension
de $\gT_{0,3}=\gS_3\times {\mathbf{Z}}/2{\mathbf{Z}}$ par $\pi_{0,3}$).}. --- \it 
Soit $X$ une surface topologique paracompacte, $G$ un groupe discret
opérant proprement sur $X$; on suppose que pour tout $x\in X$, le 
stabilisateur $G_x$ opère fidèlement sur un voisinage de $x$ (de sorte
que $G$ opère fidèlement -- en fait les deux doivent être équivalents)
en préservant l'orientation locale (donc $G_x$ est un groupe cyclique)
de sorte que $Y=X/G$ est une surface topologique paracompacte. Considérons 
l'ensemble $X^!=\{x\in X \mid G_x\ne (1)\}$ -- qui est une partie discrète de $X$
stable sous $G$ -- et l'image $Y^!\subset  Y$ de $X^!$ -- partie discrète
de $Y$ -- et la ``signature'' sur $(Y, Y^!)$, pour laquelle le coefficient
$d_y\, (y\in Y^!)$ est l'indice de ramification ord$(G_X)$ en les $x\in X$
au dessus de $y$. Pout tout $G$-revêtement $X'$ de $X$, considérons 
alors $X'/G=Y'$ comme espace au dessus de $Y$. Alors:

\noindent a) $Y'$ est une surface topologique, revêtement ramifié
de $Y$, subordonné à la signature $\underline d =\underline d(X/Y)$.

\noindent b) Le foncteur $X'\mapstochar\to Y'=X'/G$ est une 
équivalence de catégories:
$${\rm Rev}(X, G) \buildrel\sim\over\to {\rm Rev}_
{\underline d}(Y).\leqno(1)$$
}
\vskip .3cm
\noindent N.B. On peut expliciter un foncteur quasi-inverse, en associant à tout
revêtement ramifié $Y'$ de $Y$ compatible avec $\b d$, le $(X, G)$-
revêtement ``normalisé'' (en un sens topologique facile à expliciter)
de $X\times{}_Y Y'$. On a un énoncé analogue pour les schémas
localement noethériens réguliers de dimension 1, où le foncteur
quasi-inverse s'obtient en normalisant $X\times{}_Y Y'$.
\vskip .3cm
{
Corollaire. --- \it Soient $x\in X\setminus X'$, $y$ son image dans $Y$. On
suppose $Y$ connexe (i.e. que $G$ opère transitivement sur l'ensemble des
composantes connexes de $X$). Alors on a un isomorphisme canonique
$$\pi_1(X, G, x) \buildrel\sim\over\to\pi_1\bigl((Y,\b d), y\bigr)
\leqno(2)$$
d'où, pour $X$ connexe, une suite exacte canonique
$$1\to\pi_1(X,x) \to\pi_1\bigl((Y,\b d), y\bigr)
\to G\to 1. \leqno(3)$$
}
\vskip .3cm
N.B. On a un corollaire analogue dans le contexte schématique,
$X$ et $Y$ étant des schémas réguliers de dimension 1 -- ou, le cas
échéant, des schémas relatifs lisses de dimension relative 1 sur
un $S$, mais dans ce cas il faudrait (si j'en ai besoin) faire un peu
attention de préciser l'énoncé en forme raisonnable.

\noindent {\bf Exemple:} Prenons $X=\P^1(\mathbf{C})\setminus \{0, 1, \infty\} =
U_{0, 3}(\mathbf{C})$, $G= \gS_3$ opérant de fa\c con habituelle, donc (avec
les conventions du paragraphe précédent), identifiant $\P^1(\mathbf{C})/
(G=\gS_3)$ à $\P^1(\mathbf{C})$, avec:
$$ (0, 1, \infty)\mapstochar\to 0\ \ \ \ \ d_0=2$$
$$(2, -1, 1/2)\mapstochar\to 1\ \ \ \ \ d_1=2\leqno(4)$$
$$(j, \bar\jmath)\mapstochar\to \infty\ \ \ \ \ \ \ \ d_\infty = 3,
$$
on trouve $Y=\P^1(\mathbf{C})\setminus \{0\}$, et $\b d = 2\{1\}+3\{\infty\}$.
On trouve donc un isomorphisme canonique:
$$ \pi_1\bigl((U_{0, 3}(\mathbf{C}), \gS_3), x\bigr) \isom 
\pi_1\bigl((\P^1(\mathbf{C})\setminus\{0\}, \b d), y\bigr), \leqno(5)$$
où malheureusement on ne peut prendre $x=P=j$; il faut prendre
$x\in \P^1(\mathbf{C})\setminus \{0, 1, \infty, 2, -1, -1/2, j, \bar\jmath\}$ et
le plus commode sera de prendre $x$ tel que $y=f(x)$ soit le point base
typique $Q=y$ pour calculer le $\pi_1$ à ramification -- pour un tel 
$x$ on trouve donc un isomorphisme canonique
$$ \pi_1\bigl((U_{0, 3}(\mathbf{C}), \gS_3), x\bigr) \isom  \pi_{0,3}'
\leqno {(5 \ {\rm bis})}$$
et en choisissant une $\gS_3$-classe de chemins du point base 
initial $P=y$  sur $X=U_{0, 3}(\mathbf{C})$ vers $x$, d'où un isomorphisme
correspondant du premier membre de (5 bis) avec $\pi_{0, 3}$, on 
trouve un isomorphisme canonique
$$ \pi_1\bigl((U_{0, 3}(\mathbf{C}), \gS_3), P\bigr) \isom  \pi_{0,3}'
\leqno {(6)}$$
(qui change par automorphisme intérieur quand on modifie le choix 
d'une $\gS_3$-classe de chemins), d'où la suite exacte
$$1\to\pi_{0, 3}\to\pi_{0, 3}'\to\gS_3\to
1, \leqno(7)$$
qui n'est autre que (9 bis) du paragraphe précédent, mais interprétée
ici de fa\c con bien plus transparente comme suite exacte d'homotopie
pour $\gS_3$ opérant sur $U_{0, 3}(\mathbf{C})$. 
Et on trouve de même, dans le cadre arithmético-géométrique,
travaillant sur le schéma $U_{0, 3} = U_{0, 3, {\mathbf{Q}}}$ et son quotient
$U_{0,3}/\gS_3 \isom \P^1_{\mathbf{Q}}\setminus\{0\}$, avec la signature
$2\{1\}+3\{\infty\}$ dessus, le diagramme de suites exactes:
\[\begin{tikzcd}
	& 1 & 1 \\
	1 & {\hat{\pi}_{0, 3} = \hat{\gT}^+_{0, 4}} & {\hat{\pi}'_{0, 3}} & {\gS_3} & 1 \\
	1 & {E_{0, 3} = \cN^!_{0, 4}} & {E'_{0, 3}} & {\gS_3} & 1 \\
	& {\GG_{0, 4} = \GG_{\mathbf{Q}}} & {= \GG_{\mathbf{Q}}} \\
	& 1 & 1
	\arrow[from=1-2, to=2-2]
	\arrow[from=1-3, to=2-3]
	\arrow[from=2-1, to=2-2]
	\arrow[from=3-1, to=3-2]
	\arrow[from=3-2, to=3-3]
	\arrow[from=2-2, to=2-3]
	\arrow[from=2-3, to=2-4]
	\arrow[from=3-3, to=3-4]
	\arrow[from=3-2, to=4-2]
	\arrow[from=3-3, to=4-3]
	\arrow[from=4-3, to=5-3]
	\arrow[from=4-2, to=5-2]
	\arrow[from=3-4, to=3-5]
	\arrow[from=2-4, to=2-5]
	\arrow[shift left=1, shorten <=2pt, shorten >=2pt, no head, from=2-4, to=3-4]
	\arrow[shorten <=2pt, shorten >=2pt, no head, from=2-4, to=3-4]
	\arrow[from=2-3, to=3-3]
	\arrow[from=2-2, to=3-2]
\end{tikzcd}\leqno{(8)}\]

Nous voulons maintenant mettre en relation $U_{0,3}$, avec l'action
de $\gS_3$ dessus, avec la multiplicité modulaire $M_{1,1}$ pour les
courbes elliptiques (indices ${\mathbf{Q}}$ sous-entendus). C'est domage d'ailleurs
de travailler seulement sur ${\mathbf{Q}}$ - je vais travailler plutôt sur
${\mathbf{Z}}[{1\over 2}]$ - i.e. en caractéristique résiduelle différente de 2.

Au paragraphe précédent j'ai identifié, un peu vaseusement,
$U_{0,3}=M^!_{0,4}$ à $M_{1,1}[2]'$ (sous lequel $M_{1,1}[2]$
est une gerbe liée par ${\mathbf{Z}}/2$). Les ennuis techniques (plutôt les
complications conceptuelles) tiennent au fait que pour les courbes 
elliptiques (plus généralement pour les variétés abéliennes
de dimension quelconque), si $n$ est un entier $\ge 3$, la ``rigidification
de Jacobi d'échelon $n$'' est bel et bien une rigidification,
i.e. tout automorphisme d'une courbe elliptique relative $E$ qui induit
l'identité sur ${}_nE$ -- sous schéma noyau de $n\,  {\rm id}_E$ -- est
l'identité; mais il n'en est plus de même pour $n=2$, où $n$ est
l'identité ${\rm id}_E$ au-dessus d'une partie ouverte fermée
de $S$ et $-{\rm id}_E$ sur le complémentaire. Ceci suggère, pour une
meilleure compréhension, de coiffer $M_{1,1}[2]$ par un $M_{1,1}[n]$,
où $n$ est un multiple de 2; donc $M_{1,1}[n]$ sera un schéma 
ordinaire de type fini sur Spec$\,{\mathbf{Z}}$ (son image dans Spec$\,{\mathbf{Z}}$
sera l'ouvert Spec$\, {\mathbf{Z}}[{1 \over n}]$) et au-dessus de Spec$\, {\mathbf{Z}}[{1
\over n}]$, le topos modulaire $M_{1,1}$ se récupère comme
$$M_{1,1}  \isom \bigl( M_{1,1}[n], GL(2,{\mathbf{Z}}/n{\mathbf{Z}})=\Gamma_n \bigr)
\leqno(9)$$
au-dessus de Spec$\,{\mathbf{Z}}[{1\over n}]$.
\footnote{Attention, c'est bien $ GL(2,{\mathbf{Z}}/n{\mathbf{Z}})$ et {\it non}
$SL(2,{\mathbf{Z}}/n{\mathbf{Z}})$ qu'il faut prendre ici.}
Ici $M_{1,1}[n]$ est le schéma qui représente le foncteur sur 
(Sch):

%\vfill\eject
$S\mapstochar\to$ ensemble des courbes elliptiques relatives 
sur $S$

\ \ \ \ \ \ {\it munies} d'un isomorphisme ${({\mathbf{Z}}/n{\mathbf{Z}})}^2_S\to 
{}_nE$
\footnote{Ce foncteur est représentable si et seulement si
$n\ge 3$ (donc il ne l'est pas pour $n=1,2$).}

\noindent sur lequel le groupe $\Gamma_n =GL(2, {\mathbf{Z}}/n{\mathbf{Z}}) ={\rm Aut}\bigl(
({\mathbf{Z}}/n{\mathbf{Z}})^2\bigr)$ opère de fa\c con évidente. Le {\it schéma
modulaire ``grossier''} (par opposition à la multiplicité
ou topos modulaire) est décrit au-dessus de ${\mathbf{Z}}/n{\mathbf{Z}}$ par
$$ \widetilde {M_{1,1}} \isom  M_{1,1}[n]/\Gamma_n\leqno(10)$$
au-dessus de ${\mathbf{Z}}[{1\over n}]$.

\noindent N.B. Le schéma $\widetilde{M_{1,1}}$ sur Spec$\, {\mathbf{Z}}$ peut 
se décrire comme ``l'enveloppe représentable'' du foncteur
non-représentable

$S\mapstochar\to$ classes d'isomorphisme de courbes
elliptiques relatives sur S\dots

\noindent -- itou sur un schéma de base
(par exemple ${\mathbf{Q}}$ ou Spec$\, {\mathbf{Z}}[{1\over n}]$), quelconque\dots

Il faut faire attention qu'en tant que schéma sur ${\mathbf{Z}}[{1\over n}]$
(ou sur ${\mathbf{Q}}$, en passant à la fibre générique), $M_{1,1}[n]$ {\it n'est 
pas} relativement connexe (i.e. n'est pas à fibres géométriquement
connexes). En effet, un isomorphisme
$${({\mathbf{Z}}/n{\mathbf{Z}})}^2_S \isom  {}_nE$$
implique par passage à la seconde puissance extérieure, un isomorphisme 
$${({\mathbf{Z}}/n{\mathbf{Z}})}_S\buildrel\sim\over\to \bigwedge^2_{{\mathbf{Z}}/n{\mathbf{Z}}} {}_nE
 \isom  {\bf \mu}_n^{\otimes -1}(S)$$
d'où un isomorphisme
$${\mathbf{Z}}/n{\mathbf{Z}}  \isom  {\bf \mu}(S)$$
qui s'identifie (prenant l'image de 1 mod $n$) à une section de ${\bf \mu}_n^*
(S)$ où ${\bf \mu}^*_n$ est le $({\mathbf{Z}}/n{\mathbf{Z}})^*$-torseur relatif sur
Spec$\, {\mathbf{Z}}[{1\over n}]$ des ``racines primitives $n$-ièmes de 1''.
On a donc un morphisme canonique
$$M_{1,1}[n]\to{\bf \mu}_n^*\leqno(11)$$
et c'est ce morphisme qui est à fibres géométriques connexes
en caractéristique 0. Passant à la limite sur $n$ variable, on trouve
{\it sur ${\mathbf{Q}}$}
$$M_{1,1}[\infty]=\lim_\leftarrow M_{1,1}[n] \to
{\bf \mu}_\infty^* \isom  {\rm Spec}\, \Sigma \leqno(12)$$
où le dernier isomorphisme est canonique et $\Sigma\subset  \C$
est la sous-extension cyclotomique maximale de $\overline{\mathbf{Q}}_0$.

On récupère les $M_{1,1}[n']$, pour $n'|n$, $n'\ne 1,2$, à
partir de $M_{1,1}[n]$ avec l'action de $\Gamma_n$ dessus par
$$M_{1,1}[n'] \isom  M_{1,1}[n]\times_{\Gamma_n}\Gamma_{n'}
 \isom  M_{1,1}[n]/K_{n,n'}\leqno(13)$$
sur Spec$\,{\mathbf{Z}}[{1\over n}]$, où 
$$K_{n,n'}= {\rm Ker}\, (\Gamma_n\to \Gamma_{n'}).
\footnote{$K_{n,n'}$ opère {\it librement} sur le schéma si
$n'\vert n, n'\ne 1,2$ mais pas bien s\^ur si $n'=1$ ou 2.}\leqno(14)$$
Si on applique cependant cette formule dans les cas non licites
$n'$=1 ou 2, on trouve pour le cas $n'=1$, non $M_{1,1}$ mais
$\widetilde{M_{1,1}}$ (schéma modulaire grossier), et pour $n'=2$,
non $M_{1,1}[2]$ (qui n'est pas non plus un schéma), mais ce qu'on 
avait appelé au paragraphe précédent $M_{1,1}[2]'$.

On peut expliciter $M_{1,1}$ et $\widetilde{M_{1,1}}$ sur Spec$\, {\mathbf{Z}}$
tout entier en termes des $M_{1,1}[n]$ en les décrivant au-dessus des schémas
ouverts Spec$({\mathbf{Z}}[{1\over n}])$, Spec$({\mathbf{Z}}[{1\over n'}])$ qui recouvrent
Spec(${\mathbf{Z}}$) (i.e. pour ($n,n'$)=1) par (9) et (10) en y prenant
d'abord $n$ puis $n'$, et en ``recollant'' au-dessus de
${\rm Spec}({\mathbf{Z}}[{1\over {nn'}}])$, par ces mêmes formules appliquées
à $nn'$\dots

On a d'ailleurs (pour mémoire)
\vskip .3cm
{
Théorème. --- \it $\widetilde{M_{1,1}}  \isom  {\E}_{{\mathbf{Z}}}^1$
\ \ \ (sur Spec$\,{\mathbf{Z}}$).
}
\vskip .3cm
On peut épingler un tel isomorphisme (défini à priori modulo
le groupe affine entier $X\mapstochar\to\pm X +n, n\in {\mathbf{Z}}$),
en notant 
qu'il y a dans  $\widetilde{M_{1,1}}$ deux sections privilégiées
sur ${\mathbf{Z}}$, dont les valeurs au point générique correspondent aux deux
classes d'isomorphisme de courbes elliptiques en caractéristique 0
(sur $\mathbf{C}$ par exemple) qui ont des automorphismes différents de 
${\rm id}$, ${\rm -id}$ -- les deux groupes d'automorphimes qu'on 
obtient sont d'ailleurs ${\mathbf{Z}}/4{\mathbf{Z}}$ et ${\mathbf{Z}}/6{\mathbf{Z}}$ (très facile, 
par exemple par voie transcendante).  Ce sont ce qu'on appelle sauf 
erreur les courbes elliptiques ``anharmoniques''.  Sur un 

$M_{1,1}[n]$, le groupe des automorphismes rationnels sur $k$ d'une
courbe elliptique décrite par un point $x$ de $M_{1,1}[n](k)$ 
s'identifie canoniquement au stabilisateur de $x$ dans $\Gamma_n$:
$${\rm Aut}(E_x) \isom  (\Gamma_n)_x.\leqno(15)$$
Bien s\^ur on a des énoncés idoines sur un schéma de base 
quelconque, pas nécessairement un corps.  Cela montre déjà que 
les automorphismes des courbes elliptiques sont liés de fa\c con 
essentielle à la {\it ramification} de $\Gamma_n$ opérant sur 
$M_{1,1}[n]$. On trouve ainsi:
\vskip .3cm
{
Proposition (Sur Spec$\,{\mathbf{Q}}$, si $n\ne 1,2$). --- \it Il y 
a exactement deux orbites de $\Gamma_n$ opérant sur $M_{1,1}[n]$ 
qui sont ``critiques'', et qui correspondent à des stabilisateurs 
isomorphes respectivement à ${\mathbf{Z}}/4{\mathbf{Z}}$ et ${\mathbf{Z}}/6{\mathbf{Z}}$, qui sont
les groupes d'automorphismes des courbes
elliptiques (``anharmoniques'') correspondantes.
}
\vskip .2cm
On montre que $\Gamma_n$ {\it n'opère pas} fidèlement sur 
$M_{1,1}[n]$ -- le noyau de cette opération est le sous-groupe 
central $\{\pm1\}$, image du groupe correspondant de $GL(2, {\mathbf{Z}})$ -- 
il correspond au groupe des automorphismes ``universels'' 
$\pm {\rm id}_E$ des courbes elliptiques $E$.  Le groupe quotient
$$\Gamma_n' :=\Gamma_n/\{\pm 1\} \leqno(16)$$
opère fidèlement, et on peut regarder le ``topos mixte''
$(M_{1,1}[n],\Gamma_n')$ -- on trouve aussitôt qu'il ne {\it 
dépend pas} (à équivalence canonique près) {\it du choix de 
$n$} [à cela près que l'ouvert de Spec$\,{\mathbf{Z}}$ sur lequel ``il a 
un sens'' dépend de $n$\dots]
\footnote{donc à condition de se placer au dessus d'un schéma
de base tel Spec$\,({\mathbf{Z}}[{1\over {nn'}}])$ où $n, n'$ sont tous deux
inversibles\dots}
-- et en fait, on a:
$$(M_{1,1}[n],\Gamma_n') \isom  M_{1,1}'\leqno(17)$$
au-dessus de Spec$\,{\mathbf{Z}}[{1\over n}]$, où la multiplicité 
schématique modulaire $M_{1,1}'$ est décrite, en termes de 
courbes elliptiques ``définies modulo symétries'', comme
au paragraphe précédent.

\noindent {\bf Remarque:} La proposition précédente semble dépendre de $n$,
mais on voit a priori (gr\^ace au fait que les $\Gamma_{n,n'}, n'\vert n,
n'\ne 1,2$ opèrent librement) que si elle est valable pour {\it un} $n$,
elle est valable pour tous. 

Notons aussi
\vskip .2cm
{
Corollaire (de la Proposition). --- \it En caractéristique 0, les stabilisateurs
{\it dans $\Gamma_n'$} (le groupe qui opère fidèlement sur 
$(M_{1,1}[n]$) des points des deux orbites critiques
sont respectivement ${\mathbf{Z}}/4{\mathbf{Z}}$ et ${\mathbf{Z}}/6{\mathbf{Z}}$.\footnote{valable pour $n\ge 2$, cf. plus bas.}
}
\vskip .3cm
On a supposé ici que $n\ne 1,2$, mais rien ne nous empêche de 
regarder aussi
$$M_{1,1}[2]'\buildrel\rm \defeq =M_{1,1}[2m]/\Gamma_{2m,2}=
M_{1,1}[2m]/\Gamma_{2m,??} \leqno(18)$$
(où $m$ est un entier quelconque $\ge 2$). L'action de $\Gamma_2
 \isom  GL(2, {\mathbf{Z}}/2{\mathbf{Z}}) \isom  SL(2, {\mathbf{Z}}/2{\mathbf{Z}}) \isom  \gS_3 \isom  \Gamma_{2m}/
\Gamma_{2m,2}$. Il n'est plus vrai, certes, que $\Gamma_{2m,2}$ opère
librement sur $M_{1,1}[2m]$ -- car $\Gamma_{2m,2}$ contient l'élément $-1\in
\Gamma_{2m}$ correspondant à la symétrie universelle ${\rm -id}_E$ 
-- i.e. à l'élément $-1$ de $GL(2, {\mathbf{Z}})$.  Mais le résultat 
général de rigidité rappelé au début implique que
$$\Gamma_{2m,2}:= \Gamma_{2m,2}'/\pm1  \isom  {\rm Ker}\,(\Gamma_{2m}'
\to\Gamma_2' \isom \gS_3)$$
opère encore {\it librement} sur $M_{1,1}[2m]$ -- et cela implique
que le corollaire de la proposition est valable encore pour l'action
de $\Gamma_2'( \isom \Gamma_2 \isom \gS_3)$ sur $M_{1,1}[2]'$.
\vskip .2cm
\noindent {\bf Digression terminologique:} $M_{1,1}=M_{1,1}[1]$ et les 
$M_{1,1}[n]$, $n\in \N^*$, sont définies a priori comme des
{\it multiplicités} schématiques modulaire, i.e. a) comme des
topos localement annelés localement isomorphes au topos étale
d'un schéma (en l'occurence de type fini sur Spec$\,{\mathbf{Z}}$, ou sur ${\mathbf{Q}}$
si on travaille en caractéristique 0) et b) définis, à 
équivalence près définie elle-même à isomorphisme unique
près, comme ``représentant'' les foncteurs correspondants

$S\mapstochar\to$\ catégorie des courbes elliptiques 
relatives sur $S$, avec rigidification

\ \ \ \ \ \ \ d'échelon $n$, i.e. isomorphisme $({\mathbf{Z}}/n{\mathbf{Z}})_S
 \isom  {}_nE$

\ \ \ \ \ \ \ $= \Ell_n(S)$

\noindent donc par construction , on a pour tout schéma $S$ une 
équivalence:
$$ \Ell_n(S)\buildrel\sim\over\to \Hom(S,(M_{1,1}[n])$$
[où les homomorphismes sont des homomorphismes de topos localement
annelés, i.e. homomorphismes de multiplicités).]  Pour que les catégories
$Ell_n(S)$ soient {\it discrètes}, ou encore que $(M_{1,1}[n]$ soit un
schéma, ou encore que les automorphismes d'une courbe elliptique
respectant une rigidification de Jacobi d'échelon $n$ soient l'identité,
il faut et il suffit que $n\ge 3$, i.e. $n\ne 1,2$.  Donc $M_{1,1}[2]$
{\it n'est pas} un schéma, mais $M_{1,1}[2]'$ en est un, et
$M_{1,1}[2]\to M_{1,1}[2]'$ est un isomorphisme local (décrit
entièrement par une ${\mathbf{Z}}/2{\mathbf{Z}}$-gerbe sur $M_{1,1}[2]'$). Il faut quand
même prendre la peine de définir:
$$M_{1,1}[2]\buildrel\hbox{\sevenrm{``théor.''}}\over \isom 
(M_{1,1}[2m],\Gamma_{2m,2})\to M_{1,1}[2]'
\buildrel\hbox{\sevenrm{``défini''}}\over \isom 
 M_{1,1}[2m]/\Gamma_{2m,2}'\leqno(20)$$
 (on a aussi $M_{1,1}[2m]/\Gamma_{2m,2}'= M_{1,1}[2m]/\Gamma_{2m,2}$)
 comme l'homorphisme canonique de ``passage au quotient grossier''
 qui a un sens chaque fois qu'un groupe discret (disons) $G$ opère sur un
 schéma (disons) $X$, comme un morphisme
 $$(X, G)\to X/G=Y \leqno(21)$$
 dont le foncteur image inverse (faisceaux sur $Y=X/G\to$ $G$-faisceaux
 sur $X$) est évident.  Ici on a un sous-groupe invariant $z$ de $G$ qui
 opère trivialement sur $X$, tel que $G/z$ opère librement, ce qui 
 signifie que le morphisme (21) se factorise en
 $$(X,G)\to (X,G/z=:G')\to Y=X/G =X/G'
 \leqno(22)$$
 de sorte que le morphisme de multiplicité (21) s'identifie aussi,
 simplement, à:
 $$(M_{1,1}[2m],\Gamma_{2m,2})\to (M_{1,1}[2m],\Gamma_{2m,2}')
  \isom  M_{1,1}[2]'\leqno(23).$$
 Par contre, si on veut descendre jusqu'à $M_{1,1}[1]' =M_{1,1}'$,
 en passant au quotient (au sens ``grossier'') dans $M_{1,1}[n]$ par 
 $\Gamma_n'$, l'homorphisme
 $$(M_{1,1}[n], \Gamma_n')=M_{1,1}'\to M_{1,1}[n]/\Gamma_n' =\widetilde {M_{1,1}}\leqno(24)$$
 n'est plus un ``isomorphisme'', i.e. n'est plus une équivalence, car 
 $\Gamma_n'$ n'opère pas librement -- il a (même en caractéristique 0)
 de la ramification de degrés 2, 3. 
 [Si cependant en excluant les courbes elliptiques anharmoniques, \c ca
 marcherait -- ce n'est ``qu'au voisinage'' de ces courbes elliptiques
 que (24) n'est pas un isomorphisme\dots].
 \vskip .2cm
 En fait les raisonnements du paragraphe précédent établissent
 un isomorphisme (valable sur Spec${\mathbf{Z}}[{1\over 2}]$)
 \footnote{L'image de $M_{1,1}[2]'$ dans Spec$\,{\mathbf{Z}}$
 est Spec$\,{\mathbf{Z}}[{1\over 2}]$ tandis que celle de $M_{0,4}^! \isom U_{0,1}$
 est Spec$\,{\mathbf{Z}}$ tout entier.}
 $$M_{1,1}[2]' \isom  M_{0,4}^! \isom U_{0,3}\leqno(25)$$
 et cet isomorphisme est compatible avec les opérations de 
 $$\Gamma_2' \isom \Gamma_2=GL(2,{\mathbf{Z}}/2{\mathbf{Z}})=SL(2, {\mathbf{Z}}/2{\mathbf{Z}})\leqno(26)$$
 sur $M_{1,1}[2]'$ d'une part, de $\gS_3$ sur $U_{0,3}$ d'autre part,
 quand on considère l'isomorphisme 
 $$\Gamma_2=GL(2,\mathbf{F}_2)\buildrel\sim\over\to\gS_3\leqno(27)$$
 qui associe, à tout automorphisme de $\mathbf{F}_2^2$, son action sur les
 trois éléments non nuls $\underline\alpha=(1,0),
 \underline\beta=(0,1)$ et $\underline\gamma =(1,1)
 =\underline\alpha + \underline\beta$ (de sorte qu'on a d'ailleurs)
 $$\gamma=\alpha+\beta,\ \alpha=\beta+ \gamma, \ \beta=
 \gamma+\alpha.\leqno(28)$$
 Il faudrait expliciter cette compatibilité, en considérant l'action
 naturelle de $\gS_4$ sur $M_{0,4}^!$, $M_{0,n}^!$ étant la 
 multiplicité modulaire définie par 
 
 \noindent (29)\ \  ${\underline{{\rm Hom}}}_{\rm multipl}(S, M_{0,n}^!)
  \isom $ catégorie des courbes relatives sur $S$, localement 
 
\ \ \quad\quad\quad\quad\quad isomorphes à $\P^1_S$ et munies 
de $n$ sections mutuellement 
 
\ \ \quad\quad\quad\quad\quad disjointes numérotées $S_1,S_2,\ldots,S_n$.
 
 \noindent N.B. C'est représentable bel et bien par une multiplicité si et
 seulement si $n\ge 3$, i.e. si le champ des groupo\" \i des qu'on
 veut représentés est ``rigide'',
 \footnote{Si on veut une représentabilité pour $n=0,1,2$, 
 il ne suffit plus de travailler avec une topologie étale -- il faut 
 des topologies fppf par exemple.}
 et alors cette multiplicité est même
 un schéma, isomorphe d'ailleurs canoniquement au sous-schéma\hfill\break
 $\underline{{\rm Mon}}(I_{n-3},U_{0,3})\subset U_{0,3}^{I_{n-3}}$, où
 ici $I_{n-3} =\{i\in \N \mid 3\le i\le n\}$ -- ainsi $M_{0,3}^!
  \isom  {\rm Spec}\,{\mathbf{Z}}$ (schéma fini), et $M_{0,4}^! \isom U_{0,3}
 =\P^1_{\mathbf{Z}}\setminus\{{\rm sections}\ 0,1,\infty\}=
 {\rm Spec}\,{\mathbf{Z}}[T][1/\{T(T-1)\}]$.

 Mais on fera attention qu'il y a une opération naturelle de $\gS_n$ sur 
 $M_{0,n}^!$, donc sur $\underline{\rm Mon}(I_{n-3},U_{0,3})$ -- alors que
 l'on ne voit a priori que l'action de $\gS_3\times\gS_{n-3}$ sur ce dernier:
 l'isomorphisme canonique
 $$M_{0,n}^! \isom  {\rm Mon}((I_{n-3},U_{0,3})\subset  U_{0,3}^{I_{n-3}}
 \ \ (\hbox{où}\ I_{n-3} = \{i\in \N\mid 3<i\le n\})\leqno(30)$$
ne tient compte que des opérations naturelles du sous-groupe
$$\gS_3\times\gS_{n-3}\subset \gS_n\leqno(31)$$
de $\gS_n$ sur $M_{0,n}^!$, et pas de celle de $\gS_n$ tout entier.  Dans
le cas de $n=4$, l'opération de $\Gamma_2'=GL(2,\F_2) \isom \gS_3$ sur 
$M_{0,4}^!$ qui nous intéresse est celle qui correspond au sous-groupe
$\gS_1\times\gS_3$ qui fixe le premier élément (correspondant à la 
section nulle d'une courbe elliptique) et qui permute les trois 
suivants (correspondant aux trois éléments d'ordre 2 d'une courbe
elliptique générique en caractéristique différente de 2, qui
sont les trois autres points fixes de $x\mapstochar\to -x$), alors que
l'opération naïve de $\gS_3$ sur $U_{0,3}$ correspond au sous-groupe
$\gS_3\times\gS_1$ dans $\gS_4$, qui fixe les trois premiers éléments.
Petite question de gymnastique schématique (indépendante maintenant 
de la géométrie des courbes elliptiques, mais histoire de géométrie
projective de dimension 1): comment passer d'une de ces actions de $\gS_3$
à l'autre?  On a envie de prouver que ce sont les mêmes ! 

Notons que si $I$ est un ensemble à 4 éléments, alors il lui est
associé un ensemble à 3 éléments de ``partitions de type $(2,2)$'',
$P_{2,2}(I)=\tilde I$, et l'homomorphisme 
$$\gS_I \isom \gS_4\to\gS_{\tilde I} \isom \gS_3\leqno(32)$$
est surjectif -- son noyau est un sous-groupe isomorphe (non canoniquement) 
à $\F_2\times\F_2$ -- donc un groupe commutatif $V(I)$ annulé par 2, i.e.
un espace vectoriel sur $\F_2$,
et en tant que tel de dimension 2. En fait, $V(I)$
opèrant sur $I$ fait de $I$ un $V(I)$-torseur -- donc $I$ est muni
canoniquement d'une structure de plan affine sur $\F_2$, et $V(I)\subset 
\gS_I$ est le groupe de ses translations. $\gS_I$ s'interprète comme le
groupe des automorphismes affines de $I$ ({\it toute} permutation de $I$
est affine -- il y a sur  l'ensemble $I$ une
et une seule structure de plan affine sur $\F_2$),
et $\gS_{\tilde I}$ comme le groupe Aut$(V(I))$ - ce qui
suggère d'interpréter $\tilde I$ comme l'ensemble des trois éléments
non nuls de $V(I)$, et en fait, on constate que l'on a une bijection canonique
$$\tilde I\buildrel\sim\over\to V(I)^\ast \buildrel
{\rm def} \over = V(I)\setminus\{0\} \leqno(33)$$
en associant à une partition $I=I'Up I''$, avec $I', I''$ de cardinal 2,
la seule permutation de $I$ qui invarie $I',I''$ et induit sur chacun
une permutation non triviale (les éléments de $\gS_I$ obtenus ainsi
sont les {\it permutations paires d'ordre 2}, qui avec l'identité forment
donc le sous-groupe $V(I)$).

On trouve, si $i\in I$, d'où $I\setminus\{i\}$ de cardinal 3, une 
application canonique
$$I\setminus \{ i \} \to \tilde I\leqno(34)$$
en associant à $j\in I\setminus \{i\}$ la partition de $I$ en $\{i,j\}$
et en son complémentaire.  Cette bijection étant compatible avec le 
transport de structure, on en conclut que l'isomorphisme correspondant
$$\gS_{I\setminus\{i\}}  \isom \gS_{\tilde I}\leqno(35)$$
est aussi le composé
$$\gS_{I\setminus\{i\}} \hookrightarrow\gS_I
\to\gS_{\tilde I},\leqno(35)$$
($\gS_{I\setminus\{i\}}$ étant le stabilisateur de $i$ dans $\gS_I$)
donc les quatre sous-groupes symétriques d'indice 3
$\gS_{I\setminus\{i\}}$ de $\gS_I$ ($i \in I$) sont canoniquement
isomorphe à $\gS_{\tilde I}$ -- donc entre eux, par des isomorphismes
qui sont d'ailleurs déduits des bijections canoniques
$$I\setminus\{i\}\buildrel\sim\over \to I\setminus\{j\}\ \ \ 
(i\ne j, i, j\in I),\leqno(37)$$ égales à la {\it transposition}
(non à l'identité) sur $I\setminus\{i,j\}$.  Cette bijection
en effet est le composé
$$I\setminus\{i\}\buildrel\sim\over \to \tilde I
\buildrel\sim\over \to I\setminus\{j\},$$
et pour $i,j$ variables, ces isomorphismes forment un système 
transitif d'isomorphismes.
\vskip .2cm
Revenant à l'action de $\gS_4$ sur $M_{0,4}^!$, la clef de la 
compréhension est donnée par ceci:
\vskip .3cm
{
Proposition. --- \it Considérons le sous-groupe $V(I)$ de $\gS_I$
formé de l'identité et des involutions paires.  Alors l'action
de $V(I)$ sur $M_{0,I}^!$ 
\footnote{N.B. $M_{0,I}^!$ se définit en termes de $I$, pour tout
ensemble $I$ de cardinal $\ge 3$ comme $M_{0,4}^!$}
est triviale, donc $\gS_I$ agit sur $M_{0,I}^!$ par l'intermédiaire
de $\gS_I/V(I) = \gS_{\tilde I}$.
}
\vskip .3cm
{
Corollaire. --- \it Soit $i\in I$, alors l'action de 
$\gS_{I\setminus\{i\}}\subset \gS_I$ sur $M_{0,I}^!$ est déduite
de celle de $\gS_{\tilde I}$ via l'isomorphisme $\gS_{I\setminus\{i\}}
\buildrel\sim\over\to\gS_I$ provenant de $I\setminus\{i\}\buildrel
\sim\over \to\tilde I$.
}
\vskip .3cm
En particulier, pour $I=[1,2,3,4]$, pour comparer l'action de 
$\gS_{\{1,2,3\}}$  et de $\gS_{\{2,3,4\}}$ sur $M_{0,4}^! \isom U_{0,3}$ --
où la première est l'action standard de $\gS_3$ sur $U_{0,3}$, la
deuxième celle de $\gS_3$ sur $M{1,1}[2]'$, on utilise l'isomorphisme
$$\gS_{\{1,2,3\}} \isom \gS_{\{2,3,4\}}\leqno(38)$$
explicité dans (37), correspondant à la bijection
$$1\mapstochar\to4, 2\mapstochar\to3, 3\mapstochar\to2$$
des ensembles d'indices, qui correspond donc à l'automorphisme
$${\rm int}(\sigma_1):\gS_3\to\gS_3\leqno(39)$$
en identifiant maintenant les deux ensembles d'indices $[1,2,3],
[2,3,4]$ à $[0,1,\infty]$, et en désignant par $\sigma_i\in 
\gS_{\{0, 1,\infty\}}$ (pour $i\in \{0,1,\infty\})$ la transposition 
des éléments $\ne i$. Donc
\vskip .3cm
{
Corollaire. --- \it L'isomorphisme canonique
$$ U_{0,3}\buildrel\phi\over\to M_{1,1}[2]' \ \ \ ({\rm sur}
\ \ {\mathbf{Z}}[{1\over2}]),$$
quand on fait opérer $\gS_3 =\gS_{\{0,1,\infty\}}$ sur l'un et l'autre
membre, est compatible avec ces opérations, {\it via} l'automorphisme
int$(\sigma_1)$, i.e.
$$\phi(u \cdot x)=(\sigma_1 u \sigma_1^{-1}) \cdot \phi(x)\leqno(40)$$
}
\vskip .3cm
Il faut quand même démontrer la proposition 4, qui équivaut
bien s\^ur à ceci:
\vskip .3cm
{
Corollaire. --- \it Soit $X$ une droite projective relative sur
un schéma $S$, $I$ un ensemble à quatre éléments, $(S_i)_{i\in I}$
une famille de sections mutuellement disjointes, et $\sigma$ une involution
paire de $I$ (correspondant à une partition $I=\{i_1,i_2\}Up \{i_3,i_4\}$
par $\sigma i_1=i_2,\sigma i_2=i_1,\sigma i_3=i_4,\sigma i_4=i_3$).  Alors il
existe un automorphisme (évidemment unique) $u$ de $X$, tel que 
$u\circ s_i=s_{\sigma i}$.
}
\vskip.3cm
\noindent Démonstration: On peut supposer que $X=\P_S^1, s_1=0, 
s_2=\infty$; donc $s_3, s_4$
s'identifient à des sections de ${\cal O}_S^\ast$; on prendra $u$ 
défini par
$$u(z)={\lambda\over z}, \ \ \  \lambda\in \Gamma(S,{\cal O}_X^\ast)$$
qui échange 0 et $\infty$, et la condition pour échanger $s_3, s_4$
s'écrit ${\lambda\over s_3}=s_4$, ${\lambda\over s_4}=s_3$, i.e. 
$\lambda = s_3s_4$ (N.B. il suffirait que $(s_1,s_2,s_3)$ et $(s_1,s_2,s_4)$
soient mutuellement disjointes -- pas la peine que $s_3,s_4$ soient
mutuellement disjointes\dots)
\vskip .2cm
Pla\c cons nous à nouveau sur ${\mathbf{Q}}$ (pour fixer les idées), et considérons
les groupes fondamentaux des multiplicités modulaires $M_{1,1}, M_{1,1}'$.
On part de
$$\begin{cases}
M_{1,1} \isom (M_{1,1}[n],\Gamma_n)&$n \geq 3$ \\
      M_{1,1}' \isom (M_{1,1}[n],\Gamma_n'=\Gamma_n/{\pm 1})&$n \geq 2$
\end{cases}\leqno(41)$$
(où pour $n=2$ on remplace $M_{1,1}[2]$ par $M_{1,1}[2]'$),
qui donne des isomorphismes de groupes extérieurs
\[\begin{tikzcd}
	{\pi_1(M_{1, 1})} & {\isom \pi_1(M_{1, 1}[n], \Gamma_n)} & {(n \geq 3)} \\
	{\pi_1(M_{1, 1})} & {\isom \pi_1(M_{1, 1}[n], \Gamma'_n)} & {(n \geq 2)}
	\arrow[from=1-1, to=2-1]
	\arrow[from=1-2, to=2-2]
\end{tikzcd}\leqno{(42)}\]
(où comme au (41) si $n=2$), et comme $M_{1,1}[n]$ est connexe
(même s'il n'est pas géo\-mé\-trique\-ment connexe)
\footnote{c'est le théorème de Kronecker d'irréductibilité de 
l'équation cyclotomique.},
on trouve
\[\begin{tikzcd}
	1 & {\pi_1(M_{1, 1}[n])} & {\pi_1(M_{1, 1})} & {\Gamma_n} & 1 \\
	1 & {\pi_1(M_{1, 1}[n])} & {\pi_1(M'_{1, 1})} & {\Gamma'_n} & 1
	\arrow["\sim", from=1-2, to=2-2]
	\arrow[from=1-3, to=2-3]
	\arrow[from=1-4, to=2-4]
	\arrow[from=1-1, to=1-2]
	\arrow[from=2-1, to=2-2]
	\arrow[from=2-2, to=2-3]
	\arrow[from=1-2, to=1-3]
	\arrow[from=1-3, to=1-4]
	\arrow[from=2-3, to=2-4]
	\arrow[from=1-4, to=1-5]
	\arrow[from=2-4, to=2-5]
\end{tikzcd}\leqno{(43)}\]
qui permet de reconstruire $\pi_1(M_{1,1})$, en tant qu'extension de $\Gamma_n$,
quand on conna\^\i t $\pi_1(M_{1,1}')$ comme extension de $\Gamma_n'$, par
image inverse via $\Gamma_n\to \Gamma_n'$. On trouve comme de juste
\vskip .3cm
{
Proposition. --- \it $\pi_1(M_{1,1})$ est une extension centrale
de $\pi_1(M_{1,1}')$ par $\{\pm 1\}$.
}
\vskip .3cm
D'autre part, on a les suites exactes d'homotopie sur Spec$\,{\mathbf{Q}}$
\[\begin{tikzcd}
	1 & {\pi_1(\overline{M_{1, 1}})} & {\pi_1(M_{1, 1})} & {\GG_n} & 1 \\
	1 & {\pi_1(\overline{M'_{1, 1}})} & {\pi_1(M'_{1, 1})} & {\GG_n} & 1
	\arrow["\sim", from=1-2, to=2-2]
	\arrow[from=1-3, to=2-3]
	\arrow[from=1-4, to=2-4]
	\arrow[from=1-1, to=1-2]
	\arrow[from=2-1, to=2-2]
	\arrow[from=2-2, to=2-3]
	\arrow[from=1-2, to=1-3]
	\arrow[from=1-3, to=1-4]
	\arrow[from=2-3, to=2-4]
	\arrow[from=1-4, to=1-5]
	\arrow[from=2-4, to=2-5]
\end{tikzcd}\leqno{(44)}\]
(avec les isomorphismes $\pi_1(\overline{M_{1,1}}) \isom \gT_{1,1}^+$,
$\pi_1(M_{1,1}) \isom \gT_{1,1} \isom \cN_{1,1}$)
et la théorie transcendante fournit des isomorphismes extérieurs 
canoniques
\footnote{Attention; ne pas confondre $SL(2,{\mathbf{Z}})\hat{}$ avec
$SL(2,\hat{\mathbf{Z}})$ !!!}
$$\pi_1(\overline{M_{1,1}}) \isom  \Sl(2,{\mathbf{Z}})\hat{} , \  
\pi_1(\overline{M_{1,1}'}) \isom  \Sl(2,{\mathbf{Z}})'\hat{}=SL(2,{\mathbf{Z}})\hat{}/
{\pm 1},\leqno(45)$$
On a la compatibilité essentielle suivante entre (43), (44), (45):
la commutativité de
\[\begin{tikzcd}
	1 & {\pi_1(\overline{M_{1, 1}}) \isom \Sl(2, \mathbf{Z})~\hat{}} & {\pi_1(M_{1, 1})} & {\GG_n} & 1 \\
	1 & {\Sl(2, \mathbf{Z}/n\mathbf{Z})} & {\Gamma_n = \Gl(2, \mathbf{Z}/n\mathbf{Z})} & {(\mathbf{Z}/n\mathbf{Z})^*} & 1
	\arrow["{\text{surj}}", from=1-2, to=2-2]
	\arrow[from=1-3, to=2-3]
	\arrow["{Hi_n}", from=1-4, to=2-4]
	\arrow[from=1-1, to=1-2]
	\arrow[from=2-1, to=2-2]
	\arrow[hook, from=2-2, to=2-3]
	\arrow[from=1-2, to=1-3]
	\arrow[from=1-3, to=1-4]
	\arrow["{\text{det}}", from=2-3, to=2-4]
	\arrow[from=1-4, to=1-5]
	\arrow[from=2-4, to=2-5]
\end{tikzcd}\leqno{(46)}\]
(où $Hi_n$ est le caractère cyclotomique).

Passant à la limite sur $n$, ceci donne un diagramme commutatif
\[\begin{tikzcd}
	1 & {\pi_1(\overline{M_{1, 1}}) \isom \Sl(2, \mathbf{Z})~\hat{}} & {\pi_1(M_{1, 1})} & {\GG_n} & 1 \\
	1 & {\Sl(2, \hat{\mathbf{Z}})} & {\Gl(2, \hat{\mathbf{Z}})} & {(\hat{\mathbf{Z}})^*} & 1
	\arrow["{\text{surj}}", from=1-2, to=2-2]
	\arrow[from=1-3, to=2-3]
	\arrow["Hi", from=1-4, to=2-4]
	\arrow[from=1-1, to=1-2]
	\arrow[from=2-1, to=2-2]
	\arrow[hook, from=2-2, to=2-3]
	\arrow[from=1-2, to=1-3]
	\arrow[from=1-3, to=1-4]
	\arrow[from=2-3, to=2-4]
	\arrow[from=1-4, to=1-5]
	\arrow[from=2-4, to=2-5]
\end{tikzcd}\leqno{(47)}\]
diagramme sur lequel nous allons revenir.
\vskip .2cm
Pour l'instant je vais exploiter le deuxième isomorphisme (42) pour $n=2$:
$$\pi_1(M_{1,1}')=\cN_{1,1}' \isom \pi_1(M_{1,1}[2]',\Gamma_2=\gS_3)
 \isom \pi_1(U_{0,3},\gS_3),\leqno(48)$$
où le dernier isomorphisme correspond à l'isomorphisme ${\rm int}(\sigma_1):
\gS_3\buildrel\sim\over \to\gS_3$, explicité dans (40).  On trouve
donc bien, comme prévu aux paragraphe précédent -- et de fa\c con
entièrement conceptuelle, l'isomorphisme d'extension
\[\begin{tikzcd}
	1 & {\pi_1(\overline{M'_{1, 1}})} & {\pi_1(M_{1, 1})} & {\GG_n} & 1 \\
	1 & {\pi_1(\overline{U_{0, 3}}, \gS_3)} & {\pi_1(U_{0, 3}, \gS_3)} & {\GG_n} & 1
	\arrow["\sim", from=1-2, to=2-2]
	\arrow["\sim", from=1-3, to=2-3]
	\arrow[shift left=2, shorten <=2pt, shorten >=2pt, no head, from=1-4, to=2-4]
	\arrow[from=1-1, to=1-2]
	\arrow[from=2-1, to=2-2]
	\arrow[from=2-2, to=2-3]
	\arrow[from=1-2, to=1-3]
	\arrow[from=1-3, to=1-4]
	\arrow[from=2-3, to=2-4]
	\arrow[from=1-4, to=1-5]
	\arrow[from=2-4, to=2-5]
	\arrow[shorten <=2pt, shorten >=2pt, no head, from=1-4, to=2-4]
\end{tikzcd}\leqno{(49)}\]
où l'on a $\pi_1(\overline{M_{1,1}'})=\hat\gT_{1,1}'^{+}
 \isom  SL(2,{\mathbf{Z}})\hat{}$, $\pi_1(M_{1,1}') \isom \cN_{1,1}'$,
$\pi_1(\overline{U_{0,3}},\gS_3) \isom \hat\pi_{0,3}'$ et
$\pi_1(U_{0,3},\gS_3)$ $ \isom E_{0,3}'$. D'ailleurs, reprenant 
ces réflexions dans le contexte transcendant, on voit que cet 
isomorphisme
$$\hat\pi_{0,3}/\{l_1^2,l_\infty^3\}=\hat\pi_{0,3}'\buildrel\sim\over
\to\pi_1(\overline{M_{1,1}'})=SL(2,{\mathbf{Z}})'\hat{}=SL(2,{\mathbf{Z}})
\hat{}/\pm 1 \leqno(50)$$
est associé à un isomorphisme
$$ \pi_{0,3}/\{l_1^2,l_\infty^3\}=\pi_{0,3}'\buildrel\sim\over
\to SL(2,{\mathbf{Z}})'=SL(2,{\mathbf{Z}})/\pm 1 \leqno(51)$$
déduit de l'isomorphisme de multiplicités analytiques complexes
$$M_{1,1}'^{\rm an} \isom (U_{0,3}^{\rm an},\gS_3).\leqno(52)$$
Il faudrait quand même expliciter l'homomorphisme (51) -- qui n'est défini
que modulo automorphisme intérieur, a priori, par des formules explicites --
alors que sa définition ici sort de fa\c con purement conceptuelle, 
``géométrique'' (au sens de la géométrie algébrique relative,
des courbes rationnelles et elliptiques sur des espaces analytiques 
arbitraires).  C'est là un calcul clef, s\^urement instructif, qui ne
devrait pas présenter de difficulté particulière\dots
\vskip .2cm
J'ai un peu laissé tomber en chemin dans tout \c ca le schéma modulaire
``grossier'' $\widetilde{M_{1,1}}$, après avoir affirmé qu'il est
isomorphe à ${\E}_{\mathbf{Z}}^1$ et qu'il y a deux sections marquées, dont les 
les valeurs en caractéristique 0 correspondent aux deux classes 
d'isomorphisme de courbes elliptiques anharmoniques.  En termes de 
l'isomorphisme 
$$\widetilde{M_{1,1}} \isom  M_{1,1}[2]'/\Gamma_2' \isom U_{0,3}/\gS_3
\leqno(53)$$
(valable sur Spec$\,{\mathbf{Z}}[{1\over 2}]$), on trouve bien, au dessus de
$S={\rm Spec}\,{\mathbf{Z}}[{1\over 2}]$, que $\widetilde{M_{1,1}}$ se déduit
d'un schéma relatif $Y$ qui est localement (au sens étale) isomorphe
à $\P_S^1$, en enlevant une section $s$, dont l'existence implique 
déjà que $Y$ est globalement isomorphe à $\P_S^1$ -- donc
$Y\setminus s(S)$ isomorphe à ${\E}_S^1$.  Dans cette approche, on a  donc envie
de désigner par $\infty$ (non par 0, comme en théorie des cartes !) cette
section, qui correspond aux ``points à l'infini'' $(0,1,\infty)$ de
$U_{0,3}$, ou encore au ``point à l'infini'' de $\widetilde{M_{1,1}}$.
Il est d'ailleurs facile de vérifier a priori (par la compactification
de Deligne-Mumford de $M_{1,1}$ et de $\widetilde{M_{1,1}}$) que
$\widetilde{M_{1,1}}$ sur Spec$\,{\mathbf{Z}}$ tout entier est de la forme
$Y\setminus{\rm Im}\,s$, où $Y$ est lisse et propre sur Spec$\,{\mathbf{Z}}$ tout
entier, et $s$ une section -- dès lors ce qu'on conna\^\i t par exemple
sur la fibre géométrique, et le fait que ${\mathbf{Z}}$ est principal, impliquent
que $Y \isom \P_{\mathbf{Z}}^1$, et qu'on a donc $\widetilde{M_{1,1}} \isom {\E}_{\mathbf{Z}}^1$.
Pour choisir cet isomorphisme, on utilise les deux sections de ${\E}_{\mathbf{Z}}^1$,
correspondant aux courbes anharmoniques.  Ces deux sections {\it ne sont
pas disjointes}, elles se rencontrent en caratéristique 2 et en 
caractéristique 3 (pour ce qui se passe en caractéristique $\ne 2$,
i.e. sur Spec$\,{\mathbf{Z}}[{1\over 2}]$, on le voit bien par l'isomorphisme (53) -- 
car en caractéristique 3 les deux orbites $(2,-1,{1\over 2})$ et
$(j,\bar\jmath)$ coïncident et se collapsent en un seul et même point).
 Mais prenant l'une de ces sections comme section nulle (sauf erreur c'est
 celle qui correspond à la courbe elliptique la plus riche en symétries,
 à savoir le groupe ${\mathbf{Z}}/6{\mathbf{Z}}$, qu'on prend -- c'est une question de convention
 bien s\^ur -- c'est donc aussi l'orbite $\{j,\bar\jmath\}$, correspondant
 à un stabilisateur isomorphe à ${\mathbf{Z}}/3{\mathbf{Z}}$), cela détermine l'isomorphisme
 $\widetilde{M_{1,1}} \isom {\E}_{\mathbf{Z}}^1$ au signe près.
 L'autre section devient alors un entier de la forme $\pm 2^a 3^b$ ($a, b
 \in \N^\ast$ -- entiers bien déterminés dont je ne sais pas la valeur
 par coeur), et on achève de normaliser en exigeant que le signe soit
 {\it plus}. Donc
 \vskip .3cm
{
Théorème (pour mémoire -- c'est bien connu). --- \it Il y a un isomorphisme unique
 $$\widetilde{M_{1,1}} \isom {\E}_{\mathbf{Z}}^1\leqno(54)$$
 qui en caractéristique 0 donne à la courbe anharmonique de groupe
 d'automorphisme ${\mathbf{Z}}/6{\mathbf{Z}}$ l'invariant 0, et à celle de groupe ${\mathbf{Z}}/4{\mathbf{Z}}$
 un invariant $>0$ (qui sera nécessairement $2^a 3^b$, avec $a,b\in \N^\ast$
 bien déterminés, mais par moi oubliés\dots$b=3$, cf. ci-dessous).
 }
 \vskip .3cm
 En termes de l'isomorphisme
 $$\widetilde{M_{1,1}} \isom U_{0,3}/\gS_3\subset \P^1/\gS_3$$
 valable sur Spec$\,{\mathbf{Z}}[{1\over 2}]$,
 ceci correspond à un isomorphisme entre $\P^1/\gS_3$ et $\P^1$ qui
 à l'orbite $(0,1,\infty)$ associe $\infty$ (non 0), et à l'orbite
 $(j, \bar\jmath)$ associe 0 (non $\infty$), à l'orbite $(2,-1,{1\over 2})$
 le fameux $2^a3^b$ (et non 1).  La fonction (invariant modulaire)
 $$J(\lambda)\in {\mathbf{Q}}(\lambda)\leqno(55)$$
qui réalise cet isomorphisme, i.e. le morphisme
$$\P^1_{\mathbf{Z}} \xlongrightarrow{J} \P^1_{\mathbf{Z}}\leqno(56)$$
correspondant (de degré 6, avec $J\circ u =J, u\in\gS_3$)
est donc lié à celle inspirée de la théorie des cartes,
soit $f(\lambda)$, par 
$$J(\lambda)=2^a3^b f(\lambda)^{-1}\leqno(57).$$
D'ailleurs, $f(\lambda)$ (ou $J(\lambda)$) se calcule aisément par la
connaissance de ses zéros et de ses pôles, avec leurs multiplicités,
et la ``normalisation'' pour la valeur de $f$ (ou de $J$) sur une des
sections $2, -1, {1\over 2}$; on trouve immédiatement
$$f(z)= {3^3\over 2^2}{z^2(z-1)^2 \over (z^2-z+1)^3},\leqno(58)$$
donc,
$$J(z)=2^{a+2}3^{b-3}{(z^2-z+1)^3 \over z^2(z-1)^2 }.\leqno(59)$$
Comme $J(\lambda)$ doit garder un sens en caractéristique 3, et ne 
peut pas être constante, ceci montre d'ailleurs que $b=3$, donc (59)
s'écrit aussi
$$J(z)=2^{a+2}{(z^2-z+1)^3 \over z^2(z-1)^2 },$$
mais on se rappellera que cette formule n'est pertinente -- ne permet 
de calculer l'invariant d'une courbe elliptique -- que si la caractéristique
est différente de 2, heureusement!  Pour décrire
$$M_{1,1}\longrightarrow {\E}_{\mathbf{Z}}^1\leqno(60)$$
(et en particulier pour déterminer l'autre invariant modulaire critique
$2^a3^3$, i.e. pour déterminer $a$) aussi au voisinage de 2 -- disons
en caractéristique différente de 3 -- il faut une étude des courbes
elliptiques qui ne passe plus par la représentation (de Legendre, sauf
erreur) $y^2=\sqrt{x(x-1)(x-\lambda)}$, ou encore, par la rigidification de
Jacobi d'échelon 2, mais par celle d'échelon 3.

\noindent{\bf Remarques:}
\noindent on voit tout de suite que le quotient $\P^1_{\mathbf{Z}}/\gS_3$ s'identifie
à $\P^1_{\mathbf{Z}}$ de fa\c con unique en prenant comme images des orbites 
$(0,1,\infty)$ et $(j,\bar\jmath)$ les sections $0$ et $\infty$ (par 
exemple, en suivant la convention qui correspond à la théorie des 
cartes), et en prenant comme image de l'orbite $(2,-1,{1\over 2})$ un
nombre rationnel qui soit $>0$.

Ceci est possible a priori, car on note que les orbites
$(0,1,\infty)$ et $(j,\bar\jmath)$ ne coïncident en aucune caractéristique, donc 
correspondent à des sections disjointes de $Y=\P^1_{\mathbf{Z}}/\gS_3$, tandis
que l'orbite constante $(2,-1,{1\over 2})$ coïncide avec la première
en caractéristique 2, avec la deuxième en caractéristique 3 (et
ce sont là les deux seules coïncidences qui peuvent arriver). 

On voit donc a priori que la troisième section sera de la forme
$2^\alpha/3^\beta$, avec $\alpha, \beta \in \N^\ast$. Le calcul explicite
est d'ailleurs évident; l'homomorphisme composé
$$\P^1_{\mathbf{Z}}\buildrel f_1\over\longrightarrow\P^1_{\mathbf{Z}}/\gS_3 \isom \P^1_{\mathbf{Z}}
\leqno(61)$$
doit être de la forme $f_1=c\, f$, avec la constante $c\in {\mathbf{Q}}$ choisie
de telle fa\c con que $cf$ se réduise bien en toute caractéristique
(ce qui détermine $c$ modulo le signe), et que $f_1(-1)=cf(-1)=c>0$
(ce qui lève l'indétermination du signe). On trouve alors 
$$f_1(\lambda)={\lambda^2(\lambda-1)^2 \over (\lambda^2-\lambda+1)^2}
= {2^2 \over 3^3} f(\lambda),\leqno(62)$$
donc 
$$f_1(-1)={2^2 \over 3^3} \ \ \ ({\rm donc} \ \alpha=2, \beta=3).\leqno(63)$$
Mais on fera attention qu'au voisinage de la caractéristique 2, ces calculs
ne se rapportent plus à $\widetilde{M_{1,1}}$ et $\widetilde{\hat{M}_{1,1}}$,
où la configuration des trois sections n'est pas la même qu'ici\dots
\vskip .2cm
Sur la lancée de ces réflexions, il serait naturel de regarder de
plus près le schéma
$$M_{1,1}[4] \ \hbox{sur lequel agit}\ \Gamma_4=GL(2, {\mathbf{Z}}/4{\mathbf{Z}}) \ 
\hbox{via}\ \Gamma_4'=\Gamma_4/\pm1 \leqno(64)$$
et l'homomorphisme
$$M_{1,1}[4]\longrightarrow M_{1,1}[2]' \isom U_{0,3}\leqno(65)$$
compatible avec 
$$\Gamma_4'=GL(2,{\mathbf{Z}}/4{\mathbf{Z}})/\pm1\to \Gamma_2'=\Gamma_2
 \isom  GL(2,{\mathbf{Z}}/2{\mathbf{Z}}) \isom \gS_3.
\footnote{En fait le noyau de l'homomorphisme 
$GL(k,A)\to GL(k,A/J)$ est toujours un groupe commutatif --
et même un $A/J$-module isomorphe à $(A/J)^4$ -- si $J$ est
un idéal de carré nul d'un anneau $A$.  Donc ici
${\rm Ker}\bigl( GL(2,{\mathbf{Z}}/4{\mathbf{Z}})\to GL(2,{\mathbf{Z}}/2{\mathbf{Z}})\bigr)$ est un groupe
{\it commutatif}, et même un espace vectoriel sur $\mathbf{F}_2$; il est isomorphe
à $\F_2^4$.}
$$
Un calcul immédiat, compte tenu que $({\mathbf{Z}}/4{\mathbf{Z}})^\ast \isom  \{\pm1\}$
donne 
$${\rm Card}\, \Gamma_4 =2 {\rm Card}\, SL(2,{\mathbf{Z}}/4{\mathbf{Z}}) =
8{\rm Card}\, SL(2,{\mathbf{Z}}/2{\mathbf{Z}})= 16 {\rm Card}\,\Gamma_2= 16\times 6.$$ 
Donc
$${\rm Card}\bigl( \Gamma_{4,2}'={\rm Ker}(\Gamma_4'\to\Gamma_2')
\bigr)=8\leqno(66)$$
i.e. $M_{1,1}[4]$ est un revêtement galoisien de $M_{1,1}[2]'$
d'ordre 8.  Notons qu'il n'est pas géométriquement connexe sur ${\mathbf{Q}}$,
ou sur Spec$\,{\mathbf{Z}}[{1\over 2}]$, puisque $M_{1,1}[4]$ se trouve sur l'extension
quadratique ${\bf \mu}_4^\ast$, de Spec$\,\Lambda$ où $\Lambda=
{\mathbf{Z}}[{1\over 2}]$, qui n'est autre que $\Lambda(i)$.  On a une factorisation
de $M_{1,1}[4]\to M_{1,1}[2]'$:
\[\begin{tikzcd}
	{M_{1, 1}[4]} & {(U_{0, 3})_S} & {M_{1, 1}[2]' \isom U_{0, 3}} \\
	& {S^1 = \mu^*_4} & {\Spec \mathbf{Z}[\frac{1}{2}] = S}
	\arrow["2", from=1-2, to=1-3]
	\arrow["2", from=2-2, to=2-3]
	\arrow[from=1-2, to=2-2]
	\arrow[from=1-1, to=2-2]
	\arrow[from=1-1, to=1-2]
	\arrow[from=1-3, to=2-3]
\end{tikzcd}\leqno{(67)}\]
[où les flèches horizontales sont galoisiennes du degré 
indiqué et] où maintenant
$$M_{1,1}[4]\longrightarrow {M_{1,1}[2]'}_S' = (U_{0,3})_S'$$
est un revêtement galoisien dont le groupe est le noyau de
$\Gamma_{4,2}'\buildrel {\rm det}\over\to \{\pm 1\}$, ou
encore celui de $SL(2,{\mathbf{Z}}/4{\mathbf{Z}})'\to SL(2,{\mathbf{Z}}/2{\mathbf{Z}})$, que nous allons
désigner par $S\Gamma_{4,2}'$, qui est d'ordre 4.
\vskip .3cm
{
Proposition. --- \it Le groupe
$$S\Gamma_{4,2}={\rm Ker}\bigl(SL(2,{\mathbf{Z}}/4{\mathbf{Z}})\to SL(2,{\mathbf{Z}}/2{\mathbf{Z}})
\leqno(68)$$
est isomorphe à $TL(2, \mathbf{F}_2)$ (groupe des matrices de trace nulle
à coefficients dans $\F_2$), avec l'opération évidente de
$SL(2, \mathbf{F}_2)$ dessus, et le sous-groupe $\{\pm 1\}$ correspond aux
matrices scalaires -- le quotient $S\Gamma_{4,2}'$ des deux est isomorphe
à $\mathbf{F}_2^2$ (de fa\c con idiote et pas canonique du tout!).  Regardant
$M_{1,1}[4]$ comme un revêtement galoisien de ${M_{1,1}'[2]}_{S'}=
{(U_{0,3})}_{S'}$, on trouve {\it géométriquement} ``le''
revêtement {\it abélien} universel de $U_{0,3}$ de groupe
annulé par 2 (et pour cause), qui a comme groupe de Galois
$(\mathbf{F}_2)^{\{0,1,\infty\}}/$(diagonale). 
\footnote{NB Ce sont les 4 supplémentaire de ce sous-espace
$\F_2\hookleftarrow TL(2,\mathbf{F}_2)$ qui forment un torseur
sur le dual de $V= TL(2,\mathbf{F}_2)$, qui devraient correspondre
aux quatre relèvements en $\pi_{0,3}\hookrightarrow
\gT_{1,1}'^+=SL(2,{\mathbf{Z}})$; cf. plus haut sur ces questions.}
}
\vskip .3cm
Enfin, tout est essentiellement tautologique, mais un calcul amusant 
à faire sera de redécrire l'action de $SL(2,\F_2)$ sur $\cT L(2,\mathbf{F}_2)$/
(matrices scalaires), de fa\c con à l'identifier à l'action de 
$\gS_3$ sur $\mathbf{F}_2^{\{0,1,\infty\}}$/(diagonale). Comme revêtement de
$$\widetilde{{M_{1,1}}_{S'}} \isom  M_{1,1}[4]/S\Gamma_4'  \isom 
{M_{1,1}[2]'}_{S'}/\Gamma_2'\leqno(69)$$
$M_{1,1}[4]$ est un revêtement galoisien d'ordre 24 (=$4\cdot6$)
de groupe $SL(2,{\mathbf{Z}}/4{\mathbf{Z}})'$.  Ce serait bien qu'on n'ait pas un
isomorphisme
$$SL(2,{\mathbf{Z}}/4{\mathbf{Z}})' \isom  \gS_4 \leqno(70)$$
(où $\gS_4$ est le groupe des automorphismes de l'octaèdre), et
que ${M_{1,1}}_{S'}$ ne soit le revêtement de $\P^1_{S'}\setminus
\{0,1,\infty\}$, avec ramification compatible avec $2\{1\}+3\{\infty\}$, 
qui correspond à {\it l'octaèdre} dans la théorie des cartes --
on devrait avoir sur $S'={\rm Spec}\,{\mathbf{Z}}[{1\over 2}][i]$ un isomorphisme
canonique
$$ {M_{1,1}[4]}_{S'} \isom  \ {\rm courbe\ de\ Fermat}\ x^r+y^r+z^r=0
\ ({\rm sur} \ S')
\footnote{Il vaudrait mieux écrire l'équation de Fermat ici
$x_0^r+x_1^r+x_\infty^r=0$.}
\leqno(71)$$
dont le groupe d'automorphismes est justement une extension de $\gS_3$ 
par 
$${\bf \mu_2}^{\{0,1,\infty\}}/({\rm diagonale}) !$$
\vskip .2cm
Pour bien faire, il faudrait expliciter la relation entre courbes
elliptiques $E$ (sur un schéma $S$ au dessus de ${\mathbf{Z}}[{1\over 2}][i]$)
munies d'une rigidification de Jacobi d'échelon 4, $e_1,e_2$ avec
$e_1\wedge e_2=[i]^{-1}$, et solutions ``non triviales'' de l'équation
$$x^r+y^r+z^r=o\leqno(72)$$
sur $S$, i.e. les systèmes de sections $x,y,z\in \Gamma(S,O_S^\ast)$
satisfaisant (72), modulo multiplication par un ``scalaire'' $\alpha\in
\Gamma(S,O_S\ast)$ -- ou encore, rompant la symétrie en posant
$X=x/z, Y=y/z$, les solutions de l'équation
$$X^r+Y^r=-1 \ (=i^2), \ X, Y \in \Gamma(S,O_S^\ast)\dots\leqno(73)$$
On a ainsi interprété $M_{1,1}[2]'$ et (modulo des vérifications
et une étude un peu plus poussée) $M_{1,1}[4]$, en relation avec
deux cartes triangulaires pondérées régulières -- la carte
``diédrale'' ou carte triangulaire pondérée universelle
(trois sommets $0,1,\infty$, trois arêtes, deux faces qui sont des
triangles, type $(p,q)=(3,3)$), et la carte {\it octogonale}, qui est
un revêtement d'ordre 4 de celle-ci (étale en dehors de $\{0,1
\infty\}$).  

Il serait bien aussi de regarder de plus près le variétés
modulaires congruentielles $M_{1,1}[3]$ et $M_{1,1}[5]$, correspondant
à des groupes modulaires ``géométriques'':

$$
\begin{cases}
S\Gamma_3'=SL(2,\mathbf{F}_3)/\pm 1 \ ({\rm d'ordre}\ 12)
\subset  GP(1,\mathbf{F}_3)\ ({\rm d'ordre}\ 24) \  \isom \gS_4& \\
\ \ \ \ {\hbox{en fait on doit avoir}}\ S\Gamma_4' \isom  \gA_4& \\
S\Gamma_5'=SL(2,\mathbf{F}_5)/\pm 1 \ ({\rm d'ordre}\ 60) 
\subset  GP(1,\mathbf{F}_5)\ ({\rm d'ordre}\ 120) \  \isom \ {\hbox{groupe du 
bi-icosaèdre}}&
\end{cases}
\leqno(74)
$$
Sauf erreur, ces groupes sont respectivement les groupes $\gA_4$ du
{\it tétraédre} orienté et celui $\gA_5$ de l'icosaèdre orienté.
Si je me rappelle bien, les cas $n=2,3,4,5$ épuisent les cas de 
courbes modulaires congruentielles $M_{1,1}[n]$ qui soient {\it rationnelles}.
C'est une chose remarquable qu'on trouve par exemple tous les polyèdres
réguliers finis à faces des triangles [sauf un, à arêtes
repliées - celui de la figure (76), cf. plus bas\dots]; il était clair
a priori, par l'isomorphisme entre $\gT_{1,1}'$ et le ``groupe 
cartographique triangulé orienté'' $\pi_{0,3}'$, qu'on devrait 
retrouver tous les polyèdres réguliers finis à faces des triangles 
de cette fa\c con, mais non pas que ce soit des sous-groupes de
congruences, très exactement.  Le tableau obtenu est alors le suivant:
\vskip .2cm
\quad\quad\ \ \ \ \ groupe $G=S\Gamma_n'$\ \ courbe modulaire  \ \
type du polyèdre
$$
\begin{cases}
D_3 \isom \gS_3\quad\quad\ \ \ \ M_{1,1}[2]'& \text{triangle sphérique}\\
\gA_4\quad\quad\quad\quad M_{1,1}[3]& \text{tétraèdre}\\
\gS_4 \quad\quad\quad\quad M_{1,1}[4]& \text{octaèdre} \\
\gA_5\quad\quad\quad\quad M_{1,1}[5]& \text{icosaèdre}
\end{cases}
\leqno(75)
$$
\vskip .2cm
Comme autres courbes modulaires (pas congruentielles) rationnelles
galoisiennes sur $\widetilde M_{1,1}$, il y aurait encore les
quotients des précédents par les sous-groupes invariants de
$S\Gamma_n'$, distincts du groupe entier et de 1.  On trouve
comme quotients possibles:

a) Cas $\gS_3$: le quotient $\{\pm1\}$ (via signature).

d) Cas $\gA_4$: le quotient ${\mathbf{Z}}/3{\mathbf{Z}}=\gA_3$ (via $\gS_4\to\gS_3$
induisant $\gA_4\to\gA_3$).

c) Cas $\gS_4$: le quotient $\gS_3$, et le quotient $\{\pm1\}$ de celui-ci.

(pour d il n'y a rien, $\gA_5$ étant un groupe simple).
\vskip .2cm
Les courbes modulaires rationnelles obtenues se réduisent aux deux cas
a) et b) (qui sont retrouvés dans c)).  Dans le cas a), on trouve le
revêtement quadratique de $Y=\P_1(\mathbf{C})\setminus\{0,1,\infty\}$, qui
est non ramifié en $\infty$ (car l'indice de ramification devrait
diviser 2 et 3), qui correspond donc au revêtement $y=\sqrt{x(x-1)}$
et à la carte sphérique de type (2,1) (sommets d'indice 2, faces
d'indice 1), formée d'un sommet avec une arête (équateur) 
délimitant deux faces qui sont des monogones.

Le cas b) est le revêtement cyclique d'ordre 3, ramifié seulement 
en 0 et en $\infty$ (en 1, l'indice de ramification doit être diviseur
de 2 et de 3, donc est 1) avec carte sphérique de type (3,3), avec
des arêtes qui sont des boucles repliées:

\noindent (76) 1 seul sommet, 3 arêtes repliées qui en sortent,
une face triangulaire.

Les variétés modulaires sont trop proches de $\widetilde M_{1,1}$
-- avec un groupe d'automorphismes (quotient de $\Gamma_\infty'=SL(2,{\mathbf{Z}})$)
trop petit, pour pouvoir être rigidifiantes; il faudrait que le groupe
de Galois-Poincaré du revêtement soit multiple de 12 (multiple de
4 et 6!), pour que la courbe modulaire ait une chance d'être
rigidifiante.  Ces cas semblent donc nettement moins intéressants.

Il y aurait lieu par ailleurs, dans chacun des trois cas restants de
(75) (mis à part donc le premier cas, qui correspond à $M_{1,1}[2]'$
et est à peu près compris), d'expliciter la relation entre les 
points des courbes rationnelles correspondantes, et entre courbes
elliptiques à rigidification de Jacobi, d'échelon respectivement
3,4,5.  Il y a s\^urement des choses précises dans Klein, mais on
aimerait faire des choses un peu plus fines, qui soient valables
simultanément en toute caractéristique, i.e. sur des schémas
de base généraux.  Ce souci semble intuitivement lié à la question
des opérations de $\GG_{\mathbf{Q}}$ sur les classes d'isomorphismes des cartes,
et plus particulièrement des cartes sphériques régulières.  C'est
assez extraordinaire que [?], depuis trois ans que la question est là,
n'y ait pas encore touché.  Dieu sait qu'elle est juteuse!
\vskip .2cm
\noindent {\bf Complément sur les rapports anharmoniques:}

Soient $x_1,x_2,x_3,x_4$ des sections partout distinctes de $\P^1_S$, on veut 
construire la section correspondante de 
$U_{0,3}=\P^1_S\setminus \{0,1,\infty\}_S$.  Soient $a,b,c,d$ sections
locales de $\underline{O}_S$, avec $ad-bc$ inversible, telles que:
$$u=u_{a,b,c,d}:z\mapstochar\to {az+b \over cz+d}\leqno(77)$$
transforme $(x_1, x_2, x_3)$ en $(0,1,\infty)$ -- on aura donc
$$\lambda = u(x_4).\leqno(78)$$
Les conditions sur $(a,b,c,d)$ pour $ux_1=0$ etc sont
$$ax_1=0,\ \ ax_2+b=cx_2+d,\ \ cx_3+d=0$$
qui donnent, si par exemple $a$ est inversible (sinon on peut
supposer $c$ inversible), en résolvant en $b,c,d$
[ici quelques lignes de calcul élémentaire omises]:
$$ \lambda(x_1,x_2,x_3,x_4) = {(x_4-x_1)(x_2-x_3) \over 
(x_2-x_1)(x_4-x_3)}. \leqno(79)$$
En tant que fonction rationnelle en $x_1,x_2,x_3,x_4$ (à coefficients
dans un anneau intègre fixé), $\lambda$ est caractérisé par
les deux propriétés: 

a) invariance par rapport à une (même) transformation
homographique sur les variables $x_1,x_2,x_3,x_4$,
$f(ux_1, ux_2,ux_3,ux_4)=f(x_1,x_2,x_3,x_4)$, et

b)$f(0,1,\infty,x_4)=x_4$ (avec un grain de sel pour donner
un sens au premier membre).  

\noindent On a de plus

c) une propriété remarquable de symétrie, qu'on explicite en 
définissant un homomorphisme de $\gS_4$ dans le groupe homographique
(moralement, $GP(1)$)
\[\begin{tikzcd}
	{\sigma \in \gS_4} && {\GP(1)} \\
	& {\gS_3}
	\arrow[from=1-1, to=2-2]
	\arrow[from=2-2, to=1-3]
	\arrow["{\sigma \mapsto \phi_0}", from=1-1, to=1-3]
\end{tikzcd}\]
de telle fa\c con qu'on ait
$$\lambda (x_{\sigma^{-1}1},x_{\sigma^{-1}2},x_{\sigma^{-1}3},x_{\sigma^{-1}4})
= \phi_\sigma \bigl(\lambda(x_1,x_2,x_3,x_4)\bigr)\dots\leqno(80)$$









%End
